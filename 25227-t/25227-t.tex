% %%%%%%%%%%%%%%%%%%%%%%%%%%%%%%%%%%%%%%%%%%%%%%%%%%%%%%%%%%%%%%%%%%%%%%% %
%                                                                         %
% The Project Gutenberg EBook of Sur quelques applications des fonctions  %
% elliptiques, by Charles Hermite                                         %
%                                                                         %
% This eBook is for the use of anyone anywhere at no cost and with        %
% almost no restrictions whatsoever.  You may copy it, give it away or    %
% re-use it under the terms of the Project Gutenberg License included     %
% with this eBook or online at www.gutenberg.org                          %
%                                                                         %
%                                                                         %
% Title: Sur quelques applications des fonctions elliptiques              %
%                                                                         %
% Author: Charles Hermite                                                 %
%                                                                         %
% Release Date: April 30, 2008 [EBook #25227]                             %
%                                                                         %
% Language: French                                                        %
%                                                                         %
% Character set encoding: ISO-8859-1                                      %
%                                                                         %
% *** START OF THIS PROJECT GUTENBERG EBOOK SUR QUELQUES APPLICATIONS *** %
%                                                                         %
% %%%%%%%%%%%%%%%%%%%%%%%%%%%%%%%%%%%%%%%%%%%%%%%%%%%%%%%%%%%%%%%%%%%%%%% %

\def\ebook{25227}
%%%%%%%%%%%%%%%%%%%%%%%%%%%%%%%%%%%%%%%%%%%%%%%%%%%%%%%%%%%%%%%%%%%%%%%%%%%%
%% Sur Quelques Applications des Fonctions Elliptiques by Charles Hermite %%
%%                                                                        %%
%%                                                                        %%
%% Packages used:                                                         %%
%%                                                                        %%
%% amsmath  : Mathematics                                                 %%
%% amssymb  : Basic AMS symbols                                           %%
%% babel    : French language Hyphenation                                 %%
%% inputenc : Encoding                                                    %%
%%                                                                        %%
%% PDF pages: 165                                                         %%
%% one overfull hbox (math display)                                       %%
%%                                                                        %%
%% Compile sequence:                                                      %%
%%                                                                        %%
%% pdflatex x2                                                            %%
%%                                                                        %%
%% Compile history:                                                       %%
%%                                                                        %%
%% 2007-Dec-13. Compiled with pdflatex (rfrank)                           %%
%%   using MiKTeX-pdfTeX 2.6.2748 (1.40.4) (MiKTeX 2.6)                   %%
%% April 08: dcwilson.                                                    %%
%%           Compiled with pdfLaTeX TWO times.                            %%
%%           MiKTeX 2.7, Windows XP Pro                                   %%
%%                                                                        %%
%%%%%%%%%%%%%%%%%%%%%%%%%%%%%%%%%%%%%%%%%%%%%%%%%%%%%%%%%%%%%%%%%%%%%%%%%%%%

\documentclass[11pt,leqno,oneside,letterpaper]{book}[2005/09/16]
\usepackage{amsmath}[2000/07/18] % required
\usepackage{amssymb}[2002/01/22] % required
\usepackage[french]{babel}[2005/11/23] % required
\usepackage[latin1]{inputenc}[2006/05/05] % probably unnecessary: accented characters are in old TeX form
%\usepackage[T1]{fontenc} % strongly recommended for other font or paper sizes

\newcommand{\mysection}[1]{\section*{\begin{center}#1\end{center}}\setcounter{footnote}{0}}

%% Additional operators:
\DeclareMathOperator{\sn}{sn}
\DeclareMathOperator{\cn}{cn}
\DeclareMathOperator{\dn}{dn}
\DeclareMathOperator{\am}{am}
\DeclareMathOperator{\Al}{Al}
\DeclareMathOperator{\arctang}{arc tang}
\newcommand{\cntrdot}{\mathbin{.}}

%% Either the first three or the second three of the following commands, for dots representing omitted equations:
\newcommand{\dotfillalign}{\multispan{2}{\dotfill}}
\newcommand{\dotfillalignat}{\multispan{4}{\dotfill}}
\newcommand{\dotfillgather}{\makebox[15em]{\dotfill}}
%\newcommand{\dotfillalign}{&\vdots}
%\newcommand{\dotfillalignat}{&&&\vdots}
%\newcommand{\dotfillgather}{\vdots}

%% A4 PDF
\setlength{\pdfpagewidth}{210truemm}% or \paperwidth to get it from the documentclass
\setlength{\pdfpageheight}{297truemm}% or \paperheight to get it from the documentclass

\begin{document}
%-----Title.png-------------------------
\frontmatter
\thispagestyle{empty}
{\small
\begin{verbatim}
The Project Gutenberg EBook of Sur quelques applications des fonctions
elliptiques, by Charles Hermite

This eBook is for the use of anyone anywhere at no cost and with
almost no restrictions whatsoever.  You may copy it, give it away or
re-use it under the terms of the Project Gutenberg License included
with this eBook or online at www.gutenberg.org


Title: Sur quelques applications des fonctions elliptiques

Author: Charles Hermite

Release Date: April 30, 2008 [EBook #25227]

Language: French

Character set encoding: ISO-8859-1

*** START OF THIS PROJECT GUTENBERG EBOOK SUR QUELQUES APPLICATIONS ***
\end{verbatim}}
\clearpage
\thispagestyle{empty}
\null\vfil
\begin{center}
Produced by K.F. Greiner, Joshua Hutchinson and the Online
Distributed Proofreading Team at http://www.pgdp.net (This
file was produced from images generously made available
by Cornell University Digital Collections)
\end{center}
\vfil\clearpage
\thispagestyle{empty}
\begin{center}
\vspace{1cm}
{\LARGE SUR QUELQUES APPLICATIONS}
\bigskip\bigskip

{\large DES}
\bigskip\bigskip

{\Huge FONCTIONS ELLIPTIQUES,}

\bigskip\bigskip\bigskip\bigskip\bigskip\bigskip

\textsc{\LARGE Par M. Ch. HERMITE.}

\vfill

{\LARGE PARIS,}
\bigskip\bigskip

GAUTHIER-VILLARS, IMPRIMEUR-LIBRAIRE

DES COMPTES RENDUS DES S\'EANCES DE L'ACAD\'EMIE DES SCIENCES,
\bigskip

SUCCESSEUR DE MALLET-BACHELIER,

Quai des Augustins, 55.

\bigskip\bigskip

{\LARGE 1885}
\vspace{1cm}
\end{center}
\newpage
%-----006.png-----------------------
% [Blank Page]
%-----007.png-----------------------
\thispagestyle{empty}
\begin{center}
\vspace{3cm}
{\Large \`A LA M\'EMORIE} % original has no grave on A

\bigskip\bigskip\bigskip\bigskip

DE

\bigskip\bigskip\bigskip\bigskip

{\Huge C.-W. BORCHARDT.}
\end{center}
%-----008.png-----------------------
%[Blank Page]
%-----009.png-----------------------
\newpage
\mainmatter
\thispagestyle{empty}
\begin{center}
{\large SUR QUELQUES APPLICATIONS}\\[2ex]

{\small DES}\\[3ex]

{\LARGE FONCTIONS ELLIPTIQUES.}
\end{center}
\bigskip\bigskip\bigskip


La th\'eorie analytique de la chaleur donne pour l'importante question
de l'\'equilibre des temp\'eratures d'un corps solide homog\`ene, soumis \`a des
sources calorifiques constantes, une \'equation aux diff\'erences partielles
dont l'int\'egration, dans le cas de l'ellipso\"ide, a \'et\'e l'une des belles d\'ecouvertes
auxquelles est attach\'e le nom de Lam\'e. Les r\'esultats obtenus par
l'illustre g\'eom\`etre d\'ecoulent principalement de l'\'etude approfondie d'une
\'equation diff\'erentielle lin\'eaire du second ordre, que j'\'ecrirai avec les
notations de la th\'eorie des fonctions elliptiques, sous la forme suivante:
\[
\frac{d^2 y}{dx^2} = \left[ n(n+1) k^2 \sn^2 x + h \right] y,
\]
$k$ \'etant le module, $n$ un nombre entier et $h$ une constante. Lam\'e a montr\'e
que, pour des valeurs convenables de cette constante, on y satisfait par des
polyn\^omes entiers en $\sn x$:
\[
y = \sn^n x + h_1 \sn^{n-2} x + h_2 \sn^{n-4} x + \ldots,
\]
dont les termes sont de m\^eme parit\'e, puis encore par ces expressions:
\begin{alignat*}{4}
&y &&= (\sn^{n-1} x + h_1' \sn^{n-3} x &&+ h_2' \sn^{n-5} x &&+ \ldots) \cn x, \\
&y &&= (\sn^{n-1} x + h_1'' \sn^{n-3} x &&+ h_2'' \sn^{n-5} x &&+ \ldots) \dn x, \\
&y &&= (\sn^{n-2} x + h_1''' \sn^{n-4} x &&+ h_2''' \sn^{n-6} x &&+ \ldots) \cn x \dn x.
\end{alignat*}

M. Liouville a ensuite introduit, dans la question physique, la con\-sid\'era\-tion
de la seconde solution de l'\'equation diff\'erentielle, d'o\`u il a tir\'e des
th\'eor\`emes du plus grand int\'er\^et~(\footnote{
  \textit{Comptes rendus}, 1845, $1^{\text{er}}$ semestre, p.~1386
  et 1609; \textit{Journal de Math\'ematiques}, t.~XI, p.~217 et 261.}).
C'est \'egalement cette seconde solution,
dont la nature et les propri\'et\'es ont \'et\'e approfondies par M. Heine,
qui a montr\'e l'analogie de ces deux genres de fonctions de Lam\'e avec
%-----010.png-----------------------
les fonctions sph\'eriques, et leurs rapports avec la th\'eorie des fractions continues
alg\'ebriques. On doit de plus \`a l'\'eminent g\'eom\`etre une extension
de ses profondes recherches \`a des \'equations diff\'erentielles lin\'eaires du
second ordre beaucoup plus g\'en\'erales, qui se rattachent aux int\'egrales
ab\'eliennes, comme celle de Lam\'e aux fonctions elliptiques~(\footnote{
  \textit{Journal de Crelle} (\textit{Beitrag zur Theorie der Anziehung und der W\"arme}, t.~29);
\textit{Journal de M.~Borchardt} (\textit{Ueber die Lam\'eschen Functionen; Einige Eigenschaften der
Lam\'eschen Functionen}, dans le t.~56, et \textit{Die Lam\'eschen Functionen verschiedener Ordnungen}, t.~57). Le premier de ces M\'emoires, paru en 1845, mais dat\'e du 19~avril 1844, contient
une application de la seconde solution de l'\'equation de Lam\'e, qui a \'et\'e par cons\'equent
d\'ecouverte par M.~Heine, ind\'ependamment des travaux de M.~Liouville, et \`a la m\^eme \'epoque.}).

Je me suis plac\'e \`a un autre point de vue en me proposant d'obtenir,
quel que soit $h$, l'int\'egrale g\'en\'erale de cette \'equation, et c'est l'objet principal
des recherches qu'on va lire. On verra que la solution est toujours,
comme dans les cas particuliers consid\'er\'es par Lam\'e, une fonction uniforme
de la variable, mais qui n'est plus doublement p\'eriodique. Elle est,
en effet, donn\'ee par la formule
\[
y = C F(x) + C' F(-x),
\]
o\`u
la fonction $F(x)$, qui satisfait \`a ces deux conditions
\begin{alignat*}{2}
&F(x + 2 K) &&= \mu F(x),\\
&F(x + 2iK') &&= \mu' F(x),
\end{alignat*}
dans lesquelles les facteurs $\mu$ et $\mu'$ sont des constantes, s'exprime comme
il suit. Soit, pour un moment,
\[
\Phi(x) = \frac{H(x+\omega)}{\Theta(x)}\,
e^{\left[\lambda-\frac{\Theta'(\omega)}{\Theta(\omega)} \right] x} ,
\]
nous aurons
\[
F(x) = D^{n-1}_{x} \Phi(x)
- A_1 D^{n-3}_{x} \Phi(x)
+ A_2 D^{n-5}_{x} \Phi(x)
- \ldots ;
\]
les quantit\'es $\sn^2 \omega$ et $\lambda^2$ sont des fonctions rationnelles du module et de
$h$, et les coefficients $A_1$, $A_2$, \ldots, des fonctions enti\`eres. On a, par exemple,
\begin{align*}
A_1 &= \tfrac{(n-1)(n-2)}{2(2n-1)}
\left[ h + \tfrac{n(n+1)(1+k^2)}{3} \right], \\
A_2 &= \tfrac{ (n-1) (n-2) (n-3) (n-4) }{ 8 (2n-1) (2n-3) } \\
\times& \left[ h^2 + \tfrac{2n (n+1) (1+k^2)}{3} h
 + \tfrac{n^2 (n+1)^2}{9} (1+k^2)^2
  - \tfrac{2n (n+1) (2n-1)}{15} (1-k^2+k^4) \right],\\
&\qquad\dotfillgather
\end{align*}
%-----011.png----------------------------

Je m'occuperai, avant de traiter le cas g\'en\'eral o\`u le nombre $n$ est
quelconque, des cas particuliers de $n=1$ et $n=2$. Le premier s'applique
\`a la rotation d'un corps solide autour d'un point fixe, lorsqu'il n'y a point
de forces acc\'el\'eratrices, et nous conduira aux formules donn\'ees par Jacobi
dans son admirable M\'emoire sur cette question (\textit{{\OE}uvres compl\`etes}, t.~II,
p.~139, et \textit{Comptes rendus}, 30~juillet 1849). J'y rattacherai encore la
d\'etermination de la figure d'\'equilibre d'un ressort, qui a \'et\'e le sujet de travaux
de Binet et de Wantzel (\textit{Comptes rendus}, 1844, $1^{\mathrm{er}}$ semestre, p.~1115 et 1197).
Le second se rapportant au pendule sph\'erique, j'aurai ainsi r\'euni
quelques-unes des plus importantes applications qui aient \'et\'e faites jusqu'ici
de la th\'eorie des fonctions elliptiques.


\mysection{I.}


La m\'ethode que je vais exposer, pour int\'egrer l'\'equation de Lam\'e,
repose principalement sur des expressions, par les quantit\'es $\Theta(x)$, $H(x)$, \ldots,
des fonctions $F(x)$ satisfaisant aux conditions \'enonc\'ees tout \`a l'heure
\begin{alignat*}{3}
&F(x + 2 K) &&= \mu && F(x),\\
&F(x + 2iK') &&= \mu' && F(x),
\end{alignat*}
qui s'obtiennent ainsi:

Soit, en d\'esignant par $A$ un facteur constant,
\[
f(x) = A \frac{H(x+\omega) e^{\lambda x}}{H(x)} ;
\]
les relations fondamentales
\begin{alignat*}{2}
&H(x+2 K) &&= -H(x),\\
&H(x+2iK') &&= -H(x) e^{-\frac{i\pi}{K} (x+iK')}
\end{alignat*}
donneront celles-ci:
\begin{alignat*}{2}
&f(x+2 K) &&= f(x) e^{2\lambda K},\\
&f(x+2iK') &&= f(x) e^{-\frac{i\pi\omega}{K} + 2i\lambda K'}.
\end{alignat*}

Disposant donc de $\omega$ et $\lambda$ de mani\`ere \`a avoir
\begin{alignat*}{2}
&\mu &&= e^{2\lambda K} ,\\
&\mu' &&= e^{-\frac{i\pi\omega}{K} + 2i\lambda K' },
\end{alignat*}
%-----012.png----------------------
on voit que le quotient $\frac{F(x)}{f(x)}$ est ramen\'e aux fonctions doublement p\'eriodi\-ques,
d'o\`u cette premi\`ere forme g\'en\'erale et dont il sera souvent fait
usage:
\[
F(x) = f(x) \Phi(x),
\]
la fonction $\Phi(x)$ n'\'etant assujettie qu'aux conditions
\[
\Phi(x + 2 K ) = \Phi(x), \qquad
\Phi(x + 2iK') = \Phi(x).
\]

En voici une seconde, qui est fondamentale pour notre objet. Je
remarque que les relations
\begin{alignat*}{3}
& f(x + 2K ) &&= \mu && f(x),\\
& f(x + 2iK') &&= \mu' && f(x),
\end{alignat*}
ont pour cons\'equence celles-ci:
\begin{alignat*}{3}
& f(x - 2K) &&= \tfrac{1}{\mu } && f(x),\\
& f(x - 2iK') &&= \tfrac{1}{\mu'} && f(x),
\end{alignat*}
de sorte que le produit
\[
\Phi(z) = F(z) f(x-z)
\]
sera, quel que soit $x$, une fonction doublement p\'eriodique de $z$. Cela \'etant,
nous allons calculer les r\'esidus de $\Phi(z)$, pour les diverses valeurs de l'argument
qui la rendent infinie, dans l'int\'erieur du rectangle des p\'eriodes; et,
en \'egalant leur somme \`a z\'ero, nous obtiendrons imm\'ediatement l'expression
cherch\'ee. Remarquons \`a cet effet que $f(x)$ ne devient infinie qu'une
fois pour $x = 0$, et que, son r\'esidu ayant pour valeur
\[
  \frac{AH(\omega)}{H'(0)},
\]
on peut disposer de $A$, de mani\`ere \`a le faire \'egal \`a l'unit\'e. Posant donc,
en adoptant cette d\'etermination,
\[
  f(x) = \frac{ H'(0) H(x+\omega)e^{\lambda x} }{ H(\omega) H(x) },
\]
on voit que le r\'esidu correspondant \`a la valeur $z = x$ de $\Phi(z)$ sera
$-F(x)$. Ceux qui proviennent des p\^oles de $F(z)$ s'obtiennent ensuite
%-----013.png-----------------
sous la forme suivante. Soit $z = a$ l'un d'eux, et posons en cons\'equence,
pour $\varepsilon$ infiniment petit,
\begin{align*}
F(a+\varepsilon) &=
  A \varepsilon^{-1} +
  A_1 D_{\varepsilon} \varepsilon^{-1} +
  A_2 D_{\varepsilon}^2 \varepsilon^{-1} +
  \ldots +
  A_{\alpha} D_{\varepsilon}^{\alpha} \varepsilon^{-1} \\
  &\quad\ + a_0 + a_1 \varepsilon + a_2 \varepsilon^2 + \ldots, \\
f(x-a-\varepsilon) &=
  f(x-a) - \frac{\varepsilon}{1} D_x f(x-a) \\
&\quad\ +\frac{\varepsilon^2}{1 \cntrdot 2} D_x^2 f(x-a) -
  \ldots +
  \frac{(-1)^{\alpha} \varepsilon^{\alpha}}{1 \cntrdot 2 \ldots \alpha} D_x^{\alpha} f(x-a) + \ldots,
\end{align*}
le coefficient du terme en $\frac{1}{\varepsilon}$ dans le produit des seconds membres, qui est
la quantit\'e cherch\'ee, se trouve imm\'ediatement, en remarquant que
\[
D_{\varepsilon}^n \varepsilon^{-1} = (-1)^n \frac{1\cntrdot 2\ldots n}{\varepsilon^{n+1}},
\]
et a pour expression
\[
A f(x-a) + A_1 D_x f(x-a) +
A_2 D_x^2 f(x-a) + \ldots +
A_{\alpha} D_x^{\alpha} f(x-a) .
\]
La somme des r\'esidus de la fonction $\Phi(z)$, \'egal\'ee \`a z\'ero, nous conduit
ainsi \`a la relation\label{page5}
\[
F(x) = \sum \left[
A f(x-a) + A_1 D_x f(x-a) + \ldots +
A_{\alpha} D_x^{\alpha} f(x-a) \right] ,
\]
o\`u le signe $\sum$ se rapporte, comme il a \'et\'e dit, \`a tous les p\^oles de $F(z)$ qui
sont \`a l'int\'erieur du rectangle des p\'eriodes.


\mysection{II.}


La fonction $F(x)$ comprend les fonctions doublement p\'eriodiques;
en supposant \'egaux \`a l'unit\'e les multiplicateurs $\mu$ et $\mu'$, je vais imm\'ediatement
rechercher ce que l'on tire, dans cette hypoth\`ese, du r\'esultat auquel
nous venons de parvenir. Tout d'abord les relations
\[
\mu = e^{2\lambda K}, \qquad
\mu' = e^{-\frac{i\pi\omega}{K} + 2i\lambda K }
\]
donnant n\'ecessairement $\lambda=0$ et $\omega=2mK$, ou, ce qui revient au m\^eme,
$\omega=0$, le nombre $m$ \'etant entier, la quantit\'e
$f(x) = \frac{H'(0) H(x+\omega) }{ H(\omega) H(x)} e^{\lambda x}$
devient infinie et la formule semble inapplicable. Mais il arrive seulement
qu'elle subit un changement de forme analytique, qui s'obtient de la mani\`ere
la plus facile, comme on va voir. Supposons, en effet, $\lambda=0$ et $\omega$ infiniment
%-----014.png----------------------
petit, on aura, en d\'eveloppant suivant les puissances croissantes
de $\omega$,
\begin{gather*}
  \frac{H'(0)}{H(\omega)}
= \frac{1}{\omega}
+ \left( \frac{1+k^2}{6} - \frac{J}{2K}\right) \omega
+ \ldots,
\\
  \frac{H(x+\omega)}{H(x)}
= 1 + \frac{H'(x)}{H(x)}\omega + \ldots;
\end{gather*}
d'o\`u
\[
  f(x)
= \frac{1}{\omega} + \frac{H'(x)}{H(x)}
+ \left( \frac{1+k^2}{6} - \frac{J}{2K} \right) \omega
+ \ldots.
\]

D'autre part, observons que les coefficients $A$, $A_1$, \ldots\ doivent \^etre
con\-sid\'er\'es comme d\'ependants de $\omega$, et qu'on aura en particulier
\[
  A = \mathrm{a} + \mathrm{a}'\omega + \ldots,
\]
$\mathrm{a}$, $\mathrm{a}'$, \ldots\  d\'esignant les valeurs de $A$ et de ses d\'eriv\'ees par rapport \`a $\omega$ pour
$\omega = 0$. Nous obtenons donc, en n'\'ecrivant point les termes qui contiennent
$\omega$ en facteur,
\[
  A f(x-a) = \frac{\mathrm{a}}{\omega} + \mathrm{a}'
+ \mathrm{a}\frac{H'(x-a)}{H(x-a)} + \ldots
\]
et, par cons\'equent,
\[
  \sum A f(x-a)
= \frac{1}{\omega}\sum\mathrm{a} + \sum\mathrm{a}'
+ \sum\mathrm{a}\,\frac{H'(x-a)}{H(x-a)} + \ldots.
\]

Or on voit que le coefficient de $\frac{1}{\omega}$ dispara\^it, les quantit\'es $\mathrm{a}$ ayant une
somme nulle comme r\'esidus d'une fonction doublement p\'eriodique, et la
diff\'erentiation donnant imm\'ediatement, pour $\omega = 0$,
\[
  D_x f(x)  = D_x  \frac{H'(x)}{H(x)}, \qquad
  D^2_x f(x) = D^2_x \frac{H'(x)}{H(x)}, \qquad\ldots,
\]
nous parvenons \`a l'expression suivante, o\`u $\mathrm{a}$, $\mathrm{a}_1$,
\ldots, $\mathrm{a}_{\alpha}$ sont les valeurs de
$A$, $A_1$, \ldots, $A_{\alpha}$ pour $\omega = 0$:
\[
  F(x)
= \sum\mathrm{a}' + \sum\left[
  \mathrm{a}                   \tfrac{H'(x-a)}{H(x-a)}
+ \mathrm{a}_1      D_x        \tfrac{H'(x-a)}{H(x-a)} + \ldots
+ \mathrm{a}_{\alpha} D_x^{\alpha} \tfrac{H'(x-a)}{H(x-a)}
  \right].
\]

C'est la formule que j'ai \'etablie directement, pour les fonctions doublement
p\'eriodiques, dans une \textit{Note sur la th\'eorie des fonctions elliptiques},
ajout\'ee \`a la sixi\`eme \'edition du \textit{Trait\'e de Calcul diff\'erentiel et de Calcul int\'egral}
de Lacroix.
%-----015.png------------------


\mysection{III.}


Revenant au cas g\'en\'eral pour donner des exemples de la d\'etermination
de la fonction $f(x)$, qui joue le r\^ole d'\'el\'ement simple, et du calcul
des coefficients $A$, $A_1$, $A_2$, \ldots, je consid\'ererai ces deux expressions:
\begin{align*}
F(x)   &= \frac{\Theta (x+a) \Theta (x+b) \ldots \Theta(x+l) e^{\lambda x}}{\Theta^n(x)}, \\
F_1(x) &= \frac{H(x+a) H(x+b) \ldots H(x+l) e^{\lambda x}}{\Theta^n(x)},
\end{align*}
o\`u $a$, $b$, \ldots, $l$ sont des constantes au nombre de $n$. On trouve d'abord
ais\'ement leurs multiplicateurs, au moyen des relations
\begin{alignat*}{2}
&\Theta(x+2K)  &&= +\Theta(x),\\
&H(x+2K)       &&= -H(x),     \\
&\Theta(x+2iK')&&= -\Theta(x) e^{-\frac{i\pi}{K}(x+iK')}, \\
&H(x+2iK')     &&= -H(x)      e^{-\frac{i\pi}{K}(x+iK')}.
\end{alignat*}
Elles montrent qu'en posant
\[
\omega = a + b + \ldots + l, \\
\]
puis, comme pr\'ec\'edemment,
\begin{align*}
\mu &= e^{2\lambda K} , \\
\mu'&= e^{-\frac{i\pi\omega}{K} + 2i\lambda K'} ,
\end{align*}
on aura
\begin{alignat*}{5}
&F(x+2K ) &&= \mu &&F(x), &&\qquad F_1(x+2K ) &&= (-1)^n \mu F_1(x), \\
&F(x+2iK') &&= \mu'&&F(x), &&\qquad F_1(x+2iK') &&= \mu' \, F_1(x).
\end{alignat*}

Il en r\'esulte que, quand $n$ est pair, la fonction
\[
f(x) = \frac{ H'(0) H(x+\omega) e^{\lambda x}}{H(\omega) H(x) },
\]
ayant ces quantit\'es $\mu$ et $\mu'$ pour multiplicateurs, peut servir d'\'el\'ement
simple pour nos deux expressions; mais il n'en est plus de m\^eme relativement
\`a la seconde $F_1(x)$, dans le cas o\`u $n$ est impair: on voit ais\'ement qu'il
%-----016.png-----------------
faut prendre alors pour \'el\'ement simple la fonction
\[
f_1(x) =
\frac{H'(0) \Theta(x+\omega) e^{\lambda x}}{\Theta(\omega) H(x)} ,
\]
afin de changer le signe du premier multiplicateur, le r\'esidu correspondant
\`a $x = 0$ \'etant d'ailleurs \'egal \`a l'unit\'e. Cela pos\'e, comme $F(x)$ et $F_1(x)$
ne deviennent infinies que pour $x = iK'$, ce sont les quantit\'es $f(x-iK')$
et $f_1(x-iK')$ qui figureront dans notre formule. Il convient de leur
attribuer une d\'esignation particuli\`ere, et nous repr\'esenterons dor\'enavant
la premi\`ere par $\varphi(x)$\label{page8} et la seconde par $\chi(x)$, en observant que les relations
\begin{align*}
\Theta(x+iK') &=
iH(x) e^{-\frac{i\pi}{4K}(2x+iK')} ,\\
H(x+iK') &=
i\Theta(x) e^{-\frac{i\pi}{4K}(2x+iK')}
\end{align*}
donnent facilement, apr\`es y avoir chang\'e $x$ en $-x$, ces valeurs:
\begin{align*}
\varphi(x) &=
\frac{H'(0) \Theta(x+\omega) e^{\lambda x}}{\sqrt{\mu'} H(\omega) \Theta(x)}, \\
\chi(x) &=
\frac{H'(0) H(x+\omega) e^{\lambda x}}{\sqrt{\mu'} \Theta(\omega) \Theta(x)}.
\end{align*}

Nous avons maintenant \`a calculer dans les d\'eveloppements de
$F(iK'+\varepsilon)$ et $F_1(iK' + \varepsilon)$, suivant les puissances croissantes de $\varepsilon$, la partie
qui renferme les puissances n\'egatives de cette quantit\'e, et qu'on pourrait,
pour abr\'eger, nommer la partie principale. A cet effet, je remarque qu'en
faisant, pour un moment,
\[
F(x) = \frac{\Pi(x)}{\Theta^n(x)}, \qquad
F_1(x) = \frac{\Pi_1(x)}{\Theta^n(x)},
\]
on aura
\[
F(iK' +\varepsilon) = \frac{\sqrt{\mu'} \Pi_1(\varepsilon)}{H^n(\varepsilon)}, \qquad
F_1(iK' +\varepsilon) = \frac{\sqrt{\mu'} \Pi(\varepsilon)}{H^n(\varepsilon)}.
\]

Nous d\'evelopperons donc $\Pi(\varepsilon)$ et $\Pi_1(\varepsilon)$, par la formule de Maclaurin,
jusqu'aux termes en $\varepsilon^{n-1}$, et nous multiplierons par la partie principale de
$\frac{1}{H^n(\varepsilon)}$, qui s'obtient, comme on va voir, au moyen de la fonction de
M.~Weier\-strass:
\[
\Al(x)_1 = x - \frac{1+k^2}{6} x^3 + \frac{1+4k^2+k^4}{120} x^5 - \ldots.
\]
%-----017.png-----------------

On a en effet, d'apr\`es la d\'efinition m\^eme de l'illustre analyste,
\[
H(x) =
H'(0) e^{\frac{J x^2}{2K}} \Al(x)_1 ,
\]
et l'on en d\'eduit
\begin{align*}
\left[ \frac{H'(0)}{H(\varepsilon)} \right]^n
&=
e^{-\frac{n J \varepsilon^2}{2K}}
\left[ \varepsilon - \frac{1+k^2}{6} \varepsilon^3 +
\frac{1+4k^2+k^4}{120} \varepsilon^5 - \ldots \right]^{-n} \\
&=
e^{-\frac{n J \varepsilon^2}{2K}}
\left[ \frac{1}{\varepsilon^n} + \frac{n(1+k^2)}{6} \frac{1}{\varepsilon^{n-2}} + \ldots \right] \\
&=
\frac{1}{\varepsilon^n} +
n \left( \frac{1+k^2}{6} - \frac{J}{2K} \right) \frac{1}{\varepsilon^{n-2}} + \ldots .
\end{align*}


\mysection{IV.}


Je vais appliquer ce qui pr\'ec\`ede au cas le plus simple, en supposant
$n = 2$ et $\lambda = 0$, ce qui donnera
\begin{align*}
F(x) &=
\frac{\Theta(x+a) \Theta(x+b)}{\Theta^2(x)} ,\\
F_1(x) &=
\frac{H(x+a) H(x+b)}{\Theta^2(x)} ,
\end{align*}
et, par cons\'equent
\begin{alignat*}{3}
\frac{1}{\sqrt{\mu'}} \Pi(\varepsilon) &=
H(a) H(b) &&+
\left[H(a) H'(b) + H(b) H'(a)\right] \varepsilon &&+ \ldots, \\
\frac{1}{\sqrt{\mu'}} \Pi_1(\varepsilon) &=
\Theta(a) \Theta(b) &&+
\left[\Theta(a) \Theta'(b) + \Theta(b) \Theta'(a)\right] \varepsilon &&+ \ldots.
\end{alignat*}
Maintenant la partie principale de $\frac{1}{H^2(\varepsilon)}$ ne contenant que le seul terme
$\frac{1}{H'^2(0)}\frac{1}{\varepsilon^2}$, on a imm\'ediatement
\begin{alignat*}{4}
&\frac{H'^2(0)}{\sqrt{\mu'}} F(iK'+\varepsilon)
&&=
\frac{H(a) H(b)}{\varepsilon^2} &&+
\frac{H(a) H'(b) + H(b) H'(a)}{\varepsilon} &&+ \ldots, \\
&\frac{H'^2(0)}{\sqrt{\mu'}} F_1(iK'+\varepsilon)
&&=
\frac{\Theta(a) \Theta(b)}{\varepsilon^2} &&+
\frac{\Theta(a) \Theta'(b) + \Theta(b) \Theta'(a)}{\varepsilon} &&+ \ldots,
\end{alignat*}
et, par cons\'equent, ces deux relations
\begin{multline*}
\frac{H'^2(0) \Theta(x+a) \Theta(x+b)}{\sqrt{\mu'} \Theta^2(x)} \\
\shoveright{
= -H(a) H(b) \varphi'(x) +
\left[H(a) H'(b) + H(b) H'(a) \right] \varphi(x), }\\
\shoveleft{
\frac{H'^2(0) H(x+a) H(x+b)}{\sqrt{\mu'} \Theta^2(x)} } \\
=
-\Theta(a) \Theta(b) \varphi'(x) +
\left[\Theta(a) \Theta'(b) + \Theta(b) \Theta'(a) \right] \varphi(x) .
\end{multline*}
%-----018.png----------------

En y rempla\c{c}ant $\varphi(x)$ par sa valeur
\[
\frac{H'(0) \Theta(x+a+b)}{\sqrt{\mu'}H(a+b)\Theta(x)},
\]
je les \'ecrirai
sous la forme suivante, qui est plus simple:
\begin{multline*}
\frac{H'(0) H(a+b) \Theta(x+a) \Theta(x+b)}{H(a)H(b)\Theta^2(x)} \\
\shoveright{=
- D_x \frac{\Theta(x+a+b)}{\Theta(x)}
+ \left[ \frac{H'(a)}{H(a)} + \frac{H'(b)}{H(b)} \right]
\frac{\Theta(x+a+b)}{\Theta(x)}, }
\\
\shoveleft{
\frac{H'(0) H(a+b) H(x+a) H(x+b)}{\Theta(a)\Theta(b)\Theta^2(x)} }\\
=
- D_x \frac{\Theta(x+a+b)}{\Theta(x)}
+ \left[ \frac{\Theta'(a)}{\Theta(a)} + \frac{\Theta'(b)}{\Theta(b)} \right]
\frac{\Theta(x+a+b)}{\Theta(x)} .
\end{multline*}
On en tire d'abord, \`a l'\'egard des fonctions $\Theta$, cette remarque que, sous la
condition
\[
a+b+c+d=0,
\]
on a l'\'egalit\'e~(\footnote{Elle a \'et\'e donn\'ee par Jacobi, \textit{Journal de Crelle} (\textit{Formul{\ae} nov{\ae} in theoria transcendentium
ellipticarum fundamentales}, t.~15, p.~199).})
\begin{align*}
H'(0) H(a+b) H(a+c) H(b+c)
&= \Theta'(a) \Theta(b) \Theta(c) \Theta(d) \\
&+\, \Theta'(b) \Theta(c) \Theta(d) \Theta(a) \\
&+\, \Theta'(c) \Theta(d) \Theta(a) \Theta(b) \\
&+\, \Theta'(d) \Theta(a) \Theta(b) \Theta(c).
\end{align*}

Mais c'est une autre cons\'equence que j'ai en vue, et qu'on obtient en
mettant la premi\`ere, par exemple, sous la forme
\[
\Phi(x) = py - y',
\]
o\`u $\Phi(x)$ d\'esigne le premier membre, $y$ la fonction $\frac{\Theta(x+a+b)}{\Theta(x)}$ et $p$ la
constante $\frac{H'(a)}{H(a)}+\frac{H'(b)}{H(b)}$.

Si nous multiplions par $e^{-px}$, elle devient, en effet,
\[
\Phi(x)e^{-px} = -D_x (y e^{-px}),
\]
d'o\`u
\[
\int \Phi(x) e^{-px}\, dx = -y e^{-px}.
\]

Ce r\'esultat appelle l'attention sur un cas particulier des fonctions $\varphi(x)$,
%-----019.png----------------------------
o\`u, par suite d'une certaine d\'etermination de $\lambda$, elles ne renferment plus
qu'un param\`etre. On voit qu'en posant
\[
\varphi(x,a)
= \frac{H'(0) \Theta(x+a)}{\sqrt{\mu'} H(a) \Theta(x)}\:
e^{-\frac{H'(a)}{H(a)}x},
\]
ce qui entra\^ine, pour le multiplicateur $\mu'$, la valeur
\[
\mu' = e^{ -\frac{i\pi a}{K} - 2iK' \frac{H'(a)}{H(a)} },
\]
l'int\'egrale $\int\varphi(x, a)\varphi(x, b)\, dx$ s'obtient sous forme finie explicite. Un
calcul facile conduit en effet \`a la relation
\[
\int\varphi(x, a)\varphi(x, b)\, dx
= -\varphi(x,a+b)
e^{ \left[ \frac{H'(a+b)}{H(a+b)} - \frac{H'(a  )}{H(a  )}
  - \frac{H'(b  )}{H(b  )} \right](x-iK')}.
\]
Faisons, en second lieu,
\[
\chi(x,a) = \frac{ H'(0) H(x+a)  }{ \sqrt{\mu'} \Theta(a) \Theta(x) }\,
e^{ -\frac{\Theta'(a)}{\Theta(a)} x },
\]
en d\'esignant alors par $\mu'$ la quantit\'e
\[
\mu' = e^{ -\frac{i\pi a}{K} - 2iK'\frac{\Theta'(a)}{\Theta(a) }},
\]
et nous aurons semblablement
\[
\int\chi(x, a)\chi(x, b)\, dx
= -\varphi(x,a+b)
e^{ \left[ \frac{H'(a+b)}{H(a+b)} - \frac{\Theta'(a  )}{\Theta(a  )}
  - \frac{\Theta'(b  )}{\Theta(b  )} \right](x-iK') }.
\]
On en d\'eduit ais\'ement qu'en d\'esignant par $a$ et $b$ deux racines,
d'abord de l'\'equation $H'(x) = 0$, puis de l'\'equation $\Theta'(x) = 0$,
on aura, dans le premier cas,\label{page11}
\[
\int_0^{2K} \varphi(x,a)\varphi(x,b)\, dx = 0;
\]
et dans le second,
\[
\int_0^{2K} \chi(x,a)\chi(x,b)\, dx = 0,
\]
sous la condition que les deux racines ne soient point \'egales et de signes
contraires. Si l'on suppose $b = -a$, nous obtiendrons
\begin{align*}
\int_0^{2K} \varphi(x,a) \varphi(x,-a)\, dx
&= 2\left( J - \frac{K}{\sn^2 a} \right),
\\
\int_0^{2K} \chi(x,a) \chi(x,-a)\, dx
&= 2\left( J - k^2 K \sn^2 a \right).
\end{align*}
%-----020.png-----------------
On voit les recherches auxquelles ces th\'eor\`emes ouvrent la voie et que je
me r\'eserve de poursuivre plus tard; je me borne \`a les indiquer succinctement,
afin de montrer l'importance des fonctions $\varphi(x)$ et $\chi(x)$. Voici maintenant
comment on parvient \`a les d\'efinir par des \'equations diff\'erentielles.


\mysection{V.}\label{page12}


Nous remarquerons, en premier lieu, que les fonctions $\varphi(x)$ et $\chi(x)$
peuvent \^etre r\'eduites l'une \`a l'autre; leurs expressions, si l'on y remplace
le multiplicateur $\mu'$ par sa valeur, \'etant, en effet,
\begin{align*}
  \varphi(x,\omega) &=
  \frac{ H'(0)\,\Theta(x+\omega) }{ H(\omega)\,\Theta(x) }\,
  e^{- \frac{ H'(\omega) }{ H(\omega) }(x-iK')
     + \frac{ i\pi\omega }{ 2K }} ,
\\
  \chi(x,\omega) &=
  \frac{ H'(0)\,H(x+\omega) }{ \Theta(\omega)\,\Theta(x) }\,
  e^{- \frac{\Theta'(\omega) }{ \Theta(\omega) }(x-iK')
     + \frac{ i\pi\omega }{ 2K }} ,
\end{align*}
on en d\'eduit facilement les relations suivantes:
\begin{align*}
  \varphi(x,\omega+iK') &= \chi(x,\omega), \\
  \chi(x,\omega+iK') &= \varphi(x,\omega),
\end{align*}
dont nous ferons souvent usage. Cette propri\'et\'e \'etablie, nous rechercherons
le d\'eveloppement, suivant les puissances croissantes de $\varepsilon$, de $\chi(iK'+\varepsilon)$,
qui jouera plus tard un r\^ole important, et dont nous allons, comme on va
voir, tirer l'\'equation diff\'erentielle que nous avons en vue. Pour le former,
je partirai de l'\'egalit\'e
\[
  D_x \log\chi(x)
= \frac{ H'(x+\omega) }{ H(x+\omega) }
- \frac{ \Theta'(x) }{ \Theta(x) }
- \frac{ \Theta'(\omega) }{ \Theta(\omega) },
\]
d'o\`u l'on d\'eduit
\[
 D_{\varepsilon} \log\chi(iK' + \varepsilon)
= \frac{ \Theta'(\omega+\varepsilon) }{ \Theta(\omega+\varepsilon) }
- \frac{ H'(\varepsilon) }{ H(\varepsilon) }
- \frac{ \Theta'(\omega) }{ \Theta(\omega) }.
\]
Cela pos\'e, nous aurons d'abord
\[
  \frac{ \Theta'(\omega+\varepsilon) }{ \Theta(\omega+\varepsilon) }
- \frac{ \Theta'(\omega) }{ \Theta(\omega) }
= \varepsilon D_{\omega}
  \frac{ \Theta'(\omega) }{ \Theta(\omega) }
+ \frac{ \varepsilon^2 }{ 1\cntrdot 2 } D_{\omega}^2
  \frac{ \Theta'(\omega) }{ \Theta(\omega) } + \cdots ;
\]
mais, l'\'equation de Jacobi
\[
  D_x \frac{ \Theta'(x) }{ \Theta(x) }
= \frac{J}{K} - k^2 \sn^2 x
\]
donnant en g\'en\'eral
\[
   D_x^{n+1} \frac{ \Theta'(x) }{ \Theta(x) }
=- D_x^n k^2 \sn^2 x,
\]
%-----021.png----------------------------
ce d\'eveloppement prend cette nouvelle forme
\begin{multline*}
  \frac{\Theta'(\omega+\varepsilon)}{\Theta(\omega+\varepsilon)}
- \frac{\Theta'(\omega)}{\Theta(\omega)} \\
= \varepsilon \left( \frac{J}{K} - k^2\sn^2\omega \right)
- \frac{\varepsilon^2}{1\cntrdot 2}
  D_{\omega} k^2 \sn^2\omega
- \frac{\varepsilon^3}{1\cntrdot 2\cntrdot 3}
  D_{\omega}^2 k^2 \sn^2\omega - \ldots.
\end{multline*}
Joignons-y le r\'esultat qu'on tire de l'\'equation de M.~Weierstrass:
\[
  H(\varepsilon)
= H'(0)
  e^{ \frac{J\varepsilon^2}{2K} }
  \Al(\varepsilon)_1,
\]
en prenant la d\'eriv\'ee logarithmique des deux membres:
\[
  \frac{H'(\varepsilon)}{H(\varepsilon)}
= \varepsilon \frac{J}{K}
+ \frac{\Al'(\varepsilon)_1}{\Al(\varepsilon)_1},
\]
et nous aurons
\[
  D_{\varepsilon} \log \chi(iK' + \varepsilon)
= -\varepsilon k^2 \sn^2\omega
- \frac{\varepsilon^2}{1\cntrdot 2} D_{\omega} k^2 \sn^2\omega
- \cdots - \frac{\Al'(\varepsilon)_1}{\Al(\varepsilon)_1} ,
\]
d'o\`u, par cons\'equent,
\begin{multline*}
\chi(iK' + \varepsilon)
= \frac{e^{- \frac{\varepsilon^2}{2} k^2\sn^2\omega
            - \frac{\varepsilon^3}{2\cntrdot 3}
     D_{\omega} k^2\sn^2\omega - \ldots }}{\Al(\varepsilon)_1}
\\
= e^{- \frac{\varepsilon^2}{2} k^2\sn^2\omega
      - \frac{\varepsilon^3}{2\cntrdot 3}
        D_{\omega} k^2\sn^2\omega - \ldots }
   \left( \frac{1}{\varepsilon} + \frac{1+k^2}{6}\varepsilon
        + \frac{7+8k^2+7k^4}{360}\varepsilon^3 + \cdots \right),
\end{multline*}
sans qu'il soit besoin d'introduire un facteur constant dans le second
membre, puisque le premier terme de son d\'eveloppement est
$\frac{1}{\varepsilon}$, comme il le faut d'apr\`es la nature de la
fonction $\chi(x)$. Cette formule donne le r\'esultat cherch\'e par un
calcul facile; elle montre qu'en posant\label{page13}
\[
  \chi(iK' + \varepsilon)
= \frac{1}{\varepsilon} - \frac{1}{2}\Omega \varepsilon
- \frac{1}{3}\Omega_1 \varepsilon^2
- \frac{1}{8}\Omega_2 \varepsilon^3 + \ldots,
\]
on aura\label{page13a}
\begin{align*}
  \Omega\  &= k^2 \sn^2\omega - \frac{1+k^2}{3},  \\
  \Omega_1 &= k^2 \sn\omega\cn\omega\dn\omega,  \\
  \Omega_2 &= k^4 \sn^4\omega - \frac{2(k^2+k^4)}{3}\sn^2\omega
            - \frac{7-22k^2+7k^4}{45},  \\
  \dotfillalign
\end{align*}
En voici une premi\`ere application.
%-----022.png---------------------------


\mysection{VI.}


Consid\'erons, pour la d\'ecomposer en \'el\'ements simples, la fonction % not displayed in original
\[
k^2\sn^2 x\chi(x),
\]
qui a les multiplicateurs de $\chi(x)$ et ne devient infinie que pour $x = iK'$. On devra, \`a cet effet, en posant $x = iK' + \varepsilon$, former la
partie principale de son d\'eveloppement suivant les puissances croissantes
de $\varepsilon$, que nous obtenons imm\'ediatement en multipliant membre \`a membre
les deux \'egalit\'es
\begin{align*}
\chi(iK' + \varepsilon) & =
\frac{1}{\varepsilon} - \frac{1}{2} \Omega \varepsilon + \ldots, \\
\frac{1}{\sn^2 \varepsilon} & =
\frac{1}{\varepsilon^2} + \frac{1}{3} \left(1 + k^2\right) + \ldots.
\end{align*}
Il vient ainsi
\begin{align*}
k^2 \sn^2 (iK' + \varepsilon) \chi (iK' + \varepsilon) & =
\frac{1}{\varepsilon^3} + \left[\frac{1}{3}\left(1+k^{2}\right) - \frac{1}{2} \Omega \right] \frac{1}{\varepsilon} + \ldots \\
& = \frac{1}{2} D^2_{\varepsilon} \varepsilon^{-1} + \left[\frac{1}{2}\left( 1+k^2 \right) - \frac{1}{2} k^2 \sn^2 \omega \right] \varepsilon^{-1} + \ldots,
\end{align*}
et l'on en conclut la formule suivante:
\[
k^2 \sn^2 (x) \chi (x) = \frac{1}{2} D^2_x \chi (x) + \left[\frac{1}{2}\left( 1+k^2 \right) - \frac{1}{2} k^2 \sn^2 \omega \right] \chi (x).
\]
Elle montre que, en posant $y = \chi (x)$, nous obtenons une solution de
l'\'equa\-tion lin\'eaire du second ordre\label{page14}
\[
\frac{d^2 y}{dx^2} = \left(2k^2 \sn^2 x - 1 - k^2 + k^2 \sn^2 \omega \right)y,
\]
qui est celle de Lam\'e dans le cas le plus simple o\`u l'on suppose $n = 1$, la
constante $h = -1-k^2 + k^2 \sn^2 \omega$ \'etant quelconque, puisque $\omega$ est arbitraire;
et, comme cette \'equation ne change pas lorsqu'on change $x$ en
$-x$, la solution obtenue en donne une seconde, $y = \chi(-x)$, d'o\`u, par
suite, l'int\'egrale compl\`ete sous la forme
\[
y = C \chi(x) + C' \chi(-x).
\]
A ce r\'esultat il est n\'ecessaire de joindre ceux qu'on obtient quand on remplace
successivement $\omega$ par $\omega + iK'$, $\omega + K$, $\omega + K + iK'$, ce qui conduit
aux \'equations
\begin{align*}
\frac{d^2 y}{dx^2} & =  \left( 2k^2 \sn^2 x - 1 - k^2 + \frac{1}{\sn^2 \omega} \right) y, \\
\frac{d^2 y}{dx^2} & =  \left( 2k^2 \sn^2 x - 1 - k^2 + \frac{k^2 \cn^2 \omega}{\dn^2 \omega} \right) y, \\
\frac{d^2 y}{dx^2} & =  \left( 2k^2 \sn^2 x - 1 - k^2 + \frac{\dn^2 \omega}{\cn^2 \omega} \right) y.
\end{align*}
%-----023.png---------------------------
La premi\`ere, d'apr\`es l'\'egalit\'e $\chi(x,\omega + iK') = \varphi (x,\omega)$, a pour int\'egrale
\[
y = C \varphi (x) + C' \varphi (-x) ;
\]
et, en introduisant ces nouvelles fonctions, \`a savoir:
\begin{align*}
i \chi_1 (x,\omega) & =  \chi (x,\omega + K), \\
i \varphi_1 (x,\omega) & =  \varphi (x,\omega + K),
\end{align*}
nous aurons, sous une forme semblable, pour la seconde et la troisi\`eme:
\begin{align*}
y & = C \chi_1 (x) + C'  \chi_1 (-x),\\
y & = C \varphi_1 (x) + C' \varphi_1 (-x).
\end{align*}
Les expressions de $\varphi_1 (x)$ et $\chi_1 (x)$ s'obtiennent ais\'ement \`a l'aide des fonctions
$\Theta_1(x) = \Theta (x + K)$, $H_1 (x) = H(x + K)$; on trouve ainsi
\begin{align*}
\varphi_1(x,\omega) & =
\frac{H'(0)\Theta_1 (x+\omega)}{H_1(\omega)\Theta(x)}\,
e^{ -\frac{H'_1(\omega)}{H_1(\omega)} (x-iK')
    + \frac{i\pi \omega}{2K} } ,
\\
\chi_1(x,\omega) & =
\frac{H'(0) H_1(x+\omega)}{\Theta_1(\omega)\Theta(x)}\,
e^{ -\frac{\Theta'_1(\omega)}{\Theta_1(\omega)} (x-iK')
    + \frac{i\pi \omega}{2K}} .
\end{align*}
Nous allons en voir un premier usage dans la recherche des solutions de
l'\'equation de Lam\'e par des fonctions doublement p\'eriodiques.


\mysection{VII.}


Nous supposons \`a cet effet $\omega = 0$ dans les \'equations pr\'ec\'edentes,
en exceptant toutefois celle o\`u se trouve le terme $\frac{1}{\sn^2 \omega}$ qui deviendrait infini.
On obtient ainsi, pour la constante $h$, les d\'eterminations suivantes:
\[
h = 1 - k^2, \quad h = -1, \quad h = -k^2.
\]
Ce sont pr\'ecis\'ement les quantit\'es qu'on trouve en appliquant la m\'ethode de
Lam\'e; et en m\^eme temps nous tirons des valeurs des fonctions $\chi(x)$,
$\chi_1(x)$, $\varphi_1(x)$, pour $\omega = 0$, les solutions auxquelles conduit son analyse:
\[
y = \sqrt{k  } \frac{     H   (x)}{\Theta (x)}, \quad
y = \sqrt{kk'} \frac{     H_1 (x)}{\Theta (x)}, \quad
y = \sqrt{k' } \frac{\Theta_1 (x)}{\Theta (x)},
\]
ou, plus simplement, puisqu'on peut les multiplier par des facteurs constants,
\[
y = \sn x, \quad y = \cn x , \quad  y =\dn x.
\]
%-----024.png-----------------
Mais une circonstance se pr\'esente maintenant, qui demande un examen
attentif. On ne peut plus, en effet, d\'eduire de ces expressions d'autres
qui en soient distinctes par le changement de signe de la variable, et il
faut, par suite, employer une nouvelle m\'ethode pour obtenir l'int\'egrale
compl\`ete. Repr\'esentons, dans ce but, la solution g\'en\'erale de l'une quelconque
de nos trois \'equations, en laissant $\omega$ ind\'etermin\'e, par la formule
\[
  y = CF(x,\omega) + C'F(-x,\omega).
\]
Je la mettrai d'abord sous cette forme \'equivalente
\[
  y = CF(x,\omega) + C'F(x,-\omega);
\]
puis, en d\'eveloppant suivant les puissances croissantes de $\omega$, je ferai
\[
  F(x,\omega) = F_0(x) + \omega F_1(x) + \omega^2 F_2(x) + \ldots,
\]
ce qui permettra d'\'ecrire
\[
y =         (C + C')F_0(x)
  + \omega  (C - C')F_1(x)
  + \omega^2(C + C')F_2(x) + \ldots,
\]
ou encore
\[
y = C_0F_0(x) + C_1F_1(x) + \omega C_0F_2(x) + \ldots,
\]
en posant, d'apr\`es la m\'ethode de d'Alembert,
\[
C_0 = C + C',\qquad  C_1 = \omega(C - C').
\]

Si l'on suppose maintenant $\omega = 0$, on parvient \`a la formule
\[
y = C_0F_0(x) + C_1F_1(x),
\]
qu'il faudra appliquer en faisant successivement
\[
F(x,\omega) = \chi(x),\quad
F(x,\omega) = \chi_1(x), \quad
F(x,\omega) = \varphi_1(x);
\]
mais le calcul sera plus simple si l'on prend
\begin{align*}
   F(x,\omega)
&= \frac{ H(x+\omega) }{ \Theta(x) }\,
   e^{-\frac{ \Theta'(\omega) }{ \Theta(\omega) }x} ,
\\
   F(x,\omega)
&= \frac{ H_1(x+\omega) }{ \Theta(x) }\,
   e^{-\frac{ \Theta_1'(\omega) }{ \Theta_1(\omega) }x} ,
\\
   F(x,\omega)
&= \frac{ \Theta_1(x+\omega) }{ \Theta(x) }\,
   e^{-\frac{ H_1'(\omega) }{ H_1(\omega) }x} ,
\end{align*}
%-----025.png---------------------------
ces quantit\'es ne diff\'erant des pr\'ec\'edentes que par des facteurs constants.
Observant donc que, pour $\omega = 0$, on a
\[
D_{\omega} \frac{\Theta' (\omega)}{\Theta (\omega)} = \frac{J}{K},\quad
D_{\omega} \frac{\Theta'_1 (\omega)}{\Theta_1 (\omega)} = \frac{J}{K} - k^2,\quad
D_{\omega} = \frac{H'_1 (\omega)}{H_1 (\omega)} = \frac{J}{K} - 1,
\]
nous obtenons imm\'ediatement les valeurs que prennent leurs d\'eriv\'ees par
rapport \`a $\omega$, dans cette hypoth\`ese de $\omega = 0$
\begin{align*}
F_1 (x) & = \frac{H' (x)}{\Theta (x)} - \frac{J H (x)}{K \Theta (x)} x,\\
F_1 (x) & = \frac{H'_1 (x)}{\Theta (x)} - \frac{(J - k^2 K) H_1 (x)}{K \Theta (x)} x,\\
F_1 (x) & = \frac{\Theta'_1 (x)}{\Theta (x)} - \frac{(J - K) \Theta_1 (x)}{K \Theta (x)} x.
\end{align*}

La solution g\'en\'erale de l'\'equation de Lam\'e, dans les cas particuliers
que nous venons de consid\'erer, peut donc se repr\'esenter par les formules
suivantes:
\begin{align*}
\tag*{\primo} h &= -1 - k^2, &
  y &= C \sn x + C' \sn x \left[ \frac{H'(x)}{H(x)} - \frac{J}{K} x \right],
\\
\tag*{\secundo} h &= -1, &
  y &= C \cn x + C' \cn x \left[ \frac{H'_1 (x)}{H_1 (x)} - \frac{J-k^2 K}{K} x \right],
\\
\tag*{\tertio} h &= - k^2, &
  y &= C \dn x + C' \dn x \left[ \frac{\Theta'_1 (x)}{\Theta_1 (x)} - \frac{J-K}{K} x \right].
\end{align*}


\mysection{VIII.}


Un dernier point me reste \`a traiter avant d'aborder, au moyen des
r\'esultats qui viennent d'\^etre obtenus, le probl\`eme de la rotation d'un
corps autour d'un point fixe, dans le cas o\`u il n'y a point de forces acc\'el\'eratrices.
On a vu que les quantit\'es $\varphi (x)$, $\chi (x)$, $\varphi_1 (x)$, $\chi_1 (x)$ sont les
produits d'une exponentielle par les fonctions p\'eriodiques
\[
\frac{H'(0) \Theta  (x + \omega)}{H      (\omega) \Theta (x)},\quad
\frac{H'(0) H       (x + \omega)}{\Theta (\omega) \Theta (x)},\quad
\frac{H'(0) \Theta_1(x + \omega)}{H      (\omega) \Theta (x)},\quad
\frac{H'(0) H_1     (x + \omega)}{\Theta (\omega) \Theta (x)},
\]
d\'eveloppables par cons\'equent en s\'eries simples de sinus et cosinus de
multiples entiers de $\frac{\pi x}{K}$. Ces s\'eries ont \'et\'e donn\'ees pour la premi\`ere fois
par Jacobi, \`a l'occasion m\^eme de ses recherches sur la rotation; et, comme
l'observe l'illustre auteur, elles sont d'une grande importance dans la
%-----026.png-----------------
th\'eorie des fonctions elliptiques. Je vais montrer comment on peut y parvenir
au moyen de l'\'equation suivante:
\begin{multline*}
\int_0^{2K} F(x_0+x)\, dx + \int_0^{2iK'} F(x_0+2K+x)\, dx \\
-\int_0^{2K} F(x_0+2iK'+x)\, dx - \int_0^{2iK'} F(x_0+x)\, dx
=2i\pi S,
\end{multline*}
ou, les quatre int\'egrales \'etant rectilignes, $S$ repr\'esente la somme des r\'esidus
de la fonction $F(x)$ qui correspondent aux p\^oles situ\'es \`a l'int\'erieur du
rectangle dont les sommets ont pour affixes les quantit\'es $x_0$, $x_0+2K$,
$x_0+2K+2iK'$, $x_0+2iK'$. Supposons \`a cet effet qu'on ait:
\begin{alignat*}{3}
&F(x+ 2K) &&= \mu &&F(x),\\
&F(x + 2iK') &&= \mu' &&F(x),
\end{alignat*}
on obtiendra la relation
\[
(1-\mu') \int_0^{2K} F(x_0+x)\, dx -
(1-\mu) \int_0^{2iK'}F(x_0+x)\, dx = 2i\pi S,
\]
et si l'on admet en outre que le multiplicateur $\mu$ soit \'egal \`a l'unit\'e, on en
conclura le r\'esultat suivant:
\[
\int_0^{2K} F(x_0+x)\, dx = \frac{2i\pi S}{1-\mu'}.
\]
Cela pos\'e, soit, en d\'esignant par $n$ un nombre entier quelconque,
\[
F(x)= \frac{H'(0)\Theta(x+\omega)}{H(\omega)\Theta(x)}\,
e^{-\frac{i\pi n x}{K}} ,
\]
on aura
\[
\mu=1, \qquad
\mu' = e^{-\frac{i\pi}{K}(\omega+2niK')},
\]
et, en limitant la constante $x_0$ de telle sorte que le p\^ole unique de $F(x)$ qui
est \`a l'int\'erieur du rectangle soit $x=iK'$, nous obtiendrons pour le r\'esidu
correspondant, et par cons\'equent pour $S$, la valeur
\[
S= e^{-\frac{i\pi}{2K}(\omega+2niK')}.
\]
%-----027.png----------------------

De l\`a r\'esulte, pour l'int\'egrale d\'efinie, l'expression suivante:
\[
  \int_0^{2K} F(x_0+x)dx
= \frac{ 2i\pi e^{-\frac{i\pi}{2K} (\omega+2niK')} }{
           1 - e^{-\frac{i\pi}{2K} (\omega+2niK')} }
= \frac{\pi }{ \sin \frac{\pi}{2K} (\omega+2niK')},
\]
et l'on voit qu'en posant l'\'equation
\[
  \frac{H'(0) \Theta(x_0+x+\omega)}{H'(\omega) \Theta(x_0+x)}
= \sum A_n e^{ \frac{i\pi n(x_0+x)}{K} },
\]
on en d\'eduit imm\'ediatement la d\'etermination de $A_n$. Nous avons, en effet,
\[
  2KA_n = \int_0^{2K} F(x_0+x)\, dx,
\]
et, par cons\'equent,
\[
  \frac{2K}{\pi} A_n
= \frac{1}{\sin\frac{\pi}{2K} (\omega+2niK')}.
\]

La constante $x_0$ que j'ai introduite pour plus de g\'en\'eralit\'e, et aussi
pour \'eviter qu'un p\^ole de $F(x)$ se trouve sur le contour d'int\'egration, peut
maintenant sans difficult\'e \^etre suppos\'ee nulle. Nous parvenons ainsi \`a une
premi\`ere formule de d\'eveloppement:
\[
  \frac{2K}{\pi}\,
  \frac{H'(0) \Theta(x+\omega)}{H(\omega) \Theta(x)}
= \sum \frac{ e^{\frac{i\pi nx}{K}} }{
              \sin\frac{\pi}{2K} (\omega+2niK') },
\]
dont les trois autres r\'esultent, comme on va le voir. Qu'on change, en
effet, $\omega$ en $\omega+iK'$, on en conclura d'abord
\[
  \frac{2K}{\pi}\,
  \frac{H'(0) H(x+\omega)}{\Theta(\omega) \Theta(x)}\,
  e^{-\frac{i\pi x}{2K}}
= \sum \frac{ e^{\frac{i\pi nx}{K}} }{
    \sin\frac{\pi}{2K} \left[\omega+(2n+1)iK'\right] };
\]
puis, en multipliant les deux membres par l'exponentielle, et posant
$m = 2n+1$,
\[
  \frac{2K}{\pi}\,
  \frac{H'(0) H(x+\omega)}{\Theta(\omega) \Theta(x)}
= \sum
  \frac{ e^{\frac{i\pi mx}{K}} }{ \sin\frac{\pi}{2K} (\omega+miK') }.
\]

Mettons enfin, dans les deux formules que nous venons d'\'etablir,
%-----028.png----------------------
$\omega + K$ \`a la place de $K$, et l'on obtiendra les suivantes, qui nous restaient
\`a trouver:
\begin{align*}
& \frac{2K}{\pi}\,
  \frac{ H'(0) \Theta_1(x+\omega) }{ H_1(\omega) \Theta(x) }
= \sum \frac{ e^{\frac{ i\pi nx }{K}} }{
    \cos\frac{\pi}{2K} (\omega+2niK')} ,
\\
& \frac{2K}{\pi}\,
  \frac{ H'(0) H_1(x+\omega) }{ \Theta_1(\omega) \Theta(x) }
= \sum \frac{ e^{\frac{ i\pi mx }{ 2K }} }{
    \cos\frac{\pi}{2K} (\omega+ miK')} .
\end{align*}

Voici \`a leur sujet quelques remarques.


\mysection{IX.}


Elles sont d'une forme diff\'erente de celles de Jacobi et l'on peut
s'en servir utilement dans beaucoup de questions que je ne puis aborder
en ce moment. Je me contenterai, sans en faire l'\'etude, d'indiquer succinctement
comment on en tire les sommes des s\'eries suivantes:
\[
  \sum f(2niK') e^{\frac{i\pi nx}{ K}}, \qquad
  \sum f( miK') e^{\frac{i\pi mx}{2K}},
\]
o\`u $f(z)$ est une fonction rationnelle de
$\sin \frac{\pi z}{2K}$ et
$\cos \frac{\pi z}{2K}$, sans partie enti\`ere
et assujettie \`a la condition $f(z+2K) = - f(z)$. Il suffit, en effet,
d'employer la d\'ecomposition de cette fonction en \'el\'ements simples,
c'est-\`a-dire en termes tels que
$D_z^{\alpha} \dfrac{1}{\sin\frac{\pi}{2K}(z+\omega)}$,
pour obtenir imm\'ediatement
la valeur des s\'eries propos\'ees, au moyen de ces deux expressions:
\begin{align*}
& \sum D_{\omega}^{\alpha}
  \left[ \frac{1}{ \sin\frac{\pi}{2K} (\omega+2niK') } \right]
  e^{\frac{i\pi nx}{K}}
=  D_{\omega}^{\alpha} \frac{2K}{\pi}
   \frac{ H'(0) \Theta(x+\omega) }{ H(\omega) \Theta(x) } ,
\\
& \sum D_{\omega}^{\alpha}
  \left[ \frac{1}{ \sin\frac{\pi}{2K} (\omega+ miK') } \right]
  e^{\frac{i\pi mx}{2K}}
=  D_{\omega}^{\alpha} \frac{2K}{\pi}
   \frac{ H'(0) H(x+\omega) }{ \Theta(\omega) \Theta(x) } .
\end{align*}

J'ajouterai encore qu'on retrouve les r\'esultats de Jacobi, si l'on r\'eunit
les termes qui correspondent \`a des valeurs de l'indice \'egales et de signes
%-----029.png-----------------
contraires. Il vient ainsi, en effet, en d\'esignant par $m$ un nombre qu'on
fera successivement pair et impair,
\begin{multline*}
  \frac{ e^{ \frac{i\pi m x}{2K}} }{ \sin\frac{\pi}{2K}(\omega+m i K') }
+ \frac{ e^{-\frac{i\pi m x}{2K}} }{ \sin\frac{\pi}{2K}(\omega-m i K') }
\\
=
  \frac{ 2\cos\frac{m\pi x}{2K}
    \cos\frac{m\pi iK'}{2K}
    \sin\frac{\pi\omega}{2K} }{ \sin\frac{\pi}{2K}(\omega+m i K')
    \sin\frac{\pi}{2K}(\omega-m i K')} \\
- i
  \frac{ 2\sin\frac{m\pi x}{2K}
    \sin\frac{m\pi iK'}{2K}
    \cos\frac{\pi\omega}{2K} }{ \sin\frac{\pi}{2K}(\omega+m i K')
    \sin\frac{\pi}{2K}(\omega-m i K')};
\end{multline*}
employons ensuite les \'equations de la page 85 des \textit{Fundamenta}, qui donnent:
\begin{align*}
  &\cos\frac{m\pi iK'}{2K} =  \frac{1+q^m}{2\sqrt{q^m}},\\
  &\sin\frac{m\pi iK'}{2K} = i\frac{1-q^m}{2\sqrt{q^m}},
\end{align*}\[
  \sin\frac{\pi}{2K}(\omega+m i K')
  \sin\frac{\pi}{2K}(\omega-m i K')
= \frac{1-2q^m\cos\frac{\pi\omega}{K}+q^{2m}}{4q^m},
\]
et nous parviendrons \`a cette nouvelle forme:
\begin{multline*}
  \frac{ e^{ \frac{i\pi m x}{2K}} }{ \sin\frac{\pi}{2K}(\omega+m i K')}
+ \frac{ e^{-\frac{i\pi m x}{2K}} }{ \sin\frac{\pi}{2K}(\omega-m i K') }
\\[1ex]
 =
  \frac{ 4\sqrt{q^m}(1+q^m) \sin\frac{\pi\omega}{2K} }{ 1-2q^m\cos\frac{\pi\omega}{K} + q^{2m} }
  \cos\frac{m\pi x}{2K}
+
  \frac{ 4\sqrt{q^m}(1-q^m) \cos\frac{\pi\omega}{2K} }{
         1-2q^m\cos\frac{\pi\omega}{K} + q^{2m} }
  \sin\frac{m\pi x}{2K} .
\end{multline*}
C'est celle qu'on voit dans la lettre adress\'ee \`a l'Acad\'emie des Sciences et
publi\'ee dans les \textit{Comptes rendus} du 30 juillet 1849; car, en introduisant la
constante $b=\frac{i\omega}{K'}$, on peut \'ecrire
\begin{align*}
  \sin\frac{\pi\omega}{2K}
&= \frac{q^{\frac{1}{2}b} - q^{-\frac{1}{2}b} }{2i},
\\
  \cos\frac{\pi\omega}{2K}
&= \frac{q^{\frac{1}{2}b} + q^{-\frac{1}{2}b} }{2}
\end{align*}
et
\[
1-2q^m\cos\frac{\pi\omega}{K}+q^{2m} = (1-q^{m+b})(1-q^{m-b}).
\]
%-----030.png--------------------------
Mais une faute d'impression, reproduite dans les \textit{{\OE}uvres compl\`etes}, t.~II,
p.~143, et dans le \textit{Journal de Crelle}, t.~XXXIX, p.~297, s'est gliss\'ee dans ces
formules. Les \'equations (3), (4), (5), (6) renferment en effet les quantit\'es
$\sqrt{q(1+q)}$, $\sqrt{q^3(1+q^3)}$, \ldots\ et
$\sqrt{q(1-q)}$, $\sqrt{q^3(1-q^3)}$, \ldots, qui doivent \^etre
remplac\'ees par
$\sqrt{q}(1+q)$, $\sqrt{q^3}(1+q^3)$, \ldots\ et
$\sqrt{q}(1-q)$, $\sqrt{q^3}(1-q^3)$, \ldots.
On peut d'ailleurs parvenir par d'autres m\'ethodes \`a ces r\'esultats importants.
M.~Somoff les obtient en d\'ecomposant la quantit\'e
\[
\tfrac{ (1-qvz) (1-q^3vz) (1-q^5vz)\ldots
  (1-qv^{-1}z^{-1}) (1-q^3v^{-1}z^{-1}) (1-q^5v^{-1}z^{-1}) \ldots}{
  (z-1) (1-q^2z) (1-q^4z)\ldots (1-q^2z^{-1}) (1-q^4z^{-1})\ldots }
\]
en fractions simples:
\[
      \frac{A_0}{z-1}
+ \sum\frac{A_m}{1-q^{2m}z}
+ \sum\frac{B_m}{z-q^{2m}} .
\]
Le P.~Joubert m'a communiqu\'e la remarque qu'on peut, en suivant la
m\^eme marche, partir de ces expressions finies:
\begin{gather*}
\tfrac{ z(z-q^{1-b} ) (z-q^{3-b} ) \ldots (z-q^{2n-1-b} )
  (1-q^{1+b}z) (1-q^{3+b}z) \ldots (1-q^{2n-1+b}z) }{
  (z-q ) (z-q^3 ) \ldots (z-q^{2n+1} )
  (1-qz) (1-q^3z) \ldots (1-q^{2n+1}z) },
\\[1ex]
\tfrac{ z(z-q^{2-b} ) (z-q^{4-b} ) \ldots (z-q^{2n  -b} )
  (1-q^{2+b}z) (1-q^{4+b}z) \ldots (1-q^{2n  +b}z) }{
  (z-q ) (z-q^3 ) \ldots (z-q^{2n+1} )
  (1-qz) (1-q^3z) \ldots (1-q^{2n+1}z) };
\end{gather*}
et faire ensuite grandir ind\'efiniment le nombre $n$.

Enfin, et en dernier lieu, je remarque qu'au moyen de la formule
\[
  \int_0^{2K} F(x_0+x)\, dx = \frac{2i\pi S}{1-\mu'},
\]
qui a \'et\'e le point de d\'epart de mon proc\'ed\'e, nous pouvons tr\`es-simplement
d\'emontrer les relations \'etablies au \S~IV, p.~\pageref{page11}:
\begin{align*}
  \int_0^{2K} \frac{\Theta(x+a) \Theta(x+b)}{\Theta^2(x)}\, dx &= 0,  \\
  \int_0^{2K} \frac{H(x+a)H(x+b)}{\Theta^2(x)}\, dx &= 0,
\end{align*}
o\`u $a$ et $b$ d\'esignent, dans la premi\`ere, deux racines de l'\'equation $H'(x) = 0$,
et dans la seconde, deux racines de l'\'equation $\Theta'(x) = 0$. Si l'on prend,
en effet, successivement
\begin{align*}
  F(x) &= \frac{\Theta(x+a) \Theta(x+b)}{\Theta^2(x)},\\
  F(x) &= \frac{H(x+a)H(x+b)}{\Theta^2(x)},
\end{align*}
%-----031.png----------------------
on aura $\mu=1$ et $\mu'$ diff\'erant de l'unit\'e, sauf la supposition que nous
excluons de $b=-a$. On obtient d'ailleurs, dans le premier cas,
\[
  S = \frac{H(a) H'(b) + H(b) H'(a)}{H'^2(0)} \sqrt{\mu'},
\]
et, dans le second,
\[
  S = \frac{\Theta(a)\Theta'(b) + \Theta(b)\Theta'(a)}{H'^2(0)} \sqrt{\mu'},
\]
de sorte que, sous les conditions admises, les deux valeurs de $S$
s'\'evanouiss\-ent. Cela \'etant, nous pouvons, dans la relation ainsi
d\'emontr\'ee
\[
  \int_0^{2K} F(x_0+x)\, dx = 0,
\]
supposer $x_0 = 0$; car l'int\'egrale est une fonction continue de $x_0$, non-seulement
dans le voisinage de cette valeur particuli\`ere, mais dans l'intervalle
des deux parall\`eles \`a l'axe des abscisses, men\'ees \`a la m\^eme distance $K'$
au-dessus et au-dessous de cet axe.


\mysection{X.}


Dans la th\'eorie de la rotation d'un corps autour d'un point
fixe $O$, le mouvement d'un point quelconque du solide se d\'etermine en
rapportant ce point aux axes principaux d'inertie $Ox'$, $Oy'$, $Oz'$, immobiles
dans le corps, mais entra\^in\'es par lui, et dont on donne la position
\`a un instant quelconque par rapport \`a des axes fixes $Ox$, $Oy$, $Oz$, le plan
des $xy$ \'etant le plan invariable et l'axe $Oz$ la perpendiculaire \`a ce plan.
Soient donc $x$, $y$, $z$ les coordonn\'ees d'un point du corps par rapport aux
axes fixes, et $\xi$, $\eta$, $\zeta$ les coordonn\'ees par rapport aux axes mobiles; ces
quantit\'es seront li\'ees par les relations\label{page23}
\begin{alignat*}{3}
  x &= a  \xi &&+ b  \eta &&+ c  \zeta,\\
  y &= a' \xi &&+ b' \eta &&+ c' \zeta,\\
  z &= a''\xi &&+ b''\eta &&+ c''\zeta,
\end{alignat*}
et la question consiste \`a obtenir en fonction du temps les neuf coefficients
$a$, $b$, $c$, \ldots. Jacobi le premier en a donn\'e une solution compl\`ete et d\'efinitive,
qui offre l'une des plus belles applications de calcul \`a la M\'ecanique
et ouvre en m\^eme temps des voies nouvelles dans la th\'eorie des fonctions elliptiques.
C'est \`a l'\'etude des r\'esultats si importants d\'ecouverts par l'immortel
%-----032.png----------------------
g\'eom\`etre que je dois les recherches expos\'ees dans ce travail, et tout
d'abord l'int\'egration de l'\'equation de Lam\'e, dans le cas dont je viens de
m'occuper, o\`u l'on suppose $n = 1$; on va voir en effet comment la th\'eorie
de la rotation, lorsqu'il n'y a point de forces acc\'el\'eratrices, se trouve
\'etroitement li\'ee \`a cette \'equation.

Pour cela je partirai des relations suivantes, donn\'ees dans le tome~II
du \textit{Trait\'e de M\'ecanique} de Poisson, p.~135:
\begin{align*}
  \frac{da  }{dt} &= b  r - c  q,\quad
& \frac{da' }{dt} &= b' r - c' q,\quad
& \frac{da''}{dt} &= b''r - c''q,
\\
  \frac{db  }{dt} &= c  p - a  r,\quad
& \frac{db' }{dt} &= c' p - a' r,\quad
& \frac{db''}{dt} &= c''p - a''r,
\\
  \frac{dc  }{dt} &= a  q - b  p,\quad
& \frac{dc' }{dt} &= a' q - b' p,\quad
& \frac{dc''}{dt} &= a''q - b''p,
\end{align*}
dans lesquelles $p$, $q$, $r$ sont les composantes rectangulaires de la vitesse de
rotation, par rapport aux mobiles $Ox'$, $Oy'$, $Oz'$. Cela \'etant, des conditions
connues
\[
  p = \alpha a'', \quad  q = \beta b'', \quad  r = \gamma c'',
\]
o\`u $\alpha$, $\beta$, $\gamma$ sont des constantes, on tire imm\'ediatement les \'equations
\[
  \frac{da''}{dt} = (\gamma - \beta )b''c'',\quad
  \frac{db''}{dt} = (\alpha - \gamma)c''a'',\quad
  \frac{dc''}{dt} = (\beta  - \alpha)a''b'',
\]
dont une premi\`ere int\'egrale alg\'ebrique est donn\'ee par l'\'egalit\'e
\[
  a''^2 + b''^2 + c''^2 = 1,
\]
et une seconde int\'egrale par celle-ci:
\[
  \alpha a''^2 + \beta b''^2 + \gamma c''^2 = \delta,
\]
$\delta$ \'etant une constante arbitraire. Ces quantit\'es $\alpha$, $\beta$, $\gamma$, $\delta$ sont li\'ees aux
constantes $A$, $B$, $C$, $h$, $l$ du M\'emoire de Jacobi, par les relations
\[
  \alpha = \frac{l}{A},\quad
  \beta  = \frac{l}{B},\quad
  \gamma = \frac{l}{C},\quad
  \delta = \frac{h}{l};
\]
elles sont donc du signe de $l$ qui peut \^etre positif ou n\'egatif, comme
re\-pr\'esentant le moment d'impulsion dans le plan invariable. Dans ces
deux cas, $\beta$ sera compris entre $\alpha$ et $\gamma$, puisqu'on
suppose $B$ compris entre $A$ et $C$;
%-----033.png----------------------
mais j'admettrai, pour fixer les id\'ees, que $l$ soit positif. On voit de plus que,
$\delta$ \'etant une moyenne entre $\alpha$, $\beta$, $\gamma$, peut
\^etre plus grand ou plus petit que $\beta$: la premi\`ere hypoth\`ese
donne $Bh > l^2$, et Jacobi suppose alors $A > B > C$; dans la seconde,
on a $Bh < l^2$, avec $A < B < C$; ces conditions  prendront, avec nos
constantes, la forme suivante:
\begin{align*}
\tag*{I. } \alpha < \beta &< \delta < \gamma,\\
\tag*{II.} \alpha > \beta &> \delta > \gamma,
\end{align*}
et nous allons imm\'ediatement en faire usage en recherchant les expressions
des coefficients $a''$, $b''$, $c''$, par des fonctions elliptiques du temps.


\mysection{XI.}


J'observe, en premier lieu, qu'on obtient, si l'on exprime $a''$ et $c''$
au moyen de $b''$, les valeurs
\[
  (\gamma-\alpha)a''^2 = \gamma - \delta - (\gamma-\beta) b''^2,\qquad
  (\gamma-\alpha)c''^2 = \delta - \alpha - (\beta-\alpha) b''^2.
\]
Posons maintenant
\[
  a''^2 = \frac{\gamma-\delta}{\gamma-\alpha} V^2,\quad
  b''^2 = \frac{\gamma-\delta}{\gamma-\beta } U^2,\quad
  c''^2 = \frac{\delta-\alpha}{\gamma-\alpha} W^2,
\]
puis
\[
k^2 = \frac{(\beta -\alpha)(\gamma-\delta)}{(\delta-\alpha)(\gamma- \beta)};
\]
il viendra plus simplement
\[
  V^2 = 1 - U^2, \qquad W^2 = 1 - k^2 U^2.
\]
Introduisons, en outre, la quantit\'e
$n^2 = (\delta-\alpha)(\gamma-\beta)$; l'\'equation % following two displays inline in original
\[
\frac{db''}{dt} = (\alpha-\gamma)c''a''
\]
prend cette forme:
\[
\frac{dU}{dt} = nVW,
\]
et l'on en conclut, en d\'esignant par $t_0$ une constante arbitraire,
\[
  U = \sn\left[n(t-t_0), k\right], \quad
  V = \cn\left[n(t-t_0), k\right], \quad
  W = \dn\left[n(t-t_0), k\right].
\]

J'ajoute que les quantit\'es
$\frac{\gamma-\delta}{\gamma-\alpha}$,
$\frac{\gamma-\delta}{\gamma- \beta}$,
$\frac{\delta-\alpha}{\gamma-\alpha}$,
$(\delta-\alpha)(\gamma-\beta)$ sont toutes
positives et que $k^2$ est positif et moindre que l'unit\'e, sous les conditions
I et~II. A l'\'egard du module il suffit en effet de remarquer que l'identit\'e
\[
  (\delta-\alpha)(\gamma- \beta)
= (\gamma-\alpha)(\delta- \beta)
+ ( \beta-\alpha)(\gamma-\delta)
\]
%-----034.png----------------------
donne
\[
k'^2 = \frac{(\gamma-\alpha)(\delta-\beta)}{(\delta-\alpha)(\gamma-\beta)},
\]
de sorte que $k^2$ et $k'^2$, \'etant \'evidemment positifs, sont par cela m\^eme tous
deux inf\'erieurs \`a l'unit\'e. Ce point \'etabli, d\'esignons par
$\varepsilon$, $\varepsilon'$, $\varepsilon''$ des facteurs
\'egaux \`a $\pm 1$; en convenant de prendre dor\'enavant les racines carr\'ees avec
le signe $+$, nous pourrons \'ecrire
\[
  a'' = \varepsilon  \sqrt{\frac{\gamma-\delta}{\gamma-\alpha}}V,\quad
  b'' = \varepsilon' \sqrt{\frac{\gamma-\delta}{\gamma-\beta }}U,\quad
  c'' = \varepsilon''\sqrt{\frac{\delta-\alpha}{\gamma-\alpha}}W,
\]
et la substitution dans les \'equations
\[
  \frac{da''}{dt} = (\gamma-\beta ) b''c'',\quad
  \frac{db''}{dt} = (\alpha-\gamma) c''a'',\quad
  \frac{dc''}{dt} = (\beta -\alpha) a''b''
\]
donnera les conclusions suivantes. Admettons d'abord les conditions~I: les
trois diff\'erences $\gamma-\beta$, $\alpha-\gamma$, $\beta-\alpha$ seront n\'egatives, et l'on trouvera
$\varepsilon  = -\varepsilon' \varepsilon''$,
$\varepsilon' = -\varepsilon''\varepsilon  $,
$\varepsilon''= -\varepsilon  \varepsilon'$;
mais sous les conditions~II, ces m\^emes
quantit\'es \'etant positives, nous aurons
$\varepsilon  = \varepsilon' \varepsilon''$,
$\varepsilon' = \varepsilon''\varepsilon  $,
$\varepsilon''= \varepsilon  \varepsilon' $; ainsi, en
faisant, avec Jacobi,
$\varepsilon = -1$, $\varepsilon' = +1$, on voit qu'il faudra prendre
$\varepsilon'' = +1$ dans le premier cas et la valeur contraire $\varepsilon'' = -1$ dans le second.
Cela pos\'e, et en convenant toujours que les racines carr\'ees soient
positives, je dis qu'on peut d\'eterminer un argument $\omega$ par les deux conditions\label{page26}
\[
  \cn\omega = \sqrt{\frac{\gamma-\alpha}{\gamma-\delta}},\qquad
  \dn\omega = \sqrt{\frac{\gamma-\alpha}{\gamma-\beta }};
\]
d'o\`u nous tirons
$\frac{\dn\omega}{\cn\omega} = \sqrt{\frac{\gamma-\delta}{\gamma-\beta}}$;
ces quantit\'es satisfont en effet \`a la relation
\[
  \dn^2\omega - k^2\cn^2\omega = k'^2,
\]
comme on le v\'erifie ais\'ement. Je remarque, en outre, que, $\cn\omega$ et $\dn\omega$
\'etant des fonctions paires, on peut encore \`a volont\'e disposer du signe
de $\omega$. Or, ayant
$\frac{\sn^2\omega}{\cn^2\omega} = \frac{\alpha-\delta}{\gamma-\alpha}$,
nous fixerons ce signe de mani\`ere que, suivant
les conditions I ou~II, $\frac{\sn\omega}{i\cn\omega}$,
qui est une fonction impaire, soit \'egal \`a
$+\sqrt{\frac{\delta-\alpha}{\gamma-\alpha}}$ ou \`a
$-\sqrt{\frac{\delta-\alpha}{\gamma-\alpha}}$.
Nous \'eviterons, en d\'efinissant la constante $\omega$
comme on vient de le faire, les doubles signes qui figurent dans les relations
%-----035.png-----------------
de Jacobi; ainsi, \`a l'\'egard de $a''$, $b''$, $c''$, on aura, dans
tous les cas, les formules suivantes, o\`u je fais pour abr\'eger
$u = n(t-t_0)$:\label{page27}
\[
  a'' = -\frac{\cn u}{ \cn\omega},\quad
  b'' =  \frac{\dn\omega \sn u}{\cn\omega},\quad
  c'' =  \frac{\sn\omega \dn u}{i\cn\omega}.
\]
Enfin il est facile de voir que $\omega = i\upsilon$, $\upsilon$
\'etant r\'eel; de la formule \label{page27a} $\cn(i\upsilon, k) =
\frac{1}{\cn(\upsilon, k')}$, on conclut, en effet,
\[
\cn(\upsilon, k') = \sqrt{\frac{\gamma-\delta}{\gamma-\alpha}},
\]
valeur qui est dans les deux cas non-seulement r\'eelle, mais
moindre que l'unit\'e.
\medskip\enlargethispage{-0.1ex}

\mysection{XII.}


J'aborde maintenant la d\'etermination des six coefficients $a$, $b$,
$c$, $a'$, $b'$, $c'$ en introduisant les quantit\'es
\[
  A = a + ia', \quad  B = b + ib', \quad  C = c + ic',
\]
et partant des relations suivantes:\label{page27b}
\begin{align*}
  Aa'' + Bb'' + Cc'' &= 0, \\
  iA   - Bc'' + Cb'' &= 0,
\end{align*}
qu'il est facile de d\'emontrer. La premi\`ere est une suite des \'egalit\'es
\[
  aa'' + bb'' + cc'' = 0, \qquad
  a'a''+ b'b''+c'c'' = 0,
\]
et la seconde r\'esulte de celles-ci:
\[
  a  = b' c'' - c' b'', \quad
  a' = b''c   - c''b  , \quad
  a''= b  c'  - c  b' , \quad\ldots.
\]
Qu'on prenne, en effet, les valeurs de $a$ et $a'$, on en d\'eduira
\[
  a + ia' = (b' - ib)c'' - b''(c' - ic),
\]
ce qui revient bien \`a la relation \'enonc\'ee. Cela pos\'e, je fais usage des \'equations
de Poisson rappel\'ees plus haut, et qui donnent
\[
  D_t A = Br - Cq, \quad D_t B = Cp - Ar, \quad D_t C = Aq - Bp,
\]
puis, en rempla\c{c}ant $p$, $q$, $r$ par $\alpha a''$, $\beta b''$, $\gamma c''$,
\[
  D_t A = Bc''\gamma - Cb''\beta,  \quad
  D_t B = Ca''\alpha - Ac''\gamma, \quad
  D_t C = Ab''\beta  - Ba''\alpha.
\]
%-----036.png------------

Mettons maintenant dans la premi\`ere les expressions de $B$ et $C$ en $A$,
qu'on tire de nos deux relations, \`a savoir
\[
  B=\frac{a''b'' - ic''}{a''^2-1}A, \qquad
  C=\frac{a''c'' + ib''}{a''^2-1}A,
\]
on obtiendra ais\'ement
\[
  \frac{D_t A}{A} =
  \frac{ (\gamma-\beta) a''b''c'' - i (\gamma c''^2 + \beta b''^2)}{a''^2 -1} ,
\]
ou bien encore
\[
  \frac{D_t A}{A} =
  \frac{a'' D_t a'' + i(\alpha a''^2 - \delta)}{a''^2 - 1},
\]
et, par un simple changement de lettres, on en conclut, sans nouveau
calcul,
\begin{align*}
  \frac{D_t B}{B} &=
  \frac{b'' D_t b'' + i(\beta b''^2 - \delta)}{b''^2 - 1}, \\
  \frac{D_t C}{C} &=
  \frac{c'' D_t c'' + i(\gamma c''^2 - \delta)}{c''^2 - 1}.
\end{align*}
Ces formules seront plus simples si l'on fait
\[
  A = \mathrm{a} e^{i\alpha t},\quad
  B = \mathrm{b} e^{i\beta  t},\quad
  C = \mathrm{c} e^{i\gamma t};
\]
car il vient ainsi
\begin{align*}
  \frac{D_t \mathrm{a}}{\mathrm{a}} &=
  \frac{a'' D_t a'' + i(\alpha - \delta)}{a''^2 - 1}, \\
  \frac{D_t \mathrm{b}}{\mathrm{b}} &=
  \frac{b'' D_t b'' + i(\beta  - \delta)}{b''^2 - 1}, \\
  \frac{D_t \mathrm{c}}{\mathrm{c}} &=
  \frac{c'' D_t c'' + i(\gamma - \delta)}{c''^2 - 1}.
\end{align*}
Cela \'etant, j'envisage la premi\`ere, et pour un instant je pose $a''^2 -1 = \mathfrak{a}^2$,
ce qui donnera
\[
  \frac{D_t \mathrm{a}}{\mathrm{a}}
  =
  \frac{\mathfrak{a} D_t \mathfrak{a} + i (\alpha-\delta)}{\mathfrak{a}^2}
  =
  \frac{D_t \mathfrak{a}}{\mathfrak{a}} + i \frac{\alpha-\delta}{\mathfrak{a}^2} .
\]
On en conclut ensuite, en diff\'erentiant,
\[
  \frac{D_t^2 \mathrm{a}}{\mathrm{a}} - \left( \frac{D_t\mathrm{a}}{\mathrm{a}} \right)^2
  =
  \frac{D_t^2 \mathfrak{a}}{\mathfrak{a}} - \left( \frac{D_t\mathfrak{a}}{\mathfrak{a}} \right)^2
  - 2i \frac{(\alpha-\delta) D_t\mathfrak{a}}{\mathfrak{a}^{3}};
\]
puis encore, par l'\'elimination de $\frac{D_t\mathrm{a}}{\mathrm{a}}$,
\[
  \frac{D_t^2\mathrm{a}}{\mathrm{a}} =
  \frac{D_t^2\mathfrak{a}}{\mathfrak{a}} - \frac{(\alpha-\delta)^2}{\mathfrak{a}^{4}};
\]
%-----037.png-----------------
mais, comme cons\'equence de l'\'equation diff\'erentielle,
\[
(D_t a'')^2 = (\gamma-\beta)^2 b''^2c''^2
= \left[ \delta - \beta  - (\alpha-\beta )a''^2 \right]
\left[ \gamma - \delta - (\gamma-\alpha)a''^2 \right],
\]
on a la suivante:
\[
  \frac{\mathfrak{a}^2}{1+\mathfrak{a}^2} (D_t\mathfrak{a})^2
= - (\delta-\alpha)^2
  - (\delta-\alpha)(\beta+\gamma-2\alpha)\mathfrak{a}^2
  - (\beta-\gamma)(\gamma-\alpha)\mathfrak{a}^4,
\]
qui peut s'\'ecrire
\begin{multline*}
  (D_t\mathfrak{a})^2 + \frac{(\delta-\alpha)^2}{\mathfrak{a}^2}  \\
= - (\delta-\alpha)^2
  - (\delta-\alpha)(\beta+\gamma-2\alpha)(1+\mathfrak{a}^2)
  - (\beta-\gamma)(\gamma-\alpha)(\mathfrak{a}^2+\mathfrak{a}^4).
\end{multline*}
Or on en tire, en diff\'erentiant et divisant ensuite les deux membres par
$2\mathfrak{a}D_t\mathfrak{a}$,
\begin{multline*}
  \frac{D_t^2\mathfrak{a}}{\mathfrak{a}}
- \frac{(\delta-\alpha)^2}{\mathfrak{a}^4}
\\
= -[ (\delta-\alpha)(\beta+\gamma-2\alpha)
   + (\beta-\alpha)(\gamma-\alpha) ]
  - 2(\beta-\alpha)(\gamma-\alpha)\mathfrak{a}^2.
\end{multline*}
Nous avons donc, apr\`es avoir remplac\'e $\mathfrak{a}^2$ par
$a''^2-1$,\label{page29}
\[
  \frac{D_t^2 \mathrm{a}}{\mathrm{a}}
= (\beta -\alpha)(\gamma-\delta)
- (\delta-\alpha)(\gamma-\alpha)
-2(\beta -\alpha)(\gamma-\alpha)a''^2;
\]
c'est le r\'esultat que j'avais en vue d'obtenir.


\mysection{XIII.}


Deux voies s'ouvrent maintenant pour parvenir aux expressions
de $A$, $B$, $C$; voici d'abord la plus \'el\'ementaire. Revenant aux formules
\[
  B = \frac{a''b''-ic''}{a''^2-1} A, \qquad
  C = \frac{a''c''+ib''}{a''^2-1} A,
\]
je remplace $a''$, $b''$, $c''$ par les valeurs obtenues au \S~XI, page~\pageref{page27}:
\[
  a'' = -\frac{          \cn u}{ \cn\omega}, \quad
  b'' =  \frac{\dn\omega \sn u}{ \cn\omega}, \quad
  c'' =  \frac{\sn\omega \dn u}{i\cn\omega},
\]
et, au moyen des relations relatives \`a l'addition des arguments, j'obtiens
%-----038.png-----------------
ces r\'esultats:
\begin{align*}
  \frac{a'' b'' - ic''}{a''^2-1} &=
  \frac{\sn u \cn u \dn\omega + \sn\omega\cn\omega\dn u}{\sn^2 u - \sn^2\omega}
  = \frac{\cn(u-\omega)}{\sn(u-\omega)}, \\
  \frac{a'' c'' + ib''}{a''^2-1} &=
  \frac{\sn u \cn\omega\dn\omega + \sn\omega\cn u \dn u}{i(\sn^2 u - \sn^2\omega)}
  = \frac{1}{i\sn(u-\omega)},
\end{align*}
de sorte que nous pouvons \'ecrire
\[
  B=\frac{\cn(u-\omega)}{\sn(u-\omega)} A, \qquad
  C=\frac{A}{i\sn(u-\omega)} .
\]
Cela pos\'e, j'envisage l'expression
\[
  \frac{D_t\mathrm{a}}{\mathrm{a}} =
  \frac{a'' D_t a'' + i(\alpha-\delta)}{a''^2-1} =
  \frac{(\gamma-\beta)a''b''c'' + i(\alpha-\delta)}{a''^2 - 1}
\]
et je fais le m\^eme calcul, apr\`es avoir remplac\'e $\gamma-\beta$ et $\alpha-\delta$ par les
valeurs suivantes:
\[
  \gamma-\beta = in\frac{\cn\omega}{\sn\omega\dn\omega}, \qquad
  \alpha-\delta= in\frac{\sn\omega\dn\omega}{\cn\omega},
\]
qu'on tire facilement des \'equations pos\'ees page~\pageref{page26}:
\[
  \cn\omega = \sqrt{\frac{\gamma-\alpha}{\gamma-\delta}}, \quad
  \dn\omega = \sqrt{\frac{\gamma-\alpha}{\gamma-\beta}},  \quad
  \sn\omega =i\sqrt{\frac{\delta-\alpha}{\gamma-\delta}}
\]
et de $n=\sqrt{(\delta-\alpha)(\gamma-\beta)}$. L'expression \`a laquelle nous parvenons ainsi,
\[
  \frac{D_t\mathrm{a}}{\mathrm{a}} =
  n \frac{\sn u \cn u \dn u + \sn\omega\cn\omega\dn\omega}{\sn^2 u - \sn^2 \omega},
\]
nous offre une fonction doublement p\'eriodique, dont les p\'eriodes
sont $2K$, $2iK'$, et qui a deux p\^oles, $u = \omega$, $u = iK'$. Les r\'esidus correspondant
\`a ces p\^oles \'etant $+1$ et $-1$, la d\'ecomposition en \'el\'ements simples
donne imm\'ediatement
\[
  \frac{\sn u \cn u \dn u + \sn\omega\cn\omega\dn\omega}{\sn^2 u - \sn^2 \omega}
  =  \frac{H'(u-\omega)}{H(u-\omega)} - \frac{\Theta'(u)}{\Theta(u)} + C,
\]
et la constante se d\'etermine en faisant, par exemple, $u=0$; on obtient
de cette mani\`ere:
\[
  C =
  \frac{H'(\omega)}{H(\omega)} - \frac{\cn\omega\dn\omega}{\sn\omega} =
  \frac{\Theta'(\omega)}{\Theta(\omega)} .
\]
%-----039.png-----------------
Nous pouvons donc \'ecrire, apr\`es avoir pris pour variable $u=n(t-t_0)$,
\[
  \frac{D_u\mathrm{a}}{\mathrm{a}} = \frac{H'(u-\omega)}{H(u-\omega)}
  - \frac{\Theta'(u)}{\Theta(u)} + \frac{\Theta'(\omega)}{\Theta(\omega)} ,
\]
et, si l'on d\'esigne par $Ne^{i\nu}$ une nouvelle constante \`a laquelle nous donnons
cette forme, parce qu'elle doit \^etre, en g\'en\'eral, suppos\'ee imaginaire, on
aura
\[
  \mathrm{a} = Ne^{i\nu} \frac{H(u-\omega)}{\Theta(u)} \,
    e^{\frac{\Theta'(\omega)}{\Theta(\omega)}u} .
\]
De cette formule r\'esulte ensuite
\[
  A= Ne^{i(\nu+\alpha t_0)}
  \frac{H(u-\omega)}{\Theta(u)}\, e^{\left[\frac{i\alpha}{n} + \frac{\Theta'(\omega)}{\Theta(\omega)}\right]u} ,
\]
ou plus simplement, en mettant $\nu-\alpha t_0$ au lieu de $\nu$,
\[
  A= Ne^{i\nu}
  \frac{H(u-\omega)}{\Theta(u)}\, e^{\left[\frac{i\alpha}{n} + \frac{\Theta'(\omega)}{\Theta(\omega)}\right]u} ,
\]
et l'on en conclut imm\'ediatement
\begin{alignat*}{2}
  B &= \frac{\cn(u-\omega)}{\sn(u-\omega)}\: A &&=
    \sqrt{k'} Ne^{i\nu} \frac{H_1(u-\omega)}{\Theta(u)}\,
    e^{\left[\frac{i\alpha}{n} + \frac{\Theta'(\omega)}{\Theta(\omega)}\right]u} , \\
  C &= \frac{1}{i\sn(u-\omega)} A &&=
    \sqrt{k}  Ne^{i\nu} \frac{\Theta(u-\omega)}{i\Theta(u)}\,
    e^{\left[\frac{i\alpha}{n} + \frac{\Theta'(\omega)}{\Theta(\omega)}\right]u} .
\end{alignat*}
Des deux ind\'etermin\'ees $N$ et $\nu$ qui figurent dans ces expressions, la derni\`ere
seule subsistera comme quantit\'e arbitraire; $N$, qui est r\'eel et positif, se d\'etermine
comme nous allons le montrer.


\mysection{XIV.}\label{page31a}


Je fais \`a cet effet, pour plus de simplicit\'e, dans les expressions pr\'ec\'ed\-entes,\label{page31}
\[
\frac{i\alpha}{n} + \frac{\Theta'(\omega)}{\Theta(\omega)} = i\lambda,
\]
en observant que cette quantit\'e $\lambda$ est r\'eelle, car on a $\omega=i\upsilon$, ainsi que
%-----040.png-----------------
nous l'avons fait voir (p.~\pageref{page27a}). Cela \'etant, nous pouvons \'ecrire
\begin{align*}
  A &= \sqrt{k} N \frac{\Theta(u-\omega)
       e^{i(\lambda u+\nu)}}{\Theta(u)} \sn(u-\omega),
\\
  B &= \sqrt{k} N \frac{\Theta(u-\omega)
       e^{i(\lambda u+\nu)}}{\Theta(u)} \cn(u-\omega),
\\
  C &= \sqrt{k} N \frac{\Theta(u-\omega)
       e^{i(\lambda u+\nu)}}{i\Theta(u)} ,
\end{align*}
et je remarque tout d'abord que ces formules permettent de v\'erifier
facilement les conditions auxquelles doivent satisfaire les neuf coefficients
$a$, $b$, $c$,~\ldots. En premier lieu, nous en d\'eduisons:
\begin{multline*}
  Aa'' + Bb'' + Cc''
= \sqrt{k} N \frac{\Theta(u-\omega)e^{i(\lambda u+\nu)}}{\cn\omega\,\Theta(u)}
  [-\cn u \sn(u-\omega)
\\
+ \dn\omega \sn u \cn(u-\omega) - \sn\omega \dn u].
\end{multline*}
Or on a
\[
\cn u \sn(u-\omega)
- \dn\omega \sn u \cn(u-\omega)
+ \sn\omega \dn u = 0,
\]
cette \'equation \'etant l'une des relations fondamentales pour l'addition des
arguments [\textsc{Jacobi}, \textit{{\OE}uvres compl\`etes}, t.~II, p.~171, \'equation~(16)], et nous
obtenons ainsi:
\[
  a a'' + b b'' + c c'' = 0, \qquad
  a'a'' + b'b'' + c'c'' = 0.
\]
Je remarque ensuite que la somme des carr\'es $A^2 + B^2 + C^2$ s'\'evanouit
comme contenant en facteur $\sn^2(u-\omega) + \cn^2(u-\omega) - 1$, et nous en
concluons
\[
  a^2 + b^2 + c^2 = a'^2 + b'^2+ c'^2, \qquad
  aa' + bb' + cc' = 0.
\]
\medskip

Ayant d'ailleurs
\begin{multline*}
  a''^2 + b''^2 + c''^2
= \left( \frac{          \cn u}{\cn\omega} \right)^2
 + \left( \frac{\dn\omega \sn u}{\cn\omega} \right)^2
 - \left( \frac{\sn\omega \dn u}{\cn\omega} \right)^2
\\
= \frac{ 1-                \sn^2 u}{\cn^2\omega}
 + \frac{(1-k^2\sn^2\omega) \sn^2 u}{\cn^2\omega}
 - \frac{(1-k^2\sn^2 u) \sn^2\omega}{\cn^2\omega} = 1,
\end{multline*}
les six relations que nous avons en vue seront compl\`etement v\'erifi\'ees d\`es
que $N$ sera d\'etermin\'e de mani\`ere \`a obtenir
$a^2 + b^2 + c^2 = 1$~(\footnote{%start footnote
  Les \'equations
\[
  iA = Bc'' - Cb'', \qquad iB = Ca'' - Ac'', \qquad iC = Ab'' - Ba'',
\]
dont la premi\`ere a \'et\'e employ\'ee pr\'ec\'edemment,
page~\pageref{page27b}, et qui contiennent les suivantes:
\begin{align*}
  a &= b'c'' - c'b'', &
  b &= c'a'' - a'c'', &
  c &= a'b'' - b'a'';
\\
  a'&= b''c - c''b, &
  b'&= c''a - a''c, &
  c'&= a''b - b''a,
\end{align*}
se v\'erifient aussi de la mani\`ere la plus facile. Les relations auxquelles elles conduisent, \`a
savoir:
\begin{alignat*}{2}
   \cn\omega
&= \cn u      \cn(u-\omega) &&+ \dn\omega\sn u \sn (u-\omega),
\\
   \cn u
&= \cn \omega \cn(u-\omega) &&- \dn\omega\sn u \sn (u-\omega),
\\
   \dn\omega\sn u
&= \cn \omega \sn(u-\omega) &&+ \sn\omega\dn u \cn (u-\omega),
\end{alignat*}
figurent, en effet, dans le tableau donn\'e par Jacobi sous les
n\textsuperscript{os} 9, 10 et 11. Formons enfin les trois produits
\[
(b-ib')(c + ic'),\quad (c-ic')(a + ia'),\quad (a-ia')(b+ib'),
\]
nous trouverons
\begin{align*}
   (b-ib')(c+ic')
&= \frac{ \Theta(0)H_1(0)H_1(u+\omega)\Theta(u-\omega) }
        {  H_1^2(\omega)\Theta^2(u) } i,
\\
   (c-ic')(a+ia')
&= \frac{ \Theta_1(0)H_1(0)\Theta(u+\omega)H(u-\omega) }
        { iH_1^2(\omega)\Theta^2(u) },
\\
   (a-ia')(b+ib')
&= \frac{ \Theta(0)\Theta_1(0)H(u+\omega)H_1(u-\omega) }
        {  H_1^2(\omega)\Theta^2(u) };
\end{align*}
or les relations \'el\'ementaires
\begin{align*}
   \Theta(0)H_1(0)H_1(u+\omega)\Theta(u-\omega)
&= H(\omega)\Theta_1(\omega)H(u)\Theta_1(u)
 - H_1(\omega)\Theta(\omega)\Theta(u)H_1(u),
\\
   \Theta_1(0)H_1(0)\Theta(u+\omega)H(u-\omega)
&= H(\omega)\Theta(\omega)H_1(u)\Theta_1(u)
 - H_1(\omega)\Theta_1(\omega)\Theta(u)H(u),
\\
   \Theta(0)H_1(0)H(u+\omega)H_1(u-\omega)
&= \Theta(\omega)\Theta_1(\omega)H(u)H_1(u)
 + H(\omega)H_1(\omega)\Theta(u)\Theta_1(u)
\end{align*}
conduisent facilement \`a ces \'egalit\'es
\begin{align*}
(b-ib')(c + ic') & = -b''c''+ia'',\\
(c-ic')(a + ia') & = -c''a''+ib'',\\
(a-ia')(b + ib') & = -a''b''+ic'';
\end{align*}
d'o\`u l'on tire ce nouveau syst\`eme de conditions:
\begin{align*}
bc+b'c'+b''c'' &= 0, &
bc' - cb' &= a'',\\
ca+c'a'+c''a'' &= 0, &
ca' - ac' &= b'',\\
ab+a'b'+a''b'' &= 0, &
ab' - ba' &= c''.
\end{align*}
}). % end footnote
Formons pour cela les carr\'es des modules de $A$, $B$, $C$;
%-----041.png----------------
\enlargethispage{1ex}
en remarquant que, par le changement de $i$ en $-i$, $\omega$ se
change en $-\omega$, on trouve imm\'ediatement
\begin{align*}
   a^2 + a'^2
&= kN^2\frac{\Theta(u+\omega)\Theta(u-\omega)}{\Theta^2(u)}
   \sn(u+\omega)\sn(u-\omega),
\\
   b^2 + b'^2
&= kN^2\frac{\Theta(u+\omega)\Theta(u-\omega)}{\Theta^2(u)}
   \cn(u+\omega)\cn(u-\omega),
\\
   c^2 + c'^2
&= kN^2\frac{\Theta(u+\omega)\Theta(u-\omega)}{\Theta^2(u)};
\end{align*}
%-----042.png------------------
d'o\`u, en ajoutant membre \`a membre,
\[
  2=
  kN^2 \frac{\Theta(u+\omega) \Theta(u-\omega)}{\Theta^2(u)}
  [\sn(u+\omega)\sn(u-\omega) + \cn(u+\omega)\cn(u-\omega) + 1].
\]
Or les formules \'el\'ementaires
\begin{align*}
  \sn(u+\omega)\sn(u-\omega) &=
  \frac{\sn^2 u - \sn^2 \omega}{1-k^2\sn^2 u\sn^2 \omega},\\
  \cn(u+\omega)\cn(u-\omega) &=
  -1+\frac{\cn^2 u + \cn^2 \omega}{1-k^2\sn^2 u\sn^2 \omega}
\end{align*}
donnent
\[
  \sn(u+\omega)\sn(u-\omega) + \cn(u+\omega)\cn(u-\omega) + 1
  = \frac{2\cn^2 \omega}{1-k^2 \sn^2 u \sn^2 \omega};
\]
on a d'ailleurs
\[
\frac{\Theta^2(0)\Theta(u+\omega)\Theta(u-\omega)}{\Theta^2(u)\Theta^2(\omega)}
  = 1- k^2\sn^2 u \sn^2 \omega ;
\]
nous obtenons donc
\[
  1 = kN^2 \frac{\Theta^2(\omega) \cn^2 \omega}{\Theta^2 (0)},
\]
et par cons\'equent, apr\`es une r\'eduction facile,
\[
  N = \frac{\Theta_1(0)}{H_1(\omega)} .
\]
On en conclut les r\'esultats de Jacobi, que nous gardons sous la forme
suivante:
\begin{align*}
  a+ia' &=
\frac{ \Theta_1(0) H(u-\omega) e^{i(\lambda u+\nu)} }{ H_1(\omega) \Theta(u) },
\\
  b+ib' &=
\frac{ \Theta(0) H_1(u-\omega) e^{i(\lambda u+\nu)} }{ H_1(\omega) \Theta(u) },
\\
  c+ic' &=
\frac{ H_1(0) \Theta(u-\omega) e^{i(\lambda u+\nu)} }{ i H_1(\omega) \Theta(u) },
\end{align*}
et il ne nous reste plus qu'\`a y joindre les expressions des vitesses de rotation
autour des axes fixes $Ox$, $Oy$, $Oz$.

Ces quantit\'es, que je d\'esignerai par $v$, $v'$, $v''$, ont pour valeurs
\begin{alignat*}{3}
  v  &= ap  &&+ bq  &&+ cr, \\
  v' &= a'p &&+ b'q &&+ c'r, \\
  v''&= a''p&&+ b''q&&+ c''r,
\end{alignat*}
%-----043.png------------------
ou encore, en rempla\c{c}ant $p$, $q$, $r$ par $\alpha a''$, $\beta b''$, $\gamma c''$,
\begin{alignat*}{3}
  v  &= aa'' \alpha &&+ bb'' \beta &&+ cc''\gamma, \\
  v' &= a'a'' \alpha &&+ b'b'' \beta &&+ c'c'' \gamma , \\
  v''&= a''^2 \alpha &&+ b''^2 \beta &&+ c''^2 \gamma = \delta .
\end{alignat*}

Cela pos\'e, soit $v+iv' = V$, nous pouvons \'ecrire
\[
  V = A a'' \alpha + B b'' \beta + C c'' \gamma,
\]
et, si nous employons de nouveau les \'egalit\'es
\[
  B = \frac{a'' b'' -ic''}{a''^2-1} A, \qquad
  C = \frac{a'' c'' +ib''}{a''^2-1} A,
\]
on obtiendra la formule
\[
  V = \frac{(\delta-\alpha)a'' + i(\gamma-\beta)b''c''}{a''^2-1} A.
\]
Or, au moyen des relations
\[
  \delta-\alpha = -in \frac{\sn\omega\dn\omega}{\cn\omega}, \qquad
  \gamma-\beta  = in \frac{\cn\omega}{\sn\omega\dn\omega}
\]
et des valeurs de $a''$, $b''$, $c''$, il vient
\begin{align*}
  \frac{(\delta-\alpha)a'' + i(\gamma-\beta)b''c''}{a''^2-1} &=
  -in \frac{\sn u\cn u\dn\omega + \sn\omega\cn\omega\dn u}{\sn^2 u - \sn^2\omega} \\
  &= -in \frac{\dn(u-\omega)}{\sn(u-\omega)} ;
\end{align*}
l'expression pr\'ec\'edente de $A$ nous donne donc imm\'ediatement
\[
  V= -in \frac{H'(0)\Theta_1(u-\omega) e^{i(\lambda u+\nu)} }{H_1(\omega)\Theta(u)} .
\]

Voici maintenant la seconde m\'ethode que j'ai annonc\'ee pour parvenir
\`a la d\'etermination des quantit\'es $A$, $B$, $C$.


\mysection{XV.}


Je reprends l'\'equation diff\'erentielle du second ordre, obtenue
au \S~XII, p.~\pageref{page29}, \`a savoir:
\[
  D_t^2 \mathrm{a} =
  \left[(\beta-\alpha)(\gamma-\delta) - (\delta-\alpha)(\gamma-\alpha)
   - 2(\beta-\alpha)(\gamma-\alpha) a''^2 \right] \mathrm{a},
\]
%-----044.png----------------
et j'y joins les deux suivantes, qui s'en tirent par un changement de lettres:
\begin{align*}
D^2_t\mathrm{b} &= \left[ (\gamma-\beta)(\alpha-\delta) -(\delta-\beta)(\alpha-\beta) -2(\gamma-\beta)(\alpha-\beta)b''^2 \right] \mathrm{b},\\
D^2_t\mathrm{c} &= \left[ (\alpha-\gamma)(\beta-\delta) -(\delta-\gamma)(\beta-\gamma) -2(\alpha-\gamma)(\beta-\gamma)c''^2 \right] \mathrm{c}.
\end{align*}
Cela pos\'e, au moyen des expressions de $a''$, $b''$, $c''$, en fonction de $u$, et de
ces formules qu'on \'etablit sans peine,\label{page36}
\begin{align*}
\alpha-\beta  &= in \frac{k^2\sn\omega\cn\omega}{\dn\omega},&
\beta -\delta &= in \frac{k'^2\sn\omega}{\cn\omega\dn\omega},\\
\alpha-\delta &= in \frac{\sn\omega\dn\omega}{\cn\omega},&
\gamma -\beta &= in \frac{\cn\omega}{\sn\omega\dn\omega},\\
\gamma-\alpha &= in \frac{\cn\omega\dn\omega}{\sn\omega},&
\gamma-\delta &= in \frac{\dn\omega}{\sn\omega\cn\omega},
\end{align*}
nous obtenons, par un calcul facile,
\begin{align*}
&(\beta-\alpha)(\gamma-\delta) -(\delta-\alpha)(\gamma-\alpha) -2(\beta-\alpha)(\gamma -\alpha)a''^2 \\
& \qquad = n^2\left[2k^2{\sn}^2u-1-k^2+k^2{\sn}^2\omega\right],\\[1ex]
&(\gamma-\beta)(\alpha-\delta) -(\delta-\beta)(\alpha-\beta) -2(\gamma-\beta)(\alpha-\beta)b''^2 \\
& \qquad = n^2\left[2k^2{\sn}^2u-1-k^2+k^2\frac{\cn^2\omega}{\dn^2\omega} \right],\\[1ex]
&(\alpha-\gamma)(\beta-\delta) -(\delta-\gamma)(\beta-\gamma) -2(\alpha-\gamma)(\beta-\gamma)c''^2\\
& \qquad = n^2\left[2k^2{\sn}^2u-1-k^2+\frac{1}{\sn^2\omega}\right].
\end{align*}
Prenant donc pour variable ind\'ependante $u$ au lieu de $t$, on aura
\begin{align*}
D^2_u\mathrm{a} & = \left[2k^2{\sn}^2u-1-k^2+k^2{\sn}^2\omega\right]\mathrm{a},\\
D^2_u\mathrm{b} & = \left[2k^2{\sn}^2u-1-k^2+k^2\frac{\cn^2\omega}{\dn^2\omega}\right] \mathrm{b},\\
D^2_u\mathrm{c} & = \left[2k^2{\sn}^2u-1-k^2 + \frac{1}{\sn^2\omega}\right]\mathrm{c};
\end{align*}
et nous nous trouvons, par cons\'equent, amen\'es \`a trois des quatre formes
canoniques de l'\'equation de Lam\'e, qui ont \'et\'e consid\'er\'ees au \S~VI, p.~\pageref{page14}.
La solution g\'en\'erale de ces \'equations nous donne donc, en d\'esignant les
constantes arbitraires par $P$, $Q$, $R$, $P'$, $Q'$, $R'$,
\begin{alignat*}{2}
\mathrm{a} & = P\frac{H(u-\omega) e^{\frac{\Theta'(\omega)}{\Theta(\omega)}u}}{\Theta(u)} &&+
P'\frac{H(u+\omega) e^{-\frac{\Theta'(\omega)}{\Theta(\omega)}u}}{\Theta(u)},\\
\mathrm{b} & = Q\frac{H_1(u-\omega) e^{\frac{\Theta'_1(\omega)}{\Theta_1(\omega)}u}}{\Theta(u)} &&+
Q'\frac{H_1(u+\omega) e^{-\frac{\Theta'_1(\omega)}{\Theta_1(\omega)}u}}{\Theta(u)},\\
\mathrm{c} & = R\frac{\Theta(u-\omega) e^{\frac{H'(\omega)}{H(\omega)}u}}{\Theta(u)} &&+
R'\frac{\Theta(u+\omega) e^{-\frac{H'(\omega)}{H(\omega)}u}}{\Theta(u)},
\end{alignat*}
%-----045.png-----------------
et l'on en conclut, si l'on \'ecrit, pour plus de simplicit\'e, $P$, $Q$,
$R$,~\ldots\ au lieu de $Pe^{i\alpha t_0}$, $Qe^{i\beta t_0}$,
$Re^{i\gamma t_0}$,~\ldots,
\begin{alignat*}{4}
  A &= P \frac{H(u-\omega)}{\Theta(u)}
&&       e^{\left[ \frac{i\alpha}{n}
               + \frac{\Theta'(\omega)}{\Theta(\omega)} \right]u}
&&     + P'\frac{H(u+\omega)}{\Theta(u)}
&&       e^{\left[ \frac{i\alpha}{n}
               - \frac{\Theta'(\omega)}{\Theta(\omega)} \right]u},
\\
  B &= Q \frac{H_1(u-\omega)}{\Theta(u)}
&&       e^{\left[ \frac{i \beta}{n}
               + \frac{\Theta_1'(\omega)}{\Theta_1(\omega)} \right]u}
&&     + Q'\frac{H_1(u+\omega)}{\Theta(u)}
&&       e^{\left[ \frac{i \beta}{n}
               - \frac{\Theta_1'(\omega)}{\Theta_1(\omega)} \right]u},
\\
  C &= R \frac{\Theta(u-\omega)}{\Theta(u)}
&&       e^{\left[ \frac{i\gamma}{n}
               + \frac{H'(\omega)}{H(\omega)} \right]u}
&&     + R'\frac{\Theta(u+\omega)}{\Theta(u)}
&&       e^{\left[ \frac{i\gamma}{n}
               - \frac{H'(\omega)}{H(\omega)} \right]u}.
\end{alignat*}
La d\'etermination des six constantes qui entrent dans ces expressions se fait
tr\`es-facilement, comme on va le voir.

Je remarque, en premier lieu, que nous pouvons poser\label{page37}
\[
  \frac{i\alpha}{n} + \frac{\Theta'  (\omega)}{\Theta  (\omega)}
= \frac{i \beta}{n} + \frac{\Theta'_1(\omega)}{\Theta_1(\omega)}
= \frac{i\gamma}{n} + \frac{H'(\omega)}{H(\omega)}
= i\lambda,
\]
$\lambda$ d\'esignant la quantit\'e d\'ej\`a consid\'er\'ee au \S~XIV,
p.~\pageref{page31}. On a, en effet,
\begin{alignat*}{2}
  \frac{\Theta'_1(\omega)}{\Theta_1(\omega)}
- \frac{\Theta'  (\omega)}{\Theta  (\omega)}
&= D_{\omega} \log\dn\omega
&&= -\frac{k^2 \sn\omega \cn\omega}{\dn\omega},
\\
  \frac{H'(\omega)}{H(\omega)}
- \frac{\Theta'  (\omega)}{\Theta  (\omega)}
&= D_{\omega} \log\sn\omega
&&= \frac{ \cn\omega \dn\omega}{\sn\omega},
\end{alignat*}
et les \'egalit\'es pr\'ec\'edentes sont v\'erifi\'ees au moyen des relations
\[
  \alpha - \beta = in\frac{k^2 \sn\omega \cn\omega}{\dn\omega},
\qquad
  \gamma -\alpha = in\frac{    \cn\omega \dn\omega}{\sn\omega},
\]
que nous avons donn\'ees plus haut. Une cons\'equence importante d\'ecoule
de l\`a: c'est qu'en changeant $u$ en $u + 4K$, les fonctions
\[
\frac{H     (u-\omega) e^{i\lambda u}}{\Theta(u)},\quad
\frac{H_1   (u-\omega) e^{i\lambda u}}{\Theta(u)},\quad
\frac{\Theta(u-\omega) e^{i\lambda u}}{\Theta(u)}
\]
se reproduisent multipli\'ees par le m\^eme facteur
$e^{4i\lambda K}$, tandis que les quantit\'es
\[
\frac{H(u+\omega)}{\Theta(u)}\,
e^{\left[ \frac{i\alpha}{n}
        - \frac{\Theta'  (\omega)}{\Theta  (\omega)} \right]u},
\
\frac{H_1(u+\omega)}{\Theta(u)}\,
e^{\left[ \frac{i \beta}{n}
        - \frac{\Theta'_1(\omega)}{\Theta_1(\omega)} \right]u},
\
\frac{\Theta(u+\omega)}{\Theta(u)}\,
e^{\left[ \frac{i\gamma}{n}
        - \frac{H'(\omega)}{H(\omega)} \right]u},
\]
sont affect\'ees des facteurs
$e^{4i\left( \frac{\alpha}{n} - \lambda K \right)}$,
$e^{4i\left( \frac{\beta }{n} - \lambda K \right)}$,
$e^{4i\left( \frac{\gamma}{n} - \lambda K \right)}$,
essentiellement in\'egaux. Or on a obtenu, pour les quotients $\frac{B}{A}$, $\frac{C}{A}$, des fonctions
doublement p\'eriodiques, ne changeant point quand on met $u + 4K$ au
%-----046.png-----------------
lieu de $u$; il faut donc que les facteurs qui multiplient $A$, $B$, $C$, lorsqu'on
remplace $u$ par $u + 4K$, soient les m\^emes, ce qui exige qu'on fasse $P'= 0$,
$Q'= 0$, $R' = 0$. Ce point \'etabli, j'\'ecris, en modifiant convenablement la
forme des constantes $P$, $Q$,~$R$,
\begin{align*}
A &= P\frac{\Theta(u-\omega) e^{i\lambda u}}{\Theta(u)} \sn(u-\omega),
\\
B &= Q\frac{\Theta(u-\omega) e^{i\lambda u}}{\Theta(u)} \cn(u-\omega),
\\
C &= R\frac{\Theta(u-\omega) e^{i\lambda u}}{\Theta(u)};
\end{align*}
et j'emploie la condition $Aa'' + Bb'' + Cc'' = 0$, qui conduit \`a l'\'egalit\'e
\[
  -P \cn u \sn(u-\omega)
+  Q \dn\omega \sn u \cn (u-\omega)
- iR \sn\omega \dn u = 0.
\]
Or, en faisant $a = 0$ et $u = \omega$, on en d\'eduit
\[
  P = Q = iR;
\]
de sorte qu'on peut poser
\[
  P = \sqrt{k}Ne^{i\nu}, \quad
  Q = \sqrt{k}Ne^{i\nu}, \quad
  R = \frac{\sqrt{k}Ne^{i\nu}}{i},
\]
ce qui nous donne les expressions de $A$, $B$, $C$ obtenues au \S~XIV,
p.~\pageref{page31a}. Le calcul s'ach\`eve donc en d\'eterminant, ainsi qu'on
l'a fait plus haut, la valeur du facteur $N$.


\mysection{XVI.}


Les formules que nous venons d'\'etablir ont \'et\'e le sujet des travaux de
plusieurs g\'eom\`etres; M.~Somoff en a donn\'e une d\'emonstration dans un
M\'emoire du \textit{Journal de Crelle}~(\footnote{
  \textit{D\'emonstration des formules de M.~Jacobi relatives \`a la
  th\'eorie de la rotation d'un
  corps solide}, t.~42, p.~95.}),
peu diff\'erente de celle de Jacobi, et qui
repose aussi sur l'emploi des trois angles d'Euler. M.~Brill, dans un excellent
travail intitul\'e: \textit{Sul problema della rotazione dei corpi} (\textit{Annali di Matematica},
serie~II, t.~III, p.~33), a employ\'e le premier les \'equations diff\'erentielles
de Poisson et les quantit\'es $a + ia'$, $b + ib'$, $c + ic'$ dont j'ai fait usage,
mais son analyse est enti\`erement diff\'erente de la mienne. C'est \`a un autre
point de vue que s'est plac\'e M.~Chelini~(\footnote{
  \textit{Determinazione analitica della rotazione dei corpi liberi
  secundo i concetti del signor
  Poinsot} (\textit{Memorie dell'Accademia delle Scienze dell'Istituto di Bologna}, vol.~X).})
en d\'eduisant pour la premi\`ere
%-----047.png----------------
fois les cons\'equences analytiques de la belle th\'eorie de Poinsot, que son
auteur ni personne n'avait encore donn\'ees d'une mani\`ere aussi approfondie.
Je mentionnerai enfin deux r\'ecents M\'emoires de M.~Siacci, professeur
\`a l'Universit\'e de Turin, et dont l'auteur a bien voulu, dans la lettre
suivante, m'indiquer les points les plus essentiels:
\bigskip

{\small\hfill\og Turin, 24 d\'ecembre 1877.\indent}\medskip

{\fg}~Poinsot, \`a la fin de son \textit{M\'emoire sur la rotation des corps}, d\'emontre que
la section diam\'etrale de l'ellipso\"ide central, d\'etermin\'ee par le plan parall\`ele
au couple d'impulsion, a son aire constante. Ce th\'eor\`eme a \'et\'e le point
de d\'epart d'un M\'emoire~(\footnote{
  \textit{Memorie della Societ\`a italiana delle Scienze}, serie~III, t.~III.})
dont les r\'esultats se rattachent \`a la th\'eorie des
fonctions elliptiques aussi bien qu'\`a la th\'eorie de la rotation. Je me suis
d'abord propos\'e le probl\`eme de d\'eterminer le mouvement des axes de cette
section: pour abr\'eger, je l'appellerai \textit{section invariable}, et son plan, \textit{plan
invariable}. Une premi\`ere solution du probl\`eme est sugg\'er\'ee par l'homoth\'etie
de la section invariable avec l'indicatrice de Dupin, relative \`a
l'extr\'emit\'e de l'axe instantan\'e (p\^ole). La rotation d'un syst\`eme de trois
axes rectangulaires, dont les premiers co\"incident avec les axes de la section,
n'est que la r\'esultante de deux rotations, l'une due au mouvement
du p\^ole sur la polo\"ide, l'autre due au mouvement de l'ellipso\"ide. Soient,
sur ces axes, $P_1$, $P_2$, $P_3$ les composantes de la premi\`ere vitesse angulaire;
$m_1$, $m_2$, $m_3$ celles de la seconde. La r\'esultante se composera de $P_1+m_1$,
$P_2+m_2$, $P_3+m_3$; et, comme le p\^ole reste sur un plan, on aura
\[
\tag{1}
  P_1+m_1 = 0,\quad
  P_2+m_2 = 0,\quad
  P_3+m_3 = d\psi:dt,
\]
$\psi$ \'etant la longitude d'un des axes de la section. Soient $\sqrt{a_1}$, $\sqrt{a_2}$, $\sqrt{a_3}$ les
demi-axes de l'ellipso\"ide (le troisi\`eme est celui qui ne se couche jamais sur
le plan invariable); $x_1$, $x_2$, $x_3$ les coordonn\'ees du p\^ole; $\lambda_1$, $\lambda_2$, $\lambda_3$ (${\lambda_3=0}$,
$\lambda_1$, $\lambda_2$ sont les demi-axes carr\'es de la section) les racines de l'\'equation
\[
(\lambda) \equiv
\frac{x_1^2}{a_1-\lambda} +\frac{x_2^2}{a_2-\lambda} +\frac{x_3^2}{a_3-\lambda}-1=0.
\]
On aura
\[
m_r^2= \frac{(a_1-\lambda_r)(a_2-\lambda_r)(a_3-\lambda_r)
  }{(\lambda_r-\lambda_s)(\lambda_r-\lambda_{s'})},
\quad
2P_r dt=\frac{m_s m_{s'}}{\lambda_s-\lambda_{s'}}
\left(\frac{d\lambda_s}{m^2_s} +\frac{d\lambda_{s'}}{m^2_{s'}}\right)
\]
($r$, $s$, $s'$ \'etant trois nombres de la s\'erie 1, 2, 3). Comme $\lambda_1\lambda_2 = \textrm{const.}= c^2$, on
a ${m_3 = \textrm{const}}$. C'est, en effet, la distance du centre $O$ au plan fixe de contact;
%-----048.png-----------------
de m\^eme $m_1$, $m_2$ sont les distances de $O$ des plans tangents aux surfaces $(\lambda_1)$ et $(\lambda_2)$. Au moyen de ces valeurs, les \'equations~(1), qui reviennent en substance
aux \'equations d'Euler, donnent $t$ et $\psi$ en fonction de $x = \lambda_1 + \lambda_2$.
En posant $t = nu$ ($n$ expression connue), on obtient
\begin{align*}
\tag{2}
  \psi
&= \mp \frac{u}{2} \left( \frac{d\log\sn i\sigma}{d\sigma}
                        + \frac{d\log\sn i\tau  }{d\tau  } \right)
   \pm \frac{1}{2i} \left[\Pi(u, i\sigma) + \Pi(u, i\tau) \right],
\\
\tag{3}
  \psi
&= \mp \frac{u}{2} \left[ \frac{d\log H(i\sigma)}{d\sigma}
                        + \frac{d\log H(i\tau  )}{d\tau  } \right]
   \pm \frac{1}{4i} \log
          \frac{\Theta(n-i\sigma) \Theta(n-i\tau)}{\Theta(n+i\sigma) \Theta(n+i\tau)},
\end{align*}
et l'on prendra le signe sup\'erieur ou inf\'erieur, suivant que
$m_3^2 >$ ou $< a_2$.
Le module est
\[
k = \sqrt{ \frac{a_3(a_2-a_1)(c^2-a_1a_2)}{a_1(a_2-a_3)(c^2-a_2a_3)} },
\]
et $\sigma$ et $\tau$ sont ainsi donn\'es:
\begin{gather*}
  \tau   = \int_0^F \frac{d\varphi}{\sqrt{1-k'^2\sin^2\varphi}}, \qquad
  \sigma = \int_0^G \frac{d\varphi}{\sqrt{1-k'^2\sin^2\varphi}},
\\[0.7ex]
  \cos\begin{pmatrix}F\\G\end{pmatrix} = \frac{c\pm a_3}{a_3\pm c}
  \sqrt{\frac{a_3}{a_2}},
\end{gather*}
$F$ \'etant un angle aigu n\'egatif ou positif, suivant que
$m_3^2 \gtrless a_2$ et $G$ un angle
positif, qui sera $<$ ou $> \frac{1}{2}\pi$, suivant que la zone entour\'ee par la polo\"ide
comprendra deux ombilics ou aucun: c'est, en effet, ce qui revient
aux cas de $G \lessgtr \frac{1}{2}\pi$ ou de $\sigma \lessgtr K'$. La double expression
\[
c\,\frac{    H(i\sigma) \sqrt{ \Theta(u+i\tau  ) \Theta(u-i\tau  ) }
       \pm H(i\tau  ) \sqrt{ \Theta(u+i\sigma) \Theta(u-i\sigma) }}{
           H(i\sigma) \sqrt{ \Theta(u+i\tau  ) \Theta(u-i\tau  ) }
       \mp H(i\tau  ) \sqrt{ \Theta(u+i\sigma) \Theta(u-i\sigma) }}
\]
donne $\lambda_1$ et $\lambda_2$. L'\'etude de l'expression~(3) d\'emontre que le mouvement
moyen des demi-axes de la section est donn\'e par le terme multipli\'e par $u$,
et l'in\'egalit\'e par l'autre, lorsque $\sigma < K'$; lorsque $\sigma > K'$, le mouvement
moyen et l'in\'egalit\'e sont donn\'es par les m\^emes termes en y changeant $\sigma$
en $\sigma - 2K'$; et l'on trouve que, dans le second cas, le mouvement moyen
co\"incide avec celui des projections des demi-axes $\sqrt{a_1}$ et $\sqrt{a_2}$, et dans le
premier avec celui des projections de $\sqrt{a_3}$ et de l'axe instantan\'e.

{\frqq}~On peut tirer $\psi$ de l'expression de la longitude $(\mu)$ d'une droite quelconque
$OR$, dont l'extr\'emit\'e a $\xi_1$, $\xi_2$, $\xi_3$ pour coordonn\'ees. Je trouve ainsi
\begin{multline*}
\psi + \arctang \left[
  \left( \tfrac{m_2 x_1 \xi_1}{a_1 - \lambda_2}
       + \tfrac{m_2 x_2 \xi_2}{a_2 - \lambda_2}
       + \tfrac{m_2 x_3 \xi_3}{a_3 - \lambda_2}
  \right) :
  \left( \tfrac{m_1 x_1 \xi_1}{a_1 - \lambda_1}
       + \tfrac{m_1 x_2 \xi_2}{a_2 - \lambda_1}
       + \tfrac{m_1 x_3 \xi_3}{a_3 - \lambda_1}
  \right)
\right] \\
= (\mu),\quad
\end{multline*}
et je donne aussi l'expression d\'evelopp\'ee de $(\mu)$. Comme $\xi_1$, $\xi_2$, $\xi_3$ sont
fonctions arbitraires de $u$, on voit l'infinit\'e de formes qu'on peut donner
\`a l'expression~(2) de~$\psi$.
%-----049.png-----------------------------

En faisant co\"incider $OR$ avec $\sqrt{a_1}$, $\sqrt{a_2}$, $\sqrt{a_3}$ et avec l'axe instantan\'e,
on obtient leurs longitudes $\mu_1$, $\mu_2$, $\mu_3$, $\mu$ et l'on a
\[
\tag{4}
\psi =\mu_r-\arctang\frac{m_2}{m_1}\frac{a_r-\lambda_1}{a_r-\lambda_2}
=\mu-\arctang\frac{m_2}{m_1}.
\]

Ces quatre expressions de $\psi$ contiennent les principaux th\'eor\`emes
sur la transformation et sur l'addition des param\`etres des int\'egrales elliptiques
de troisi\`eme esp\`ece, mais sous une forme nouvelle, \`a cause des
termes circulaires.

Le mouvement des projections des axes du corps et de l'axe instantan\'e
a \'et\'e d\'etermin\'e par Jacobi: leurs in\'egalit\'es sont donn\'ees au moyen
d'une constante $a$, qui se trouve li\'ee avec nos quantit\'es par l'\'equation
$\sigma+\tau = 2a$; mais aux expressions des mouvements moyens concourent les
moments d'inertie du corps. Au moyen des quantit\'es $\sigma$ et $\tau$, elles acqui\`erent,
comme on a vu, une forme plus homog\`ene. Si nous posons $\sigma-\tau = 2b$,
les constantes du probl\`eme $a_1$, $a_2$, $a_3$, $m_3$ se transforment en $a$, $b$, $c$, $k$.
Ainsi l'on a
\begin{align*}
  \frac{a_1}{c}&=\frac{\sn ia \dn ia \cn ib}{\sn ib \dn ib \cn ia},
& \frac{a_2}{c}&=\frac{\sn ia \cn ib \dn ib}{\sn ib \cn ia \dn ia},
& \frac{a_3}{c}&=\frac{\sn ib \cn ib \dn ia}{\sn ia \cn ia \dn ib},
\\[0.7ex]
  \frac{x_1^2}{a_1}&= \frac{\cn^2  u}{\cn^2 ib},
& \frac{x_2^2}{a_2}&= \frac{\dn^2 ib}{\cn^2 ib}\sn^2 u,
& \frac{x_3^2}{a_3}&=-\frac{\sn^2 ib}{\cn^2 ib}\dn^2 u;
\end{align*}
en changeant $x_r^2:a_r$ en $m_3^2x_r^2:a_r^2$, on change $b$ en $a$.

J'ajouterai aux r\'esultats de mon M\'emoire les cosinus de direction des
axes de la section invariable par rapport \`a l'axe instantan\'e et aux axes du
corps; ils sont:
\begin{gather*}
\tfrac{m_1}{\sqrt{m_1^2+m_2^2}} =\mp\tfrac{Y\dn(u+ia)-X\dn(u-ia)}{2i\sqrt{XY\dn(u+ia)\dn(u-ia)}},
\\
\tfrac{m_2}{\sqrt{m_1^2+m_2^2}} =-\tfrac{Y\dn(u+ia)+X\dn(u-ia)}{2\sqrt{XY\dn(u+ia)\dn(u-ia)}},
\\[0.7ex]
\begin{aligned}
  \tfrac{m_1x_1}{a_1-\lambda_1}
&=-\tfrac{Y\sn(u+ia)+X\sn(u-ia)}{2\cn ia\sqrt{XYZ}},
& \tfrac{m_2x_1}{a_1-\lambda_2}
&=\mp\tfrac{Y\sn(u+ia)-X\sn(u-ia)}{2i\cn ia\sqrt{XYZ}},
\\
\tfrac{m_1x_2}{a_2-\lambda_1}
&=-\tfrac{Y\cn(u+ia)+X\cn(u-ia)}{2\cn ia\sqrt{XYZ}},
& \tfrac{m_2x_2}{a_2-\lambda_2}
&=\mp\tfrac{Y\cn(u+ia)-X\cn(u-ia)}{2i\cn ia\sqrt{XYZ}},
\\
\tfrac{m_1x_3}{a_3-\lambda_1}
&=-\mp\tfrac{Y-X}{2i\cn ia\sqrt{XYZ}},
& \tfrac{m_2x_3}{a_3-\lambda_2}
&=-\tfrac{Y+X}{2\cn ia\sqrt{XYZ}},
\end{aligned}
\end{gather*}
%-----050.png------------------
o\`u
\begin{gather*}
  X^2= 1-k^2 \sn^2 ib \sn^2(u + ia),\qquad
  Y^2= 1-k^2 \sn^2 ib \sn^2(u - ia),\\
  Z(1-k^2 \sn^2 ia \sn^2 u) =1, \\
  \frac{n}{\sqrt{c}} =
    \pm\frac{2\sn i\sigma \sn i\tau}{\sqrt{\sn^2 i\tau - \sn^2 i\sigma}} .
\end{gather*}
Les doubles signes se rapportent aux cas de $m_3^2\gtrless a_2$, avec la convention
que, suivant que $a+b >$ ou $<K'$, $X$, $Y$, ou bien $X\sn(u-ia)$, $Y\sn(u+ia)$
imaginaires conjugu\'es, aient leur partie r\'eelle positive. On tire ces expressions
de (4). La substitution directe des valeurs $x_1$, $x_2$, $x_3$; $m_1$, $m_2$; $\lambda_1$, $\lambda_2$
donne des expressions assez simples, mais tout \`a fait diff\'erentes, et leur
comparaison donne lieu \`a des formules remarquables.\fg % closing quote missing in original
\bigskip

Les r\'esultats dont on vient de voir l'indication succincte sont les premiers
qui aient \'et\'e ajout\'es aux travaux de Jacobi dans la th\'eorie de la
rotation; mais je dois signaler encore, en raison de l'int\'er\^et que j'y attache,
un point non mentionn\'e dans le r\'esum\'e pr\'ec\'edent. Rempla\c{c}ons, dans le
plan invariable, les axes fixes $Ox$, $Oy$ par deux autres \'egalement rectangulaires,
mais mobiles, $Ox_1$, $Oy_1$, dont le premier soit constamment parall\`ele
\`a la direction du rayon vecteur de l'erpolo\"ide; M.~Chelini a introduit,
en suivant la m\'ethode de Poinsot, les angles des axes d'inertie avec
les droites $Ox_1$, $Oy_1$, $Oz$, et donn\'e ce syst\`eme de formules, o\`u $\iota$ d\'esigne
le rayon vecteur de l'erpolo\"ide:
\begin{align*}
  \cos(x_1 x') &= \frac{(\alpha-\delta)a''}{\iota}, &
  \cos(y_1 x') &= \frac{(\gamma-\beta )b''c''}{\iota}, &
  \cos(z_1 x') &= a'',\\
  \cos(x_1 y') &= \frac{( \beta-\delta)b''}{\iota}, &
  \cos(y_1 y') &= \frac{(\alpha-\gamma)c''a''}{\iota}, &
  \cos(z_1 y') &= b'',\\
  \cos(x_1 z') &= \frac{(\gamma-\delta)c''}{\iota}, &
  \cos(y_1 z') &= \frac{(\beta -\alpha)a''b''}{\iota}, &
  \cos(z_1 z') &= c''.
\end{align*}
C'est le passage des neuf cosinus de M.~Chelini \`a ceux de Jacobi, qu'il \'etait
important d'effectuer pour compl\'eter la d\'eduction analytique de la th\'eorie
de Poinsot, alors m\^eme que, par cette voie, on ne d\^ut peut-\^etre pas y arriver
de la mani\`ere la plus rapide. Je renverrai, sur ce point essentiel, aux
beaux M\'emoires de M.~Siacci, en me bornant \`a remarquer les relations
suivantes, dans lesquelles $V_1 = v-iv'$:
\begin{align*}
  \cos(x_1 x') +i\cos(y_1 x') &= \frac{1}{\iota}AV_1, \\
  \cos(x_1 y') +i\cos(y_1 y') &= \frac{1}{\iota}BV_1, \\
  \cos(x_1 z') +i\cos(y_1 z') &= \frac{1}{\iota}CV_1,
\end{align*}
et j'y ajouterai quelques formules relatives \`a l'erpolo\"ide.

%-----051.png------------------------------

\mysection{XVII.}


Si l'on met, au lieu de $\xi$, $\eta$, $\zeta$, dans les \'equations du \S~X,
p.~\pageref{page23}, les quantit\'es suivantes:
\[
  \xi = p\rho, \quad  \eta = q\rho, \quad  \zeta = r\rho,
\]
o\`u $p$, $q$, $r$ sont les composantes de la vitesse et $\rho$ une ind\'etermin\'ee, on aura,
pour d\'eterminer la position de l'axe instantan\'e de rotation par rapport
aux axes fixes, les formules
\begin{alignat*}{4}
  x &=  (a  p && + b  q && + c  r) &\rho &= v  \rho, \\
  y &=  (a' p && + b' q && + c' r) &\rho &= v' \rho, \\
  z &=  (a''p && + b''q && + c''r) &\rho &= v''\rho,
\end{alignat*}
dont la derni\`ere est simplement $z = \delta\rho$. Or, l'erpolo\"ide \'etant la trace de cet
axe mobile sur le plan tangent \`a l'ellipso\"ide central, $z = \delta$, on voit qu'il
suffit de faire $\rho = 1$ pour obtenir les coordonn\'ees de cette courbe, exprim\'ees
en fonction du temps, ou de la variable $u$. Nous avons ainsi $x = v$,
$y = v'$; mais ce sont plut\^ot les quantit\'es $x + iy$ et $x - iy$ qu'il convient
de consid\'erer, et je poserai en cons\'equence
\begin{alignat*}{2}
  x + iy
&= -in\frac{ H'(0) \Theta_1(u-\omega) e^{ i(\lambda u+\nu)} }{ H_1(\omega) \Theta(u)}
&&= \Phi(u),\\
  x - iy
&= +in\frac{ H'(0) \Theta_1(u+\omega) e^{-i(\lambda u+\nu)} }{ H_1(\omega) \Theta(u)}
&&= \Phi_1(u),
\end{alignat*}
ce qui permettra d'employer les conditions caract\'eristiques
\begin{align*}
  \Phi(u+2 K ) &= \mu \Phi(u),
& \Phi(u+2iK') &=-\mu'\Phi(u),
\\
  \Phi_1(u+2 K ) &= \frac{1}{\mu} \Phi_1(u),
& \Phi_1(u+2iK') &=-\frac{1}{\mu'} \Phi_1(u),
\end{align*}
o\`u j'ai fait
\[
  \mu  = e^{2i\lambda K}, \qquad
  \mu' = e^{\frac{i\pi\omega}{K} - 2\lambda K'} .
\]
Elles montrent, en effet, que les produits
$\Phi(u)\Phi_1(u)$, $D_u \Phi(u) D_u \Phi_1(u)$, et en
g\'en\'eral $D_u^m \Phi(u) D_u^n \Phi_1(u)$, quels que soient $m$ et $n$, sont des fonctions doublement
p\'eriodiques, ayant $2K$ et $2iK'$ pour p\'eriodes. En particulier,
nous envisagerons l'expression
$D_u \Phi(u) D_u \Phi_1(u) = x'^2 + y'^2$, puis les
%-----052.png-----------------
coefficients de $i$ dans les suivantes:
\begin{alignat*}{5}
  &D_u\Phi(u)\;\Phi_1(u)      &&= xx' &&+ yy' &&+ i(xy' &&- yx'), \\[1ex]
  &D^2_u\Phi(u)D_u\Phi_1(u) &&= x'x'' &&+y'y'' &&+ i(x'y'' &&-y'x''),
\end{alignat*}
ces fonctions doublement p\'eriodiques donnant, par les formules connues,
les \'el\'ements de l'arc, du secteur et le rayon de courbure. J'emploierai,
pour les obtenir, la formule de d\'ecomposition en \'el\'ements simples,
rappel\'ee au commencement de ce travail (\S~I, p.~\pageref{page5}), et dont l'application
sera facile, $\Phi(u)$ et $\Phi_1(u)$ ayant pour p\^ole unique $u = iK'$. N'ayant
ainsi \`a consid\'erer qu'un seul \'el\'ement simple, $\frac{\Theta'(u)}{\Theta(u)}$, il suffit d'avoir les
d\'eveloppements suivant les puissances croissantes de $\varepsilon$ de $\Phi(iK'+ \varepsilon)$ et
$\Phi_1(iK'+ \varepsilon)$; ils s'obtiennent comme on va voir.

Je remarque d'abord que, au moyen de la fonction $\varphi_1(x,\omega)$, d\'efinie
au \S~III, p.~\pageref{page8}, on peut \'ecrire
\[
\Phi(u) = C  \varphi_1(u,-\omega) e^{\frac{i\delta u}{n}}, \qquad
\Phi_1(u)=C_1\varphi_1(u, \omega) e^{-\frac{i\delta u}{n}},
\]
$C$ et $C_1$ d\'esignant des constantes. C'est ce que l'on voit en joignant aux
relations pr\'ec\'edemment employ\'ees,
\[
  i\lambda
= \frac{i\alpha}{n} + \frac{\Theta'(\omega)}{\Theta(\omega)}
= \frac{i \beta}{n} + \frac{\Theta_1'(\omega)}{\Theta_1(\omega)}
= \frac{i\gamma}{n} + \frac{H'(\omega)}{H(\omega)} ,
\]
la suivante:
\[
  i\lambda = \frac{i\delta}{n} + \frac{H_1'(\omega)}{H_1(\omega)} ,
\]
qui r\'esulte de la condition $\alpha-\delta = in\frac{\sn\omega\dn\omega}{\cn\omega}$ (\S~XV, p.~\pageref{page36}), en la mettant
sous la forme
\[
  \frac{i\alpha}{n}-\frac{i\delta}{n} = D_{\omega} \log\cn\omega
  = \frac{H_1'(\omega)}{H_1(\omega)}
  - \frac{\Theta'(\omega)}{\Theta(\omega)} .
\]
Cela pos\'e, l'\'equation $i\varphi_1(u, \omega) = \chi(u, \omega+K+iK')$ montre qu'on a le
d\'eveloppement de $\varphi_1(iK'+\varepsilon, \omega)$ en changeant simplement $\omega$ en $\omega+K+iK'$
dans la formule de la page~\pageref{page13}:
\[
  \chi(iK'+\varepsilon,\omega) = \frac{1}{\varepsilon}
  - \frac{1}{2}\Omega\varepsilon
  - \frac{1}{3}\Omega_1\varepsilon^2
  - \frac{1}{8}\Omega_2\varepsilon^3 + \ldots ,
\]
et il vient ainsi, en nous bornant aux seuls termes n\'ecessaires,
\[
  i\varphi_1(iK'+\varepsilon,\omega) = \frac{1}{\varepsilon}
  - \left( \frac{k'^2}{\cn^2\omega}
    + \frac{2k^2-1}{3} \right) \frac{\varepsilon}{2}
  - \frac{k'^2 \sn\omega\dn\omega}{\cn^3\omega}
    \frac{\varepsilon^2}{3} - \ldots.
\]

D\'esignons par $S_1$, pour abr\'eger, la s\'erie du second membre, et par $S$
%-----053.png-----------------
ce qu'elle devient lorsqu'on change $i$ en $-i$, c'est-\`a-dire $\omega$ en $-\omega$;
puisqu'on a $\omega = i\upsilon$, on aura les expressions
\[
  \Phi(iK'+\varepsilon) = R  S  e^{ \frac{i\delta\varepsilon}{n}}, \qquad
  \Phi_1(iK'+\varepsilon)=R_1S_1e^{-\frac{i\delta\varepsilon}{n}},
\]
o\`u $R$ et $R_1$ sont deux nouvelles constantes, dont la signification se montre
d'elle-m\^eme. Il est clair, en effet, que ces quantit\'es sont les r\'esidus des
fonctions $\Phi(u)$ et $\Phi_1(u)$ pour $u=iK'$, de sorte qu'on trouve imm\'ediatement
les valeurs
\[
  R=-ne^{ \frac{i\pi\omega}{2K}-\lambda K'+i\nu} ,\qquad
  R=+ne^{-\frac{i\pi\omega}{2K}+\lambda K'-i\nu}
\]
et par suite la relation $RR_1=-n^2$. Voici maintenant les applications de
nos formules.


\mysection{XVIII.}


Je pars des \'equations suivantes:
\begin{alignat*}{2}
&  D_{\varepsilon}\Phi(iK'+\varepsilon) D_{\varepsilon}\Phi_1(iK'+\varepsilon) &&=
    -n^2\left( S' + \frac{i\delta}{n} S \right) \left( S'_1-\frac{i\delta}{n} S_1\right), \\
&  D_{\varepsilon}\Phi(iK'+\varepsilon) \;\Phi_1(iK'+\varepsilon) &&=
    -n^2\left( S' + \frac{i\delta}{n} S \right) S_1, \\
&  D^2_{\varepsilon}\Phi(iK'+\varepsilon) D_{\varepsilon}\Phi_1(iK'+\varepsilon) &&=
    -n^2\left( S''+ \frac{2i\delta}{n} S' - \frac{\delta^2}{n^2} S \right) \left( S'_1-\frac{i\delta}{n} S_1\right),
\end{alignat*}
et je me borne \`a la partie principale des d\'eveloppements en faisant, dans
les deux derni\`eres, abstraction des termes r\'eels; le calcul donne pour
r\'esultats
\[
   \frac{P}{\varepsilon^2} - \frac{n^2}{\varepsilon^4}, \qquad
  -\frac{n\delta}{\varepsilon^2}, \qquad
  -\frac{Q}{\varepsilon^2},
\]
si l'on \'ecrit, pour abr\'eger,
\begin{align*}
  P &=
  \frac{n^2 k'^2}{\cn^2\omega} + \frac{n^2(2k^2-1)}{3} + \delta^2, \\
  Q &=
  \frac{2n^3 k'^2 \sn\omega \dn\omega}{i \cn^3\omega} +
  \frac{3\delta n^2 k'^2}{\cn^2\omega} +
  \delta n^2(2k^2-1) + \delta^3 .
\end{align*}
Rempla\c{c}ant donc $\frac{1}{\varepsilon^2}$ et $\frac{1}{\varepsilon^4}$ par
$-D_{\varepsilon}\frac{1}{\varepsilon}$,
$-\frac{1}{6} D^3_{\varepsilon}\frac{1}{\varepsilon}$,
on obtiendra, en d\'esignant
%-----054.png---------------------------
par $C$, $C'$, $C''$ des constantes,
\begin{align*}
x'^2 + y'^2 &= C - PD_u\frac{\Theta'(u)}{\Theta(u)} +\frac{1}{6}n^2 D^3_u\frac{\Theta'(u)}{\Theta(u)},\\
xy' - yx' &= C' + n\delta D_u\frac{\Theta'(u)}{\Theta(u)},\\
x'y'' - y'x'' &= C'' +\frac{Q}{n}D_u\frac{\Theta'(u)}{\Theta(u)}.
\end{align*}
Employons enfin la relation $D_u\frac{\Theta'(u)}{\Theta(u)} = \frac{J}{K}-k^2\sn^2u$, et nous parviendrons,
en modifiant convenablement les constantes, aux expressions suivantes:
\begin{align*}
  x'^2 + y'^2
&= C
 + \left(n^2-\delta^2-\frac{n^2 k'^2}{\cn^2\omega}\right) k^2\sn^2u
 - n^2k^4\sn^4u,
\\
  xy'  - yx'  &= C'  - \delta n   k^2\sn^2u,\\
  xy'' - yx'' &= C'' - \frac{Q}{n}k^2\sn^2u.
\end{align*}
Pour d\'eterminer $C$, $C'$, $C''$, je supposerai $u = 0$; il suffira ainsi de conna\^itre
les valeurs des fonctions $\Phi(u)$, $\Phi_1(u)$ et de leurs premi\`eres d\'eriv\'ees
quand on pose $u = 0$; or on obtient, par un calcul facile dont je me
borne \`a donner le r\'esultat,
\begin{align*}
e^{-i\nu}\Phi(u)
&= -in  \frac{\dn\omega}{\cn\omega}
 + \beta\frac{\dn\omega}{\cn\omega}u
 + i\frac{ n^2k^2\cn^2\omega + \beta^2\dn^2\omega }{ n\cn\omega\dn\omega }\,
   \frac{u^2}{2} + \ldots,
\\
e^{+i\nu}\Phi_1(u)
&= +in  \frac{\dn\omega}{\cn\omega}
 + \beta\frac{\dn\omega}{\cn\omega}u
 - i\frac{ n^2k^2\cn^2\omega + \beta^2\dn^2\omega }{ n\cn\omega\dn\omega }
   \frac{u^2}{2} + \ldots;
\end{align*}
on en conclut
\[
C   =  \beta^2\frac{\dn^2\omega}{\cn^2\omega},\quad
C'  = n\beta  \frac{\dn^2\omega}{\cn^2\omega},\quad
C'' =  \beta\frac{n^2k^2\cn^2\omega+\beta^2\dn^2\omega}{n\cn^2\omega}.
\]
Soit donc $S$ l'aire d'un secteur, $s$ la longueur de l'arc et $R$ le rayon de courbure
de l'erpolo\"ide, nous aurons
\begin{align*}
   D_uS
&= n\left( \beta\frac{\dn^2\omega}{\cn^2\omega}
         - \delta k^2\sn^2u \right),
\\
   (D_us)^2
&= \beta^2\frac{\dn^2\omega}{\cn^2\omega}
 + \left(n^2 - \delta^2 - \frac{n^2k'^2}{\cn^2\omega}\right) k^2\sn^2u
 - n^2k^4\sn^4u,
\\
R &= \frac{ n\cn^2\omega
  \left[ \beta^2\frac{\dn^2\omega}{\cn^2\omega}
   + \left(n^2-\delta^2-\frac{n^2k'^2}{\cn^2\omega^2}\right) k^2\sn^2u
   - n^2k^4\sn^4u \right]^{\frac{3}{2}} }
         { \beta(n^2k^2\cn^2\omega+\beta^2\dn^2\omega)
         - Qk^2\cn^2\omega\sn^2u } .
\end{align*}
Ces formules donnent lieu \`a quelques remarques.

%-----055.png---------------------------

J'observerai, en premier lieu, qu'on tire de la premi\`ere, en comptant
l'aire \`a partir de $t = t_0$ ou $u = 0$,
\begin{align*}
S &= n\beta\frac{\dn^2\omega}{\cn^2\omega}u
   - n\delta\left[\frac{J}{K}u-\frac{\Theta'(u)}{\Theta(u)}\right]\\
&= nu\left(\beta\frac{\dn^2\omega}{\cn^2\omega}-\delta\frac{J}{K}\right)
 + n\delta\frac{\Theta'(u)}{\Theta(u)};
\end{align*}
il en r\'esulte que, $u$ devenant $u + 2K$, le secteur s'accro\^it de la quantit\'e
constante
\[
  2n\left(\beta\frac{\dn^2\omega}{\cn^2\omega}K - \delta J\right),
\]
ou, sous une autre forme,
\[
  2\sqrt{\frac{\delta-\alpha}{\gamma-\beta}}
  [(\gamma-\delta)\beta K - (\gamma-\beta)\delta J ].
\]
Je d\'emontrerai ensuite que le trin\^ome en $\sn u$ qui se pr\'esente dans l'\'el\'ement
de l'arc, et dont les racines sont r\'eelles et de signes contraires, a sa
racine positive comprise entre $1$ et $\frac{1}{k}$. En faisant, en effet, $\sn u = 1$, puis
$\sn u = \frac{1}{k}$, nous trouvons pour r\'esultats les quantit\'es
\[
  \frac{\alpha^2(\gamma-\delta)(\delta-\beta)}{(\gamma-\beta)(\delta-\alpha)},
  \qquad
  \frac{\gamma^2(\beta-\delta)}{\gamma-\beta},
\]
dont la premi\`ere est positive et la seconde n\'egative. On verra sans peine
aussi qu'en introduisant $\dn u$ au lieu de $\sn u$, il prend la forme suivante,
qui est assez simple:
\[
  \frac{\gamma^2(\beta-\delta)}{\gamma-\beta}
- [\gamma(\alpha+\beta-2\delta) - \alpha\beta]\dn^2u
- (\gamma-\beta)(\delta-\alpha)\dn^4u.
\]
Enfin, et en dernier lieu, je remarquerai que les constantes qui entrent
dans le d\'enominateur du rayon de courbure peuvent s'\'ecrire ainsi:
\begin{gather*}
Q = -4\delta^3 + 4(\alpha+\beta+\gamma)\delta^2
  - 3(\alpha\beta+\alpha\gamma+\beta\gamma)\delta
 + 2\alpha\beta\gamma,
\\[0.7ex]
\frac{\beta(n^2k^2\cn^2\omega+\beta^2\dn^2\omega)}{\cn^2\omega}
= \frac{\beta(\gamma-\delta)(\beta\alpha+\beta\gamma-\alpha\gamma)}{\gamma-\beta};
\end{gather*}
mais, malgr\'e cette simplification, il para\^it difficile de d\'eduire de la formule
qui d\'etermine les points stationnaires,
\[
  k^2\sn^2u
= \frac{\beta(n^2k^2\cn^2\omega + \beta^2\dn^2\omega)}{Q\cn^2\omega},
\]
les conditions sous lesquelles ces points seront r\'eels ou imaginaires, et je
ne m'y arr\^eterai pas.
%-----056.png------------------

\mysection{XIX.}


Apr\`es l'erpolo\"ide, je consid\`ere encore la courbe sph\'erique d\'ecrite
par un point d\'etermin\'e du corps pendant la rotation, et dont les
\'equations sont
\begin{alignat*}{3}
  x &= a  \xi &&+ b  \eta &&+ c  \zeta, \\
  y &= a' \xi &&+ b' \eta &&+ c' \zeta, \\
  z &= a''\xi &&+ b''\eta &&+ c''\zeta.
\end{alignat*}
Je remarquerai tout d'abord que les \'el\'ements g\'eom\'etriques, qui conservent
la m\^eme valeur quand on passe d'un syst\`eme de coordonn\'ees rectangulaires
\`a un autre quelconque, seront des fonctions doublement p\'eriodiques
du temps. Si l'on pose, en effet,
\begin{alignat*}{3}
  D^n_t x &= a  \xi_n &&+ b  \eta_n &&+ c\zeta_n, \\
  D^n_t y &= a' \xi_n &&+ b' \eta_n &&+ c'\zeta_n, \\
  D^n_t z &= a''\xi_n &&+ b''\eta_n &&+ c''\zeta_n,
\end{alignat*}
les \'equations de Poisson donnent facilement
\begin{alignat*}{3}
  \xi_{n+1}  &= D_t \xi_n   &&+ q\zeta_n &&- r\eta_n, \\
  \eta_{n+1} &= D_t \eta_n  &&+ r\xi_n   &&- p\zeta_n, \\
  \zeta_{n+1}&= D_t \zeta_n &&+ p\eta_n  &&- q\xi_n,
\end{alignat*}
et ces relations permettent d'exprimer de proche en proche, pour toute
valeur de $n$, les quantit\'es $\xi_n$, $\eta_n$, $\zeta_n$ par des fonctions rationnelles et enti\`eres
de $a''$, $b''$, $c''$. On trouvera, en particulier,
\[
  \xi_1  = b''\beta\zeta - c''\gamma\eta, \quad
  \eta_1 = c''\gamma\xi  - a''\alpha\zeta, \quad
  \zeta_1= a''\alpha\eta - b''\beta\xi,
\]
et, par cons\'equent, en d\'esignant par $s$ l'arc de la courbe, nous aurons la
formule
\[
(D_t s)^2 = \xi_1^2 + \eta_1^2 + \zeta_1^2.
\]
On obtient ensuite, pour le rayon de courbure $R$ et le rayon de torsion $R_1$,
les expressions suivantes:
\[
R^2 = \frac{(\xi_1^2 + \eta_1^2 + \zeta_1^2)^3}{u^2 + v^2 + w^2},
\qquad
R_1 = \frac{u^2 + v^2 + w^2}{\Delta},
\]
%-----057.png------------------------------
o\`u j'ai fait, pour abr\'eger,
\begin{gather*}
  u =  \eta_1\zeta_2 - \zeta_1\eta_2,  \quad
  v = \zeta_1  \xi_2 - \zeta_2 \xi_1,  \quad
  w =   \xi_1 \eta_2 -   \xi_2\eta_1,
\\
  \Delta =
  \begin{vmatrix}
    \xi_1   & \xi_2   & \xi_3   \\
    \eta_1  & \eta_2  & \eta_3  \\
    \zeta_1 & \zeta_2 & \zeta_3
  \end{vmatrix}
.
\end{gather*}
C'est \`a l'\'el\'ement de l'arc que je m'arr\^eterai un moment, afin de tirer
quelques cons\'equences de la forme analytique remarquable que pr\'esente
la quantit\'e $\xi_1^2 + \eta_1^2 + \zeta_1^2$. Nous avons, en effet, la relation
\[
  \xi\xi_1 + \eta\eta_1 + \zeta\zeta_1 = 0,
\]
qui donne facilement
\[
  (\xi^2 + \zeta^2) (D_t s)^2
= (\xi^2 + \eta^2 + \zeta^2)\eta_1^2 + (\zeta\xi_1 - \xi\zeta_1)^2,
\]
et, par suite, cette d\'ecomposition en facteurs imaginaires conjugu\'es, o\`u
j'\'ecris, pour abr\'eger, $\rho^2 = \xi^2 + \eta^2 + \zeta^2$,
\[
  (\xi^2 + \zeta^2) (D_t s)^2
= (\zeta\xi_1 - \xi\zeta_1 + i\rho\eta_1)
  (\zeta\xi_1 - \xi\zeta_1 - i\rho\eta_1) .
\]
Or les valeurs de $a''$, $b''$, $c''$, \`a savoir:
\[
  a'' = -\sqrt{\frac{\gamma-\delta}{\gamma-\alpha}} \cn u,  \quad
  b'' =  \sqrt{\frac{\gamma-\delta}{\gamma-\beta }} \sn u,  \quad
  c'' =  \sqrt{\frac{\delta-\alpha}{\gamma-\alpha}} \dn u,
\]
conduisent \`a l'expression suivante:
\begin{multline*}
  \zeta\xi_1 - \xi\zeta_1 + i\rho\eta_1
= \alpha\sqrt{\frac{\gamma-\delta}{\gamma-\alpha}}
   (\xi\eta + i\rho\zeta) \cn u \\
+ \beta \sqrt{\frac{\gamma-\delta}{\gamma-\beta }}
   (\xi^2 + \zeta^2)      \sn u
- \gamma\sqrt{\frac{\delta-\alpha}{\gamma-\alpha}}
   (\eta\zeta - i\rho\xi) \dn u,
\end{multline*}
et nous allons facilement en d\'eduire les valeurs particuli\`eres des coordonn\'ees
$\xi$, $\eta$, $\zeta$, pour lesquelles l'arc de la courbe sph\'erique, au lieu de d\'ependre
d'une transcendante compliqu\'ee, s'obtient sous forme finie explicite.
Je me fonderai, \`a cet effet, sur cette remarque, que le produit de deux
fonctions lin\'eaires
\[
  \Pi(u) = (A\cn u + B\sn u + C\dn u) (A'\cn u + B'\sn u + C'\dn u)
\]
devient le carr\'e d'une fonction uniforme si l'on a
\[
  A^2 k'^2 + B^2  - C^2 k'^2 = 0, \qquad
  A'^2k'^2 + B'^2 - C'^2k'^2 = 0.
\]
%-----058.png------------------------------
A cet effet, j'observe que les formules
\begin{align*}
  \sn 2u &= \frac{2\sn u \cn u \dn u          }{1 - k^2\sn^4 u},  \\
  \cn 2u &= \frac{1 - 2   \sn^2 u + k^2\sn^4 u}{1 - k^2\sn^4 u},  \\
  \dn 2u &= \frac{1 - 2k^2\sn^2 u + k^2\sn^4 u}{1 - k^2\sn^4 u}
\end{align*}
permettent d'\'ecrire
\begin{multline*}
  A\cn 2u + B\sn 2u + C\dn 2u  \\
= \frac{A+C - 2(A+Ck^2)\sn^2 u +
    (A+C)k^2\sn^4 u + 2B\sn u\cn u\dn u}{1 - k^2\sn^4 u}.
\end{multline*}
Cela \'etant, soit, en d\'esignant par $g$ et $h$ deux constantes,
\begin{multline*}
  A+C - 2(A+Ck^2)\sn^2 u + (A+C)k^2\sn^4 u  \\
+ 2B\sn u\cn u\dn u = (g\sn u + h\cn u\dn u)^2,
\end{multline*}
on verra que les quatre \'equations r\'esultant de l'identification se r\'eduisent
aux trois suivantes:
\[
  A + C = h^2,  \quad
  2(A+Ck^2) = h^2(1+k^2) - g^2,  \quad
  B = gh;
\]
or l'\'elimination de $g$ et $h$ conduit imm\'ediatement \`a la condition
\[
  Ak'^2 + B^2 - C^2k'^2 = 0.
\]
Soit de m\^eme ensuite
\[
  A'\cn 2u + B'\sn 2u + C'\dn 2u
= \frac{(g'\sn u + h'\cn u\dn u)^2}{1 - k^2\sn^4 u},
\]
sous la condition semblable
\[
  A'k'^2 + B'^2 - C'^2 k'^2 = 0;
\]
nous en conclurons, pour $\sqrt{\Pi(2u)}$, l'expression suivante:
\[
  \sqrt{\Pi(2u)} = \frac{(g \sn u + h \cn u\dn u)
                         (g'\sn u + h'\cn u\dn u)}{1 - k^2\sn^4 u} ,
\]
ou, en d\'eveloppant,
\begin{multline*}
  \sqrt{\Pi(2u)} = \\
  \frac{ gg'\sn^2 u + hh' \left[1 - (1+k^2)\sn^2 u + k^2\sn^4 u \right]
    + (gh'+hg')\sn u\cn u\dn u }{1 - k^2\sn^4 u} ;
\end{multline*}
on en d\'eduit ensuite facilement, si l'on change $u$ en $\frac{u}{2}$,
\[
  2\sqrt{\Pi(u)} = \frac{1}{k'^2} gg'(\dn u - \cn u)
                 + (gh'+hg')\sn u
                 + hh'(\dn u + \cn u).
\]
Voici maintenant l'application de la remarque que nous venons d'\'etablir.
%-----059.png------------------------------


\mysection{XX.}


Revenant \`a l'expression pr\'ec\'edemment donn\'ee des facteurs de $(D_t s)^2$,
je pose
\begin{align*}
   A &= \alpha\sqrt{ \tfrac{\gamma-\delta}{\gamma-\beta } }(\xi\eta + i\rho\zeta),
& A'&= \alpha\sqrt{ \tfrac{\gamma-\delta}{\gamma-\beta } }(\xi\eta - i\rho\zeta), \\
   B &= \beta \sqrt{ \tfrac{\gamma-\delta}{\gamma-\beta } }(\xi^2  + \zeta^2),
& B'&= \beta \sqrt{ \tfrac{\gamma-\delta}{\gamma-\beta } }(\xi^2  + \zeta^2), \\
   C &=-\gamma\sqrt{ \tfrac{\delta-\alpha}{\gamma-\alpha} }(\eta\zeta + i\rho\xi),
& C'&=-\gamma\sqrt{ \tfrac{\delta-\alpha}{\gamma-\alpha} }(\eta\zeta - i\rho\xi),
\end{align*}
et j'observe que, au moyen de la valeur
$k'^2 = \frac{(\alpha-\gamma)(\beta -\delta)}{(\beta -\gamma)(\alpha-\delta)}$,
nos conditions se pr\'esentent sous la forme suivante:
\begin{align*}
  \frac{\alpha^2}{\alpha-\delta} (\xi\eta   + i\rho\zeta)^2
+ \frac{ \beta^2}{ \beta-\delta} (\xi^2     + \zeta^2   )^2
+ \frac{\gamma^2}{\gamma-\delta} (\eta\zeta - i\rho\xi  )^2 &= 0,
\\
  \frac{\alpha^2}{\alpha-\delta} (\xi\eta   - i\rho\zeta)^2
+ \frac{ \beta^2}{ \beta-\delta} (\xi^2     + \zeta^2   )^2
+ \frac{\gamma^2}{\gamma-\delta} (\eta\zeta + i\rho\xi  )^2 &= 0.
\end{align*}
Elles donnent imm\'ediatement $\xi\eta\zeta = 0$; et nous poserons en cons\'equence:
\begin{align*}
\tag*{\primo}
  \xi   = 0, & \quad
  \left( \frac{\gamma^2}{\gamma-\delta}
       - \frac{\alpha^2}{\alpha-\delta} \right)\eta^2
+ \left( \frac{ \beta^2}{ \beta-\delta}
       - \frac{\alpha^2}{\alpha-\delta} \right)\zeta^2 = 0,
\\
\tag*{\secundo}
  \eta  = 0, & \quad
  \left( \frac{\alpha^2}{\alpha-\delta}
       - \frac{ \beta^2}{ \beta-\delta} \right)\zeta^2
+ \left( \frac{\gamma^2}{\gamma-\delta}
       - \frac{ \beta^2}{ \beta-\delta} \right)\xi^2 = 0,
\\
\tag*{\tertio}
  \zeta = 0, & \quad
  \left( \frac{ \beta^2}{ \beta-\delta}
       - \frac{\gamma^2}{\gamma-\delta} \right)\xi^2
+ \left( \frac{\alpha^2}{\alpha-\delta}
       - \frac{\gamma^2}{\gamma-\delta} \right)\eta^2 = 0.
\end{align*}
\medskip

Soit, pour abr\'eger,
\begin{align*}
  \mathrm{a} &= (\alpha-\delta)(\gamma-\beta )
       (\gamma\delta + \beta \delta - \gamma\beta ),
\\
  \mathrm{b} &= (\beta -\delta)(\alpha-\gamma)
       (\alpha\delta + \gamma\delta - \alpha\gamma),
\\
  \mathrm{c} &= (\gamma-\delta)(\beta -\alpha)
       (\beta \delta + \alpha\delta - \beta \alpha);
\end{align*}
au moyen de ces quantit\'es, qu'on verra facilement v\'erifier les relations
\[
  \mathrm{a} + \mathrm{b} + \mathrm{c} = 0,  \qquad
  \frac{\mathrm{a}\alpha^2}{\alpha-\delta} +
  \frac{\mathrm{b} \beta^2}{\beta -\delta} +
  \frac{\mathrm{c}\gamma^2}{\gamma-\delta} = 0,
\]
nous obtenons les trois syst\`emes de valeurs
\begin{align*}
\tag*{\primo}
  \xi   &= 0, &  \eta^2 &= \mathrm{c}, & \zeta^2 &= \mathrm{b},
\\
\tag*{\secundo}
  \eta  &= 0, & \zeta^2 &= \mathrm{a}, &   \xi^2 &= \mathrm{c},
\\
\tag*{\tertio}
  \zeta &= 0, &   \xi^2 &= \mathrm{b}, &  \eta^2 &= \mathrm{a}.
\end{align*}
%-----060.png------------------------------
Maintenant je vais d\'emontrer que, de ces diverses solutions, la premi\`ere
est seule r\'eelle et r\'epond \`a la question propos\'ee.

Pour cela, je rappelle que les constantes $\alpha$, $\beta$, $\gamma$, $\delta$ satisfont aux conditions
\[
\tag*{(I)}
\alpha < \beta < \delta < \gamma,
\]
ou \`a celles-ci
\[
\tag*{(II)}
\alpha > \beta > \delta > \gamma,
\]
et j'observe qu'on aura, dans les deux cas,
\[
(\alpha - \delta)(\gamma - \beta) < 0, \quad
(\beta  - \delta)(\alpha - \gamma) < 0, \quad
(\gamma - \delta)(\beta  - \alpha) < 0.
\]
J'ajoute \`a ces r\'esultats les suivants:
\[
\gamma\delta + \beta\delta - \gamma\beta > 0, \quad
\alpha\delta + \gamma\delta - \alpha\gamma > 0, \quad
\beta\delta  + \alpha\delta - \beta\alpha > 0,
\]
qui donneront, comme on voit,
\[
\mathrm{a} < 0, \quad
\mathrm{b} > 0, \quad
\mathrm{c} > 0.
\]
On peut \'ecrire, en effet,
\begin{align*}
\gamma\delta + \beta \delta - \gamma\beta  &= \beta \delta + (\delta - \beta )\gamma, \\
\alpha\delta + \gamma\delta - \gamma\alpha &= \alpha\delta + (\delta - \alpha)\gamma, \\
\beta \delta + \alpha\delta - \beta \alpha &= \alpha\delta + (\delta - \alpha)\beta,
\end{align*}
et, dans le premier syst\`eme de conditions, on voit ainsi que les premiers
mem\-bres sont tous positifs. Nous ferons ensuite, en passant au second
syst\`eme,
\begin{align*}
\gamma\delta + \beta \delta - \gamma\beta  &= \gamma\delta + (\delta - \gamma)\beta, \\
\alpha\delta + \gamma\delta - \alpha\gamma &= \gamma\delta + (\delta - \gamma)\alpha;
\end{align*}
mais ces transformations faciles ne suffisent plus, \`a l'\'egard de la troisi\`eme
quantit\'e $\beta\delta + \alpha\delta - \beta\alpha$, pour reconna\^itre qu'elle est toujours positive
comme les autres. Il est n\'ecessaire, en effet, d'introduire une condition
nouvelle, $\frac{1}{\alpha} + \frac{1}{\beta} > \frac{1}{\gamma}$, ayant son origine dans la d\'efinition des quantit\'es $\frac{1}{\alpha}$,
$\frac{1}{\beta}$, $\frac{1}{\gamma}$, qui sont proportionnelles aux moments principaux d'inertie. Nous
\'ecrirons, dans ce cas,
\[
\beta\delta + \alpha\delta - \alpha\beta = \frac{1}{\alpha\beta\delta}
\left[\left(\frac{1}{\alpha} + \frac{1}{\beta} - \frac{1}{\gamma}\right) +
      \left(\frac{1}{\gamma} - \frac{1}{\delta}\right)\right],
\]
et le dernier r\'esultat qui nous restait \`a \'etablir se trouve d\'emontr\'e. Les valeurs
r\'eelles ainsi obtenues pour les coordonn\'ees $\xi$, $\eta$, $\zeta$, \`a savoir $\xi = 0$, $\eta = \sqrt{\mathrm{b}}$,
%-----061.png-------------------
$\zeta=\sqrt{\mathrm{c}}$, donnent, en prenant les radicaux avec le double signe, quatre points
qui d\'ecrivent des courbes rectifiables, ou plut\^ot deux droites remarquables:
$\xi = 0$, $\eta = \pm \sqrt{\frac{\mathrm{b}}{\mathrm{c}}}\zeta$,
dont tous les points d\'ecrivent pendant la rotation du
corps de telles courbes. Pour former l'expression de l'arc $s$, observons que,
d'apr\`es l'\'egalit\'e $\mathrm{a}+\mathrm{b}+\mathrm{c} = 0$, on peut \'ecrire $i\rho = \sqrt{\mathrm{a}}$, ce qui donne les
valeurs suivantes:
\[
  A=\zeta\alpha\sqrt{\frac{\gamma-\delta}{\gamma-\alpha}\mathrm{a}}, \quad
  B=\zeta\beta \sqrt{\frac{\gamma-\delta}{\gamma-\alpha}\mathrm{b}}, \quad
  C=\zeta\gamma\sqrt{\frac{\delta-\alpha}{\gamma-\alpha}\mathrm{c}}.
\]
On a ensuite
\[
  A'=-A,\quad B'=B,\quad C'=C,
\]
et nous en concluons
\begin{multline*}
(A\cn u + B\sn u + C\dn u)(A'\cn u + B'\sn u + C'\dn u) \\
= (B\sn u + C\dn u)^2 - A^2\cn^2 u.
\end{multline*}
La condition $A^2 k'^2 + B^2 - C^2 k'^2 = 0$ conduit enfin \`a cette nouvelle transformation
\begin{align*}
\lefteqn{  (B\sn u + C\dn u)^2 - A^2\cn^2 u} \\
&\qquad = (B\sn u+C\dn u)^2 - \frac{C^2 k'^2-B^2}{k'^2} (\dn^2 u-k'^2\sn^2 u) \\
&\qquad = \left(Ck'\sn u +\frac{B}{k'}\dn u \right)^2 ,
\end{align*}
et il vient, en d\'efinitive, apr\`es quelques r\'eductions, pour l'expression de
l'arc de la courbe sph\'erique,
\[
  s=\gamma\sqrt{\tfrac{\beta-\delta}{\beta-\gamma}(\beta\delta+\alpha\delta-\beta\alpha)}
  \int k\sn u\,du
  + \beta\sqrt{\tfrac{\gamma-\delta}{\gamma-\alpha}(\alpha\delta+\gamma\delta-\alpha\gamma)}
  \int \dn u\, du ,
\]
puis, en effectuant les int\'egrations,
\begin{align*}
  s&= \gamma\sqrt{\frac{\beta-\delta}{\beta-\gamma}(\beta\delta+\alpha\delta-\beta\alpha)}
  \log(\dn u-k\cn u) \\
  &{}+ \beta\sqrt{\frac{\gamma-\delta}{\gamma-\alpha}(\alpha\delta+\gamma\delta-\alpha\gamma)}
  \am u.
\end{align*}
Il en r\'esulte que, $u$ devenant $u+4K$, l'arc s'accro\^it de la quantit\'e constante
\[
2\pi\beta\sqrt{\frac{\gamma-\delta}{\gamma-\alpha}(\alpha\delta+\gamma\delta-\alpha\gamma)}.
\]
%-----062.png----------------------------


\mysection{XXI.}


Je terminerai cette \'etude de la rotation en indiquant encore un
point de vue sous lequel on peut traiter la question et o\`u l'on \'evitera
le d\'efaut de sym\'etrie des m\'ethodes pr\'ec\'edemment expos\'ees, qui donnent
d'abord les quantit\'es $A$, $B$, $C$; puis, par un calcul diff\'erent, la quantit\'e $V$,
en s\'eparant ainsi des expressions compos\'ees de la m\^eme mani\`ere avec les
quatre fonctions fondamentales de Jacobi. Des transformations alg\'ebriques
faciles des \'equations de la rotation, lorsqu'on suppose en g\'en\'eral le corps
sollicit\'e par des forces quelconques, permettent, en effet, d'associer les
composantes de la vitesse aux neuf cosinus; elles seront le point de d\'epart
du nouveau proc\'ed\'e que je vais donner pour le cas o\`u il n'y a point de
forces acc\'el\'eratrices. Avant de les exposer, je rappelle d'abord les \'equations
d'Euler
\begin{align*}
  \mathrm{a}\,D_tp &= (\mathrm{b-c})qr + P,  \\
  \mathrm{b}\,D_tq &= (\mathrm{c-a})rp + Q,  \\
  \mathrm{c}\,D_tr &= (\mathrm{a-b})pq + R,
\end{align*}
o\`u les moments d'inertie sont d\'esign\'es par $\mathrm{a}$, $\mathrm{b}$, $\mathrm{c}$, et celles de Poisson,
dont j'ai d\'ej\`a fait usage,
\[
  D_ta'' = b''r - c''q,  \quad
  D_tb'' = c''p - a''r,  \quad
  D_tc'' = a''q - b''p,
\]
puis
\[
  D_tA = Br - Cq,  \quad
  D_tB = Cp - Ar,  \quad
  D_tC = Aq - Bp.
\]
Cela \'etant, soit, comme pr\'ec\'edemment,
\begin{alignat*}{4}
& v  &&= ap  &&+ bq   &&+ cr,\\
& v' &&= a'p &&+ b'q  &&+ c'r,\\
& v''&&= a''p&&+ b''q &&+ c''r,\\
& V  &&= Ap  &&+ Bb &&+ Cr;
\end{alignat*}
en \'ecrivant, pour abr\'eger,
\[
  \Delta = pD_tp + qD_tq + rD_tr
- (a''p + b''q + c''r) (a''D_tp + b''D_tq + c''D_tr),
\]
nous aurons, comme cons\'equence, les relations suivantes, que je vais d\'e\-mont\-rer:
\begin{gather*}
\begin{array}{c@{\qquad}c}
\text{I.} & \text{II.} \\
  A\Delta = V(D_tp - a''D_tv'') + iD_tV \cntrdot D_ta'', &
  Va'' = Av'' + iD_tA,
\\
  B\Delta = V(D_tq - b''D_tv'') + iD_tV \cntrdot D_tb'', &
  Vb'' = Bv'' + iD_tB,
\\
  C\Delta = V(D_tr - c''D_tv'') + iD_tV \cntrdot D_tc''; &
  Vc'' = Cv'' + iD_tC;
\end{array}
\\
%-----063.png------------------------------
\begin{array}{c@{\qquad}c}
  \text{III.} & \text{IV.} \\
  iCD_t b'' = Br + ic'' D_t B,  &
  iBD_t c'' = Cq + ib'' D_t C,\\
  iAD_t c'' = Cp + ia'' D_t C,  &
  iCD_t a'' = Ar + ic'' D_t A,\\
  iBD_t a'' = Aq + ib'' D_t A;  &
  iAD_t b'' = Bp + ia'' D_t B.
\end{array}
\end{gather*}

A cet effet, je remarque que, en \'ecrivant $\Delta$ sous la forme
\[
\Delta = \tfrac{1}{2} D_t(p^2+q^2+r^2) - v''D_tv'',
\]
la condition $p^2 + q^2 + r^2 = v^2 + v'^2 + v''^2$ donne imm\'ediatement
\[
\Delta = v D_t v + v'D_tv'.
\]
Observons encore qu'on tire des \'equations
\[
v = ap + bq + cr, \qquad v' = a'p + b'q + c'r,
\]
en employant les \'egalit\'es $ab' - ba' = c''$, $ca' - ac' = b''$ l'expression suivante:
\[
a'v - av' = b''r - c''q = D_ta''.
\]
On a d'ailleurs imm\'ediatement
\[
D_t p - a''D_t v'' = aD_tv + a' D_t v',
\]
et ces r\'esultats transforment l'\'equation
\[
A\Delta = V(D_tp - a''D_tv'') + i D_t V D_t a''
\]
dans la suivante:
\begin{multline*}
(a + ia')(v D_t v + v' D_t v')\\
= (v + iv')(a D_tv + a'D_tv') + i(D_tv + i D_t v')(a'v - av'),
\end{multline*}
qui est une identit\'e.

Passons \`a l'\'egalit\'e $Va'' = Av'' + i D_t A$; il suffit d'y remplacer les quantit\'es
$V$, $v''$, $D_t A$ par leurs expressions en $A$, $B$, $C$, $p$, $q$, $r$, ce qui donne
\[
(Ap + Bq + Cr) a'' = A(a''p + b''q + c''r) + i(Br - Cq),
\]
et par cons\'equent encore une identit\'e, en l'\'ecrivant ainsi:
\[
q(Ba'' - Ab'' + iC) + r(Ca'' - Ac'' - iB) = 0.
\]
%-----064.png-----------------
Enfin les \'equations
$iAD_t c'' = Cp+ iD_t Ca''$,
$iAD_t b'' = Bp + iD_t Ba''$ des
syst\`emes III et IV conduisent, par un calcul semblable, en se servant des
expressions de $D_tc''$ et $D_tb''$, aux m\^emes \'egalit\'es
\[
Ab'' - Ba'' = iC,\qquad
Ac'' - Ca'' = - iB;
\]
elles se trouvent donc encore v\'erifi\'ees; or toutes les autres \'equations, dans
les quatre syst\`emes, se d\'emontreraient de m\^eme, ou se d\'eduisent de celles
que nous venons d'\'etablir par un simple changement de lettres.


\mysection{XXII.}


J'applique maintenant ces r\'esultats au cas o\`u il n'y a point de forces
acc\'el\'eratrices, et je pose \`a cet effet
$p=\alpha a''$, $q=\beta b''$, $r=\gamma c''$, $v''=\delta$, ce qui donne d'abord
\[
\Delta = \alpha^2 a'' D_t a'' + \beta^2 b'' D_t b'' + \gamma^2 c'' D_t c''
  = (\alpha-\beta)(\beta-\gamma)(\gamma-\alpha) a'' b'' c''.
\]
Ayant ensuite $D_t p - a'' D_t v'' = \alpha(\gamma-\beta)b''c''$, on voit que, en supprimant le
facteur $(\gamma-\beta)b''c''$, l'\'equation
\[
A\Delta = V (D_t p - a'' D_t v'') + i D_t V D_t a''
\]
devient simplement
\[
Aa'' (\alpha-\beta)(\alpha-\gamma) = V\alpha + i D_t V.
\]
Dans les trois autres syst\`emes, les r\'eductions sont encore plus faciles, et
nous nous trouvons ainsi amen\'es aux relations suivantes:
\begin{gather*}
\begin{array}{c@{\qquad}c}
\text{I.} & \text{II.} \\
Aa''(\alpha -  \beta)(\alpha-\gamma) = V\alpha + iD_t V,
  & Va'' = A\delta + iD_t A, \\
Ab''(\beta  - \gamma)(\beta -\alpha) = V\beta  + iD_t V,
  & Vb'' = B\delta + iD_t B, \\
Ac''(\gamma - \alpha)(\gamma-\beta ) = V\gamma + iD_t V;
  & Vc'' = C\delta + iD_t C;
\end{array}
\\
\begin{array}{c@{\qquad}c}
\text{III.} & \text{IV.} \\
iCa'' (\alpha-\gamma) = B\gamma + iD_t B,
  &  iBa'' ( \beta -\alpha) = C\beta  + iD_t C, \\
iAb'' (\beta -\alpha) = C\alpha + iD_t C,
  &  iCb'' ( \gamma- \beta) = A\gamma + iD_t A, \\
iBc'' (\gamma-\beta ) = A\beta + iD_t A;
  &  iAc'' ( \alpha-\gamma) = B\alpha + iD_t B.
\end{array}
\end{gather*}
%-----065.png---------------------
La question est maintenant d'obtenir quatre fonctions $A$, $B$, $C$, $V$, qui
v\'erifient \`a la fois ces douze \'equations. Nous ferons un premier pas vers
notre but, par un changement d'inconnues, en posant
\[
A = \frac{i}{k\cn\omega}\mathfrak{a},
\quad
B = \frac{\dn\omega}{k\cn\omega}\mathfrak{b},
\quad
C =-\frac{\sn\omega}{\cn\omega} \mathfrak{c},
\quad
V = -in\mathfrak{v};
\]
nous prendrons aussi la quantit\'e $u$ pour variable ind\'ependante \`a la place
de $t$; enfin, en employant les expressions de $a''$, $b''$, $c''$, on trouvera les
transform\'ees suivantes de nos \'equations:
\begin{align*}
  &\mspace{30mu}\text{I.}  &&\mspace{25mu}\text{II.}
\\
  ik\cn u \,\mathfrak{a}
&= \frac{i\alpha}{n}\mathfrak{v} - D_u\mathfrak{v},
& ik\cn u \,\mathfrak{v}
&= \frac{i\delta}{n}\mathfrak{a} - D_u\mathfrak{a},
\\
   k\sn u \,\mathfrak{b}
&= \frac{i\beta} {n}\mathfrak{v} - D_u\mathfrak{v},
&  k\sn u \,\mathfrak{v}
&= \frac{i\delta}{n}\mathfrak{b} - D_u\mathfrak{b},
\\
  i \dn u \,\mathfrak{c}
&= \frac{i\gamma}{n}\mathfrak{v} - D_u\mathfrak{v};
& i \dn u \,\mathfrak{v}
&= \frac{i\delta}{n}\mathfrak{c} - D_u\mathfrak{c};
\\[2ex]
  &\mspace{25mu}\text{III.}  &&\mspace{25mu}\text{IV.}
\\
  ik\cn u \,\mathfrak{b}
&= \frac{i\beta} {n}\mathfrak{c} - D_u\mathfrak{c},
& ik\cn u \,\mathfrak{c}
&= \frac{i\gamma}{n}\mathfrak{b} - D_u\mathfrak{b},
\\
   k\sn u \,\mathfrak{c}
&= \frac{i\gamma}{n}\mathfrak{a} - D_u\mathfrak{a},
&  k\sn u \,\mathfrak{a}
&= \frac{i\alpha}{n}\mathfrak{c} - D_u\mathfrak{c},
\\
   i\dn u \,\mathfrak{a}
&= \frac{i\alpha}{n}\mathfrak{b} - D_u\mathfrak{b};
& i \dn u \,\mathfrak{b}
&= \frac{i\beta}{n}\mathfrak{a} - D_u\mathfrak{a}.
\end{align*}

Je ne m'arr\^eterai point aux calculs faciles qui donnent ces r\'esultats,
et je remarque imm\'ediatement qu'il convient de les disposer dans ce nouvel
ordre, \`a savoir:
\begin{align*}
  ik\cn u \,\mathfrak{a}
&= \frac{i\alpha}{n}\mathfrak{v} - D_u\mathfrak{v},
&  k\sn u \,\mathfrak{a}
&= \frac{i\alpha}{n}\mathfrak{c} - D_u\mathfrak{c},
& i \dn u \,\mathfrak{a}
&= \frac{i\alpha}{n}\mathfrak{b} - D_u\mathfrak{b},
\\
  ik\cn u \,\mathfrak{b}
&= \frac{i\beta}{n}\mathfrak{c} - D_u\mathfrak{c},
&  k\sn u \,\mathfrak{b}
&= \frac{i\beta}{n}\mathfrak{v} - D_u\mathfrak{v},
& i \dn u \,\mathfrak{b}
&= \frac{i\beta}{n}\mathfrak{a} - D_u\mathfrak{a},
\\
  ik\cn u \,\mathfrak{c}
&= \frac{i\gamma}{n}\mathfrak{b} - D_u\mathfrak{b},
&  k\sn u \,\mathfrak{c}
&= \frac{i\gamma}{n}\mathfrak{a} - D_u\mathfrak{a},
& i \dn u \,\mathfrak{c}
&= \frac{i\gamma}{n}\mathfrak{v} - D_u\mathfrak{v},
\\
  ik\cn u \,\mathfrak{v}
&= \frac{i\delta}{n}\mathfrak{a} - D_u\mathfrak{a},
&  k\sn u \,\mathfrak{v}
&= \frac{i\delta}{n}\mathfrak{b} - D_u\mathfrak{b},
& i \dn u \,\mathfrak{v}
&= \frac{i\delta}{n}\mathfrak{c} - D_u\mathfrak{c}.
\end{align*}
Par l\`a se trouvent mises en \'evidence trois substitutions remarquables, qui
%-----066.png-----------
correspondent aux multiplications des quatre fonctions par $\cn u$, $\sn u$, $\dn u$,
\`a savoir:
\[
\begin{pmatrix}
\mathfrak{a}, & \mathfrak{b}, & \mathfrak{c}, & \mathfrak{v} \\
\mathfrak{v}, & \mathfrak{c}, & \mathfrak{b}, & \mathfrak{a}
\end{pmatrix},
\quad
\begin{pmatrix}
\mathfrak{a}, & \mathfrak{b}, & \mathfrak{c}, & \mathfrak{v} \\
\mathfrak{c}, & \mathfrak{v}, & \mathfrak{a}, & \mathfrak{b}
\end{pmatrix},
\quad
\begin{pmatrix}
\mathfrak{a}, & \mathfrak{b}, & \mathfrak{c}, & \mathfrak{v} \\
\mathfrak{b}, & \mathfrak{a}, & \mathfrak{v}, & \mathfrak{c}
\end{pmatrix};
\]
elles ont la propri\'et\'e caract\'eristique de laisser invariables les quantit\'es
du type $(\mathfrak{a}-\mathfrak{b}) (\mathfrak{c}-\mathfrak{v})$,
et, si on les applique deux fois, chacune d'elles
donne la substitution identique. Repr\'esentons les quatre lettres $\mathfrak{a}$, $\mathfrak{b}$, $\mathfrak{c}$, $\mathfrak{v}$
par $X_s$ pour les valeurs $0, 1, 2, 3$ de l'indice, en convenant de prendre cet
indice suivant le module 4; elles s'expriment comme il suit:
\[
\begin{pmatrix}
X_s \\ X_{3-s}
\end{pmatrix},
\quad
\begin{pmatrix}
X_s \\ X_{2+s}
\end{pmatrix},
\quad
\begin{pmatrix}
X_s \\ X_{1-s}
\end{pmatrix}.
\]

Si l'on adopte un autre ordre, en supposant que $Z_s$ donne $\mathfrak{c}$, $\mathfrak{a}$, $\mathfrak{b}$, $\mathfrak{v}$
pour $s = 0, 1, 2, 3$, on retrouvera encore, sauf un certain \'echange, les
m\^emes fonctions de l'indice, \`a savoir:
\[
\begin{pmatrix}
Z_s \\ Z_{2+s}
\end{pmatrix},
\quad
\begin{pmatrix}
Z_s \\ Z_{1-s}
\end{pmatrix},
\quad
\begin{pmatrix}
Z_s \\ Z_{3-s}
\end{pmatrix}.
\]
C'est cette disposition qu'il convient de garder, et semblablement nous d\'e\-signerons
les constantes $\frac{i\gamma}{n}$, $\frac{i\alpha}{n}$, $\frac{i\beta}{n}$, $\frac{i\delta}{n}$
par $\varepsilon_s$ pour $s = 0, 1, 2, 3$; cela \'etant,
nous pouvons comprendre, dans ces trois seules \'equations, le syst\`eme de
nos douze relations:
\[
\tag*{(I)}
\left\{
\begin{array}{r}
ik\cn u Z_s = \varepsilon_s Z_{2+s} - D_u Z_{2+s}, \\
 k\sn u Z_s = \varepsilon_s Z_{1-s} - D_u Z_{1-s}, \\
 i\dn u Z_s = \varepsilon_s Z_{3-s} - D_u Z_{3-s}.
\end{array}\right.
\]

Le r\'esultat relatif aux quantit\'es $X_s$ ne diff\`ere de celui-ci qu'en ce que
$ik\cn u$, $k\sn u$, $i\dn u$ se trouvent remplac\'es respectivement par $i\dn u$, $ik\cn u$,
$k\sn u$; en d\'esignant $\frac{i\alpha}{n}$, $\frac{i\beta}{n}$, $\frac{i\gamma}{n}$, $\frac{i\delta}{n}$ par
$\eta_s$ pour $s = 0, 1, 2, 3$, nous aurons, en effet,
\[
\tag*{(II)}
\left\{
\begin{array}{r}
ik\cn u X_s = \eta_s X_{3-s} - D_u X_{3-s}, \\
 k\sn u X_s = \eta_s X_{2+s} - D_u X_{2+s}, \\
 i\dn u X_s = \eta_s X_{1-s} - D_u X_{1-s}.
\end{array}\right.
\]
Avant d'aller plus loin, je crois devoir montrer comment ces deux syst\`emes
%-----067.png--------------
d'\'equations se ram\`enent l'un \`a l'autre, par un changement tr\`es-simple de
la variable et des constantes.

Je me fonderai, \`a cet effet, sur les formules de la transformation du
premier ordre:
\[
  \cn \left( iku, \frac{ik'}{k} \right) = \frac{  1    }{\dn u}, \quad
  \sn \left( iku, \frac{ik'}{k} \right) = \frac{ik\sn u}{\dn u}, \quad
  \dn \left( iku, \frac{ik'}{k} \right) = \frac{  \cn u}{\dn u},
\]
en les \'ecrivant de la mani\`ere suivante, o\`u j'ai fait, pour abr\'eger, $l=\dfrac{ik'}{k}$,
\begin{align*}
  k'\cn(iku, l) &= -\dn(u - K + 2iK'),\\
   l\sn(iku, l) &= +\cn(u - K + 2iK'),\\
    \dn(iku, l) &= -\sn(u - K + 2iK').
\end{align*}

Changeons, en effet, $u$ en $u - K + 2iK'$, et d\'esignons par $Z'_s$ ce que
devient ainsi $Z_s$; les \'equations (I) donneront celles-ci:
\begin{align*}
  ikl \sn (iku, l)Z'_s &= \varepsilon_s Z'_{2+s} - D_u Z'_{2+s},\\
  -k' \dn (iku, l)Z'_s &= \varepsilon_s Z'_{1-s} - D_u Z'_{1-s},\\
 -ik' \cn (iku, l)Z'_s &= \varepsilon_s Z'_{3-s} - D_u Z'_{3-s}.
\end{align*}
Soit encore $Z''_s$ le r\'esultat de la substitution de $\frac{u}{ik}$, au lieu de $u$, on trouvera,
si l'on remarque que $il = -\frac{k'}{k}$,
\begin{align*}
  l \sn(u, l)Z''_s
&= \frac{\varepsilon_s}{ik} Z''_{2+s} - D_u Z''_{2+s},\\
  i \dn(u, l)Z''_s
&= \frac{\varepsilon_s}{ik} Z''_{1-s} - D_u Z''_{1-s},\\
 il \cn(u, l)Z''_s
&= \frac{\varepsilon_s}{ik} Z''_{3-s} - D_u Z''_{3-s},
\end{align*}
nous sommes donc ainsi ramen\'es aux \'equations (II), en y rempla\c{c}ant
les constantes $\eta_s$ par $\frac{\varepsilon_s}{ik}$, ce qui entra\^ine le changement de $k$ en $l$.

Je vais montrer maintenant comment la th\'eorie des fonctions elliptiques
donne la solution de ces nouvelles \'equations auxquelles nous a conduit
le probl\`eme de la rotation.
%-----068.png--------------------


\mysection{XXIII.}


Je repr\'esenterai dans ce qui va suivre les fonctions $\Theta(u)$, $H(u)$, $H_1(u)$,
$\Theta_1(u)$ par $\theta_0(u)$, $\theta_1(u)$, $\theta_2(u)$, $\theta_3(u)$, en adoptant une notation employ\'ee
pour la premi\`ere fois par Jacobi dans ses le\c{c}ons \`a l'Universit\'e de K{\oe}nigsberg,
et dont plusieurs auteurs ont depuis fait usage. L'une quelconque
des quatre fonctions fondamentales sera ainsi d\'esign\'ee par $\theta_s(u)$, et je
ferai de plus la convention que l'indice sera pris suivant le module 4, afin
de pouvoir lui supposer une valeur enti\`ere quelconque. Cela pos\'e, soit $R_s$
le r\'esidu correspondant au p\^ole $u = ik'$ de la quantit\'e $\dfrac{\theta_s(u + a) e^{\lambda u}}{\theta_0 (u)}$, o\`u $a$
et $\lambda$ sont des constantes quelconques, et posons
\[
\Phi_s(u) = \frac{\theta_s(u + a) e^{\lambda u}}{R_s \theta_0 (u)}.
\]

Nous d\'efinissons ainsi un syst\`eme de quatre fonctions comprenant
comme cas particuliers $\sn u$, $\cn u$, $\dn u$, lorsqu'on suppose $a = 0$, $\lambda = 0$,
mais qui, en g\'en\'eral, ne sont point doublement p\'eriodiques, et se reproduisent
multipli\'ees par des constantes, lorsqu'on change $u$ en $u + 2K$ et
en $u + 2iK'$~(\footnote{
  Peut-\^etre pourrait-on, afin d'abr\'eger, convenir de d\'esigner les quantit\'es de cette
  nature sous le nom de \textit{fonctions doublement p\'eriodiques de seconde esp\`ece}, les fonctions
  p\'eriodiques de premi\`ere esp\`ece correspondant au cas o\`u les multiplicateurs seraient \'egaux
  \`a l'unit\'e. Enfin les quantit\'es telles que $\Theta(u)$, $H(u)$,
\ldots, les fonctions interm\'ediaires de
  MM.~Briot et Bouquet, o\`u les multiplicateurs sont des exponentielles, recevraient par
  analogie le nom de \textit{fonctions p\'eriodiques de troisi\`eme esp\`ece}.}).
On a en effet, en posant $\mu = e^{2 \lambda K}$,
$\mu' = e^{-\frac{i\pi a}{K} + 2i \lambda K'}$, les
relations suivantes:
\begin{alignat*}{2}
& \Phi_s(u + 2K) &&= \mu (-1)^{\frac{1}{2}s(s + 1)} \Phi_s (u), \\
& \Phi_s(u + 2iK') &&= \mu'(-1)^{\frac{1}{2}s(s - 1)} \Phi_s (u),
\end{alignat*}
et, en passant aux valeurs particuli\`eres de l'indice, les multiplicateurs seront
indiqu\'es comme il suit:
\[
\begin{matrix}
\Phi_0 (s), & +\mu, & +\mu',\\
\Phi_1 (s), & -\mu, & +\mu',\\
\Phi_2 (s), & -\mu, & -\mu',\\
\Phi_3 (s), & +\mu, & -\mu'.
\end{matrix}
\]
%-----069.png-----------------------------

L'\'etude de leurs propri\'et\'es pourrait peut-\^etre former un chapitre
nouveau dans la th\'eorie des fonctions elliptiques, mais en ce moment je
dois me borner \`a en tirer la solution que j'ai en vue du probl\`eme de la rotation.
Je partirai de ce que les expressions $\Phi_s(u)$, ayant un seul p\^ole $u = iK'$
\`a l'int\'erieur du rectangle des p\'eriodes et pour r\'esidu correspondant l'unit\'e,
peuvent jouer le r\^ole d'\'el\'ements simples \`a l'\'egard des fonctions qui ont
les m\^emes multiplicateurs. Telles seront, par exemple, les quantit\'es
\[
\cn u\, \Phi_s(u), \quad \sn u\, \Phi_s(u), \quad \dn u\, \Phi_s(u);
\]
si l'on remarque qu'en mettant $2+s$, $1-s$, $3-s$ au lieu de $s$, le facteur
$(-1)^{\frac{1}{2}s(s+1)}$ se reproduit multipli\'e par $-1$, $-1$, $+1$, tandis que
$(-1)^{\frac{1}{2}s(s-1)}$ est multipli\'e successivement par $-1$, $+1$, $-1$, on reconna\^it
en effet qu'elles ont respectivement les multiplicateurs des fonctions
\[
\Phi_{2+s}(u), \quad \Phi_{1-s}(u), \quad \Phi_{3-s}(u).
\]

Nous voyons aussi qu'elles n'admettent que le p\^ole $u = iK'$, dans le
rectangle des p\'eriodes, de sorte que la d\'ecomposition en \'el\'ements
simples s'obtiendra imm\'ediatement au moyen de la partie principale des
trois d\'eveloppe\-ments
\[
\cn(iK' + \varepsilon) \Phi_s(iK' + \varepsilon), \quad
\sn(iK' + \varepsilon) \Phi_s(iK' + \varepsilon), \quad
\dn(iK' + \varepsilon) \Phi_s(iK' + \varepsilon).
\]
Or on a, sans aucun terme constant dans les seconds membres,
\[
ik \cn(iK' + \varepsilon) = \frac{1}{\varepsilon}, \quad
 k \sn(iK' + \varepsilon) = \frac{1}{\varepsilon}, \quad
 i \dn(iK' + \varepsilon) = \frac{1}{\varepsilon},
\]
et par cons\'equent il suffit de calculer les deux premiers termes du d\'eveloppe\-ment
de l'autre facteur $\Phi_s(iK' + \varepsilon)$, c'est-\`a-dire le terme en $\frac{1}{\varepsilon}$, et le
terme constant. J'emploie \`a cet effet la relation, sur laquelle je reviendrai
tout \`a l'heure,
\[
\theta_s(u+iK') = \sigma \theta_{1-s}(u) e^{-\frac{i\pi}{4K} (2u + iK')},
\]
%-----070.png-----------------------------
o\`u $\sigma$ est \'egal \`a $i$ pour $s = 0$, $s = 1$, et \`a l'unit\'e, si l'on suppose $s = 2$,
$s = 3$, de sorte qu'on peut faire $\sigma = -e^{-\frac{i\pi}{4} (s+1)(s+2)(2s+1)}$. On en conclut
l'expression suivante:
\[
\Phi_s(iK' + \varepsilon) = A\frac{\theta_{1-s}(a + \varepsilon) e^{\lambda \varepsilon}}{\theta_1(\varepsilon)},
\]
$A$ d\'esignant un facteur constant, et par suite ce d\'eveloppement, que je
limite \`a ses deux premiers termes:
\[
\Phi_s(iK' + \varepsilon) = \frac{A\theta_{1-s}(a)}{\theta'_1(0)}
  \left[ \frac{1}{\varepsilon} + \lambda + D_a \log \theta_{1-s}(a) \right].
\]
Mais $A$ doit \^etre tel que le coefficient de $\frac{1}{\varepsilon}$ soit
l'unit\'e; nous avons donc simplement
\[
\Phi_s(iK' + \varepsilon) = \frac{1}{\varepsilon} + \lambda + D_a \log \theta_{1-s}(a),
\]
et l'on voit que les parties principales des d\'eveloppements des fonctions
\begin{align*}
ik\sn(iK' + \varepsilon) &\Phi_s(iK' + \varepsilon), \\
 k\sn(iK' + \varepsilon) &\Phi_s(iK' + \varepsilon), \\
 i\dn(iK' + \varepsilon) &\Phi_s(iK' + \varepsilon)
\end{align*}
se r\'eduisent \`a cette seule et m\^eme expression dans les trois cas, \`a savoir:
\[
\frac{1}{\varepsilon^2} + \left[\lambda + D_a \log \theta_{1-s}(a)\right] \frac{1}{\varepsilon}.
\]
\medskip

La formule g\'en\'erale de d\'ecomposition en \'el\'ements simples nous
donne en cons\'equence les relations suivantes:\label{page62}
\begin{align*}
ik \cn u\, \Phi_s(u) &= \left[\lambda + D_a \log \theta_{1-s}(a) \right] \Phi_{2+s}(u) - D_u \Phi_{2+s}(u),\\
 k \sn u\, \Phi_s(u) &= \left[\lambda + D_a \log \theta_{1-s}(a) \right] \Phi_{1-s}(u) - D_u \Phi_{1-s}(u),\\
 i \dn u\, \Phi_s(u) &= \left[\lambda + D_a \log \theta_{1-s}(a) \right] \Phi_{3-s}(u) - D_u \Phi_{3-s}(u);
\end{align*}
et l'on voit qu'on les identifiera aux \'equations (I), obtenues dans le paragraphe
pr\'ec\'edent, en disposant des ind\'etermin\'ees $a$ et $\lambda$ de mani\`ere \`a
avoir
\[
\varepsilon_s = \lambda + D_a \log \theta_{1-s}(a).
\]
%-----071.png---------------------------

Reprenons, \`a cet effet, les \'egalit\'es donn\'ees, p.~\pageref{page36}, \S~XV,
\[
\alpha-\beta = in\frac{k^2\sn\omega\cn\omega}{\dn\omega},
  \quad \alpha-\delta = in\frac{\sn\omega\dn\omega}{\cn\omega},
  \quad \gamma-\alpha = in\frac{\cn\omega\dn\omega}{\sn\omega},
\]
en les \'ecrivant d'abord de cette mani\`ere (voir p.~\pageref{page37}):
\[
\frac{i\alpha}{n} + \frac{\Theta'(\omega)}{\Theta(\omega)} =
  \frac{i\beta}{n} + \frac{\Theta'_1(\omega)}{\Theta_1(\omega)} =
  \frac{i\gamma}{n} + \frac{H'(\omega)}{H(\omega)} =
  \frac{i\delta}{n} + \frac{H'_1(\omega)}{H_1(\omega)}.
\]

Rappelons ensuite que les constantes $\frac{i\gamma}{n}$, $\frac{i\alpha}{n}$, $\frac{i\beta}{n}$, $\frac{i\delta}{n}$ ont \'et\'e d\'esign\'ees
par $\varepsilon_s$ pour $s = 0, 1, 2, 3$, et elles prendront, en introduisant les quantit\'es
$\theta_s(\omega)$, cette nouvelle forme
\begin{align*}
\varepsilon_1 + D_{\omega}\log\theta_0(\omega) &= \varepsilon_2 + D_{\omega}\log\theta_3(\omega) \\
  &= \varepsilon_0 + D_{\omega}\log\theta_1(\omega) = \varepsilon_3 + D_{\omega}\log\theta_2(\omega).
\end{align*}
Il en r\'esulte que l'expression
\[
\varepsilon_s + D_{\omega}\log\theta_{1-s}(\omega)
\]
reste la m\^eme pour toutes les valeurs de $s$; par cons\'equent on satisfait imm\'ediatement
\`a la condition pos\'ee en faisant
\[
a = -\omega \quad\text{et}\quad \lambda = \varepsilon_s + D_{\omega}\log\theta_{1-s}(\omega).
\]


\mysection{XXIV.}


Les r\'esultats que nous venons d'obtenir montrent encore par un
nouvel exemple combien la question de la rotation se trouve intimement
li\'ee \`a la th\'eorie des fonctions elliptiques. C'est m\^eme \`a l'\'etude d'un probl\`eme
de M\'ecanique qu'est due la consid\'eration de ces nouveaux \'el\'ements
analytiques $\Phi_s(u)$, tr\`es-voisins des fonctions $\varphi(x,\omega)$, $\varphi_1(x,\omega)$,
$\chi(x,\omega)$, $\chi_1(x,\omega)$, employ\'ees au commencement de ce travail pour int\'egrer
l'\'equation de Lam\'e, mais qui en sont n\'eanmoins distincts et offrent un
ensemble de propri\'et\'es propres. Il est n\'ecessaire, en effet, d'attribuer \`a la
constante $\lambda$ quatre valeurs particuli\`eres pour en d\'eduire ces derni\`eres
%-----072.png---------------------------
fonctions, et de l\`a r\'esultent, pour les multiplicateurs de chacune d'elles,
des d\'eterminations essentiellement diff\'erentes, tandis que la propri\'et\'e
essentielle qui r\'eunit en un seul syst\`eme les fonctions $\Phi_s(u)$, c'est d'avoir,
sauf le signe, les m\^emes multiplicateurs. Je me bornerai \`a leur \'egard \`a
consid\'erer, pour en donner l'int\'egrale compl\`ete, les \'equations diff\'erentielles
auxquelles elles satisfont, \'equations lin\'eaires et du second ordre
comme celle de Lam\'e; mais auparavant je dois d'abord montrer comment
les formules de Jacobi r\'esultent de l'expression \`a laquelle nous venons
de parvenir, $Z_s=N\Phi_s(u)$, o\`u $N$ d\'esigne une constante. J'emploie, \`a cet
effet, la valeur de $R_s$, qu'on obtient facilement sous la forme
\[
R_s = \frac{\sigma\theta_{1-s}(a) e^{-\frac{i\pi a}{2K}+i\lambda K'}}{i\theta'_1(0)}
\]
et o\`u l'on doit faire $a = -\omega$. En se rappelant la d\'etermination du facteur
$\sigma$, et \'ecrivant pour un moment
\[
\Omega = \sigma\frac{e^{\frac{i\pi\omega}{2K}+i\lambda K'}}{i\theta'_1(0)},
\]
nous obtenons ainsi
\[
R_0 = -i\Omega\theta_1(\omega),\quad
  R_1 = i\Omega\theta_0(\omega),\quad
  R_2 = \Omega\theta_3(\omega),\quad
  R_3 = \Omega\theta_2(\omega).
\]
Or on a
\[
A = \frac{i}{k\cn\omega}Z_1,\quad
  B = \frac{\dn\omega}{k\cn\omega}Z_2,\quad
  C = -\frac{\sn\omega}{\cn\omega}Z_0,\quad
  V = -inZ_3;
\]
de l\`a r\'esultent, si l'on remplace $N$ par $\Omega N$ et les quantit\'es $\theta_s$ par $\Theta$, $H$, \ldots,
les valeurs suivantes:
\begin{gather*}
\begin{alignedat}{2}
A &= \frac{iN}{k\cn\omega}\frac{H(u-\omega)e^{\lambda u}}{i\Theta(\omega)\Theta(u)}
  &&= \frac{N}{\sqrt{kk'}}\frac{H(u-\omega)e^{\lambda u}}{H_1(\omega)\Theta(u)},\\
B &= \frac{\dn\omega N}{k\cn\omega}\frac{H_1(u-\omega)e^{\lambda u}}{\Theta_1(\omega)\Theta(u)}
  &&= \frac{N}{\sqrt{k}}\frac{H_1(u-\omega)e^{\lambda u}}{H_1(\omega)\Theta(u)},\\
C &= \frac{\sn\omega N}{\cn\omega}\frac{\Theta(u-\omega)e^{\lambda u}}{iH(\omega)\Theta(u)}
  &&= \frac{N}{\sqrt{k'}}\frac{\Theta(u-\omega)e^{\lambda u}}{iH_1(\omega)\Theta(u)},
\end{alignedat}\\
V = -inN\frac{\Theta_1(u-\omega)e^{\lambda u}}{H_1(\omega)\Theta(u)}.
\end{gather*}
%-----073.png-----------------
Je ne m'arr\^ete pas \`a la d\'etermination de la constante $N$, qui
s'obtient comme on l'a d\'ej\`a vu au \S~XIV, p.~\pageref{page31a};
elle a pour valeur $H'(0)e^{i\nu}$, et nous retrouvons bien, sauf le
changement de $\lambda$ en $i\lambda$, les r\'esultats qu'il fallait
obtenir.

Je reviens encore un moment sur la d\'esignation par $\theta_s(u)$ des quatre
fonctions fondamentales de Jacobi, afin de la rapprocher de la notation
qui r\'esulte de la d\'efinition m\^eme de ces fonctions, par la s\'erie
\[
  \theta_{\mu,\nu}(u) =
  e^{-\frac{\mu\nu i\pi}{2}}
  \sum (-1)^{m\nu}
    e^{\frac{i\pi}{K}\left[ (2m+\mu)u + \frac{1}{4}(2m+\mu)^2 iK' \right]} .
\]
Supposant $\mu$ et $\nu$ \'egaux \`a z\'ero ou \`a l'unit\'e, on a donc en m\^eme temps
\begin{align*}
  \Theta(u)  &= \theta_0 (u) = \theta_{0,1}(u), \\
   H    (u)  &= \theta_1 (u) = \theta_{1,1}(u), \\
   H_1  (u)  &= \theta_2 (u) = \theta_{0,1}(u), \\
  \Theta_1(u)&= \theta_3 (u) = \theta_{0,0}(u);
\end{align*}
et, en premier lieu, je remarquerai que le syst\`eme des quatre \'equations
fondamentales
\begin{align*}
  \Theta(u+iK') &=
    iH(u)       e^{-\frac{i\pi}{4K}(2u+iK')} , \\
  H     (u+iK') &=
    i\Theta(u)  e^{-\frac{i\pi}{4K}(2u+iK')} , \\
  H_1   (u+iK') &=
    \Theta_1(u)e^{-\frac{i\pi}{4K}(2u+iK')} , \\
 \Theta_1(u+iK')&=
    H_1(u)     e^{-\frac{i\pi}{4K}(2u+iK')}
\end{align*}
peut \^etre remplac\'e par la relation unique dont j'ai d\'ej\`a fait usage, \`a savoir
\[
  \theta_s (u+iK') = \sigma\theta_{1-s}(u)
    e^{-\frac{i\pi}{4K}(2u+iK')} .
\]
On doit y joindre les suivantes:
\begin{alignat*}{2}
  \theta_s(u+K) &=
    \sigma' &&\theta_{3-s}(u) e^{-\frac{i\pi}{4K}(2u+iK')} , \\
  \theta_s(u+K+iK') &=
    \sigma''&&\theta_{2+s}(u) e^{-\frac{i\pi}{4K}(2u+iK')} ,
\end{alignat*}
%-----074.png------------------------------
les facteurs $\sigma$, $\sigma'$, $\sigma''$ ayant pour valeurs
\[
\sigma   = -e^{-\frac{i \pi}{4}(s + 1)(s + 2)(2s + 1)}, \quad
\sigma'  =  e^{ \frac{i \pi}{2} s (s^2 -1)}, \quad
\sigma'' =  e^{-\frac{i \pi}{4} s (s - 1)};
\]
puis celles-ci:
\begin{align*}
\theta_s(u + 2K)    &= \phantom{-}(-1)^{\frac{1}{2}s(s+1)} \theta_s(u), \\
\theta_s(u + 2iK')  &=           -(-1)^{\frac{1}{2}s(s-1)} \theta_s(u) e^{-\frac{i\pi}{K}(u + iK')} .
\end{align*}

Je remarquerai enfin qu'en passant du syst\`eme de deux indices \`a un
indice unique on est amen\'e \`a exprimer, d'une mani\`ere g\'en\'erale, $s$ au
moyen de $\mu$ et $\nu$. Si nous avons \'egard \`a la convention admise que $s$ est
pris suivant le module 4, on trouve ais\'ement l'expression
\[
s \equiv - 1 - \mu + \nu + 2\mu\nu.
\]
Cela \'etant, soit de m\^eme
\[
s' \equiv -1 - \mu' + \nu' + 2\mu'\nu',
\]
et d\'esignons par $S$ la quantit\'e relative aux sommes $\mu+\mu'$ et $\nu+\nu'$. Les
admirables travaux de M. Weierstrass ayant montr\'e de quelle importance
est, pour la th\'eorie des fonctions ab\'eliennes, l'addition des indices dans les
fonctions $\theta$ \`a $n$ variables, o\`u entrent $2n$ quantit\'es analogues \`a $\mu$ et $\nu$, on
est amen\'e, dans le cas le plus simple des fonctions elliptiques, \`a chercher
l'expression de $S$ en $s$ et $s'$. M.~Lipschitz m'a communiqu\'e la solution de
cette question par la formule \'el\'egante
\[
S \equiv -1 - s - s' - 2ss' \pmod{4},
\]
et voici comment l'\'eminent g\'eom\`etre la d\'emontre. \'Ecrivons l'\'egalit\'e pr\'e\-c\'edemment
donn\'ee: $2s \equiv -1 - \mu + \nu + 2\mu\nu$ sous cette forme
\[
2s + 1 \equiv (2\mu + 1)(2\nu - 1) \pmod{8},
\]
et remarquons qu'on peut poser, $\mu$ et $\nu$ \'etant z\'ero ou l'unit\'e,
\[
2 \mu + 1 \equiv  3^{\mu}, \quad
2 \nu - 1 \equiv -7^{\nu} \pmod{8}.
\]
%-----075.png---------------------------
On en conclura
\[
2s + 1 \equiv -3^{\mu} 7^{\nu} \pmod{8};
\]
or les relations analogues
\[
2s'+1 \equiv -3^{\mu'}7^{\nu'},\quad 2S+1\equiv -3^{\mu+\mu'}7^{\nu+\nu'} \pmod{8}
\]
donneront imm\'ediatement:
\[
2S+1 \equiv -(2s+1)(2s'+1)\pmod{8},
\]
et l'on en conclut l'\'equation qu'il s'agissait d'obtenir.


\mysection{XXV.}


Nous avons vu que le syst\`eme des quatre fonctions repr\'esent\'ees, en faisant
$s = 0, 1, 2, 3$, par l'expression
\[
\Phi_s(u) = \frac{\theta_s(u+a)e^{\lambda u}}{R_s\theta_0(u)},
\]
o\`u $a$ et $\lambda$ sont des constantes quelconques et $R_s$ le r\'esidu correspondant
au p\^ole $u = iK'$ de $\frac{\theta_s(u+a)e^{\lambda u}}{\theta_0(u)}$, conduit aux \'equations diff\'erentielles suivantes
(\S~XXIII, p.~\pageref{page62}):
\begin{align*}
ik\cn u\,\Phi_s(u) &= [\lambda+D_a\log\theta_{1-s}(a)]\Phi_{2+s}(u)-D_u\Phi_{2+s}(u),\\
k\sn u\,\Phi_s(u) &= [\lambda+D_a\log\theta_{1-s}(a)]\Phi_{1-s}(u)-D_u\Phi_{1-s}(u),\\
i\dn u\,\Phi_s(u) &= [\lambda+D_a\log\theta_{1-s}(a)]\Phi_{3-s}(u)-D_u\Phi_{3-s}(u).
\end{align*}
Ces relations me paraissent appeler l'attention, comme donnant d'elles-m\^emes
des \'equations lin\'eaires du second ordre, dont la solution compl\`ete
s'obtient, ainsi que celle de Lam\'e, dans le cas de $n = 1$, par des fonctions
doublement p\'eriodiques de seconde esp\`ece, ayant la demi-p\'eriode $iK'$ pour
infini simple. Pour y parvenir facilement, il convient de repr\'esenter les
quantit\'es $ik \cn u$, $k \sn u$, $i \dn u$ par $U_1$, $U_2$, $U_3$, de mani\`ere \`a avoir sous
forme enti\`erement sym\'etrique:
\[
D_u U_1 = -U_2 U_3,\quad
D_u U_2 = -U_1 U_3,\quad
D_u U_3 = -U_1 U_2.
\]
%-----076.png---------------------------
Cela \'etant, si nous changeons successivement $s$ en $2+s$, $1-s$, $3-s$, on
obtiendra, en \'ecrivant, pour abr\'eger, $\Phi_s$ au lieu de $\Phi_s(u)$ et $\varepsilon_s$ pour
$\lambda+ D_a\log\theta_{1-s}(a)$, ces trois groupes de deux \'equations, \`a savoir:
\begin{gather*}
\left\{
\begin{alignedat}{5}
&U_1\Phi_s     &&= \varepsilon_s    &&\Phi_{2+s} && - D_u\Phi_{2+s},\\
&U_1\Phi_{2+s} &&= \varepsilon_{2+s}&&\Phi_s     && - D_u\Phi_s,
\end{alignedat}
\right.\\
\left\{
\begin{alignedat}{5}
&U_2\Phi_s     &&= \varepsilon_s    &&\Phi_{1-s} && - D_u\Phi_{1-s},\\
&U_2\Phi_{1-s} &&= \varepsilon_{1-s}&&\Phi_s     && - D_u\Phi_s,
\end{alignedat}
\right.\\
\left\{
\begin{alignedat}{5}
&U_3\Phi_s     &&= \varepsilon_s    &&\Phi_{3-s} && - D_u\Phi_{3-s},\\
&U_3\Phi_{3-s} &&= \varepsilon_{3-s}&&\Phi_s     && - D_u\Phi_s.
\end{alignedat}
\right.
\end{gather*}
L'\'elimination successive des quantit\'es $\Phi_{2+s}$, $\Phi_{1-s}$, $\Phi_{3-s}$ donne ensuite
\begin{gather*}
\tag*{(I)}
\begin{split}
D_u^2\Phi_s-(\varepsilon_s&+\varepsilon_{2+s}+D_u\log U_1)D_u\Phi_s \\ &+
  (\varepsilon_s\varepsilon_{2+s}+\varepsilon_{2+s}D_u\log U_1-U_1^2)\Phi_s=0,
\end{split} \\
\tag*{(II)}
\begin{split}
D_u^2\Phi_s-(\varepsilon_s&+\varepsilon_{1-s}+D_u\log U_2)D_u\Phi_s \\ &+
  (\varepsilon_s\varepsilon_{1-s}+\varepsilon_{1-s}D_u\log U_2-U_2^2)\Phi_s=0,
\end{split} \\
\tag*{(III)}
\begin{split}
D_u^2\Phi_s-(\varepsilon_s&+\varepsilon_{3-s}+D_u\log U_3)D_u\Phi_s \\&+
  (\varepsilon_s\varepsilon_{3-s}+\varepsilon_{3-s}D_u\log U_3-U_3^2)\Phi_s=0.
\end{split}
\end{gather*}
Nous avons donc trois \'equations du second ordre dont une solution particuli\`ere
est la fonction $\Phi_s(u)$; voici comment on parvient \`a les int\'egrer compl\`etement.
\medskip

Faisons successivement dans (I), (II) et~(III)
\begin{align*}
\Phi_s &= X_1e^{\frac{u}{2}(\varepsilon_s+\varepsilon_{2+s})},\\
\Phi_s &= X_2e^{\frac{u}{2}(\varepsilon_s+\varepsilon_{1-s})},\\
\Phi_s &= X_3e^{\frac{u}{2}(\varepsilon_s+\varepsilon_{3-s})};
\end{align*}
on aura pour transform\'ees:
\begin{align*}
D_u^2X_1-D_u\log U_1D_uX_1-(\delta_1^2+\delta_1D_u\log U_1+U_1^2)X_1&=0,\\
D_u^2X_2-D_u\log U_2D_uX_2-(\delta_2^2+\delta_2D_u\log U_2+U_2^2)X_2&=0,\\
D_u^2X_3-D_u\log U_3D_uX_3-(\delta_3^2+\delta_3D_u\log U_3+U_3^2)X_3&=0,
\end{align*}
en posant, pour abr\'eger l'\'ecriture,
\[
\delta_1 = \tfrac{1}{2}(\varepsilon_s-\varepsilon_{2+s}),\quad
  \delta_2 = \tfrac{1}{2}(\varepsilon_s-\varepsilon_{1-s}),\quad
  \delta_3 = \tfrac{1}{2}(\varepsilon_s-\varepsilon_{3-s}).
\]

Je remarque maintenant que ces \'equations ne changent pas si, en rempla\c{c}ant
%-----077.png------------------------------
dans la premi\`ere, la deuxi\`eme et la troisi\`eme, $s$ par $2+s$, $1-s$
et $3-s$, on \'ecrit dans toutes en m\^eme temps $-u$ au lieu de $u$. Par
cons\'equent, on peut, d'une solution, en tirer une autre: la premi\`ere, par
exemple, qui est v\'erifi\'ee en prenant
\[
X_1 = \Phi_s(u) e^{-\frac{u}{2}(\varepsilon_s + \varepsilon_{2+s})} , \]
le sera encore si l'on fait
\[
X_1 = \Phi_{2+s}(-u)e^{+\frac{u}{2}(\varepsilon_s + \varepsilon_{2+s})}.
\]
En employant les formules
\[
\varepsilon_s = \lambda + D_a \log \theta_{1-s}(a), \qquad
\varepsilon_{2+s} = \lambda + D_a \log \theta_{3-s}(a),
\]
et mettant pour abr\'eger $\theta_s$ au lieu de $\theta_s(a)$, on en conclut pour l'int\'egrale
g\'en\'erale
\[
X_1 = \frac{C \theta_s     (u + a)}{\theta_0(u)} e^{-\frac{u}{2} D_a \log \theta_{1-s} \theta_{3-s} }
+     \frac{C'\theta_{2+s} (u - a)}{\theta_0(u)} e^{ \frac{u}{2} D_a \log \theta_{1-s} \theta_{3-s} } .
\]
Les solutions des deux autres \'equations seront semblablement\label{page69}
\begin{align*}
X_2 &= \frac{C  \theta_s     (u + a)}{\theta_0(u)} e^{-\frac{u}{2}D_a \log \theta_s \theta_{1-s} }
     + \frac{C' \theta_{1-s} (u - a)}{\theta_0(u)} e^{ \frac{u}{2}D_a \log \theta_s \theta_{1-s} },\\
X_3 &= \frac{C  \theta_s     (u + a)}{\theta_0(u)} e^{-\frac{u}{2}D_a \log \theta_s \theta_{2+s} }
     + \frac{C' \theta_{3-s} (u - a)}{\theta_0(u)} e^{ \frac{u}{2}D_a \log \theta_s \theta_{2+s} }.
\end{align*}


\mysection{XXVI.}


Les relations qui nous ont servi de point de d\'epart donnent
lieu \`a d'autres combinaisons dont se tirent de nouvelles \'equations du
second ordre analogues aux pr\'ec\'edentes, et qu'il est important de former.
On a, par exemple, comme on le voit facilement,
\[
U_1(\varepsilon_s \Phi_{1-s} - D_u \Phi_{1-s}) = U_2 ( \varepsilon_s \Phi_{2+s} - D_u \Phi_{2+s} ),
\]
et l'on en conclut, en changeant $s$ en $1-s$,
\[
U_1(\varepsilon_{1-s} \Phi_s - D_u \Phi_s) = U_2(\varepsilon_{1-s} \Phi_{3-s} - D_u \Phi_{3-s}).
\]

Joignons \`a cette \'equation la suivante:
\[
U_3 \Phi_{3-s} = \varepsilon_{3-s} \Phi_s - D_u \Phi_s,
\]
%-----078.png------------------------------
et l'on trouvera, par l'\'elimination de $\Phi_{3-s}$,
\begin{align*}
D_u^2 \Phi_s &- (\varepsilon_{1-s} + \varepsilon_{3-s} + D_u \log U_2 U_3) D_u \Phi_s \\
             &+ (\varepsilon_{1-s}\varepsilon_{3-s} + \varepsilon_{1-s} D_u \log U_2 + \varepsilon_{3-s} D_u \log U_3 ) \Phi_s = 0.
\end{align*}
De simples changements de lettres donneront ensuite
\begin{align*}
D_u^2 \Phi_s &- (\varepsilon_{3-s} + \varepsilon_{2+s} + D_u \log U_1 U_3) D_u \Phi_s \\
             &+ (\varepsilon_{3-s}\varepsilon_{2+s} + \varepsilon_{3-s} D_u \log U_3 + \varepsilon_{2+s} D_u \log U_1) \Phi_s = 0, \\
D_u^2 \Phi_s &- (\varepsilon_{1-s} + \varepsilon_{2+s} + D_u \log U_1 U_2) D_u \Phi_s \\
             &+ (\varepsilon_{1-s}\varepsilon_{2+s} + \varepsilon_{1-s} D_u \log U_2 + \varepsilon_{2+s} D_u \log U_1) \Phi_s = 0.
\end{align*}

Cela pos\'e, je fais dans la premi\`ere, la deuxi\`eme et la troisi\`eme de
ces \'equations, les substitutions
\begin{align*}
\Phi_s &= Y_1\, e^{ \frac{u}{2} (\varepsilon_{1-s} + \varepsilon_{3-s})},\\
\Phi_s &= Y_2\, e^{ \frac{u}{2} (\varepsilon_{3-s} + \varepsilon_{2+s})},\\
\Phi_s &= Y_3\, e^{ \frac{u}{2} (\varepsilon_{1-s} + \varepsilon_{2+s})}.
\end{align*}

J'\'ecris aussi, pour abr\'eger,
\[
\delta'_1 = \tfrac{1}{2}( \varepsilon_{1-s} - \varepsilon_{3-s}), \quad
\delta'_2 = \tfrac{1}{2}( \varepsilon_{2+s} - \varepsilon_{3-s}), \quad
\delta'_3  = \tfrac{1}{2}( \varepsilon_{2+s} - \varepsilon_{1-s});
\]
les transform\'ees qui en r\'esultent, savoir:
\begin{align*}
D_u^2 Y_1 - D_u \log U_2 U_3 D_u Y_1 - \left( \delta_1'^2 - \delta_1' D_u \log \frac{U_2}{U_3} \right) Y_1 &= 0, \\
D_u^2 Y_2 - D_u \log U_1 U_3 D_u Y_2 - \left( \delta_2'^2 - \delta_2' D_u \log \frac{U_1}{U_2} \right) Y_2 &= 0, \\
D_u^2 Y_3 - D_u \log U_1 U_2 D_u Y_3 - \left( \delta_3'^2 - \delta_3' D_u \log \frac{U_3}{U_1} \right) Y_3 &= 0,
\end{align*}
se reproduisent comme les \'equations en $X$, lorsqu'on change $s$ en $2+s$,
$1-s$, $3-s$ et $u$ en $-u$, les quantit\'es $\delta$ et $\delta'$, ainsi que les d\'eriv\'ees logarithmiques,
changeant de signe. On en conclut imm\'ediatement pour les int\'egrales
compl\`etes les formules
\begin{alignat*}{2}
Y_1 &= \frac{ C  \theta_s     (u + a) }{ \theta_0(u) } e^{-\frac{u}{2} D_a \log \theta_s     \theta_{2+s}}
     &&+ \frac{ C' \theta_{2+s} (u - a) }{ \theta_0(u) } e^{ \frac{u}{2} D_a \log \theta_s     \theta_{2+s}} , \\
Y_2 &= \frac{ C  \theta_s     (u + a) }{ \theta_0(u) } e^{-\frac{u}{2} D_a \log \theta_{2+s} \theta_{3-s}}
     &&+ \frac{ C' \theta_{1-s} (u - a) }{ \theta_0(u) } e^{ \frac{u}{2} D_a \log \theta_{2+s} \theta_{3-s}} , \\
Y_3 &= \frac{ C  \theta_s     (u + a) }{ \theta_0(u) } e^{-\frac{u}{2} D_a \log \theta_s     \theta_{3-s}}
     &&+ \frac{ C' \theta_{3-s} (u - a) }{ \theta_0(u) } e^{ \frac{u}{2} D_a \log \theta_s     \theta_{3-s}} .
\end{alignat*}
%-----079.png------------------------------

Ce sont donc les m\^emes quotients des fonctions $\theta$ qui figurent dans
les valeurs de $X_1$ et $Y_1$, $X_2$ et $Y_2$, $X_3$ et $Y_3$, les exponentielles qui multiplient
ces quotients \'etant seules diff\'erentes. Cette circonstance fait pr\'esumer
l'existence d'\'equations lin\'eaires du second ordre plus g\'en\'erales, dont la
solution s'obtiendrait en rempla\c{c}ant, dans les expressions $CA + C'B$ des
quantit\'es $X$ et $Y$, les fonctions d\'etermin\'ees $A$ et $B$ par $A e^{pu}$ et $B e^{-pu}$, o\`u $p$
est une constante quelconque; voici comment on les obtient.


\mysection{XXVII.}


Consid\'erons en g\'en\'eral une \'equation lin\'eaire du second ordre \`a laquelle
nous donnerons la forme suivante:
\[
PX''-P'X'+QX = 0,
\]
o\`u $P$ et $Q$ sont des fonctions quelconques de la variable $u$, et dont l'int\'egrale
soit
\[
X = CA+C'B.
\]

Je dis que, si l'on conna\^it le produit de deux solutions particuli\`eres,
et qu'on fasse en cons\'equence
\[
AB = R,
\]
nous pourrons obtenir l'\'equation qui aurait pour solution l'expression
plus g\'en\'erale
\[
\mathfrak{X} = CA\, e^{pu} + C'B\, e^{-pu}.
\]
J'observe \`a cet effet que, le r\'esultat de l'\'elimination des constantes $C$ et $C'$
\'etant
\[
\begin{vmatrix}
\mathfrak{X}     & A                & B                  \\
\mathfrak{X}'    & Ap + A'          & -Bp + B'           \\
\mathfrak{X}''   & Ap^2 + 2A'p + A''  & Bp^2 - 2B'p + B''
\end{vmatrix}
= 0,
\]
le d\'eveloppement du d\'eterminant donne pour l'\'equation cherch\'ee
\[
\mathfrak{PX'' - P'X' + QX} = 0,
\]
%-----080.png---------------------------
les nouvelles fonctions $\mathfrak{P}$ et $\mathfrak{Q}$ ayant pour expressions
\begin{align*}
\mathfrak{P} &= AB'-BA'-2ABp,\\
\mathfrak{Q} &= A'B''-B'A''+(AB''-4A'B'+BA'')p-3(AB'-BA')p^2+2ABp^3.
\end{align*}
Or on a, quelles que soient les solutions particuli\`eres $A$ et $B$, la relation
\[
AB'-BA'=Pg,
\]
en d\'esignant par $g$ une constante dont voici la d\'etermination.

Donnons \`a la variable une valeur $u = u_0$ qui annule $B$ dans cette
\'equation et la suivante:
\[
AB'+BA'= R',
\]
et soient $P_0$ et $R'_0$ les valeurs que prennent $P$ et $R$; on trouvera imm\'ediate\-ment
la condition
\[
P_0g=R'_0.
\]
La constante $g$ \'etant ainsi connue, nous avons d\'ej\`a la formule
\[
\mathfrak{P}=Pg-2Rp.
\]
Pour obtenir $\mathfrak{Q}$, je remarque d'abord qu'on peut \'ecrire
\[
A'B''-B'A''=\frac{P'B'-QB}{P}A'-\frac{P'A'-QA}{P}B'=Qg,
\]
puis semblablement
\[
AB''+BA''=\frac{P'B'-QB}{P}A+\frac{P'A'-QA}{P}B=\frac{P'R'-2QR}{P};
\]
nous avons d'ailleurs
\[
AB''+2A'B'+BA''=R'',
\]
par cons\'equent
\[
AB''-4A'B'+BA''= -\frac{2PR''-3P'R'+6QR}{P},
\]
et l'on en conclut la valeur cherch\'ee:
\[
\mathfrak{Q}=Qg-\frac{2PR''-3P'R'+6QR}{P}p-3Pgp^2+2Rp^3.
\]
%-----081.png------------------------------

Ce point \'etabli, j'envisage, dans les \'equations diff\'erentielles en $X_1$,
$X_2$, $X_3$, les expressions du produit $AB$, que je d\'esignerai successivement
par $R_1(u)$, $R_2(u)$, $R_3(u)$, en faisant
\begin{align*}
R_1(u) &= \frac{\theta'^2_1(0) \theta_s(u + a) \theta_{2+s}(u - a)}{
                \theta_0^2(u) \theta_{1-s}(a) \theta_{3-s}(a)},  \\
R_2(u) &= \frac{\theta'^2_1(0) \theta_s(u + a) \theta_{1-s}(u - a)}{
                \theta_0^2(u) \theta_s(a)     \theta_{1-s}(a)},  \\
R_3(u) &= \frac{\theta'^2_1(0) \theta_s(u + a) \theta_{3-s}(u - a)}{
                \theta_0^2(u) \theta_{1-s}(a) \theta_{2+s}(a)}.
\end{align*}

Les formules \'el\'ementaires concernant les fonctions $\theta$ donneraient ces
quantit\'es pour chaque valeur de $s$, mais j'y parviendrai par une autre
voie en conservant l'indice variable. Et d'abord, au moyen des relations
\begin{alignat*}{2}
&\theta_s(u + 2K)   &&= (-1)^{\frac{s(s+1)}{2}}    \theta_s(u), \\
&\theta_s(u + 2iK') &&= (-1)^{\frac{(s+1)(s+2)}{2}} \theta_s(u) e^{-\frac{i\pi}{K}(u + iK')} ,
\end{alignat*}
on obtient
\begin{align*}
R_1(u + 2K) &= -R_1(u), & R_1(u + 2iK') &= -R_1(u), \\
R_2(u + 2K) &= -R_2(u), & R_2(u + 2iK') &= +R_2(u), \\
R_3(u + 2K) &= +R_3(u), & R_3(u + 2iK') &= -R_3(u).
\end{align*}
Les fonctions $R_1(u)$, $R_2(u)$, $R_3(u)$ poss\`edent ainsi la m\^eme p\'eriodicit\'e que
$\cn u$, $\sn u$, $\dn u$, par cons\'equent les quantit\'es proportionnelles $U_1$, $U_2$, $U_3$,
ayant le seul p\^ole $u = iK'$ \`a l'int\'erieur du rectangle des p\'eriodes $2K$, $2iK'$,
et pour r\'esidu correspondant l'unit\'e, peuvent servir, \`a leur \'egard, d'\'el\'ements
simples. Employons maintenant l'\'equation
\[
\theta_s(u + iK') = \sigma \theta_{1-s}(u) e^{- \frac{i\pi}{4K}(2u + iK')} ,
\]
o\`u j'ai pos\'e
\[
\sigma = -e^{ -\frac{i\pi}{4} (s+1)(s+2)(2s+1) } ,
\]
et d\'esignons par $\sigma_1$, $\sigma_2$, $\sigma_3$ ce que devient $\sigma$, et, changeant $s$ en $2+s$,
%-----082.png----------------------------
$1-s$, $3-s$, nous trouverons~(\footnote{On d\'emontre facilement qu'on a
\[
\sigma\sigma_1 = -(-1)i^{\frac{s(s-1)}{2}},  \quad
\sigma\sigma_2 =  1,  \quad
\sigma\sigma_3 = -i.
\]}):
\begin{align*}
R_1(iK'+\varepsilon) &= -\sigma\sigma_1
\frac{ \theta_1'^2(0)
       \theta_{1-s}( a+\varepsilon)
       \theta_{3-s}(-a+\varepsilon) }{
       \theta_1^2(\varepsilon)
       \theta_{1-s}(a)
       \theta_{3-s}(a) },
\\
R_2(iK'+\varepsilon) &= -\sigma\sigma_2
\frac{ \theta_1'^2(0)
       \theta_{1-s}( a+\varepsilon)
       \theta_{  s}(-a+\varepsilon) }{
       \theta_1^2(\varepsilon)
       \theta_{1-s}(a)
       \theta_{  s}(a) },
\\
R_3(iK'+\varepsilon) &= -\sigma\sigma_3
\frac{ \theta_1'^2(0)
       \theta_{1-s}( a+\varepsilon)
       \theta_{2+s}(-a+\varepsilon) }{
       \theta_1^2(\varepsilon)
       \theta_{1-s}(a)
       \theta_{2+s}(a) }.
\end{align*}
Cela \'etant, comme on peut introduire \`a volont\'e un facteur constant dans
la fonction $R$, je prends, au lieu des expressions pr\'ec\'edentes, celles-ci, qui
en diff\`erent seulement par le signe ou le facteur $\pm i$, savoir:
\begin{align*}
R_1(iK'+\varepsilon) &=
\frac{ \theta_1'^2(0)
          \theta_{1-s}( a+\varepsilon)
          \theta_{3-s}( a-\varepsilon) }{
          \theta_1^2(\varepsilon)
          \theta_{1-s}(a)
          \theta_{3-s}(a) },
\\
R_2(iK'+\varepsilon) &=
\frac{ \theta_1'^2(0)
       \theta_{1-s}( a+\varepsilon)
       \theta_{  s}( a-\varepsilon) }{
       \theta_1^2(\varepsilon)
       \theta_{1-s}(a)
       \theta_{  s}(a) },
\\
R_3(iK'+\varepsilon) &=
\frac{ \theta_1'^2(0)
       \theta_{1-s}( a+\varepsilon)
       \theta_{2+s}( a-\varepsilon) }{
       \theta_1^2(\varepsilon)
       \theta_{1-s}(a)
       \theta_{2+s}(a) }.
\end{align*}
D\'eveloppant donc suivant les puissances de $\varepsilon$ et faisant usage des quantit\'es
$\delta$, pr\'ec\'edemment introduites, qui donnent:
\begin{alignat*}{2}
\frac{\theta'_{1-s}(a)}{\theta_{1-s}(a)}
&- \frac{\theta'_{3-s}(a)}{\theta_{3-s}(a)} &= 2\delta_1 ,
\\
\frac{\theta'_{1-s}(a)}{\theta_{1-s}(a)}
&- \frac{\theta'_{  s}(a)}{\theta_{  s}(a)} &= 2\delta_2 ,
\\
\frac{\theta'_{1-s}(a)}{\theta_{1-s}(a)}
&- \frac{\theta'_{2+s}(a)}{\theta_{2+s}(a)} &= 2\delta_3 ,
\end{alignat*}
nous obtenons, pour les parties principales, les quantit\'es
\[
\frac{1}{\varepsilon^2} + \frac{2\delta_1}{\varepsilon}, \quad
\frac{1}{\varepsilon^2} + \frac{2\delta_2}{\varepsilon}, \quad
\frac{1}{\varepsilon^2} + \frac{2\delta_3}{\varepsilon},
\]
et l'on en conclut les valeurs suivantes, qu'il s'agissait d'obtenir:
\begin{align*}
R_1(u) &= 2\delta_1 U_1 - D_u U_1 , \\
R_2(u) &= 2\delta_2 U_2 - D_u U_2 , \\
R_3(u) &= 2\delta_3 U_3 - D_u U_3 .
\end{align*}
%-----083.png----------------------------

Ces r\'esultats nous permettent de former les fonctions $\mathfrak P$ et $\mathfrak Q$; mais
pour la deuxi\`eme, le calcul est un peu long, et je me bornerai \`a en
retenir cette conclusion, que dans les trois cas on parvient, en d\'esignant
par $U$ une quantit\'e qui soit successivement $U_1$, $U_2$, $U_3$, \`a des expressions
de cette forme:
\begin{align*}
  \mathfrak{P} &= \alpha U + \alpha' D_u U,  \\
  \mathfrak{Q} &= \beta  U + \beta'  D_u U + \beta'' D_u^2 U,
\end{align*}
o\`u les coefficients $\alpha$ et $\beta$ sont des constantes. Leur complication tient \`a ce
qu'ils sont exprim\'es au moyen des quantit\'es $a$ et $p$ qui figurent explicitement
dans l'int\'egrale, et nous allons voir comment l'introduction d'autres
\'el\'ements conduit \`a des valeurs beaucoup plus simples.


\mysection{XXVIII.}


Soient $U$ et $U_1$ deux fonctions doublement p\'eriodiques de seconde
esp\`ece ayant chacune un p\^ole unique $u = 0$, et repr\'esent\'ees par les
formules
\[
  U  = \frac{H(u+\alpha) e^{pu}}{H(u)} , \qquad
  U_1= \frac{H(u+\beta ) e^{qu}}{H(u)} ;
\]
je me propose de former en g\'en\'eral l'\'equation du second ordre, admettant
pour int\'egrale l'expression
\[
  \mathfrak{X} = CU + C'U_1,
\]
qui est
\[
\begin{vmatrix}
  \mathfrak{X}  & U   & U_1 \\
  \mathfrak{X'} & U'  & U'_1 \\
  \mathfrak{X''}& U'' & U''_1
\end{vmatrix}
= \mathfrak{PX'' - P'X' + Q X} = 0,
\]
en posant
\[
  \mathfrak{P} = UU'_1   - U_1U',  \qquad
  \mathfrak{Q} = U'U''_1 - U'_1U'' .
\]

Nommons pour un moment $\mu$ et $\mu'$ les multiplicateurs de $A$, $\nu$ et $\nu'$
ceux de $B$; on voit d'abord que les coefficients $\mathfrak P$ et $\mathfrak Q$ sont des fonctions
de seconde esp\`ece aux multiplicateurs $\mu\nu$ et $\mu'\nu'$, ayant de m\^eme pour seul
p\^ole $u = 0$, qui est un infini double pour $\mathfrak P$ et un infini triple pour $\mathfrak Q$.
L'\'equation $\mathfrak P = 0$ n'admet ainsi \`a l'int\'erieur du rectangle des p\'eriodes
que deux racines, $u = a$ et $u = b$, et, en d\'ecomposant en \'el\'ements
simples les fonctions de premi\`ere esp\`ece,
$\frac{\mathfrak{P'}}{\mathfrak{P}}$ et
$\frac{\mathfrak{Q}}{\mathfrak{P}}$, on aura les expressions
%-----084.png----------------------------
suivantes:
\begin{alignat*}{4}
\mathfrak{ \frac{P'}{P} } &=  \frac{H'(u-a)}{H(u-a)}  &&+
    \frac{H'(u-b)}{H(u-b)} &&-
    2\frac{H'(u)}  {H(u)}   &&+ \lambda, \\
\mathfrak{ \frac{Q}{P}  } &=  \frac{PH'(u-a)}{H(u-a)} &&+
    \frac{QH'(u-b)}{H(u-b)} &&+
    \frac{RH'(u)}{H(u)}    &&+ S,
\end{alignat*}
o\`u $P$, $Q$, \ldots\ sont des constantes assujetties \`a la condition $P+Q+R=0$.

Les quantit\'es $a$ et $b$, que nous venons d'introduire, repr\'esentent donc,
\`a l'\'egard de l'\'equation diff\'erentielle, des points que M. Weierstrass nomme
\textit{\`a apparence singuli\`ere}, $u = 0$ \'etant seul un point singulier. Ce sont les v\'eritables
\'el\'ements qu'il convient d'employer comme appropri\'es \`a la formation
de l'\'equation diff\'erentielle, au lieu des constantes $\alpha$, $\beta$, $p$, $q$ qui
entrent dans les fonctions $A$ et $B$. Je me fonderai, \`a cet effet, sur le lemme
suivant, qui donnera, par un calcul facile, la d\'etermination des coefficients
$P$, $Q$,~\ldots.

Consid\'erons l'\'equation diff\'erentielle
\[
y'' - f(u)y' + g(u)y = 0,
\]
o\`u les fonctions uniformes $f(u)$, $g(u)$ admettent seulement des infinis
simples qui soient, d'une part, $u = 0$ et, de l'autre, $u=a,b,c,$ \ldots.
Posons d'abord, en d\'eveloppant suivant les puissances croissantes de~$\varepsilon$,
\[
f(\varepsilon) = -\frac{2}{\varepsilon} + F + \ldots, \qquad
g(\varepsilon) =  \frac{G}{\varepsilon} + \ldots,
\]
et en second lieu, pour les diverses quantit\'es $a$, $b$, $c$, \ldots,
\[
f(a + \varepsilon) = \frac{1}  {\varepsilon} + f_a + \ldots, \qquad
g(a + \varepsilon) = \frac{g_a}{\varepsilon} + g_a^1 + \ldots.
\]
Si l'on a, d'une part,
\[
F+G = 0,
\]
puis, pour toutes les quantit\'es $a$, $b$, $c$, \ldots,
\[
g_a^1 = g_a(f_a - g_a),
\]
l'int\'egrale de l'\'equation propos\'ee sera une fonction uniforme ayant pour
seul point singulier $u=0$, et, dans le domaine de ce point, les int\'egrales
nomm\'ees \textit{fondamentales} par M.~Fuchs seront de la forme $\varphi_1(u)$
et $\frac{1}{u} + \varphi_2(u)$, o\`u $\varphi_1(u)$ et $\varphi_2(u)$ repr\'esentent des s\'eries qui proc\`edent suivant
les puissances ascendantes enti\`eres et positives de la variable.
%-----085.png----------------------------


\mysection{XXIX.}


Ce sont ces belles et importantes d\'ecouvertes de M.~Fuchs dans
la th\'eorie g\'en\'erale des \'equations diff\'erentielles lin\'eaires qui permettent
ainsi d'obtenir les conditions n\'ecessaires et suffisantes pour que l'int\'egrale
compl\`ete de l'\'equation consid\'er\'ee soit une fonction uniforme de la variable.
Il n'est pas inutile, \`a l'\'egard de ces conditions, de remarquer
qu'elles se conservent, comme on le v\'erifie ais\'ement, dans les transform\'ees
auxquelles conduit la substitution $y = ze^{-\alpha x}$, \`a savoir
\[
z'' - \left[2\alpha + f(u)\right] z' + \left[\alpha^2 + \alpha f(u) + g(u)\right] z = 0.
\]
J'observe encore que l'on peut supposer doublement p\'eriodiques les
fonctions $f(u)$ et $g(u)$, en convenant que les quantit\'es $u = 0$, $u = a$,
$u = b$, \ldots, au lieu de repr\'esenter tous leurs p\^oles, d\'esigneront seulement
ceux de ces p\^oles qui sont \`a l'int\'erieur du rectangle des p\'eriodes. Soit
donc, en nous pla\c{c}ant dans ce cas,
\[
f(u) = \frac{\mathfrak{P'}}{\mathfrak{P}}, \qquad
g(u) = \frac{\mathfrak{Q}}{\mathfrak{P}},
\]
ou bien, d'apr\`es la remarque qui vient d'\^etre faite,
\[
f(u) = 2\alpha  + = \frac{\mathfrak{P'}}{\mathfrak{P}}, \qquad
g(u) = \alpha^2 + \alpha \frac{\mathfrak{P'}}{\mathfrak{P}} +
\frac{\mathfrak{Q}}{\mathfrak{P}},
\]
$\alpha$ \'etant une constante arbitraire. Je disposerai de cette constante de sorte
qu'on ait
\[
f(u) = \frac{H'(u-a)}{H(u-a)} +
       \frac{H'(u-b)}{H(u-b)} -
       2 \frac{H'(u)}{H(u)}   +
       \frac{\Theta'(a)}{\Theta(a)} +
       \frac{\Theta'(b)}{\Theta(b)},
\]
et par cons\'equent, d'apr\`es les formules connues,
\[
f(u) = \frac{ \sn a }{ \sn u \sn (u-a) } +
       \frac{ \sn b }{ \sn u \sn (u-b) }.
\]
%-----086.png---------------------------
Cela \'etant, il est clair qu'on peut \'ecrire, avec trois ind\'etermin\'ees $A$, $B$, $C$,
\[
g(u) = \frac{ A\sn a }{ \sn u \sn(u-a) } +
       \frac{ B\sn b }{ \sn u \sn(u-b) } + C,
\]
et nous tirerons sur-le-champ de ces expressions les valeurs suivantes:
\begin{align*}
F     &= -\frac{ \cn a \dn a }{ \sn a } - \frac{ \cn b \dn b }{ \sn b },     \\
G     &= -A - B,                                                             \\
f_a   &= -\frac{ \cn a \dn a }{ \sn a } + \frac{ \sn b }{ \sn a \sn (a-b) }, \\
g_a   &= A,                                                                  \\
g_a^1 &= -\frac{ A\cn a \dn a }{ \sn a } + \frac{ B \sn b }{ \sn a \sn (a-b)} + C.
\end{align*}
Or la condition
\[
g_a^1 = g_a(f_a - g_a)
\]
conduit \`a
\[
\frac{ \sn b (A-B) }{ \sn a \sn (a-b) } - A^2 - C = 0;
\]
le second p\^ole $u = b$ donne semblablement
\[
\frac{ \sn a (B-A) }{\sn b \sn (b-a)} - B^2 - C = 0,
\]
et l'on conclut enfin de l'\'equation $F+G=0$
\[
  \frac{ \cn a \dn a }{ \sn a } +
  \frac{ \cn b \dn b }{ \sn b } + A + B = 0.
\]

Je remarque imm\'ediatement que cette derni\`ere relation n'est point distincte
des deux autres et qu'elle en r\'esulte en les retranchant membre \`a
membre et divisant par $A-B$. En l'employant avec la premi\`ere, nous trouvons,
par l'\'elimination de $B$,
\[
A^2 - 2A\frac{ \sn b }{ \sn a \sn(a-b) }
    - \frac{ \sn^2 a - \sn^2 b }{ \sn^2 a \sn^2 (a-b) } + C = 0,
\]
ou encore
\[
\left[ A - \frac{ \sn b }{ \sn a \sn (a-b) } \right]^2
- \frac{1}{ \sn^2 (a-b) } + C = 0.
\]

Rempla\c{c}ant d\'esormais $C$ par $\dfrac{1}{ \sn^2 (a-b) } - C^2$, on voit qu'on aura
\[
A = \frac{ \sn b }{ \sn a \sn (a-b)} + C,
\]
%-----087.png----------------------------
et par cons\'equent
\[
B = \frac{ \sn a }{ \sn b \sn (b-a) } - C.
\]

Telles sont donc, exprim\'ees au moyen de la nouvelle ind\'etermin\'ee $C$,
les valeurs tr\`es-simples des constantes $A$ et $B$ pour lesquelles, d'apr\`es les
principes de M. Fuchs, l'int\'egrale compl\`ete de l'\'equation
\begin{align*}
y''  &- \left[ \frac{ \sn a }{ \sn u \sn (u-a) } +
                \frac{ \sn b }{ \sn u \sn (u-b) } \right] y' \\
     &+ \left[ \frac{ A \sn a }{ \sn u \sn (u-a) } +
                \frac{ B \sn b }{ \sn u \sn (u-b) } +
                \frac{ 1 }{ \sn^2 (a-b) } - C^2 \right] y = 0
\end{align*}
est une fonction uniforme de la variable avec le seul p\^ole $u = 0$.

Nous sommes assur\'es de plus, par une proposition g\'en\'erale de M. Picard
(\textit{Comptes rendus} du 21~juillet 1879, p.~140, et du 19~janvier 1880, p.~128), que
cette int\'egrale s'exprime d\`es lors par deux fonctions doublement p\'eriodiques
de seconde esp\`ece. Si donc on restitue, en faisant la substitution $y = ze^{\alpha x}$,
une constante arbitraire dont il a \'et\'e dispos\'e pour simplifier les calculs,
il est certain que la nouvelle \'equation diff\'erentielle contiendra, comme
cas particuliers, toutes celles dont il a \'et\'e pr\'ec\'edemment question. C'est,
en effet, ce que je ferai bient\^ot voir; mais je veux auparavant obtenir une
confirmation de l'important th\'eor\`eme du jeune g\'eom\`etre en effectuant
directement l'int\'egration de cette \'equation et donner ainsi, avant d'aborder
des cas plus g\'en\'eraux, un nouvel exemple du proc\'ed\'e d\'ej\`a employ\'e pour
l'\'equation de Lam\'e dans le cas le plus simple de $n = 1$.


\mysection{XXX.}\label{page79}


Consid\'erons la fonction doublement p\'eriodique de seconde esp\`ece
la plus g\'en\'erale, admettant pour seul p\^ole $u = 0$, \`a savoir
\[
f(u) = \frac{ H'(0)\Theta(u + \omega) }{ H(u) }\,
  e^{ \left[ \lambda - \frac{\Theta'(\omega)}{\Theta(\omega)} \right] u }
\]
et proposons-nous de d\'eterminer $\omega$ et $\lambda$ de telle sorte qu'elle soit une solution
de l'\'equation propos\'ee. Soit, \`a cet effet, $\Phi(u)$ le r\'esultat de la substitution
de $f(u)$ dans son premier membre. Les coefficients de l'\'equation
ayant pour p\'eriodes $2K$ et $2iK'$, on voit que cette quantit\'e est une fonction
de seconde esp\`ece, ayant les m\^emes multiplicateurs que $f(u)$, qui pourra,
%-----088.png----------------------------
par cons\'equent, remplir \`a son \'egard le r\^ole d'\'el\'ement simple. On voit
aussi que les p\^oles de $\Phi(u)$ sont $u = a$, $u = b$, $u = 0$, les deux premiers
repr\'esentant des infinis simples et le troisi\`eme un infini triple. Nous aurons
donc
\[
\Phi(u) =
  \mathfrak{A}   f(u - a) +
  \mathfrak{B}   f(u - b) +
  \mathfrak{C}   f(u) +
  \mathfrak{C'}  f'(u) +
  \mathfrak{C''} f''(u),
\]
et la condition $\Phi(u) = 0$ entra\^ine ces cinq \'equations
\[
\mathfrak{A}  = 0, \quad
\mathfrak{B}  = 0, \quad
\mathfrak{C}  = 0, \quad
\mathfrak{C'} = 0, \quad
\mathfrak{C''}= 0,
\]
qu'il est ais\'e de former, comme on va voir.

Nous avons pour cela \`a d\'ecomposer en \'el\'ements simples les produits
de $f(u)$ et $f'(u)$ par deux quantit\'es de la m\^eme forme $\frac{ \sn p }{ \sn u \sn (u-p) }$, c'est-\`a-dire
\`a chercher les parties principales des d\'eveloppements de ces produits,
d'abord suivant les puissances de $u$, puis, en posant $u = p+\varepsilon$,
suivant les puissances de $\varepsilon$. Or il r\'esulte de l'expression de $f(u)$ qu'on a
\[
f(u) = \chi( iK' + u ) e^{ \lambda u },
\]
$\chi(u)$ d\'esignant la fonction consid\'er\'ee au \S~V, p.~\pageref{page12},
et par cons\'equent
\begin{align*}
f(u) &= \left[ \frac{1}{u}
  - \frac{1}{2} \left( k^2 \sn^2 \omega - \frac{1 + k^2}{3} \right) u
  + \ldots \right] e^{\lambda u} \\
&= \frac{1}{u} + \lambda
  + \frac{1}{2} \left( \lambda^2 - k^2 \sn^2 \omega
  + \frac{1 + k^2}{3} \right) u + \ldots.
\end{align*}

On trouve ensuite
\[
\frac{ \sn p }{ \sn u \sn (u-p) } =
  -\frac{1}{u} - \frac{ \cn p \dn p }{ \sn p }
  -\left( \frac{1}{ \sn^2 p } - \frac{1 + k^2}{2} \right) u
  + \ldots
\]
et sans nouveau calcul, en rempla\c{c}ant $u$ par $-\varepsilon$,
\[
\frac{ \sn p }{ \sn ( p + \varepsilon ) \sn \varepsilon } =
  \frac{1}{\varepsilon}
  - \frac{ \cn p \dn p }{ \sn p }
  + \left( \frac{1}{ \sn^2 p } - \frac{1 + k^2}{2} \right) \varepsilon
  + \ldots.
\]
Ces d\'eveloppements nous donnent les formules
\begin{multline*}
\frac{ \sn p }{ \sn u \sn (u-p) } f(u)  = f(p)f(u-p)
  - \left( \lambda + \frac{ \cn p \dn p }{ \sn p } \right) f(u)
  + f'(u), \\
\shoveleft{\frac{ \sn p }{ \sn u \sn (u-p) } f'(u) = f'(p) f(u-p) }\\
  {}- \frac{1}{2} \left( \lambda^2 - k^2\sn^2 \omega - \frac{2}{\sn^2 p} + 1 + k^2 \right) f(u)
  -\frac{ \cn p \dn p }{ \sn p } f'(u) + \frac{1}{2} f''(u),
\end{multline*}
%-----089.png-----------------------------
et l'on en conclut, en faisant successivement $p=a$, $p=b$, les expressions
cherch\'ees
\begin{align*}
\mathfrak{A}   &= Af(a) - f'(a), \\
\mathfrak{B}   &= Bf(b) - f'(b), \\
\mathfrak{C}   &= \lambda^2
               - A \left( \lambda + \frac{ \cn a \dn a }{ \sn a } \right)
               - B \left( \lambda + \frac{ \cn b \dn b }{ \sn b } \right)
               - C^2 + \frac{1}{ \sn^2(a-b) } \\
               &\quad + k^2 \sn^2 \omega
                  - \frac{1}{\sn^2 a} - \frac{1}{\sn^2 b} + 1 + k^2, \\
\mathfrak{C'}  &= A + B + \frac{ \cn a \dn a }{ \sn a }
                        + \frac{ \cn b \dn b }{ \sn b }, \\
\mathfrak{C''} &= 0.
\end{align*}
\medskip

Ces r\'esultats obtenus, nous observons d'abord que $\mathfrak{C}'$ s'\'evanouit, d'apr\`es
une des relations trouv\'ees entre $A$ et $B$; j'ajoute que l'\'equation $\mathfrak{C} = 0$
est une cons\'equence des deux premi\`eres; par cons\'equent, les cinq conditions
se r\'eduisent, comme il est n\'ecessaire, \`a deux seulement qui serviront
\`a d\'eterminer $\omega$ et $\lambda$. Nous recourrons, pour l'\'etablir, \`a la transformation
suivante de la valeur de $\mathfrak{C}$. Soit, pour abr\'eger l'\'ecriture,
\begin{align*}
G &= \left( \lambda - C + \frac{\cn b \dn b}{\sn b} \right)
     \left( \lambda + C + \frac{\cn a \dn a}{\sn a} \right), \\
H &= \left(       A - C + \frac{\cn b \dn b}{\sn b} \right)
     \left(       B + C + \frac{\cn a \dn a}{\sn a} \right);
\end{align*}
on a identiquement
\[
\mathfrak{C} = G - H + (A-C)(B+C)
             - k^2\sn^2 \omega + \frac{1}{\sn^2 (a-b)}
             - \frac{1}{\sn^2 a} - \frac{1}{\sn^2 b} + 1 + k^2
\]
et plus simplement d\'ej\`a
\[
\mathfrak{C} = G - H - k^2 \sn^2 \omega - \frac{1}{\sn^2 a} - \frac{1}{\sn^2 b} + 1 + k^2,
\]
les valeurs de $A$ et $B$ que je rappelle,
\[
A = \frac{ \sn b }{ \sn a \sn (a-b) } + C, \qquad
B = \frac{ \sn a }{ \sn b \sn (b-a) } - C,
\]
donnant
\[
(A-C)(B+C) = - \frac{1}{\sn^2 (a-b)}.
\]
Nous obtenons ensuite, en faisant usage de ces expressions,
\begin{align*}
H &= \left[ \tfrac{ \sn b }{ \sn a \sn (a-b) } + \tfrac{ \cn b \dn b }{ \sn b } \right]
     \left[ \tfrac{ \sn a }{ \sn b \sn (b-a) } + \tfrac{ \cn a \dn a }{ \sn a } \right] \\
  &= -\tfrac{1}{\sn^2(a-b)} + \tfrac{1}{\sn (a-b)}
     \left( \tfrac{ \sn b \cn a \dn a }{ \sn^2 a }
           - \tfrac{ \sn a \cn b \dn b }{ \sn^2 b } \right)
     + \tfrac{ \cn a \dn a \cn b \dn b }{ \sn a \sn b }.
\end{align*}
%-----090.png-----------------------------
On a d'ailleurs
\begin{multline*}
\frac{1}{\sn(a-b)} \left( \frac{\sn b \cn a \dn a}{\sn^2 a}
                         - \frac{\sn a \cn b \dn b}{\sn^2 b} \right) \\
\begin{aligned}
&= \left( \frac{ \sn a \cn b \dn b + \sn b \cn a \dn a }{
                  \sn^2 a - \sn^2 b } \right)
   \left( \frac{ \sn^3 b \cn a \dn a - \sn^3 a \cn b \dn b }{
                  \sn^2 a \sn^2 b } \right) \\
&= -\frac{ \sn^2 a + \sn^2 b }{ \sn^2 a \sn^2 b }
   -\frac{ \cn a \dn a \cn b \dn b }{ \sn a \sn b } + 1 + k^2,
\end{aligned}
\end{multline*}
et la valeur de $H$ qui en r\'esulte, \`a savoir
\[
H = -\frac{1}{\sn^2 (a-b)} - \frac{1}{\sn^2 a} - \frac{1}{\sn^2 b} + 1 + k^2,
\]
donne cette nouvelle r\'eduction:
\[
\mathfrak{C} = G - k^2 \sn^2 \omega + \frac{1}{\sn^2 (a-b) }.
\]

C'est maintenant qu'il est n\'ecessaire d'introduire les conditions $\mathfrak{A} = 0$,
$\mathfrak{B} = 0$, c'est-\`a-dire
\[
A = \frac{f'(a)}{f(a)},\qquad
B = \frac{f'(b)}{f(b)}.
\]
Or, au moyen des valeurs de $A$,
de $B$ et de l'expression
\begin{align*}
\frac{f'(x)}{f(x)}
&= \frac{\Theta'(x + \omega)}{\Theta(x+\omega)}
  -\frac{H'(x)}{H(x)} - \frac{\Theta'(\omega)}{\Theta(\omega)} + \lambda \\
&= - k^2 \sn x \sn \omega \sn ( x + \omega ) - \frac{\cn x \dn x}{\sn x} + \lambda,
\end{align*}
on en tire
\begin{alignat*}{3}
\lambda - C &= \frac{\sn b}{\sn a \sn (a-b)} &&+ \frac{\cn a \dn a}{\sn a}
  &&+ k^2 \sn a \sn \omega \sn ( a + \omega ), \\
\lambda + C &= \frac{\sn a}{\sn b \sn (b-a)} &&+ \frac{\cn b \dn b}{\sn b}
  &&+ k^2 \sn b \sn \omega \sn ( b + \omega ).
\end{alignat*}

Cela \'etant, une r\'eduction qui se pr\'esente facilement donne
\begin{alignat*}{3}
\lambda - C &+ \frac{ \cn b \dn b }{\sn b} &&= \frac{\sn a}{\sn b \sn (a-b)}
&&  + k^2 \sn a \sn \omega \sn (a + \omega), \\
\lambda + C &+ \frac{ \cn a \dn a }{\sn a} &&= \frac{\sn b}{\sn a \sn (b-a)}
&&  + k^2 \sn b \sn \omega \sn (b + \omega),
\end{alignat*}
et nous pouvons \'ecrire en cons\'equence
\begin{multline*}
G =  \left[ \frac{ \sn a }{ \sn b \sn (a-b) } + k^2 \sn a \sn \omega \sn (a+\omega) \right]\\
             \times \left[ \frac{ \sn b }{ \sn a \sn (b-a) } + k^2 \sn b \sn \omega \sn (b+\omega) \right].
\end{multline*}
%-----091.png------------------------------

Je consid\'ererai cette expression comme une fonction doublement p\'eriodi\-que
de $\omega$, ayant pour infinis simples $\omega = iK' - a$, $\omega = iK' - b$ et pour
infini double $\omega = iK'$. Elle pr\'esente cette circonstance que les r\'esidus qui
correspondent aux infinis simples sont nuls. En effet, des deux facteurs
dont elle se compose, le premier s'\'evanouit en faisant $\omega = iK' - b$ et le
second pour $\omega = iK' - a$. Il en r\'esulte que le r\'esidu relatif au troisi\`eme
p\^ole $\omega = iK'$ est \'egalement nul, de sorte qu'en d\'ecomposant en \'el\'ements
simples on obtient
\[
G = -D_{\omega} \frac{\Theta'(\omega)}{\Theta(\omega)} + \textrm{const.}
  = k^2 \sn^2 \omega + \textrm{const.}
\]

Posons, afin de d\'eterminer la constante, $\omega = 0$; nous trouverons finalement
\[
G = k^2 \sn^2 \omega - \frac{1}{\sn^2 (a-b)},
\]
et de l\`a r\'esulte, comme il importait essentiellement de le d\'emontrer,
que l'\'equation $\mathfrak{C} = 0$ est une cons\'equence des relations $\mathfrak{A} = 0$ et
$\mathfrak{B} = 0$.


\mysection{XXXI.}


La d\'etermination des constantes $\omega$ et $\lambda$ s'effectue au moyen des deux
\'equations
\begin{align*}
\lambda - C &= \frac{\sn b}{\sn a \sn (a-b)} + \frac{\cn a \dn a}{\sn a}
               + k^2 \sn a \sn \omega \sn (a+\omega), \\
\lambda + C &= \frac{\sn a}{\sn b \sn (b-a)} + \frac{\cn b \dn b}{\sn b}
               + k^2 \sn b \sn \omega \sn (b+\omega),
\end{align*}
que nous avons maintenant \`a traiter. En les retranchant et apr\`es une r\'educ\-tion
qui s'offre facilement, elles donnent d'abord
\begin{multline*}
k^2 \sn \omega \left[ \sn b \sn (b+\omega) - \sn a \sn (a + \omega) \right] \\
-2 \frac{\sn a \cn a \dn a + \sn b \cn b \dn b}{\sn^2 a - \sn^2 b} - 2C = 0,
\end{multline*}
et nous d\'emontrerons imm\'ediatement, le premier membre \'etant une fonction
doublement p\'eriodique, qu'on n'aura, dans le rectangle des p\'eriodes $2K$
et $2iK'$, que deux valeurs pour l'inconnue. En effet, la fonction, qui au
%-----092.png------------------------------
premier abord para\^it avoir les trois p\^oles $\omega = iK' - a$, $\omega = iK' - b$,
$\omega = iK'$, ne poss\`ede en r\'ealit\'e que les deux premiers, le r\'esidu relatif au
troisi\`eme, qui est un infini simple, \'etant nul, comme on le v\'erifie ais\'ement.
Ce point \'etabli, nous donnerons, pour \'eviter des longueurs de calcul, une
autre forme \`a l'\'equation, en employant l'identit\'e suivante,
\begin{multline*}
\sn b \sn (b + \omega) - \sn a \sn(a + \omega) \\
   = \sn (b - a) \sn (a + b + \omega)
     \left[ 1 - k^2 \sn a \sn b \sn (a + \omega) \sn (b + \omega) \right],
\end{multline*}
\`a laquelle je m'arr\^ete un moment. Elle est la cons\'equence imm\'ediate de la
relation m\'emorable obtenue par Jacobi, dans un article intitul\'e \textit{Formul{\ae}
nov{\ae} in theoria transcendentium ellipticarum fundamentales} (\textit{Journal de Crelle},
t.~XV, p.~201), \`a savoir
\begin{multline*}
E(u) + E(a) + E(b) - E(u + a + b) \\
  = k^2 \sn(u+a) \sn(u+b) \sn(a+b)
    \left[1 - k^2 \sn u \sn a \sn b \sn (u+a+b) \right].
\end{multline*}

Qu'on change en effet $a$ en $- a$, puis $u$ en $a + \omega$, on aura
\begin{multline*}
E(a+\omega) - E(a) + E(b) - E(b+\omega) \\
   = k^2 \sn \omega \sn(b-a) \sn(a+b+\omega)
     \left[1 - k^2 \sn a \sn b \sn(a+\omega) \sn(b+\omega) \right],
\end{multline*}
et il suffit de remarquer que le premier membre, \'etant la diff\'erence des
quantit\'es $E(a + \omega) - E(a) - E(\omega)$, $E(b + \omega) - E(b) - E(\omega)$, peut \^etre
remplac\'e par $k^2 \sn \omega \left[\sn b \sn(b + \omega) - \sn a \sn(a + \omega)\right]$.
\medskip

On y parvient encore d'une autre mani\`ere au moyen de la relation
pr\'ec\'edemment d\'emontr\'ee,
\begin{multline*}
G =      \left[ \frac{\sn a}{\sn b \sn(a-b)} + k^2 \sn a \sn \omega \sn(a+\omega) \right] \\
  \times \left[ \frac{\sn b}{\sn a \sn(b-a)} + k^2 \sn b \sn \omega \sn(b+\omega) \right]
   = k^2 \sn^2 \omega - \frac{1}{\sn^2 (a-b)},
\end{multline*}
car on en tire
\begin{multline*}
\sn b \sn(a+\omega) - \sn a \sn(b+\omega) \\
  = \sn \omega \sn(b-a) \left[ 1 - k^2 \sn a \sn b \sn(a+\omega) \sn(b+\omega) \right],
\end{multline*}
ce qui donne la formule propos\'ee en changeant $a$ en $-a$, $b$ en $-b$ et $\omega$
en $\omega + a + b$.
%-----093.png---------------------------

Cela pos\'e, soit $\upsilon=\omega+\frac{a+b}{2}$; faisons aussi, pour abr\'eger, $\alpha=\frac{a+b}{2}$,
$\beta=\frac{a-b}{2}$; nous trouverons, par cette formule,
\begin{multline*}
\sn\omega \left[\sn b\sn(b+\omega)-\sn a\sn(a+\omega) \right] =
{} -\sn 2\beta\sn(\upsilon+\alpha)\sn(\upsilon-\alpha) \\\times
  \left[1-k^2\sn(\alpha+\beta)\sn(\alpha-\beta)\sn(\upsilon+\beta)
      \sn(\upsilon-\beta) \right].
\end{multline*}

Or on voit que le second membre devient ainsi une fonction rationnelle
de $\sn^2\upsilon$; on peut, en outre, supprimer au num\'erateur et au d\'enominateur
le facteur $1-k^2\sn^2\upsilon\sn^2\alpha$, de sorte qu'il se r\'eduit \`a l'expression
\[
-\frac{\sn2\beta(1-k^2\sn^4\beta)(\sn^2\upsilon-\sn^2\alpha)}{
  (1-k^2\sn^2\alpha\sn^2\beta)(1-k^2\sn^2\upsilon\sn^2\beta)}.
\]

Remarquant encore que l'on a
\[
\sn2\beta(1-k^2\sn^4\beta)=2\sn\beta\cn\beta\dn\beta,
\]
nous poserons, pour simplifier l'\'ecriture,
\[
L=\frac{1-k^2\sn^2\alpha\sn^2\beta}{k^2\sn\beta\cn\beta\dn\beta}
  \left(\frac{\sn a\cn a\dn a+\sn b\cn b\dn b}{\sn^2a-\sn^2b}+C\right),
\]
et l'\'equation en $\sn\upsilon$ sera simplement
\[
\frac{\sn^2\upsilon-\sn^2\alpha}{1-k^2\sn^2\upsilon\sn^2\beta}=-L.
\]
On en tire
\begin{gather*}
\sn^2\upsilon=\frac{\sn^2\alpha-L}{1-k^2\sn^2\beta L},\qquad
  \cn^2\upsilon=\frac{\cn^2\alpha+\dn^2\beta L}{1-k^2\sn^2\beta L},\\
  \dn^2\upsilon=\frac{\dn^2\alpha+k^2\cn^2\beta L}{1-k^2\sn^2\beta L},
\end{gather*}
et, si l'on fait
\[
\mathfrak{L}= \left(\sn^2\alpha-L \right)
  \left(\cn^2\alpha+\dn^2\beta L \right)
  \left(\dn^2\alpha+k^2\cn^2\beta L \right)
  \left(1-k^2\sn^2\beta L \right),
\]
ces valeurs donnent
\[
\sn\upsilon\cn\upsilon\dn\upsilon =\frac{\sqrt{\mathfrak{L}}}{(1-k^2\sn^2\beta L)^2}.
\]

Nous ferons usage de cette expression pour le calcul de $\lambda$, qui nous
reste \`a d\'eterminer. A cet effet je reprends, pour les ajouter membre \`a
membre, les \'equations
\begin{alignat*}{3}
\lambda-C &= \frac{\sn b}{\sn a\sn(a-b)} &&+\frac{\cn a\dn a}{\sn a}
  &&+ k^2\sn a\sn\omega\sn(a+\omega),\\
\lambda+C &= \frac{\sn a}{\sn b\sn(b-a)} &&+\frac{\cn b\dn b}{\sn b}
  &&+ k^2\sn b\sn\omega\sn(b+\omega),
\end{alignat*}
%-----094.png------------------------------
et j'obtiens, comme on le voit facilement,
\[
2\lambda = k^2 \left[\sn a \sn \omega \sn(a+\omega) + \sn b \sn \omega \sn(b+\omega) \right],
\]
ou bien encore
\[
2\lambda = k^2 \left[\sn(\alpha+\beta) \sn(\upsilon-\alpha) \sn(\upsilon+\beta)
              +\sn(\alpha-\beta) \sn(\upsilon-\alpha) \sn(\upsilon-\beta) \right].
\]

Maintenant, un calcul sans difficult\'e donne en premier lieu l'expression
\begin{multline*}
\lambda
= \frac{\sn\alpha \cn\alpha \dn\alpha \left(\sn^2 \upsilon - \sn^2 \beta \right)}{
    \left(1 - k^2 \sn^2 \upsilon \sn^2 \alpha \right)
    \left(1 - k^2 \sn^2 \alpha   \sn^2 \beta \right)} \\
+ \frac{\sn\upsilon\cn\upsilon\dn\upsilon
    \left(\sn^2\beta - \sn^2\alpha \right)}{
    \left(1 - k^2 \sn^2 \upsilon \sn^2 \alpha \right)
    \left(1 - k^2 \sn^2 \upsilon \sn^2 \beta \right)};
\end{multline*}
on en conclut ensuite la valeur cherch\'ee, \`a savoir
\begin{multline*}
\lambda = \frac{k^2 \sn \alpha \cn \alpha \dn \alpha \left[\sn^2 \alpha - \sn^2 \beta - (1 - k^2 \sn^4 \beta)L\right]}{
                \left(1-k^2 \sn^2 \alpha \sn^2 \beta\right) \left[1 - k^2 \sn^4 \alpha + k^2 \left(\sn^2 \alpha - \sn^2 \beta\right)L\right]} \\
        + \frac{k^2 \left(\sn^2 \beta - \sn^2 \alpha \right) \sqrt{\mathfrak{L}}}{
                \left(1-k^2 \sn^2 \alpha \sn^2 \beta\right) \left[1 - k^2 \sn^4 \alpha + k^2 (\sn^2 \alpha - \sn^2 \beta)L \right]}.
\end{multline*}

Cette expression devient illusoire lorsqu'on suppose d'abord
\[
1 - k^2 \sn^2 \alpha \sn^2 \beta = 0,
\]
c'est-\`a-dire $\alpha + \beta = a = iK'$ ou bien $\alpha - \beta = b = iK'$,
puis en faisant
\[
1 - k^2 \sn^4 \alpha + k^2 (\sn^2 \alpha - \sn^2 \beta)L = 0.
\]

La premi\`ere condition, ayant pour effet de rendre infinis les coefficients
de l'\'equation diff\'erentielle, doit \^etre \'ecart\'ee; mais la seconde appelle l'attention,
et je m'y arr\^eterai un moment, afin d'obtenir la nouvelle forme
analytique que prend l'int\'egrale dans ce cas singulier.


\mysection{XXXII.}


Remarquons en premier lieu que cette condition se trouve en posant
\[
\sn^2 \upsilon = \frac{\sn^2 \alpha - L}{1 - k^2 \sn^2 \beta L}
  = \frac{1}{ k^2 \sn^2 \alpha},
\]
c'est-\`a-dire $\upsilon = \alpha + iK'$, et donne par cons\'equent $\omega = iK'$. Cela \'etant,
je fais dans la solution de l'int\'egrale, qui est repr\'esent\'ee par la formule
\[
\frac{\Theta(u + \omega)}{H(u)}\,
e^{\left[\lambda - \frac{\Theta'(\omega)}{\Theta(\omega)}\right]u},
\]
$\omega = iK' + \varepsilon$, $\varepsilon$ \'etant infiniment petit, et je d\'eveloppe
suivant les puissances croissantes de $\varepsilon$ la diff\'erence $\lambda - \frac{\Theta'(\omega)}{\Theta(\omega)}$. Or l'expression
%-----095.png------------------------------
pr\'ec\'edemment employ\'ee
\[
2 \lambda = k^2 \left[ \sn a \sn \omega \sn(a + \omega) + \sn b \sn \omega \sn(b+\omega) \right]
\]
donne facilement
\[
\lambda = \frac{1}{\varepsilon} - \frac{\cn a \dn a}{2 \sn a} - \frac{\cn b \dn b}{2 \sn b} + \ldots;
\]
nous avons d'ailleurs
\[
\frac{\Theta'(\omega)}{\Theta (\omega)}
  = \frac{H'(\varepsilon)}{H (\varepsilon)} - \frac{i\pi}{2K}
  = \frac{1}{\varepsilon} - \frac{i\pi}{2K} + \ldots,
\]
et l'on en conclut, pour $\varepsilon = 0$, la limite finie
\[
\lambda - \frac{\Theta'(\omega)}{\Theta(\omega)}
  = \frac{i\pi}{2K} - \frac{\cn a\dn a}{2 \sn a} - \frac{\cn b \dn b}{2 \sn b}.
\]
Rempla\c{c}ant donc $\Theta(u + iK')$ par $iH(u)e^{-\frac{i\pi}{4K}(2u+iK')}$, on voit qu'au lieu de
la fonction doublement p\'eriodique de seconde esp\`ece nous obtenons l'exponentielle
\[
e^{-\left( \tfrac{\cn a \dn a}{2\sn a} + \tfrac{\cn b\dn b}{2\sn b}\right) u},
\]
qui devient ainsi une des solutions de l'\'equation
diff\'erentielle. Nous parvenons \`a l'autre solution en employant, au
lieu de $\upsilon = \alpha + iK'$, la valeur \'egale et de signe contraire $\upsilon = -\alpha - iK'$,
d'o\`u l'on tire $\omega = -2\alpha - iK' = -a-b-iK'$, et par cons\'equent
\[
\lambda = \frac{\sn^2 a + \sn^2 b}{2\sn (a+b)\sn a\sn b}, \qquad
\frac{\Theta'(\omega)}{\Theta(\omega)} = -\frac{H'(a + b)}{H(a + b)} + \frac{i\pi}{2K}.
\]

Des r\'eductions qui s'offrent d'elles-m\^emes en employant la formule
\[
\frac{H'(a + b)}{H(a + b)} = \frac{H'(a)}{H(a)}
                           + \frac{H'(b)}{H(b)}
                           - \frac{\sn b}{\sn a \sn(a+b)}
                           - \frac{\cn b\dn b}{\sn b}
\]
donnent ensuite
\[
\lambda - \frac{\Theta'(\omega)}{\Theta(\omega)}
  = \frac{H'(a)}{H(a)} + \frac{H'(b)}{H(b)}
  - \frac{\cn a \dn a}{2\sn a} - \frac{\cn b \dn b}{2\sn b}
  + \frac{i\omega}{2K}.
\]

La seconde int\'egrale devient donc
\[
\frac{H(u - a - b)}{H(a)}\,
  e^{\textstyle\left[ \frac{H'(a)}{H(a)} + \frac{H'(b)}{H(b)} -
    \frac{\cn a \dn a}{2\sn a} - \frac{\cn b \dn b}{2\sn b}\right] u},
\]
et l'on voit que, pour le cas singulier consid\'er\'e, la solution g\'en\'erale est
repr\'esent\'ee par la relation suivante:
\[
y\, e^{ \textstyle\left( \frac{\cn a \dn a}{2\sn a} + \frac{\cn b\dn b}{2\sn b} \right) u}
  = C + C'\frac{H(u-a-b)}{H(u)} \,
    e^{\textstyle\left[ \frac{H'(a)}{H(a)} + \frac{H'(b)}{H(b)}\right] u}.
\]
%-----096.png-------------------------------


\mysection{XXXIII.}


Un dernier point me reste maintenant \`a traiter; j'ai encore \`a montrer
comment les \'equations diff\'erentielles obtenues aux \S\S~XVII et XVIII se
tirent comme cas particulier de l'\'equation que nous venons de consid\'erer,
ou plut\^ot de celle qui en r\'esulte si l'on change $u$ en $u + iK'$, \`a savoir,
\[
\begin{split}
y'' &- \left[k^2 \sn u \sn a \sn (u-a) + k^2 \sn u \sn b \sn(u-b) \right]y' \\
    &+\left[Ak^2 \sn u \sn a \sn (u-a) \vphantom{\frac{1}{\sn^2(a-b)}} \right. \\
    & \left. \qquad {} + Bk^2 \sn u \sn b \sn (u-b) + \frac{1}{\sn^2(a-b)} - C^2 \right] y = 0.
\end{split}
\]

Je me fonde, \`a cet effet, sur ce que les deux d\'eterminations de la
quantit\'e $\upsilon = \omega + \frac{a+b}{2}$ peuvent \^etre suppos\'ees \'egales et de signes contraires,
de sorte que, en d\'esignant par $\omega$ et $\omega'$ les valeurs correspondantes
de $\omega$, on a la condition $\omega + \omega' = -a-b$. Qu'on se reporte maintenant
aux expressions donn\'ees au \S~XXV, p.~\pageref{page69}:
\begin{alignat*}{2}
X_1 &= \frac{C \theta_s    (u+a)}{\theta_0(u)}\, e^{-\frac{u}{2}D_a \log \theta_{1-s} \theta_{3-s}}
&&     + \frac{C'\theta_{2+s}(u-a)}{\theta_0(u)}\, e^{ \frac{u}{2}D_a \log \theta_{1-s} \theta_{3-s}}, \\
X_2 &= \frac{C \theta_s    (u+a)}{\theta_0(u)}\, e^{-\frac{u}{2}D_a \log \theta_s     \theta_{1-s}}
&&     + \frac{C'\theta_{1-s}(u-a)}{\theta_0(u)}\, e^{ \frac{u}{2}D_a \log \theta_s     \theta_{1-s}}, \\
X_3 &= \frac{C \theta_s    (u+a)}{\theta_0(u)}\, e^{-\frac{u}{2}D_a \log \theta_s     \theta_{2+s}}
&&     + \frac{C'\theta_{3-s}(u-a)}{\theta_0(u)}\, e^{ \frac{u}{2}D_a \log \theta_s     \theta_{2+s}}.
\end{alignat*}
On voit ais\'ement que les quantit\'es qui jouent le r\^ole des constantes $\omega$ et $\omega'$
ont pour somme, successivement, $K + iK'$, $iK'$, $K$. C'est, en effet, la cons\'equence
des relations d\'ej\`a remarqu\'ees:
\begin{alignat*}{2}
\theta_s(u + iK') &= \sigma  &&\theta_{1-s}(u)\,
  e^{-\frac{i\pi}{4K}(2u+iK')},\\
\theta_s(u +  K ) &= \sigma' &&\theta_{3-s}(u)\,
  e^{-\frac{i\pi}{4K}(2u+iK')},\\
\theta_s(u+K+iK') &= \sigma'' &&\theta_{2+s}(u)\,
  e^{-\frac{i\pi}{4K}(2u+iK')}.
\end{alignat*}

D'apr\`es cela, je ferai successivement $a + b = K + iK'$, $iK'$, $K$; je poserai
%-----097.png-------------------------------
en outre, en changeant d'inconnue dans ces divers cas,
\[
y =
  ze^{-\frac{u}{2}D_a \log \cn a}, \quad
  ze^{-\frac{u}{2}D_a \log \sn a}, \quad
  ze^{-\frac{u}{2}D_a \log \dn a}.
\]
Or, en consid\'erant, pour abr\'eger, seulement le premier de ces cas,
voici le calcul et le r\'esultat auquel il conduit. La condition suppos\'ee
$b = K + iK' -a$ donne d'abord
\[
\sn     b =  \frac{\dn a}{k\cn a}    , \quad
\sn (u-b) = -\frac{\dn(u+a)}{k\cn (u+a)}, \quad
\sn (a-b) = -\frac{\dn 2a}{k\cn 2a},
\]
et nous obtenons, pour la transform\'ee en $z$, l'\'equation suivante,
\[
\begin{split}
z'' &- \left[ k^2 \sn u \sn a \sn(u-a)
              - \frac{\sn u\dn u\dn(u+a)}{\cn a\cn(u+a)}
              - \frac{\sn a\dn a}{\cn a} \right] z' \\
    &+ \left[ Pk^2 \sn u \sn a \sn(u-a)
              - Q\frac{\sn u\dn a\dn(u+a)}{\cn a\cn(u+a)}
              + R \right] z = 0,
\end{split}
\]
o\`u j'ai fait, pour abr\'eger,
\begin{gather*}
P = A - \frac{\sn a\dn a}{2\cn a}, \qquad
Q = B - \frac{\sn a\dn a}{2\cn a}, \\
R = \frac{\sn^2 a \dn^2 a}{4\cn^2 a} + \frac{k^2\cn^2 2a}{\dn^2 2a} - C^2.
\end{gather*}

Soit maintenant
\[
\mathfrak{P} = \cn(u+a) \left( 1-k^2\sn^2 u \sn^2 a \right)
  = \cn a \cn u - \sn a \dn a \sn u \dn u,
\]
on trouvera d'abord que le coefficient de $z'$ est simplement $D_u \log \mathfrak{P} = \frac{\mathfrak{P'}}{\mathfrak{P}}$.

Repr\'esentons ensuite par $\frac{\mathfrak{Q}}{\mathfrak{P}}$ le coefficient de $z$; au moyen de la formule
\'el\'ementaire
\[
\sn(u-a)\cn(u+a) = \frac{\sn u\cn u\dn a - \dn u\sn a\cn a}
                        {1 - k^2\sn^2 u\sn^2 a},
\]
nous obtiendrons
\[
\begin{split}
\mathfrak{Q} &= P k^2 \sn u \sn a \left(\sn u \cn u \dn a - \dn u \sn a \cn a \right) \\
             &\qquad {} - Q\frac{\sn a\dn a}{\cn a} \left(\dn u\dn a - k^2\sn u\cn u\sn a\cn a \right)\\
             &\qquad {} + R \left(\cn u \cn a - \sn u \dn u \sn a \dn a \right),
\end{split}
\]
ou bien, en r\'eunissant les termes semblables,
\begin{multline*}
\mathfrak{Q} = (P+Q) k^2 \sn a \dn a \sn^2 u \cn u \\
     -\left( Pk^2 \sn^2 a \cn a + Q\frac{\sn a}{\cn a} + R\sn a\dn a \right)
        \sn u \dn u + R \cn a \cn u.
\end{multline*}
%-----098.png-------------------------------

Soit maintenant $C = \delta - \frac{\sn a\dn a}{2\cn a}$, cette nouvelle forme de la constante
donnera, apr\`es quelques r\'eductions,
\begin{multline*}
\mathfrak{Q} = -k^2\cn a \sn^2 u \cn u
  +\left[ \sn a \cn a \, \delta^2  \vphantom{\frac{k^2}{\dn^2}} \right.\\
  \left. + \cn a \left(1 - 2k^2\sn^2 a\right) \delta  + k^2 \sn^3 a \dn a
    - \frac{ k^2 \cn^2 2a }{\dn^2 2a} \sn a \dn a \right] \sn u \dn u \\
  -\left[ \cn a \, \delta^2 - \sn a \dn a \, \delta
         - \frac{k^2 \cn^2 2a}{\dn^2 2a}\cn a \right] \cn u.
\end{multline*}

Or, en faisant successivement $a = 0$, puis $a = k$, on tire de l\`a les
\'equations
\begin{gather*}
\cn u \, z'' - D_u \cn u \, z' - \left[k^2 \sn^2 u \cn u - \sn u \dn u \, \delta
  + (\delta^2 - k^2)\cn u\right] z = 0, \\
\sn u \dn u \, z'' - D_u \sn u \dn u \, z'
  - \left[ \cn u \, \delta + \sn u \dn u \, \delta^2 \right] z = 0;
\end{gather*}
ce sont pr\'ecis\'ement les relations en $X_1$ et $Y_1$ des \S\S~XXV et XXVI, en supposant
dans la premi\`ere $\delta = \delta_1$ et dans la seconde $\delta = -\delta'_1$.


\mysection{XXXIV.}


Les fonctions doublement p\'eriodiques de seconde esp\`ece avec un
p\^ole simple, qu'on pourrait nommer \textit{unipolaires}, donnent, comme nous
l'avons vu, la solution d\'ecouverte par Jacobi du probl\`eme de la rotation
d'un corps autour d'un point fixe, lorsqu'il n'y a point de forces acc\'el\'eratrices.
Ces m\^emes quantit\'es s'offrent encore dans une autre question
m\'ecanique importante, la recherche de la figure d'\'equilibre d'un ressort
soumis \`a des forces quelconques, que je vais traiter succinctement. On sait
que Binet a r\'eussi le premier \`a ramener aux quadratures l'expression des
coordonn\'ees de l'\'elastique, dans le cas le plus g\'en\'eral o\`u la courbe est \`a
double courbure (\textit{Comptes rendus}, t.~XVIII, p.~1115, et t.~XIX, p.~1). Son
analyse et ses r\'esultats ont \'et\'e imm\'ediatement beaucoup simplifi\'es par
Wantzel~(\footnote{
Wantzel, enlev\'e \`a la Science par une mort pr\'ematur\'ee \`a l'\^age de trente-sept ans,
en 1849, a laiss\'e d'excellents travaux, parmi lesquels un M\'emoire extr\^emement remarquable
sur les nombres incommensurables, publi\'e dans le \textit{Journal de l'\'Ecole Polytechnique}
(t.~XV, p.~151), et une Note sur l'int\'egration des \'equations de la courbe \'elastique
\`a double courbure (\textit{Comptes rendus}, t.~XVIII, p.~1197).
}), et j'adopterai la marche de l'\'eminent g\'eom\`etre en me proposant
%-----099.png-------------------------------
de conduire la question \`a son terme et d'obtenir explicitement les
coordonn\'ees de la courbe en fonction de l'arc. Mais d'abord je crois devoir
consid\'erer le cas particulier o\`u l'\'elastique est suppos\'ee plane et o\`u l'on a,
en d\'esignant l'arc par $s$ (\textit{M\'ecanique} de Poisson, t.~I, p.~598),
\[
ds = \frac{2c^2 \: dx}{\sqrt{4c^4 - (2ax - x^2)^2}}, \qquad
dy = \frac{(2ax - x^2)\: dx}{\sqrt{4c^4 - (2ax - x^2)^2}}.
\]

Soit alors
\[
x   = a - \sqrt{2c^2 + a^2}\sqrt{1 - X^2}, \qquad
k^2 = \frac{1}{2} + \frac{a^2}{4c^2},
\]
on obtient facilement
\[
ds = \frac{ c \: dX }{ \sqrt{ (1 - X^2)(1 - k^2 X^2) } },
\]
de sorte qu'on peut prendre $X = \sn \left( \frac{s-s_0}{c} \right)$, $s_0$ \'etant une constante arbitraire.
Mais il est pr\'ef\'erable de faire $X = \sn \left(\frac{s-s_0}{c} + K \right)$; nous parviendrons
ainsi \`a des expressions mieux appropri\'ees au cas important qui a \'et\'e
consid\'er\'e par Poisson, o\`u $c$ est suppos\'e une ligne dont la longueur est
tr\`es grande par rapport \`a $a$, $s$ et $x$. En premier lieu, les formules
\[
\cn(z+K) = -k' \frac{\sn z}{\dn z}, \qquad
k^2 = \frac{1}{2} - \frac{a^2}{4c^2}
\]
donnent, pour l'abscisse,
\[
x = a + \frac{ \sqrt{4c^4-a^4} }{2c}
  \frac{ \sn \left( \frac{s-s_0}{c} \right) }{ \dn \left( \frac{s-s_0}{c} \right) } .
\]

La valeur de l'ordonn\'ee, \`a savoir
\[
2c^2y = \int \left(2ax - x^2 \right) ds
      = \int \left[a^2 - (2c^2+a^2)\cn^2 \left( \tfrac{s-s_0}{c} + K \right) \right] ds,
\]
s'obtient ensuite imm\'ediatement en employant la relation
\[
\int^z_0 k^2 \cn^2 (z+K)\, dz = k^2 z + D_z \log \Al(z)_3.
\]

Or ces formules conduisent comme il suit aux d\'eveloppements de $x$
et $y$ suivant les puissances d\'ecroissantes de $c$. J'emploie \`a cet effet la s\'erie
\[
\frac{\sn z}{\dn z} = z + \frac{k^2 - k'^2}{6}z^3 + \frac{1 - 16k^2 k'^2}{120}z^5 + \ldots,
\]
et je remarque qu'en d\'esignant par $F_n(k)$ le coefficient de $z^{2n+1}$, qui est un
%-----100.png-------------------
polyn\^ome de degr\'e $n$ en $k^2$, on a la relation suivante:
\[
F_n(k') = (-1)^n F_n(k).
\]
Nous en concluons facilement pour $n$ pair l'expression
\[
F_n(k) = \alpha_0 + \alpha_1 (kk')^2 + \alpha_2 (kk')^4 + \ldots + \alpha_{\frac{1}{2}n} (kk')^n,
\]
et pour $n$ impair,
\[
F_n(k) = (k^2 - k'^2) \left[ \beta_0 + \beta_1 (kk')^2 + \ldots + \beta_{\frac{n-1}{2}} (kk')^{n-1} \right] .
\]
Cela \'etant, les formules
\[
k^2 k'^2 = \frac{1}{4} - \frac{a^4}{16c^4} \quad \text{et} \quad
     k^2 - k'^2 = \frac{a^2}{2c^2}
\]
montrent que le terme g\'en\'eral $F_n(k)z^{2n+1}$, qui est de l'ordre $\frac{1}{c^{2n+1}}$, lorsqu'on
remplace $z$ par $\frac{s-s_0}{c}$, devient, si l'on suppose $n$ impair, de l'ordre
$\frac{1}{c^{2n+3}}$. Nous pourrons donc \'ecrire, en n\'egligeant $\frac{1}{c^3}$ dans la parenth\`ese,
\[
x = a + \frac{\sqrt{4c^4 - a^4}}{2c^2}
  \left[s-c+\frac{a^2(s-c)^3}{12c^4} %sic
   - \frac{(s-c)^5}{40c^4} \right].
\]
Rempla\c{c}ons enfin le facteur $\frac{\sqrt{4c^4-a^4}}{2c^2}$ par $1 - \frac{a^4}{8c^4}$, et prenons $s_0 = a$; il
viendra, avec le m\^eme ordre d'approximation,
\[
x = s - \frac{s-a}{120c^4} \left[ 3 (s-a)^4 - 10a^2(s-a)^2 + 15a^4 \right].
\]
Le d\'eveloppement de $c^2 y$ r\'esulte ensuite de l'\'equation
\begin{multline*}
\int_0^z k^2 \cn^2(z+K) \,dz \\
  = \frac{k^2 k'^2}{3}z^3
    + \frac{k^2 k'^2(k^2-k'^2)}{3 \cntrdot 5}z^5
    + \frac{k^2 k'^2 (2 - 17k^2 k'^2)}{3 \cntrdot 5 \cntrdot 7} z^7
    + \ldots;
\end{multline*}
mettant $\frac{s-a}{c}$ au lieu de $z$ et d\'eterminant la constante amen\'ee par l'int\'egra\-tion
de mani\`ere qu'on ait $y = 0$ pour $s = a$, on en tire, par un calcul
facile,
\[
2 c^2 y = as^2 - s^3 + \frac{(s-a)^3}{420c^4}
\left[ 9(s-a)^4 - 14a^2(s-a)^2 + 140a^4 \right].
\]
Le second membre, dans cette expression de l'ordonn\'ee, est exact aux
termes pr\`es de l'ordre $\frac{1}{c^3}$, comme la valeur trouv\'ee pour l'abscisse.
%-----101.png------------------


\mysection{XXXV.}


Les \'equations diff\'erentielles de l'\'elastique, dans le cas le plus g\'en\'eral
o\`u la courbe est \`a double courbure, se ram\`enent par un choix convenable
de coordonn\'ees, comme l'a remarqu\'e Wantzel, \`a la forme suivante,
\begin{align*}
y'z'' - y''z' &= \alpha x' + \beta y, \\
z'x'' - z''x' &= \alpha y' - \beta x, \\
x'y'' - x''y' &= \alpha z' + \gamma,
\end{align*}
o\`u $x'$, $y'$, $z'$, $x''$, $y''$, $z''$ d\'esignent les d\'eriv\'ees par rapport \`a l'arc $s$ de
$x$, $y$, $z$ et $\alpha$, $\beta$, $\gamma$ des constantes dont les deux premi\`eres sont essentiellement
positives.

Cela \'etant, j'observerai en premier lieu que, si on les ajoute apr\`es les
avoir multipli\'ees respectivement, d'abord par $x'$, $y'$, $z'$, puis par $x''$, $y''$, $z''$,
on obtient
\begin{align*}
\alpha(x'^2  + y'^2  + z'^2)  + \beta(x'y  - xy')  + \gamma z' &= 0, \\
\alpha(x'x'' + y'y'' + z'z'') + \beta(x''y - xy'') + \gamma z''&= 0.
\end{align*}
Or la premi\`ere de ces relations donne, par la diff\'erentiation,
\[
2\alpha(x'x'' + y'y'' + z'z'') + \beta(x''y - xy'') + \gamma z'' = 0;
\]
nous avons donc
\[
x'x'' + y'y'' + z'z'' = 0,
\]
d'o\`u
\[
x'^2 + y'^2 + z'^2 = \textrm{const.},
\]
et l'on voit que, en prenant la constante \'egale \`a l'unit\'e, on satisfera \`a la
condition que l'arc $s$ soit, comme on l'a admis, la variable ind\'ependante.

Cela pos\'e, et apr\`es avoir \'ecrit les \'equations pr\'ec\'edentes de cette mani\`ere,
\[
\beta(xy' - x'y) = \gamma z' + \alpha, \qquad
\beta(xy''-x''y) = \gamma z'',
\]
j'en d\'eduis
\[
\beta \left[ (xy'-x'y)z'' - (xy'' - x''y)z' \right] = \alpha z'';
\]
mais le premier membre, \'etant \'ecrit ainsi,
\[
\beta \left[ (y'z'' - y''z')x + (z'x'' - z''x')y \right],
\]
se r\'eduit \`a
\[
\beta \left[ (\alpha x' + \beta y)x + (\alpha y' - \beta x)y \right] =
   \alpha \beta (xx' + yy'),
\]
%-----102.png------------------
de sorte que nous avons
\[
\beta(xx' + yy') = z'',
\]
puis par l'int\'egration, en d\'esignant par $\delta$ une constante arbitraire,
\[
\beta(x^2 + y^2) = 2(z' - \delta).
\]

Soit maintenant $z' = \zeta$; nous remplacerons le syst\`eme des \'equations \`a
int\'egrer par celles-ci:
\begin{align*}
\beta(x^2 + y^2) &= 2(\zeta - \delta), \\
\beta(xx' + yy') &= \zeta', \\
     x'^2 + y'^2 &= 1 - \zeta^2, \\
\beta(xy' - x'y) &= \gamma\zeta + \alpha.
\end{align*}
Or l'identit\'e
\[
(x^2 + y^2)(x'^2 + y'^2) = (xx' + yy')^2 + (xy' - x'y)^2
\]
donne en premier lieu
\[
\zeta'^2 = 2\beta(\zeta - \delta)(1 - \zeta^2) - (\gamma\zeta + \alpha)^2,
\]
et l'on trouve ensuite facilement
\[
\frac{x' + iy'}{x + iy} = \frac{\zeta' + i(\gamma\zeta + \alpha)}
                               {2(\zeta - \delta)};
\]
ces r\'esultats obtenus, les expressions des coordonn\'ees en fonction de l'arc
s'en d\'eduisent comme il suit.

Soient $a$, $b$, $c$ les racines de l'\'equation
\[
2 \beta (\zeta - \delta)(1 - \zeta^2) - (\gamma\zeta + \alpha)^2 = 0,
\]
de sorte qu'on ait
\[
\zeta'^2 = -2\beta(\zeta - a)(\zeta - b)(\zeta - c).
\]
D\'esignons aussi par $\zeta_0$ une des valeurs de $\zeta$, qu'on doit, d'apr\`es la condition
$x'^2 + y'^2 + \zeta^2 = 1$, supposer comprise entre $+1$ et $-1$. Le facteur $\beta$
\'etant positif, comme nous l'avons dit, le polyn\^ome $2\beta(\zeta-a)(\zeta-b)(\zeta-c)$
sera n\'egatif en faisant $\zeta = \zeta_0$. Mais il prend pour $\zeta=+1$ et $\zeta=-1$ les
valeurs positives $(\gamma + \alpha)^2$ et $(\gamma-\alpha)^2$; par cons\'equent, les racines $a$, $b$, $c$
sont r\'eelles, et, si on les suppose rang\'ees par ordre d\'ecroissant de grandeur,
$a$ sera compris entre $+1$ et $\zeta_0$, $b$ entre $\zeta_0$ et $-1$, et $c$ entre $-1$ et $-\infty$.
Remarquons aussi que, ayant pour $z = \zeta$ un r\'esultat positif, il est n\'ecessaire
que cette constante $\delta$ soit sup\'erieure \`a $a$ ou comprise entre $b$ et $c$. Mais la
relation $x^2 + y^2 = 2(\zeta-\delta)$ montre que la seconde hypoth\`ese est seule
%-----103.png------------------
possible, car dans la premi\`ere $x^2+y^2$ serait n\'egatif. Cela pos\'e, puisque $\zeta$
a pour limites $a$ et $b$, nous ferons
\[
\zeta = a - (a - b) U^2;
\]
soit encore
\[
k^2 = \frac{a-b}{a-c}, \qquad k'^2 = \frac{b-c}{a-c},
\]
on aura
\[
(\zeta - a)(\zeta - b)(\zeta - c) = -(a - b)^2(a - c)U^2(1 - U^2)(1 - k^2U^2),
\]
et de l'\'equation
\[
\zeta'^2 = -2\beta(\zeta - a)(\zeta - b)(\zeta - c)
\]
nous conclurons
\[
U'^2 = \frac{(a - c)\beta}{2}(1 - U^2)(1 - k^2U^2).
\]

Faisons donc $n = \sqrt{ \frac{(a-c)\beta}{2} }$; puis, en d\'esignant par $s_0$ une constante
$u = n(s - s_0)$, on aura
\[
U = \sn u, \qquad
\zeta = a - (a - b)\sn^2 u,
\]
et par cons\'equent
\[
n(z - z_0) = \int^u_0 \zeta \,du = \left[ a - (a - c)\frac{J}{K} \right] u
  + (a - c) \frac{ \Theta'(u) }{ \Theta(u) },
\]
$z_0$ \'etant la valeur arbitraire de $z$ pour $u = 0$.

Consid\'erons, pour obtenir la valeur de $x+iy$, l'expression
$\frac{\zeta' + i(\gamma\zeta + \alpha)}{2(\gamma - \delta)}$, qui en repr\'esente la d\'eriv\'ee logarithmique. C'est une fonction
doublement p\'eriodique de la variable $u$, ayant pour p\^oles, d'une part
$u = iK'$ et de l'autre les racines de l'\'equation $\zeta-\delta=0$. Mais des deux
solutions $u=\pm \omega$ qu'on en tire, une seule est en effet un p\^ole, comme le
montre la relation $\zeta'^2 + (\gamma\zeta + \alpha)^2 = 2\beta(\zeta - \delta)(1 - \zeta^2)$, d'o\`u l'on d\'eduit
\[
\zeta' = \pm i(\gamma\delta + \alpha),
\]
en faisant $\zeta = \delta$. Il en r\'esulte que, si nous prenons pour $u = \omega$ la valeur
$\zeta' = +i(\gamma\delta + \alpha)$, on aura $\zeta' = -i(\gamma\delta + \alpha)$ pour $u=-\omega$, la d\'eriv\'ee
changeant de signe avec la variable. En m\^eme temps on voit que le r\'esidu
de la fonction qui correspond au p\^ole $u = \omega$ est $+n$; le r\'esidu relatif \`a
l'autre p\^ole $u = iK'$ est donc $-n$ et, par la d\'ecomposition en \'el\'ements
simples, nous obtenons
\[
\frac{ \zeta' + i(\gamma\zeta + \alpha) }
     { 2(\zeta - \delta) } =
  \frac{1}{n} \left[ \lambda - \frac{\Theta'(u)}{\Theta(u)}
                      + \frac{H'(u - \omega)}{H(u - \omega)} \right].
\]
%-----104.png---------------------------
La constante $\lambda$ se d\'etermine en supposant $u = 0$ ou $\zeta=a$, ce qui donne
imm\'ediatement
\[
\lambda = \frac{in(a\gamma + \alpha)}{a - \delta}+\frac{H'(\omega)}{H(\omega)},
\]
et l'expression cherch\'ee se conclut de la relation
\[
D_s\log(x+iy) = \tfrac{1}{n}D_u\log(x+iy)
  = \frac{1}{n}\left[\lambda-\frac{\Theta'(u)}{\Theta(u)}+\frac{H'(u-\omega)}{H(u-\omega)}\right]
\]
au moyen d'une fonction doublement p\'eriodique de seconde esp\`ece:
\[
x+iy = (x_0+iy_0) \frac{\Theta(0)H(\omega-u)e^{\lambda u}}{\Theta(u)H(\omega)}.
\]
Dans cette formule, $x_0$ et $y_0$ d\'esignent les valeurs que prennent $x$ et $y$ pour
$u = 0$; elles sont li\'ees par l'\'equation
\[
\beta(x_0^2+y_0^2)=2(a-\delta)
\]
et ne contiennent, par cons\'equent, qu'une seule ind\'etermin\'ee. En y joignant
les constantes $z_0$, $s_0$ et $\delta$, on a donc quatre quantit\'es arbitraires dans l'expression
g\'en\'erale des coordonn\'ees de l'\'elastique. A l'\'egard de $\delta$, nous avons
vu que sa valeur doit rester comprise entre $b$ et $c$; de l\`a r\'esulte que $\sn^2 \omega$,
d\'etermin\'e par la formule $\sn^2\omega=\frac{a-\delta}{a-b}$, a pour limites 1 et $\frac{1}{k^2}$. On peut
\'ecrire par suite $\omega=K+i\upsilon$, $\upsilon$ \'etant r\'eel, et poser
\[
x+iy = (x_0+iy_0)\frac{\Theta(0)H_1(i\upsilon-u)e^{\lambda u}}{\Theta(u)H_1(i\upsilon)}.
\]
Changeons $i$ en $-i$, ce qui change $\lambda$ en $-\lambda$, on aura
\[
x-iy = (x_0-iy_0)\frac{\Theta(0)H_1(i\upsilon+u)e^{-\lambda u}}{\Theta(u)H_1(i\upsilon)},
\]
et ces relations, jointes \`a celle qui a \'et\'e pr\'ec\'edemment obtenue, \`a savoir:
\[
n(z-z_0) = \left[a-(a-c)\frac{J}{K}\right]u + (a-c)\frac{\Theta'(u)}{\Theta(u)},
\]
donnent la solution compl\`ete de la question propos\'ee.
%-----105.png-------------------------

\mysection{XXXVI.}


Les expressions des rayons de courbure et de torsion, $R$ et $r$, se calculent
facilement, sans qu'il soit besoin d'employer les valeurs des coordonn\'ees,
et comme cons\'equence imm\'ediate des \'equations diff\'erentielles
\begin{align*}
  y'z'' - y''z' &= \alpha x' + \beta y,\\
  z'x'' - z''x' &= \alpha y' - \beta x,\\
  x'y'' - x''y' &= \alpha z' + \gamma.
\end{align*}

On trouve, en effet, apr\`es les r\'eductions qui s'offrent d'elles-m\^emes,
\begin{align*}
  \frac{1}{R^2} &= (\alpha x' + \beta y)^2 +
                   (\alpha y' - \beta x)^2 +
                   (\alpha z' + \gamma)^2       \\
  &= 2\beta(\zeta - \delta) + \gamma^2 - \alpha^2 =
     2\beta \left[a - \delta - (a - b) \sn^2 u \right] + \gamma^2 - \alpha^2,
\end{align*}
puis
\[
  \begin{vmatrix}
    x' & x'' & x''' \\
    y' & y'' & y''' \\
    z' & z'' & z'''
  \end{vmatrix}
  = \alpha\beta(\zeta-\delta)-\beta(\alpha\delta+\gamma)+\alpha(\gamma^2-\alpha^2),
\]
et, par cons\'equent,
\[
  \frac{1}{r} = \frac{\alpha\beta(\zeta-\delta)-\beta(\alpha\delta+\gamma)+\alpha(\gamma^2-\alpha^2)}
                {2\beta(\zeta-\delta)+\gamma^2-\alpha^2}.
\]
Cette expression du rayon de torsion conduit naturellement \`a envisager
le cas particulier o\`u elle devient ind\'ependante de $\zeta$ et a la valeur constante
$r=\frac{2}{\alpha}$. La condition \`a remplir \`a cet effet \'etant
\[
  2\beta(\alpha\delta+\gamma)-\alpha(\gamma^2-\alpha^2)=0,
\]
je remarque que, en rempla\c{c}ant l'ind\'etermin\'ee $\zeta$ par $-\frac{\gamma}{\alpha}$, dans l'\'egalit\'e
\[
  2\beta(\zeta-\delta)(1-\zeta^2)-(\gamma\zeta+\alpha)^2 =
    -2\beta(\zeta-a)(\zeta-b)(\zeta-c),
\]
le r\'esultat peut s'\'ecrire ainsi:
\[
  (\gamma^2-\alpha^2)\left[2\beta(\alpha\delta+\gamma)-\alpha(\gamma^2-\alpha^2)\right] =
    2\beta(\gamma+a\alpha)(\gamma+b\alpha)(\gamma+c\alpha),
\]
par o\`u l'on voit que l'une des racines $a$, $b$, $c$ est alors \'egale \`a
$-\frac{\gamma}{\alpha}$. Mais notre condition donne
\[
  \delta + \frac{\alpha^2-\gamma^2}{2\beta} = -\frac{\gamma}{\alpha};
\]
%-----106.png-------------------------
ainsi l'on doit poser
\[
\delta + \frac{\alpha^2-\gamma^2}{2\beta} = a, b \text{ ou } c,
\]
et voici la cons\'equence remarquable qui r\'esulte de l\`a. Nous avons trouv\'e
tout \`a l'heure
\[
\frac{1}{R^2}=2\beta\left[a-\delta-(a-b)\sn^2 u\right]+\gamma^2-\alpha^2,
\]
ou plut\^ot
\[
\frac{1}{R^2}=2\beta\left(a-\delta-\frac{\alpha^2-\gamma^2}{2\beta}\right)
  - 2\beta(a-b)\sn^2 u;
\]
or cette expression montre que le premier cas, o\`u l'on suppose
\[
\delta + \frac{\alpha^2-\gamma^2}{2\beta} = a,
\]
doit \^etre rejet\'e, comme conduisant \`a une valeur n\'egative pour $R^2$. Mais les
deux autres peuvent avoir lieu et donnent successivement, en employant
la valeur du module $k^2=\frac{a-b}{a-c}$,
\begin{align*}
  \frac{1}{R^2} &= 2\beta(a-b)\cn^2u, \\
  \frac{1}{R^2} &= 2\beta(a-c)\dn^2u.
\end{align*}

Le rayon de courbure devient donc, comme les coordonn\'ees elles-m\^emes,
une fonction uniforme de l'arc, en m\^eme temps que le rayon de
torsion prend une valeur constante. Ces circonstances remarquables me
semblent appeler l'attention sur la courbe qui les pr\'esente, mais ce serait
trop m'\'etendre d'essayer d'en suivre les cons\'equences et je reviens \`a
mon objet principal, en donnant une derni\`ere remarque sur la formation
des \'equations lin\'eaires d'ordre quelconque dont les int\'egrales sont des
fonctions doublement p\'eriodiques de seconde esp\`ece, unipolaires~(\footnote{
  On doit \`a M. de Saint-Venant un travail important sur les flexions consid\'erables
  des verges \'elastiques, que l'\'eminent g\'eom\`etre a publi\'e dans le \textit{Journal de Math\'ematiques}
  de M. Liouville (t.~IX, 1844), et auquel je dois renvoyer; je citerai aussi, sur la m\^eme
  question, un M\'emoire r\'ecemment publi\'e par M. Adolph Steen, sous le titre: \textit{Der elastike
  Kurve, og dens anvendelse i b\"ojningstheorien}. Copenhague, 1879.}).
%-----107.png-------------------------


\mysection{XXXVII.}


Soit, comme au \S~XXX (p.~\pageref{page79}),
\[
f(u)=\frac{H'(0)\Theta(u+\omega)}{H(u)\Theta(\omega)}\,
  e^{\left[\lambda-\frac{\Theta'(\omega)}{\Theta(\omega)}\right]u};
\]
d\'esignons par $f_i(u)$ ce que devient cette fonction quand on y remplace les
quantit\'es $\omega$, $\lambda$ par $\omega_i$, $\lambda_i$, nommons enfin $\mu_i$ et $\mu'_i$ ses multiplicateurs. Si l'on
pose
\[
y = C_1 f_1(u)+C_2 f_2(u)+\ldots+C_n f_n(u),
\]
l'\'equation diff\'erentielle lin\'eaire d'ordre $n$, admettant cette expression analytique
pour int\'egrale, se pr\'esente sous la forme suivante:
\[
\begin{vmatrix}
  y     & f_1(u)   & f_2(u)   & \ldots & f_n(u)   \\
  y'    & f'_1(u)  & f'_2(u)  & \ldots & f'_n(u)  \\
  \ldots & \ldots    & \ldots    & \ldots & \ldots    \\
  y^n   & f^n_1(u) & f^n_2(u) & \ldots & f^n_n(u)
\end{vmatrix} = 0.
\]

D'apr\`es cela, j'observe que, le d\'eterminant \'etant mis sous la forme
\[
\Phi_0(u)y^n+\Phi_1(u)y^{n-1}+\ldots+\Phi_n(u)y,
\]
les coefficients $\Phi_i(u)$ sont des fonctions de seconde esp\`ece, aux multiplicateurs
$\mu_1\mu_2\ldots\mu_n$, $\mu'_1\mu'_2\ldots\mu'_n$, ayant le p\^ole $u = 0$, avec l'ordre de
multiplicit\'e $n + 1$, sauf le premier $\Phi_0(u)$, o\`u l'ordre de multiplicit\'e est $n$.
C'est ce que l'on voit imm\'ediatement en retranchant la seconde colonne
du d\'eterminant de celles qui suivent, attendu que les diff\'erences $f_2(u)-f_1(u)$,
$f_3(u)-f_1(u)$, \ldots, ainsi que leurs d\'eriv\'ees, ne sont plus infinies pour $u = 0$.
Nous pouvons donc poser, comme je l'ai fait voir ailleurs (\textit{Sur l'int\'egration
de l'\'equation diff\'erentielle de Lam\'e}, dans le \textit{Journal de M.~Borchardt},
t.~LXXXIX, p.~10),
\[
\Phi_0(u)=\frac{G_0H(u-a_1)H(u-a_2)\ldots H(u-a_n)e^{g_0u}}{H^n(u)},
\]
les quantit\'es $G_0$, $g_0$, $a_i$ \'etant des constantes, puis d'une mani\`ere semblable,
pour les coefficients suivants:
\[
\Phi_i(u)=\frac{G_iH(u-a^i_1)H(u-a^i_2)\ldots H(u-a^i_{n+1})e^{g_iu}}{H^{n+1}(u)}.
\]
Il en r\'esulte qu'en d\'ecomposant en \'el\'ements simples les quotients $\frac{\Phi_i(u)}{\Phi_0(u)}$,
%-----108.png-------------------
qui sont des fonctions doublement p\'eriodiques de premi\`ere esp\`ece, on aura
\begin{multline*}
\frac{\Phi_i(u)}{\Phi_0(u)} = \text{const.} +
\frac{A_1 H'(u-a_1) }{ H(u-a_1) } \\ {}+
\frac{A_2 H'(u-a_2) }{ H(u-a_2) } + \ldots +
\frac{A_n H'(u-a_n) }{ H(u-a_n) } +
\frac{A_0 H'(u    ) }{ H(u    ) } ,
\end{multline*}
avec la condition
\[
  A_0 = -(A_1 + A_2 + \ldots + A_n).
\]
C'est donc la g\'en\'eralisation du r\'esultat trouv\'e
au \S~XXI (p.~\pageref{page106}) %sic!
pour les \'equations du second ordre, et il est clair qu'on peut
encore \'ecrire
\[
\frac{\Phi_i(u)}{\Phi_0(u)} = \text{const.} +
\frac{A_1 \sn a_1 }{\sn u \sn(u-a_1) } +
\frac{A_2 \sn a_2 }{\sn u \sn(u-a_2) } + \ldots +
\frac{A_n \sn a_n }{\sn u \sn(u-a_n) } .
\]

La d\'etermination des constantes $A_1$, $A_2$, \ldots, qui entrent dans ces expressions
des coefficients de l'\'equation lin\'eaire, par la condition que les
solutions soient des fonctions uniformes, est une question difficile et importante,
que je n'ai pas abord\'ee au del\`a du cas le plus simple de $n = 2$;
je me borne \`a donner la forme analytique g\'en\'erale de ces coefficients et \`a
observer que, chacune des fonctions $f_i(u)$ contenant deux arbitraires, l'\'equation
diff\'erentielle en renferme en tout $2n$. Les remarques que j'ai \`a pr\'esenter
ont un autre objet, comme on va le voir. Je me suis attach\'e \`a cette circonstance
que pr\'esente l'\'equation de Lam\'e, $y'' = (2k^2 \sn^2 u + h)y$, de ne
contenir aucun point \`a apparence singuli\`ere; elle m'a paru donner l'indication
d'un type sp\'ecial, \`a distinguer et \`a caract\'eriser, de mani\`ere qu'on
ait ses analogues, si je puis dire, pour un ordre quelconque. Introduisons
donc la condition $\Phi_0(u)=\textrm{const.}$ pour amener la disparition des points
\`a apparence singuli\`ere $u=a_1, a_2, \ldots, a_n$, et posons, \`a cet effet, les $n + 1$
conditions
\[
  a_1=0, \quad a_2=0, \quad \ldots, \quad a_n=0, \quad g_0=0.
\]
\medskip

J'observerai, en premier lieu, que, dans ce type particulier d'\'equations,
le nombre des arbitraires se trouve r\'eduit \`a $2n-(n+1)$, c'est-\`a-dire
\`a $n-1$. Je remarque ensuite que, les fonctions $\Phi_i(u)$ ayant toutes les m\^emes
multiplicateurs, ces multiplicateurs seront n\'ecessairement l'unit\'e, puisque
l'une d'elles, $\Phi_0(u)$, est une constante. C'est dire qu'elles deviennent des
fonctions doublement p\'eriodiques de premi\`ere esp\`ece, ayant pour p\^ole
unique $u=0$, avec l'ordre de multiplicit\'e maximum $n+1$. Nous avons,
par cons\'equent, l'expression
\[
  \Phi_i(u) = a + b\, \frac{1}{\sn^2 u} + c\, D_u \frac{1}{\sn^2 u}
    + \ldots + h\, D^{n-1}_u \frac{1}{\sn^2 u},
\]
que la consid\'eration suivante va nous permettre encore de simplifier.
%-----109.png----------------------------

Et, d'abord, il r\'esulte des expressions de $\Phi_0(u)$ et $\Phi_1(u)$, sous forme
de d\'eterminants, qu'on a, en g\'en\'eral,
\[
\Phi_1(u) = -D_u\Phi_0(u).
\]
La condition $\Phi_0(u) = \textrm{const.}$ donne donc
\[
\Phi_1(u) = 0,
\]
et l'on voit que l'\'equation d'ordre $n$, analogue \`a celle de Lam\'e, a la forme
\[
y^n + \Phi_2(u)y^{n-2} + \ldots + \Phi_n(u)y = 0.
\]

Je ferai maintenant un nouveau pas en appliquant l'un des beaux
th\'eo\-r\`emes donn\'es par M.~Fuchs, \`a savoir que le point singulier effectif $u = 0$
doit \^etre, dans le coefficient $\Phi_i(u)$, un p\^ole dont l'ordre de multiplicit\'e ne
d\'epasse pas $i$, pour que l'int\'egrale de l'\'equation diff\'erentielle soit une fonction
uniforme de la variable. On a, en cons\'equence, les expressions suivantes
des coefficients, en rempla\c{c}ant $u$ par $u + iK'$, afin de nous rapprocher
autant que possible de l'\'equation de Lam\'e:
\begin{align*}
  \Phi_2(u) &= \alpha_0 + \alpha_1 \sn^2 u, \\
  \Phi_3(u) &= \beta_0 + \beta_1 \sn^2 u + \beta_2 D_u \sn^2 u, \\
  \Phi_4(u) &= \gamma_0 + \gamma_1 \sn^2 u + \gamma_2 D_u \sn^2 u +
    \gamma_3 D^2_u \sn^2 u,\\
\dotfillalign
\end{align*}

La question de d\'eterminer les constantes $\alpha_0$, $\alpha_1$, \ldots, de mani\`ere \`a
r\'ealiser compl\`etement la condition que l'int\'egrale soit une fonction uniforme,
offre, comme on le voit, beaucoup d'int\'er\^et. Elle a fait le sujet
des recherches d'un jeune g\'eom\`etre du talent le plus distingu\'e, M.~Mittag-Leffler,
professeur \`a l'Universit\'e d'Helsingfors, et je vais exposer les r\'esultats
auxquels il est parvenu.


\mysection{XXXVIII.}


Consid\'erons en premier lieu les \'equations du troisi\`eme ordre, que
nous savons devoir contenir deux constantes arbitraires. Elles pr\'esentent
deux types distincts, et l'un d'eux, d\'ecouvert ant\'erieurement par M.~Picard,
a offert le premier et m\'emorable exemple de l'int\'egration au moyen des
fonctions elliptiques d'une \'equation diff\'erentielle d'ordre sup\'erieur au
%-----110.png-------------------------------
second~(\footnote{\textit{Sur une classe d'\'equations diff\'erentielles} (\textit{Comptes rendus}, t.~XC, p.~128).}).
C'est l'\'equation
\[
y''' + \left(\alpha - 6 k^2 \sn^2 u \right) y' + \beta y = 0,
\]
\`a laquelle on satisfait de la mani\`ere suivante.

Soit
\[
y = \frac{H(u + \omega)}{\Theta(u)}
    e^{ \left[ \lambda - \frac{\Theta'(\omega)}{\Theta(\omega)} \right] u},
\]
et posons, comme au \S~V (p.~\pageref{page13a}),
\begin{alignat*}{2}
&\Omega   &&= k^2 \sn^2 \omega - \dfrac{1 + k^2}{3},  \\
&\Omega_1 &&= k^2 \sn \omega \cn \omega \dn \omega, \\
&\Omega_2 &&= k^2 \sn^4 \omega - \frac{2(k^2 + k^4)}{3} \sn^2 \omega - \frac{ 7 - 22k^2 + 7k^4 }{45}, \\
\dotfillalignat,
\end{alignat*}
de sorte que l'on ait, pour $u = iK' + \varepsilon$,
\[
y = Ce^{\lambda \varepsilon} \left( \frac{1}{\varepsilon}
    - \frac{1}{2} \Omega   \varepsilon
    - \frac{1}{3} \Omega_1 \varepsilon^2
    - \frac{1}{8} \Omega_2 \varepsilon^3 - \ldots \right),
\]
$C$ d\'esignant un facteur constant. Les quantit\'es $\omega$ et $\lambda$ se d\'eterminent au
moyen des relations
\begin{align*}
3(\lambda^2 - \Omega) + \alpha - 2(1 + k^2)     &= 0,  \\
2\lambda^3 - 6\lambda\Omega - 4\Omega_1 - \beta &= 0,
\end{align*}
et il a \'et\'e d\'emontr\'e par M.~Picard qu'elles admettent trois syst\`emes de
solutions, d'o\`u se tirent trois int\'egrales particuli\`eres et par cons\'equent
l'int\'egrale compl\`ete de l'\'equation consid\'er\'ee.

Le second type qu'il faut joindre au pr\'ec\'edent pour avoir, dans le
trois\-i\`eme ordre, toutes les \'equations analogues \`a celle de Lam\'e, est
\[
y''' + (\alpha - 3k^2 \sn^2 u)y' +
  (\beta + \gamma k^2 \sn^2 u - 3k^2 \sn u \cn u \dn u) y = 0,
\]
avec la condition
\[
3(\alpha - 1 - k^2) + \gamma^2 = 0.
\]
Il pr\'esente cette circonstance bien remarquable que, dans les trois
int\'egrales particuli\`eres, la constante $\lambda$ a la m\^eme valeur, \`a savoir: $\lambda = -\frac{\gamma}{3}$.
%-----111.png---------------------------------
Cela \'etant, $\omega$ s'obtient par la relation
\[
2\lambda^3-\lambda(3\Omega-1-k^2)-\Omega_1-\beta = 0.
\]

En passant maintenant au quatri\`eme ordre, on obtient quatre \'equations
A, B, C, D avec trois constantes arbitraires, et pour chacune d'elles
les constantes $\omega$ et $\lambda$ se d\'eterminent ainsi que je vais l'indiquer.

\begin{gather*}
\text{A.} \\
y^{\textsc{IV}} + (\alpha-12k^2 \sn^2 u)y'' + \beta y' + (\gamma+\delta k^2\sn^2 u)y = 0,
\end{gather*}
avec la condition
\[
2\alpha - 8(1+k^2)+\delta = 0.
\]
Les relations entre $\omega$ et $\lambda$ sont
\begin{gather*}
4\lambda^3-\lambda(12\Omega+\delta)-8\Omega_1+\beta = 0, \\
\begin{split}
18\lambda^4-3\lambda^2(36\Omega + \gamma)-144\lambda\Omega_1-54\Omega_2-3\delta\Omega \\
{}-6\gamma-2\delta(1+k^2)+16(1-k^2+k^4) &= 0.
\end{split}
\end{gather*}

\begin{gather*}
\text{B.} \\
\begin{split}
y^{\textsc{IV}}+(\alpha-8 k^2 sn^2 u)y''
   &+ (\beta  + \gamma      k^2 \sn^2 u - 8      k^2 \sn u \cn u \dn u)y'  \\
   &+ (\delta + \varepsilon k^2 \sn^2 u - \gamma k^2 \sn u \cn u \dn u)y  = 0,
\end{split}
\end{gather*}
sous les conditions
\[
4\varepsilon=\gamma^2, \qquad \gamma^3+8\gamma(\alpha-2-2k^2)+16\beta = 0.
\]
On a ensuite
\begin{gather*}
48(\lambda^2-\Omega)+12\lambda\gamma+24\alpha+3\gamma^2-64(1+k^2)=0, \\
\begin{split}
120\lambda^4 - 720\lambda^2\Omega - 960\lambda\Omega_1 - 360\Omega_2
       - 60(\lambda^3 - 3\lambda\Omega - 2\Omega_1) \gamma \\
  {} -15(\lambda^2-\Omega)\gamma^2-120\delta-10(1+k^2)\gamma^2+64(1-k^2+k^4)&=0.
\end{split}
\end{gather*}

\begin{gather*}
\text{C.} \\
y^{\textsc{IV}} + (\alpha-6k^2 \sn^2 u) y''+(\beta-12 k^2 \sn u \cn u \dn u)y'+(\gamma+\delta k^2 \sn^2 u)y = 0,
\end{gather*}
avec la relation
\[
12\gamma-\delta^2-2\delta [ \alpha-4(1+k^2) ] = 0.
\]
Les \'equations en $\omega$ et $\lambda$ sont
\begin{align*}
6(\lambda^2-\Omega) + 2\alpha + \delta - 4(1+k^2)        &=0, \\
2\lambda^3 - \lambda(6\Omega+\delta) - 4\Omega_1 - \beta &=0.
\end{align*}
%-----112.png-----------------------------

\begin{gather*}
\text{D.} \\
\begin{split}
  y^{\textsc{IV}} + (\alpha - 4k^2 \sn^2 u)y'' +
    (\beta + \gamma k^2 \sn^2 u - 8k^2 \sn u \cn u \dn u) y' &\\
{} + (\delta + \varepsilon k^2 \sn^2 u - 8k^4 \sn ^4 u +
    \gamma k^2 \sn u \cn u \dn u) y &= 0.
\end{split}
\end{gather*}
On a entre les constantes les deux conditions
\begin{align*}
8\alpha - 32(1 + k^2) + 4\varepsilon + \gamma ^2 &= 0, \\
4\beta + \gamma \left[ \varepsilon - 4(1 + k^2 ) \right] &= 0.
\end{align*}

Ce dernier cas pr\'esente un second exemple de la circonstance remarquable
qui s'est offerte dans l'une des \'equations du troisi\`eme ordre, la
quantit\'e $\lambda$ ayant dans toutes les int\'egrales particuli\`eres la m\^eme valeur, \`a
savoir $\lambda = -\frac{\gamma}{4}$. L'\'equation en $\omega$ est ensuite
\begin{align*}
  90\lambda^4 - 15(\lambda^2 - \Omega) \left[3\varepsilon - 8(1+k^2) \right]
    -360\lambda^2\Omega - 360\lambda\Omega_1 &\\
{} -90\Omega_2 - 90\delta - 30\varepsilon(1+k^2) +
    16(11 + 4k^2 + 11k^4) &= 0.
\end{align*}


\mysection{XXXIX.}


Les recherches dont je viens d'\'enoncer succinctement les premiers
r\'esul\-tats ont \'et\'e \'etendues par M.~Mittag-Leffler aux \'equations lin\'eaires
d'ordre quelconque, dans un travail qui para\^itra prochainement. Il sera
ainsi \'etabli que la th\'eorie des fonctions elliptiques conduit aux premiers
types g\'en\'eraux, apr\`es celui des \'equations \`a coefficients constants, dont la
solution est connue sous forme explicite. L'\'equation de Lam\'e
\[
D_x^2 y = \left[ n(n+1)k^2 \sn^2 x + h \right]y,
\]
ayant \'et\'e l'origine et le point de d\'epart de ces recherches, doit d'autant
plus appeler notre attention, et j'y reviens pour aborder un second cas,
celui de $n=2$, en me proposant d'en faire l'application \`a la th\'eorie du
pendule. Je traiterai ce cas par une m\'ethode sp\'eciale que j'expose avant
d'arriver au cas g\'en\'eral o\`u le nombre $n$ est quelconque, afin de r\'eunir
divers points de vue sous lesquels peut \^etre trait\'ee la m\^eme question. Reprenons
\`a cet effet l'\'equation consid\'er\'ee au \S~XXX (p.~\pageref{page79}) et dont nous
avons obtenu la solution compl\`ete, \`a savoir:
\begin{align*}
  D_u^2 y &- \left[ \frac{\sn a}{\sn u \sn(u-a)} +
         \frac{\sn b}{\sn u \sn(u-b)} \right] D_u y \\
    &+ \left[ \frac{A\sn a}{\sn u \sn(u-a)} +
         \frac{B\sn b}{\sn u \sn(u-b)} +
         \frac{1}{\sn^2(a-b)} - C^2 \right] y = 0.
\end{align*}
%-----113.png--------------------------------

Soit $u = x + iK'$, et changeons aussi $a$ et $b$ en $a + iK'$ et $b + iK'$, de
sorte que les constantes $A$ et $B$ deviennent
\begin{align*}
A &= \frac{\sn a}{\sn b \sn(a - b)} + C, \\
B &= \frac{\sn b}{\sn a \sn(b - a)} - C.
\end{align*}
L'\'equation prendra la forme suivante:
\[
\begin{split}
D^2_x y &- \left[ \frac{  \sn x}{\sn a \sn(x-a)}
                 + \frac{  \sn x}{\sn b \sn(x-b)} \right] D_x y \\
        &+ \left[ \frac{A \sn x}{\sn a \sn(x-a)}
                 + \frac{B \sn x}{\sn b \sn(x-b)}
                 + \frac{1}{\sn^2 (a-b)} - C^2 \right] y = 0,
\end{split}
\]
et aura pour solution la fonction de seconde esp\`ece
\[
y = \frac{H(x + \omega)}{\Theta(x)}\,
  e^{ \left[\lambda - \frac{\Theta'(\omega)}{\Theta(\omega)} \right] x},
\]
les quantit\'es $\omega$ et $\lambda$ \'etant d\'etermin\'ees maintenant par les conditions
\begin{align*}
\lambda - C &= \frac{\sn a}{\sn b \sn(a-b)}
  - \frac{\cn a \dn a}{\sn a} + \frac{\sn \omega}{\sn a \sn (a+\omega)}, \\
\lambda + C &= \frac{\sn b}{\sn a \sn(b-a)}
  - \frac{\cn b \dn b}{\sn b} + \frac{\sn \omega}{\sn b \sn (b+\omega)}.
\end{align*}

Cela pos\'e, consid\'erons le cas o\`u $b =-a$; on trouve ais\'ement, en
chassant le d\'enominateur $\sn^2 x - \sn^2 a$, l'\'equation
\begin{multline*}
(\sn^2 x - \sn^2 a) D^2_x y - 2\sn x \cn x \dn x\, D_x y \\
   + \left[ \frac{2A \cn a \dn a}{\sn a} \sn^2 x
           + \left( \frac{1}{\sn^2 2a} - C^2 \right) (\sn^2 x - \sn^2 a)
   \right] y = 0.
\end{multline*}

Particularisons encore davantage et, observant qu'on a
\[
A = -\frac{1}{\sn 2a} + C,
\]
faisons dispara\^itre le terme en $\sn^2 x$ dans le coefficient de $y$, en posant
\[
\frac{2 \cn a \dn a}{\sn a} = \frac{1}{\sn 2a} + C.
\]
Ce coefficient se r\'eduisant \`a une constante, l'\'equation pr\'ec\'edente devient
\begin{multline*}
(\sn^2 x - \sn^2 a) D^2_x y - 2 \sn x \cn x \dn x\, D_x y \\
  + 2\left[3 k^2 \sn^4 a - 2(1+k^2) \sn^2 a + 1 \right] y = 0.
\end{multline*}
%-----114.png--------------------------------
\label{page106}
Soit donc, pour un moment,
\[
\Phi(x) = \sn^2 x - \sn^2 a;
\]
on voit qu'on peut l'\'ecrire ainsi:
\[
\Phi(x) D^2_x y - \Phi'(x) D_x y + \Phi''(a) y = 0,
\]
et l'on en conclut, par la diff\'erentiation,
\[
\Phi(x) D^3_x y - \left[ \Phi''(x) - \Phi''(a) \right] D_x y = 0.
\]

Ce r\'esultat remarquable donne, en rempla\c{c}ant $D_x y$ par $z$,
\[
D^2_x z = \left[ \frac{ \Phi''(x) - \Phi''(a) }{ \Phi(x) } \right] z
  = \left( 6k^2 \sn^2 x + 6k^2 \sn^2 a - 4 - 4k^2\right) z \colon
\]
c'est pr\'ecis\'ement l'\'equation de Lam\'e dans le cas de $n = 2$, la constante qui
y figure \'etant $h = 6k^2 \sn^2 a - 4 - 4k^2$. Nous n'avons donc plus, pour parvenir
\`a notre but, qu'\`a former l'int\'egrale de l'\'equation en $y$, c'est-\`a-dire
\`a d\'eterminer les quantit\'es $\omega$ et $\lambda$ au moyen des \'equations rappel\'ees plus
haut. Introduisons, a cet effet, les conditions
\[
b = -a,\qquad
C = \frac{2 \cn a \dn a}{\sn a} - \frac{1}{\sn 2a};
\]
on en tirera successivement, en les retranchant et les ajoutant,
\begin{align*}
\frac{\sn^2 \omega}{\sn^2 a- \sn^2 \omega} &= \frac{\sn^2 a \left(2k^2 \sn^2 a - 1 - k^2 \right)}{\cn^2 a \dn^2 a}, \\
\lambda &= \frac{\sn \omega \cn \omega \dn \omega}{\sn^2 a - \sn^2 \omega}.
\end{align*}
De l\`a nous concluons d'abord, pour $\omega$, les expressions suivantes:
\begin{align*}
\sn^2 \omega &=\phantom{-}\frac{\sn^4 a \left(2k^2 \sn^2 a - 1 - k^2 \right)}{
    3 k^2 \sn^4 a - 2(1 + k^2) \sn^2 a + 1}, \\
\cn^2 \omega &= -         \frac{\cn^4 a \left(2k^2 \sn^2 a - 1\right)}{
    3 k^2 \sn^4 a - 2(1 + k^2) \sn^2 a + 1}, \\
\dn^2 \omega &= -         \frac{\dn^4 a \left(2\sn^2 a - 1\right)}{
    3 k^2 \sn^4 a - 2(1+k^2)\sn^2 a + 1}.
\end{align*}
On a ensuite
\[
\lambda^2 = \frac{\sn^2 \omega \cn^2 \omega \dn^2 \omega}{ (\sn^2 a - \sn^2 \omega)^2 }
   = \frac{ (2 k^2 \sn^2 a - 1 - k^2)(2 k^2 \sn^2 a - 1)(2 \sn^2 a - 1) }
          { 3k^2 \sn^4 a - 2(1+k^2) \sn^2 a + 1},
\]
et l'on voit que les constantes $\sn^2 \omega$ et $\lambda^2$ sont des fonctions rationnelles
de $\sn^2 a$ ou de $h$. Nous remarquerons en m\^eme temps que, $\sn \omega$
et, par cons\'equent, $\omega$ ayant deux d\'eterminations \'egales et de signes contraires,
%-----115.png--------------------------------
le signe de $\lambda$ est donn\'e par celui de $\omega$, en vertu de la relation
$\lambda = \frac{\sn \omega \cn \omega \dn \omega}{\sn^2 a - \sn^2 \omega}$. Aucune ambigu\"it\'e ne s'offre donc dans la formule
\[
y = C \frac{H(x + \omega)}{\Theta(x)}
    e^{ \left[ \lambda - \frac{\Theta'(\omega)}{\Theta(\omega)} \right] x }
  + C'\frac{H(x - \omega)}{\Theta(x)}
    e^{-\left[ \lambda - \frac{\Theta'(\omega)}{\Theta(\omega)} \right] x },
\]
et l'on en conclut, pour l'int\'egrale de l'\'equation de Lam\'e,
\[
D^2_x y = \left(6k^2 \sn^2 x + 6k^2 \sn^2 a - 4 - 4k^2 \right) y,
\]
l'expression
\[
y = C D_x \frac{H(x+\omega)}{\Theta(x)}
    e^{ \left[ \lambda - \frac{\Theta'(\omega)}{\Theta(\omega)} \right] x } +
    C'D_x \frac{H(x-\omega)}{\Theta(x)}
    e^{-\left[ \lambda - \frac{\Theta'(\omega)}{\Theta(\omega)} \right] x }.
\]
Voici les remarques auxquelles elle donne lieu.


\mysection{XL.}


Nous allons supposer nulle ou infinie la quantit\'e $\lambda$, en nous proposant
d'\'etudier les circonstances qu'offre alors la solution de l'\'equation
diff\'eren\-tielle.

Et d'abord, on voit, par l'expression de $\lambda^2$, que le premier cas a lieu
en posant les conditions
\begin{align*}
2 k^2 \sn^2 a - 1 - k^2 &= 0, \\
      2 k^2 \sn^2 a - 1 &= 0, \\
          2 \sn^2 a - 1 &= 0,
\end{align*}
qui donnent successivement $\sn \omega = 0$, $\cn \omega = 0$, $\dn \omega = 0$. Les valeurs
de $\omega$ qui en r\'esultent, \`a savoir, $\omega = 0$, $\omega = K$, $\omega = K + iK'$, conduisent
aux solutions consid\'er\'ees par Lam\'e, qui sont des fonctions doublement
p\'eriodiques de la variable, avec la p\'eriodicit\'e caract\'eristique de $\sn x$,
$\cn x$, $\dn x$. Nous avons, en effet, pour $\omega = 0$ et $\omega = K$: $y = D_x \sn x$,
$y = D_x \cn x$. Il suffit ensuite d'employer les relations
\begin{align*}
H(x + K + iK') &= \Theta_1(x) e^{ -\frac{i\pi}{4K}(2x + iK') }, \\
\frac{ \Theta'(K + iK') }{ \Theta(K + iK') } &= - \frac{i\pi}{2K},
\end{align*}
pour conclure de la valeur $\omega = K + iK'$ l'expression $y = D_x \dn x$.
%-----116.png-------------------------------

Supposons maintenant $\lambda$ infini, et soit \`a cet effet
\[
3 k^2 \sn^4 a - 2(1 + k^2) \sn^2 a + 1 = 0;
\]
en d\'esignant une solution de cette \'equation par $a = \alpha$, je ferai $a = \alpha + \eta$,
$\omega = iK' + \varepsilon$, les quantit\'es $\eta$ et $\varepsilon$ \'etant infiniment petites. D'apr\`es la relation
\[
\sn^2 \omega = \frac{ \sn^4 a \left(2k^2 \sn^2 a - 1 - k^2 \right) }{
  3 k^2 \sn^4 a - 2(1+k^2) \sn^2 a + 1},
\]
on voit d'abord qu'on aura, en d\'eveloppant en s\'erie,
\[
\varepsilon^2 = p\eta + q\eta^2 + \ldots,
\]
$p$, $q$ \'etant des constantes. Cela \'etant, nous d\'evelopperons aussi $\lambda$ suivant
les puissances croissantes de $\varepsilon$, au moyen de l'expression
\[
\lambda = \frac{\sn \omega \cn \omega \dn \omega}{\sn^2 a - \sn^2 \omega}
  = \frac{\cn \varepsilon \dn \varepsilon}{\sn \varepsilon}\,
    \frac{1}{1 - k^2 \sn^2 (\alpha + \eta) \sn^2 \varepsilon}.
\]

Or, ayant
\begin{align*}
\frac{\cn \varepsilon \dn \varepsilon}{\sn \varepsilon}
  &= \frac{1}{\varepsilon} - \frac{1+k^2}{3}\varepsilon + \ldots,\\
\frac{1}{1 - k^2 \sn^2 (\alpha + \eta) \sn^2 \varepsilon}
  &= 1 + k^2 \sn^2 \alpha \cntrdot \varepsilon^2 + \ldots,
\end{align*}
on en conclut
\[
\lambda = \frac{1}{\varepsilon} +
    \left( k^2 \sn^2 \alpha - \frac{1+k^2}{3} \right) \varepsilon + \ldots.
\]

Employons maintenant l'\'equation
\[
\frac{\Theta'(iK' + \varepsilon)}{\Theta (iK' + \varepsilon)}
  = \frac{H'(\varepsilon)}{H (\varepsilon)} - \frac{i\pi}{2K}
  = \frac{1}{\varepsilon} - \frac{i\pi}{2K}
    + \left( \frac{J}{K} - \frac{1+k^2}{3} \right) \varepsilon + \ldots,
\]
nous obtenons cette expression, qui est finie, pour $\varepsilon = 0$, \`a savoir
\[
\lambda - \frac{\Theta'(iK' + \varepsilon)}{\Theta(iK' + \varepsilon)}
  = \frac{i\pi}{2K} + \left( k^2 \sn^2 \alpha - \frac{J}{K} \right) \varepsilon + \ldots.
\]

Enfin, je remplace, dans la solution de l'\'equation diff\'erentielle, la
quantit\'e $H(x + iK + \varepsilon)$ par
\[
i\Theta(x + \varepsilon) e^{ -\frac{i\pi}{4K} (2x + 2\varepsilon + iK')};
\]
il viendra ainsi
\[
\frac{H(x + \omega)}{\Theta(x)} \,
  e^{ \left[ \lambda - \frac{\Theta'(\omega)}{\Theta(\omega)} \right] x }
= ie^{ \frac{\pi K'}{4K} } \,\frac{\Theta(x + \varepsilon)\,
  e^{g\varepsilon} }{\Theta(x)},
\]
%-----117.png-------------------------------
en faisant, pour abr\'eger,
\[
g = -\frac{i\pi}{2K} + \left( k^2 \sn^2 \alpha - \frac{J}{K} \right) x.
\]

Or, en d\'eveloppant suivant les puissances de $\varepsilon$, on obtient, si l'on se
borne aux deux premiers termes,
\[
\frac{\Theta(x - \varepsilon) e^{g\varepsilon}}{\Theta(x)}
   = 1 + \left[ \frac{\Theta'(x)}{\Theta(x)} + g \right] \varepsilon\,;
\]
il suffira donc de remplacer la constante arbitraire $C$ par $\frac{C}{\varepsilon}$, pour avoir
la limite cherch\'ee, lorsqu'on pose $\varepsilon = 0$. Nous trouvons ainsi
\[
\frac{1}{\varepsilon} D_x \left[ \frac{\Theta(x+\varepsilon) e^{g\varepsilon}}{\Theta(x)} \right]
   = D_x \left[ \frac{\Theta'(x)}{\Theta(x)} + g \right]
   = k^2 (\sn^2 \alpha - \sn^2 x),
\]
o\`u la constante $\sn^2 \alpha$ est d\'etermin\'ee par l'\'equation
\[
3 k^2 \sn^4 \alpha - 2(1+k^2) \sn^2 \alpha + 1 = 0.
\]

Ces deux solutions de l'\'equation diff\'erentielle, r\'eunies \`a celles qui
ont \'et\'e obtenues pr\'ec\'edemment, compl\`etent l'ensemble des cinq
solutions de Lam\'e, qui sont des fonctions doublement p\'eriodiques, ces
deux derni\`eres ayant, comme on voit, la p\'eriodicit\'e de $\sn^2 x$.
\smallskip


\mysection{XLI.}


La th\'eorie du pendule conique ou du mouvement d'un point pesant
sur une sph\`ere conduit \`a une application imm\'ediate de l'\'equation qui
vient de nous occuper. C'est M.~Tissot qui a le premier trait\'e cette
question importante, par une analyse semblable \`a celle de Jacobi dans le
probl\`eme de la rotation, et donn\'e explicitement, en fonction du temps,
les coordonn\'ees du point mobile (\textit{Th\`ese de M\'ecanique}, \textit{Journal de M.~Liouville},
t.~XVII, p.~88). En suivant une autre marche, nous trouvons une
autre forme analytique de la solution que j'ai indiqu\'ee, sans d\'emonstration,
dans une Lettre adress\'ee \`a M. H.~Gyld\'en et publi\'ee dans le \textit{Journal
de Borchardt}, t.~85, p.~246. Ces r\'esultats s'\'etablissent de la mani\`ere suivante.

Soient $x$, $y$, $z$ les coordonn\'ees rectangulaires d'un point pesant, assujetti
\`a rester sur une sph\`ere de rayon \'egal \`a l'unit\'e; les \'equations du mouvement,
%-----118.png-------------------------------
si l'on d\'esigne par $g$ la pesanteur et $N$ la force acc\'el\'eratrice,
seront~(\footnote{\textit{Trait\'e de M\'ecanique} de Poisson, t.~I, p.~386.})
\begin{align*}
\frac{d^2x}{dt^2} + Nx &= 0,\\
\frac{d^2y}{dt^2} + Ny &= 0,\\
\frac{d^2z}{dt^2} + Nz &= g,
\end{align*}
\[
x^2 + y^2 + z^2 = 1.
\]
Elles donnent d'abord, comme on sait, en d\'esignant par $c$ et $l$ des constantes:
\begin{gather*}
\left( \frac{dx}{dt} \right)^2 +
\left( \frac{dy}{dt} \right)^2 +
\left( \frac{dz}{dt} \right)^2 = 2g(z + c), \\
y \frac{dx}{dt} - x \frac{dy}{dt} = l.
\end{gather*}
Cela \'etant, j'emploie la combinaison suivante:
\[
(x + iy) \left( \frac{dx}{dt} - i\frac{dy}{dt} \right)
   = x\frac{dx}{dt} + y\frac{dy}{dt} + i\left( y\frac{dx}{dt} - x\frac{dy}{dt}\right)
   = -z \frac{dz}{dt} + il,
\]
et je remarque que le carr\'e du module du premier membre,
\[
(x^2 + y^2) \left[ \left( \frac{dx}{dt} \right)^2
                  + \left( \frac{dy}{dt} \right)^2 \right],
\]
s'exprime par
\[
(1 - z^2) \left[ 2g(z+c) - \left( \frac{dz}{dt} \right)^2 \right],
\]
de sorte qu'on obtient, en l'\'egalant au carr\'e du module du second membre,
\[
(1-z^2) \left[ 2g (z+c) - \left( \frac{dz}{dt} \right)^2 \right]
  = z^2 \left( \frac{dz}{dt} \right)^2 + l^2,
\]
ou bien
\[
\left( \frac{dz}{dt} \right)^2 = 2g(z+c)(1-z^2) - l^2.
\]
La variable $z$ \'etant d\'etermin\'ee par cette relation, une premi\`ere m\'ethode
pour obtenir les deux autres coordonn\'ees consiste \`a diviser membre \`a
%-----119.png-------------------------------
membre les \'equations
\begin{gather*}
(x + iy)\left( \frac{dx}{dt} - i\frac{dy}{dt} \right)
= -z \frac{dz}{dt} + il,  \\
x^2 + y^2 = 1 - z^2.
\end{gather*}

On obtient facilement ainsi les expressions qui conduisent aux r\'esultats
de M. Tissot, \`a savoir:
\[
x - iy = e^{\textstyle-\int \frac{z \,dz - il \,dt}{1 - z^2}},
\]
puis, en changeant $i$ en $-i$,
\[
x + iy = e^{\textstyle-\int \frac{z \,dz + il \,dt}{1 - z^2}}.
\]
Mais j'op\'ererai diff\'eremment; je d\'eduis d'abord des \'equations diff\'erentielles,
et les ajoutant apr\`es les avoir multipli\'ees respectivement par $x$, $y$, $z$,
\[
x\frac{d^2x}{dt^2} + y\frac{d^2y}{dt^2} + z\frac{d^2z}{dt^2} + N = gz,
\]
puis de l'\'equation de la sph\`ere, diff\'erenti\'ee deux fois,
\[
x\frac{d^2x}{dt^2} + y\frac{d^2y}{dt^2} + z\frac{d^2z}{dt^2}
  = -\left( \frac{dx}{dt} \right)^2
    -\left( \frac{dy}{dt} \right)^2
    -\left( \frac{dz}{dt} \right)^2
  = -2g(z+c).
\]
Nous avons donc
\[
N = g(3z+2c),
\]
et, par cons\'equent,
\[
\frac{d^2(x+iy)}{dt^2} = -g(3z+2c)(x+iy);
\]
or on est ainsi amen\'e \`a l'\'equation de Lam\'e, dans le cas de $n = 2$, comme
nous allons le voir.

Formons pour cela l'expression de $z$, et soit \`a cet effet
\[
2g(z+c)(1-z^2) - l^2 = -2g(z-\alpha)(z-\beta)(z-\gamma),
\]
ce qui donne les relations suivantes:
\begin{align*}
\alpha + \beta + \gamma &= -c, \\
\alpha\beta + \beta\gamma + \gamma\alpha &= -1, \\
\alpha\beta\gamma &= c - \frac{l^2}{2g}.
\end{align*}

On sait que les racines $\alpha$, $\beta$, $\gamma$ sont n\'ecessairement r\'eelles, et qu'en les
%-----120.png-------------------------------
rangeant par ordre d\'ecroissant de grandeur $\alpha$ sera positive, $\beta$ positive ou
n\'egative, et toutes deux moindres en valeur absolue que l'unit\'e, tandis que
$\gamma$ sera n\'egative et sup\'erieure \`a l'unit\'e en valeur absolue. Soit donc
\begin{align*}
k^2 &= \frac{\alpha - \beta}{\alpha - \gamma}, \\
  u &= n(t - t_0), \\
  n &= \sqrt{ \frac{ g(\alpha - \gamma) }{2} },
\end{align*}
on aura
\[
z = \alpha - (\alpha - \beta) \sn^2 (u, k),
\]
$t_0$ \'etant une constante et le coefficient $n$ \'etant pris positivement. Introduisons
maintenant la variable $u$ dans l'\'equation du second ordre, elle deviendra
\[
D_u^2(x+iy) = \frac{g}{n^2} \left[3(\alpha - \beta)\sn^2 u - 3\alpha + 2c \right](x + iy)
\]
et, en simplifiant,
\[
D_u^2(x+iy) = \left( 6k^2\sn^2 u - 2\frac{\alpha - 2\beta - 2\gamma}{\alpha - \gamma} \right) (x+iy).
\]

C'est donc l'\'equation de Lam\'e dont nous avons donn\'e la solution
com\-pl\`ete au moyen de deux fonctions doublement p\'eriodiques de seconde
esp\`ece \`a multiplicateurs r\'eciproques. Or une seule de ces fonctions doit
figurer dans l'expression de $x+iy$, comme le montre la formule obtenue
tout \`a l'heure
\[
x + iy = e^{\textstyle -\int \frac{z \,dz + il \,dt}{1 + z^2}};
\]
par cons\'equent, nous pouvons imm\'ediatement \'ecrire
\[
x + iy = CD_u \frac{H(u+\omega)}{\Theta(u)}\,
  e^{ -\left[ \lambda - \frac{\Theta'(\omega)}{\Theta(\omega)} \right] u}
\]
ou, sous une autre forme, en modifiant la constante arbitraire,
\[
x + iy = AD_u \frac{H'(0) H(u + \omega)}{\Theta(\omega)\Theta(u)}\,
  e^{ -\left[ \lambda - \frac{\Theta'(\omega)}{\Theta(\omega)} \right] u};
\]
maintenant il nous faut d\'eterminer cette constante, ainsi que les
quantit\'es $\omega$ et $\lambda$.
%-----121.png-------------------------------


\mysection{XLII.}


En posant la condition
\[
6k^2 \sn^2 a - 4 - 4k^2 = -2 \frac{\alpha - 2\beta - 2\gamma}{\alpha - \gamma},
\]
et employant l'expression du module $k^2 = \frac{\alpha-\beta}{\alpha-\gamma}$, on trouve d'abord
\[
\sn^2 a = \frac{\alpha}{\alpha-\beta}.
\]
De l\`a se tirent ensuite, apr\`es quelques r\'eductions faciles o\`u l'on fera
usage de la relation
\[
\alpha\beta + \beta\gamma + \gamma\alpha = -1,
\]
les formules suivantes:
\begin{align*}
\sn^2 \omega &= -\frac{\alpha^2 (\beta  + \gamma)}
                      {\alpha - \beta}, \\
\cn^2 \omega &= +\frac{\beta^2  (\alpha + \gamma)}
                      {\alpha - \beta}, \\
\dn^2 \omega &= +\frac{\gamma^2 (\alpha + \beta)}
                      {\alpha - \gamma},\\
\lambda^2 &= -\frac{(\alpha + \beta)(\beta + \gamma)(\gamma + \alpha)}
                   {\alpha - \gamma}.
\end{align*}

Cela \'etant, nous remarquerons en premier lieu que, d'apr\`es les limites
entre lesquelles sont comprises les quantit\'es $\alpha$, $\beta$, $\gamma$, on obtient pour
$\sn^2 \omega$ et $\dn^2 \omega$ des valeurs positives, tandis que $\cn^2 \omega$ est n\'egatif. Il en r\'esulte
que $\sn^2 \omega$ est plus grand que l'unit\'e et moindre que $\frac{1}{k^2}$, de sorte qu'on
doit supposer
\[
\omega = \pm K + i\upsilon,
\]
$\upsilon$ \'etant r\'eel et donn\'e par ces expressions
\begin{align*}
\sn^2(\upsilon, k')
&= \frac{\beta^2 (\gamma^2 - \alpha^2)}{\alpha^2(\gamma^2 - \beta^2)}, \\
\cn^2(\upsilon, k')
&= \frac{\gamma^2(\beta^2  - \alpha^2)}{\alpha^2(\beta^2 - \gamma^2)}, \\
\dn^2(\upsilon, k')
&= \frac{\beta - \alpha}{\alpha^2(\beta + \gamma)}.
\end{align*}
%-----122.png-------------------------------

J'observe ensuite qu'ayant $n^2 = \frac{g(\alpha - \gamma)}{2}$ nous pouvons \'ecrire la
valeur de $\lambda^2$ de cette mani\`ere:
\[
\lambda^2 = -\frac{g(\alpha + \beta)(\beta + \gamma)(\gamma + \alpha)}{2n^2},
\]
d'o\`u l'on conclut facilement
\[
\lambda^2 = -\frac{l^2}{4n^2}.
\]
Les constantes $\omega$ et $\lambda$ se trouvent ainsi d\'etermin\'ees, mais seulement au
signe pr\`es, et deux autres relations sont encore n\'ecessaires pour lever
toute ambigu\"it\'e. La premi\`ere r\'esulte d'abord de la condition qui a \'et\'e
donn\'ee pour la solution g\'en\'erale de l'\'equation de Lam\'e, \`a savoir:
\[
\lambda = \frac{\sn \omega \cn \omega \dn \omega}{\sn^2 a - \sn^2 \omega},
\]
et l'on en tire imm\'ediatement
\[
\lambda = -\frac{(\alpha - \beta)\sn \omega \cn \omega \dn \omega}{\alpha\beta\gamma}.
\]
Nous obtiendrons tout \`a l'heure la seconde comme cons\'equence de l'\'equation
consid\'er\'ee plus haut:
\[
(x + iy)\left( \frac{dx}{dt} - i\frac{dy}{dt} \right) = -z\frac{dz}{dt} + il.
\]
Mais voici d'abord la d\'etermination de la constante $A$ qui entre dans la
formule
\[
x + iy = AD_u\frac{H'(0) H(u + \omega)}{\Theta(\omega) \Theta(u)}\,
         e^{ \left[ \lambda - \frac{\Theta'(\omega)}{\Theta(\omega)} \right] u }.
\]
Soit, pour abr\'eger,
\[
F(u) = \frac{H'(0) H(u + \omega)}{\Theta(\omega)\Theta(u)}\,
       e^{ \left[ \lambda - \frac{\Theta'(\omega)}{\Theta(\omega)} \right] u }.
\]
D\'esignons par $F_1(u)$ ce que devient cette fonction lorsqu'on change $i$
en $-i$, et par $A_1$ la quantit\'e conjugu\'ee de $A$, de sorte qu'on ait
\begin{align*}
x + iy &= AF'(u),\\
x - iy &= A_1 F'_1(u),
\end{align*}
et, par cons\'equent,
\[
x^2 + y^2 = A A_1 F'(u) F'_1(u).
\]
%-----123.png-----------------------------

Nous supposerons $u=0$, ce qui donne $z=\alpha$, dans l'\'equation
$x^2+y^2{+z^2=1}$; il viendra ainsi
\[
A A_1 F'(0) F'_1(0) = 1 - \alpha^2,
\]
ou encore, au moyen de la condition
$\alpha\beta + \beta\gamma + \gamma\alpha = -1$,
\[
A A_1 F'(0) F'_1(0) = -(\alpha + \beta)(\alpha + \gamma).
\]

J'emploie maintenant, pour y faire $u = 0$, la relation
\[
\frac{F'(u)}{F(u)} = \frac{H'(u+\omega)}{H(u+\omega)}
    - \frac{\Theta'(u)}{\Theta(u)}
    - \frac{\Theta'(\omega)}{\Theta(\omega)} + \lambda;
\]
on en tire d'abord
\[
\frac{F'(0)}{F(0)} = \frac{\cn \omega \dn \omega}{\sn \omega} + \lambda,
\]
puis, au moyen de la valeur donn\'ee pr\'ec\'edemment de $\lambda$,
\[
\frac{F'(0)}{F(0)} = \frac{\cn \omega \dn \omega}{\sn \omega}
 - \frac{\alpha - \beta}{\alpha\beta\gamma}\sn \omega \cn \omega \dn \omega
 = \frac{\cn \omega \dn \omega}{\sn \omega}
   \left( 1 - \frac{\alpha - \beta}{\alpha\beta\gamma}\sn^2 \omega \right)
\]
et enfin
\[
\frac{F'(0)}{F(0)} = -\frac{\cn \omega \dn \omega}{\beta \gamma \sn \omega},
\]
comme cons\'equence de la formule
\[
\sn^2 \omega = -\frac{\alpha^2 (\beta + \gamma)}{\alpha-\beta};
\]
mais l'expression de $F(u)$ donne imm\'ediatement
\[
F(0) = \frac{H'(0) H(\omega)}{\Theta(0)\Theta(\omega)} = k\sn \omega,
\]
et nous en concluons l'expression cherch\'ee, \`a savoir
\[
F'(0) = -\frac{k \cn \omega \dn \omega}{\beta\gamma}.
\]

Changeons enfin $i$ en $-i$; la constante $\omega = \pm K + i\upsilon$ deviendra
\[
\omega' = \pm K - i\upsilon;
\]
on a donc
\[
\sn \omega' = \sn \omega, \qquad
\cn \omega' \dn \omega' = -\cn \omega \dn \omega,
\]
et par suite
\[
F'(0) F'_1(0) = -\frac{k^2 \cn^2 \omega \dn^2 \omega}{\beta^2 \gamma^2}
              = -\frac{(\alpha + \beta)(\alpha + \gamma)}{(\alpha - \gamma)^2}.
\]
%-----124.png------------------------------

De cette expression nous tirons
\begin{align*}
  A A_1          &= (\alpha - \gamma)^2, \\
\intertext{de sorte qu'on peut \'ecrire}
  A\phantom{A_1} &= (\alpha - \gamma) e^{i \varphi},
\end{align*}
$\varphi$ d\'esignant un angle arbitraire.

Ce point \'etabli, je reprends l'\'equation
\[
(x + iy)\left( \frac{dx}{dt} - i \frac{dy}{dt} \right) = -z \frac{dz}{dt} + il,
\]
qui devient, si l'on introduit, au lieu de $t$, la variable $u$,
\[
(x + iy)\left( \frac{dx}{du} - i\frac{dy}{du} \right) = -z\frac{dz}{du} + \frac{il}{n},
\]
et j'y fais $u = 0$. En remarquant qu'alors $\frac{dz}{du}$ s'\'evanouit, on trouve
\[
(\alpha - \gamma)^2 F'(0) F''_1(0) = \frac{il}{n},
\]
ce qui nous m\`ene \`a chercher la valeur de $F''_1 (0)$. Pour cela, je d\'eduis de la
relation employ\'ee tout \`a l'heure
\[
\frac{F'(u)}{F(u)} = \frac{H'(u + \omega)}{H(u + \omega)}
    - \frac{\Theta'(u)}{\Theta(u)}
    - \frac{\Theta'(\omega)}{\Theta(\omega)} + \lambda,
\]
la suivante:
\[
\frac{F''(u)}{F(u)} - \frac{F'^2(u)}{F^2(u)} = -\frac{1}{\sn^2(u+\omega)} + k^2\sn^2 u,
\]
et j'en tire d'abord
\[
\frac{F''(0)}{F(0)} = \frac{F'^2(0)}{F^2(0)} - \frac{1}{\sn^2 \omega}
   = \frac{\cn^2 \omega \dn^2 \omega}{\beta^2 \gamma^2 \sn^2 \omega} - \frac{1}{\sn^2 \omega},
\]
puis, apr\`es une r\'eduction facile et au moyen de la valeur obtenue pour
$F(0)$,
\[
F''(0) = -\frac{2k\sn \omega}{\alpha(\alpha - \gamma)}.
\]

Cette expression restant la m\^eme lorsqu'on change $i$ en $-i$, nous pouvons
\'ecrire
\[
F''_1(0) = -\frac{2k\sn \omega}{\alpha(\alpha - \gamma)},
\]
%-----125.png------------------------------
et, comme on a d\'ej\`a trouv\'e
\[
F'(0) = -\frac{k\cn \omega \dn \omega}{\beta \gamma},
\]
nous en concluons
\[
F'(0) F''_1(0) = \frac{2k^2\sn \omega \cn \omega \dn \omega}{\alpha\beta\gamma(\alpha-\gamma)},
\]
et, en employant la valeur de $k^2$, l'\'equation suivante:
\[
(\alpha - \gamma)^2 F'(0) F''_1(0) = \frac{2(\alpha - \beta) \sn \omega \cn \omega \dn \omega}{\alpha \beta \gamma} = \frac{il}{n}.
\]

Si on se rapproche maintenant de la relation d\'ej\`a donn\'ee
\[
\lambda = -\frac{(\alpha - \beta)\sn \omega \cn \omega \dn \omega}{\alpha \beta \gamma},
\]
on trouve imm\'ediatement
\[
\lambda = -\frac{il}{2n};
\]
c'est le r\'esultat que j'ai principalement en vue d'obtenir, afin d'avoir la d\'etermination
pr\'ecise de la constante $\lambda$, qui n'\'etait encore connue qu'au
signe pr\`es.

En dernier lieu, et \`a l'\'egard de $\omega$, on remarquera que la fonction $F(u)$
change seulement de signe ou se reproduit quand on met $\omega + 2K$ et
$\omega + 2iK'$ \`a la place de $\omega$. Et comme on peut obtenir un tel changement
de signe pour la valeur de $x+iy$, en rempla\c{c}ant $\varphi$ par $\varphi + \pi$ dans l'argument
du facteur constant $A$, il en r\'esulte qu'il est permis de faire $\omega = K + i\upsilon$,
au lieu de $\omega = \pm K + i\upsilon$, et de d\'eterminer une valeur de $\upsilon$, comprise entre
$-K'$ et $+K'$.

Or, de la relation
\[
\sn^2(\upsilon, k') = \frac{\beta^2 (\gamma^2 - \alpha^2)}{\alpha^2 (\gamma^2 - \beta^2)},
\]
se tirent deux valeurs \'egales et de signes contraires de cette quantit\'e entre
lesquelles il reste \`a choisir. C'est \`a quoi l'on parvient au moyen de la condition
\[
\frac{il}{2n} = \frac{(\alpha - \beta) \sn \omega \cn \omega \dn \omega}{\alpha\beta\gamma},
\]
qui prend, si l'on y fait $\omega = K + i\upsilon$, la forme suivante,
\[
\frac{l}{2n}
= -\frac{(\alpha-\beta)k'^2 \sn(\upsilon, k') \cn(\upsilon, k')}{
    \alpha\beta\gamma \dn^3(\upsilon, k')};
\]
%-----126.png-----------------------------
or, $\gamma$ \'etant n\'egatif, on voit ainsi que $\upsilon$ aura le signe de $l$ ou un signe contraire,
suivant que la racine moyenne $\beta$ sera positive ou n\'egative. Dans le
cas de $\beta = 0$, on a donc
\[
  \omega = K
\]
et, par suite,
\[
  F(u) = kD_u e^{\frac{ilu}{2n}} \cn u \colon
\]
c'est un exemple de ces fonctions particuli\`eres de seconde esp\`ece qui ont
\'et\'e consid\'er\'ees par M. Mittag-Leffler dans un article intitul\'e, \textit{Sur les fonctions
doublement p\'eriodiques de seconde esp\`ece} (\textit{Comptes rendus}, t.~XC, p.~177).


\mysection{XLIII.}


Je terminerai par une remarque sur l'\'equation
\[
  \frac{il}{n} + \frac{\Theta'(\omega)}{\Theta(\omega)} = 0,
\]
qui exprime que les coordonn\'ees $x$ et $y$ se reproduisent sauf le signe, lorsqu'on
change $u$ en $u + 2K$. Soit $\omega = K + i\upsilon$ et posons
\[
  i\,\Pi(\upsilon) = \frac{il}{n} + \frac{\Theta'(K+i\upsilon)}{\Theta(K+i\upsilon)};
\]
cette fonction $\Pi(\upsilon)$, \'evidemment r\'eelle, finie et continue pour toute valeur
r\'eelle de $\upsilon$, a pour d\'eriv\'ee l'expression
\[
  \Pi'(\upsilon) = \frac JK - k^2 \sn^2(K+i\upsilon),
\]
qui est toujours n\'egative. On a, en effet,
\[
  J < k^2 K,
\]
comme cons\'equence des formules
\[
  K = \int_0^1 \frac{        dx}{\sqrt{(1-x^2)(1-k^2x^2)}},  \quad
  J = \int_0^1 \frac{k^2x^2\,dx}{\sqrt{(1-x^2)(1-k^2x^2)}},
\]
et l'on sait d'ailleurs que $\sn^2(K + i\upsilon)$ est sup\'erieur \`a l'unit\'e. La fonction
$\Pi(\upsilon)$, \'etant d\'ecroissante, ne peut s'\'evanouir qu'une fois; or on a, en
d\'esignant par $a$ un nombre entier,
\[
  \frac{ \Theta'(K+2i aK') }{ \Theta(K+2i aK') } = -\frac{ia\pi}{K},
\]
%-----127.png-------------------------
et par cons\'equent
\[
\Pi(0) = \frac{l}{n},\qquad \Pi(2aK')=\frac{l}{n}-\frac{a\pi}{K}.
\]
Nous \'etablissons ainsi l'existence d'une racine, puisqu'on peut disposer
de $a$ de mani\`ere que $\frac{l}{n} - \frac{a\pi}{K}$ soit de signe contraire \`a $\frac{l}{n}$. Mais c'est en d\'eterminant
les quantit\'es $c$ et $l$ qu'il serait surtout important d'obtenir les
cas o\`u le mouvement du pendule est p\'eriodique, ces constantes repr\'esentant
les \'el\'ements essentiels de la question. N'ayant pu surmonter les
difficult\'es qui s'offrent alors, je me borne \`a donner de l'\'equation pr\'ec\'edente
une transform\'ee o\`u ces constantes se trouvent plus explicitement en
\'evidence. Soit, \`a cet effet,
\[
R(z) = 2g(z+c)(1-z^2)-l^2;
\]
on aura, en premier lieu,
\[
K = \int_{\beta}^{\alpha} \frac{n\,dz}{\sqrt{R(z)}},\qquad
J = \int_{\beta}^{\alpha} \frac{n(\alpha-z)\,dz}{(\alpha-\gamma)\sqrt{R(z)}};
\]
on trouvera ensuite
\[
z=\alpha - (\alpha-\beta)\sn^2\omega = -\alpha\beta\gamma,
\]
d'o\`u
\[
\omega=\int_{-\alpha\beta\gamma}^{\alpha} \frac{n\,dz}{\sqrt{R(z)}},
\qquad
\int_0^{\omega} k^2\sn^2 x\,dx =
\int_{-\alpha\beta\gamma}^{\alpha} \frac{n(\alpha-z)\,dz}{(\alpha-\gamma)\sqrt{R(z)}}.
\]
Enfin, en partageant l'intervalle compris entre les limites, en deux parties,
l'une de $-\alpha\beta\gamma$ \`a $\beta$, et l'autre de $\beta$ \`a $\alpha$, l'\'equation se pr\'esentera, apr\`es
une r\'eduction facile, sous la forme suivante:
\[
\frac{2l}{g} \int_{\beta}^{\alpha} \frac{dz}{\sqrt{R(z)}} =
  \int_{\beta}^{\alpha} \frac{z\,dz}{\sqrt{R(z)}}
  \int_{-\alpha\beta\gamma}^{\beta} \frac{dz}{\sqrt{-R(z)}} -
  \int_{\beta}^{\alpha} \frac{dz}{\sqrt{R(z)}}
  \int_{-\alpha\beta\gamma}^{\beta} \frac{z\,dz}{\sqrt{-R(z)}}.
\]

La question qui vient d'\^etre trait\'ee termine les applications \`a la M\'ecani\-que
que j'ai annonc\'ees au commencement de ce travail, et j'arrive
maintenant, pour la consid\'erer dans toute sa g\'en\'eralit\'e, \`a l'\'equation
\[
D_x^z y = \left[ n(n+1)k^2\sn^2x+h \right] y,
\]
dont la solution n'a encore \'et\'e obtenue que pour $n=1$ et $n = 2$. Au moyen
des m\'ethodes de M.~Fuchs, permettant de reconna\^itre que l'int\'egrale
est une fonction uniforme de la variable, et de l'importante proposition de
M.~Picard, que cette int\'egrale est d\`es lors une fonction doublement p\'eriodique
%-----128.png--------------------------
de seconde esp\`ece, la solution de l'\'equation de Lam\'e est donn\'ee
directement par l'application de principes g\'en\'eraux s'appliquant aux \'equations
lin\'eaires d'un ordre quelconque. J'exposerai n\'eanmoins une m\'ethode
ind\'ependante de ces principes; je m'attacherai ensuite, et ce sera mon principal
but, \`a la question difficile de la d\'etermination, sous forme enti\`erement
explicite, des \'el\'ements de la solution. La consid\'eration du d\'eveloppement
en s\'erie, qu'on tire de l'\'equation propos\'ee lorsqu'on suppose
$x=iK'+\varepsilon$, aura, dans ce qui va suivre, une grande importance; voici,
en premier lieu, comment on l'obtient.


\mysection{XLIV.}


Soit, pour abr\'eger,
\[
\frac{1}{\sn^2\varepsilon} = \frac{1}{\varepsilon^2} + s_0 + s_1 \varepsilon ^2
  + \ldots + s_i \varepsilon^{2i} + \ldots ,
\]
les expressions des premiers coefficients \'etant
\begin{align*}
  s_0 &= \frac{1+k^2}{3}, \\
  s_1 &= \frac{1-k^2+k^4}{15}, \\
  s_2 &= \frac{2-3k^2-3k^4+2k^6}{189},\\
  s_3 &= \frac{2(1-k^2+k^4)^2}{675}.
\end{align*}
Je dis qu'on v\'erifie l'\'equation
\[
D_{\varepsilon}^2y = \left[\frac{n(n+1)}{\sn^2\varepsilon}+h\right]y,
\]
en posant
\[
y = \frac{1}{\varepsilon^n} + \frac{h_1}{\varepsilon^{n-2}} + \ldots + \frac{h_i}{\varepsilon^{n-2i}} + \ldots .
\]
La substitution donne en effet les conditions
\begin{align*}
  (n-1)(n-2)h_1 &= h\phantom{h_1} + n(n+1)(h_1+s_0), \\
  (n-3)(n-4)h_2 &= hh_1 + n(n+1)(h_2+s_0h_1+s_1), \\
\dotfillalign
\end{align*}
et nous allons voir qu'elles d\'eterminent de proche en proche les coefficients
$h_1$, $h_2$, \ldots. Mettons-les d'abord sous une forme plus simple; en \'eliminant
%-----129.png--------------------------
la quantit\'e $h$ au moyen de la premi\`ere, on aura, apr\`es une r\'eduction
facile,
\[
i(2n-2i+1)h_i = (2n-1)h_1h_{i-1}+m(s_1h_{i-2}+s_2h_{i-3}+\ldots+s_{i-1}),
\]
o\`u j'ai \'ecrit, pour abr\'eger, $n(n+1) = 2m$.

Or, le facteur $2n-2i+1$ ne pouvant jamais \^etre nul, on voit que le
coefficient de rang quelconque $h_i$ s'obtient au moyen des pr\'ec\'edents, $h_{i-1}$,
$h_{i-2}$, \ldots. En particulier, on trouve
\begin{align*}
  h_2 &= \frac{(2n-1)h_1^2}{2(2n-3)} - \frac{ms_1}{2(2n-3)},\\
  h_3 &= \frac{(2n-1)^2h_1^3}{6(2n-3)(2n-5)} -
         \frac{m(6n-7)s_1h_1}{6(2n-3)(2n-5)} -
         \frac{ms_2}{3(2n-5)}.
\end{align*}

Ce premier d\'eveloppement obtenu, nous en concluons imm\'ediatement
un second. Effectivement, le coefficient $n(n+1)$ ne change pas si l'on
remplace $n$ par $-(n + 1)$, de sorte qu'en d\'esignant par $h'_1$, $h'_2$, \ldots\ ce que
deviennent $h_1$, $h_2$, \ldots\ par ce changement, l'\'equation diff\'erentielle sera de
m\^eme satisfaite en prenant
\begin{align*}
y &= \varepsilon^{n+1}+h'_1\varepsilon^{n+3}+h'_2\varepsilon^{n+5}+\ldots,
\\
\intertext{ou bien}
y &= \varepsilon^{n+1}(1+h'_1\varepsilon^2+h'_2\varepsilon^4+\ldots).
\end{align*}
Je remarque enfin qu'en substituant dans l'expression
\[
D_{\varepsilon}^2 y - \left[\frac{n(n+1)}{\sn^2\varepsilon}+h\right]y
\]
la partie de la premi\`ere s\'erie repr\'esent\'ee par
\[
y=\frac{1}{\varepsilon^n}+\frac{h_2}{\varepsilon^{n-2}} +\ldots+\frac{h_i}{\varepsilon^{n-2i}},
\]
tous les termes en $\frac{1}{\varepsilon^{n+2}}$, \label{page121}
$\frac{1}{\varepsilon^n}$, \ldots, $\frac{1}{\varepsilon^{n-2i+2}}$
disparaissent, de sorte que le r\'esultat
ordonn\'e suivant les puissances croissantes de $\varepsilon$ commence par un terme en
$\frac{1}{\varepsilon^{n-2i}}$. On en conclut qu'en supposant $n$ pair et \'egal \`a $2\nu$, ou bien $n=2\nu-1$,
on n'aura aucun terme en $\frac{1}{\varepsilon}$, si l'on prend dans le premier cas
\[
y = \frac{1}{\varepsilon^{2\nu}} + \frac{h_1}{\varepsilon^{2\nu-2}}+\ldots +\frac{h_{\nu-1}}{\varepsilon^2}+h_{\nu},
\]
%-----130.png-----------------------------
et dans le second
\[
y = \frac{1}{\varepsilon^{2\nu-1}} + \frac{h_1}{\varepsilon^{2\nu-3}}
  + \ldots + \frac{h_{\nu-1}}{\varepsilon} + h_{\nu} \varepsilon.
\]
Ce point \'etabli, nous obtenons facilement, comme on va le voir, la solution
g\'en\'erale de l'\'equation de Lam\'e.


\mysection{XLV.}


Je consid\`ere l'\'el\'ement simple des fonctions doublement p\'eriodiques
de seconde esp\`ece, en le prenant sous la forme suivante:
\[
f(x) = e^{\lambda(x - iK')}\chi(x),
\]
o\`u l'on a, comme au \S~V,
\[
\chi(x) = \frac{H'(0)H(x + \omega)}{\Theta(\omega)\Theta(x)} \,
    e^{-\frac{\Theta'(\omega)}{\Theta(\omega)} (x - iK')
      + \frac{i\pi\omega}{2K} }.
\]

Le r\'esidu qui correspond au p\^ole unique $x = iK'$ sera ainsi \'egal \`a
l'unit\'e, et nous pourrons \'ecrire
\[
f(iK' + \varepsilon) = \frac{1}{\varepsilon} + H_0 + H_1\varepsilon + \ldots + H_i\varepsilon^i + \ldots.
\]
Cela pos\'e, je dis que les expressions
\begin{align*}
F(x) &= -   \frac{D_x^{2\nu-1} f(x)}{\Gamma(2\nu    )}
    -h_1\frac{D_x^{2\nu-3} f(x)}{\Gamma(2\nu - 2)}
    - \ldots - h_{\nu-1} D_x f(x), \\
F(x) &= +   \frac{D_x^{2\nu-2} f(x)}{\Gamma(2\nu - 1)}
    +h_1\frac{D_x^{2\nu-4}   f(x)}{\Gamma(2\nu - 3)}
    + \ldots + h_{\nu-1} f(x)
\end{align*}
satisferont, suivant les cas de $n = 2\nu$ et $n = 2\nu - 1$, \`a l'\'equation diff\'erentielle
en d\'eterminant convenablement les constantes $\omega$ et~$\lambda$.

Pour le d\'emontrer, je remarque que, si l'on pose $x = iK' + \varepsilon$, les
parties principales de leurs d\'eveloppements proviendront du seul terme $\frac{1}{\varepsilon}$
qui entre dans $f(iK' + \varepsilon)$, et seront, par cons\'equent,
\[
\frac{1}{\varepsilon^{2\nu}} + \frac{h_1}{\varepsilon^{2\nu-2}} + \ldots + \frac{h_{\nu-1}}{\varepsilon^2}
\]
et
\[
\frac{1}{\varepsilon^{2\nu-1}} + \frac{h_1}{\varepsilon^{2\nu-3}} + \ldots + \frac{h_{\nu-1}}{\varepsilon}.
\]
%-----131.png-----------------------------

Disposons maintenant de $\omega$ et $\lambda$, de telle sorte que dans le premier
cas le terme constant soit \'egal \`a $h_{\nu}$ et le coefficient de $\varepsilon$, dans le suivant,
\'egal \`a z\'ero; nous poserons pour cela les conditions
\begin{align*}
H_{2\nu-1} + h_1 H_{2\nu-3} + h_2 H_{2\nu-5} + \ldots + h_{\nu-1} H_1 + h_{\nu} &= 0, \\
2\nu H_{2\nu} + (2\nu-2) h_1 H_{2\nu-2} + (2\nu-4) h_2 H_{2\nu-4} + \ldots + 2 h_{\nu-1} H_2 &= 0.
\end{align*}
Et semblablement, dans le second cas, faisons en sorte que le terme constant
soit nul et le coefficient de $\varepsilon$ \'egal \`a $h_{\nu}$, en \'ecrivant\label{page123a}
\begin{align*}
H_{2\nu-2} + h_1 H_{2\nu-4} + h_2 H_{2\nu-6} + \ldots + h_{\nu-1} H_0 &= 0,\\
(2\nu-1) H_{2\nu-1} + (2\nu-3) h_1 H_{2\nu-3} + \ldots + h_{\nu-1} H_1 - h_{\nu} &= 0.
\end{align*}
On a donc ces deux d\'eveloppements, \`a savoir:
\[
F(iK' + \varepsilon) = \frac{1}{\varepsilon^{2\nu}}
    + \frac{h_1}{\varepsilon^{2\nu-2}} + \ldots
    + \frac{h_{\nu-1}}{\varepsilon^2} + h_{\nu} + \ldots,
\]
puis
\[
F(iK' + \varepsilon) = \frac{1}{\varepsilon^{2\nu-1}}
    + \frac{h_1}{\varepsilon^{2\nu-3}} - \ldots
    + \frac{h_{\nu-2}}{\varepsilon} + h_{\nu} \varepsilon + \ldots;
\]
il en r\'esulte que les deux fonctions doublement p\'eriodiques de seconde
esp\`ece
\[
D_x^2 F(x) - \left[n(n+1) k^2 \sn^2 x + h \right] F(x),
\]
\'etant finies pour $x=iK'$, sont par cons\'equent nulles. Nous avons ainsi
d\'emontr\'e que l'\'equation se trouve v\'erifi\'ee en faisant $y=F(x)$, de sorte
que l'expression
\[
y = CF(x) + C'F(-x)
\]
en donne l'int\'egrale g\'en\'erale.


\mysection{XLVI.}\label{page123}


La question qui s'offre maintenant est d'obtenir $\omega$ et $\lambda$ au moyen des
relations pr\'ec\'edentes, qui sont alg\'ebriques en $\sn\omega$ et $\lambda$. Or, on est de la
sorte amen\'e \`a un probl\`eme d'Alg\`ebre dont la difficult\'e se montre au premier
coup d'{\oe}il et r\'esulte de la complication des coefficients $H_0$, $H_1$, \ldots.

Revenons, en effet, au d\'eveloppement d\'ej\`a donn\'e \S~V, \`a savoir:
\[
\chi(iK' + \varepsilon) = \frac{1}{\varepsilon} -
\frac{1}{2}\Omega\varepsilon
    - \frac{1}{3} \Omega_1\varepsilon^2
    - \frac{1}{8} \Omega_2\varepsilon^3
    - \frac{1}{30}\Omega_3\varepsilon^4 - \ldots,
\]
%-----132.png--------------------------
o\`u l'on a
\begin{alignat*}{2}
  &\Omega &&= k^2 \sn^2 \omega - \frac{1+k^2}{3}, \\
  &\Omega_1 &&= k^2 \sn \omega \cn \omega \dn \omega, \\
  &\Omega_2 &&= k^4 \sn^4 \omega -
              \frac{2(k^2+k^4)}{3} \sn^2\omega -
              \frac{7-22k^2+7k^4}{45}, \\
  &\Omega_3 &&= k^2 \sn \omega \cn \omega \dn \omega
              \left( k^2 \sn^2 \omega - \frac{1+k^2}{3}\right),\\
\dotfillalignat
\end{alignat*}

Les coefficients $H_0$, $H_1$, \ldots\ r\'esultant de l'identit\'e
\[
\frac{1}{\varepsilon} + H_0 + H_0\varepsilon + \ldots =
  \left( 1+\lambda\varepsilon + \frac{\lambda^2\varepsilon}{2} + \ldots\right)
  \left( \frac{1}{\varepsilon}-\frac{1}{2}\Omega\varepsilon - \ldots\right)
\]
seront
\begin{align*}
H_0 &= \lambda, \\
H_1 &= \tfrac{1}{2}(\lambda^2 - \Omega), \\
H_2 &= \tfrac{1}{6}(\lambda^3 - 3\Omega\lambda - 2\Omega_1), \\
H_3 &= \tfrac{1}{24}(\lambda^4 - 6\Omega\lambda^2 - 8\Omega_1\lambda - 3\Omega_2), \\
\dotfillalign
\end{align*}
et l'on voit que, $H_n$ \'etant du degr\'e $n+1$ en $\lambda$, l'une de nos deux \'equations
est, par rapport \`a cette quantit\'e, du degr\'e $n$, et la seconde du degr\'e $n+1$.
A l'\'egard de $\sn\omega$, une nouvelle complication se pr\'esente en raison du
facteur irrationnel $\cn\omega\dn\omega$, qui entre dans $\Omega_1$, $\Omega_3$, $\Omega_5$, \ldots; aussi para\^it-il
impossible de conclure de leur forme actuelle qu'elles ne donnent
pour $\lambda^2$ et $\sn^2\omega$ qu'une seule et unique d\'etermination. Et si l'on consid\`ere
ces quantit\'es comme des coordonn\'ees, en se pla\c{c}ant au point de
vue de la G\'eom\'etrie, on verra ais\'ement que les courbes repr\'esent\'ees par
nos deux \'equations n'ont aucun point d'intersection ind\'ependant de la
constante $h$ qui entre sous forme rationnelle et enti\`ere dans les coefficients.
Il n'est donc pas possible d'employer les m\'ethodes si simples de
Clebsch et de Chasles qui permettent de reconna\^itre, \emph{a priori} et sans calcul,
que les points d'un lieu g\'eom\'etrique se d\'eterminent individuellement en
fonction d'un param\`etre. Le cas de $n = 3$, qui sera trait\'e tout \`a l'heure,
fera voir en effet que les intersections des deux courbes se trouvent, \`a
l'exception d'une seule, rejet\'ees \`a l'infini. Mais, avant d'y arriver, je ferai
encore cette remarque, qu'on peut joindre aux \'equations d\'ej\`a obtenues
une infinit\'e d'autres, dont voici l'origine.
%-----133.png--------------------------

Nous avons vu au \S~XLIV %sic!
que l'\'equation de Lam\'e donne, en faisant
$x=iK+\varepsilon$, ces deux d\'eveloppements, \`a savoir:
\begin{align*}
y &= \frac{1}{\varepsilon^n} + \frac{h_1}{\varepsilon^{n-2}} +
  \frac{h_2}{\varepsilon^{n-4}} + \ldots, \\
y &= \varepsilon^{n+1} + h'_1\varepsilon^{n+3} +
  h'_2\varepsilon^{n+5} + \ldots.
\end{align*}
Il en r\'esulte que, si l'on pose de m\^eme $x=iK'+\varepsilon$ dans la solution repr\'esent\'ee
par $F(x)$, nous aurons, en d\'esignant par $C$ une constante dont on
obtiendra bient\^ot la valeur,
\[
  F(iK'+\varepsilon) = \frac{1}{\varepsilon^n} +
    \frac{h_1}{\varepsilon^{n-2}} +
    \frac{h_2}{\varepsilon^{n-4}} + \ldots
  + C\left(\varepsilon^{n+1} + h'_1\varepsilon^{n+3} + h'_2\varepsilon^{n+5} + \ldots\right).
\]
On peut donc identifier ce d\'eveloppement avec celui que donnent l'une ou
l'autre des deux formules
\begin{align*}
  F(x) &=   -\frac{D_x^{2\nu-1}f(x)}{\Gamma(2\nu)} -
          h_1\frac{D_x^{2\nu-3}f(x)}{\Gamma(2\nu-2)} - \ldots -
          h_{\nu-1}D_xf(x), \\
  F(x) &=   +\frac{D_x^{2\nu-2}f(x)}{\Gamma(2\nu-1)} +
          h_1\frac{D_x^{2\nu-4}f(x)}{\Gamma(2\nu-3)} + \ldots +
          h_{\nu-1}f(x)
\end{align*}
lorsqu'on pose $x=iK'+\varepsilon$. Bornons-nous, pour abr\'eger, au cas de $n=2\nu$,
et repr\'esentons la partie qui proc\`ede, suivant les puissances positives de $\varepsilon$,
par
\[
\sum_{i\geq 0} \mathfrak{H}_i\varepsilon^i.
\]
On trouve facilement, si l'on \'ecrit
\[
m_i = \frac{m(m-1)\ldots(m-i+1)}{1\cntrdot 2\ldots i},
\]
l'expression\label{page125}
\begin{align*}
  \mathfrak{H}_i =& -(i+2\nu-1)_i H_{i+2\nu-1} - (i+2\nu-3)h_1H_{i+2\nu-3} \\
  & - (i+2\nu-5)_i h_2H_{i+2\nu-5} - \ldots - (i+1)h_{\nu-1}H_{i+1}.
\end{align*}
Nous aurons donc, pour $i = 1, 3, 5, \ldots, 2\nu-1$, les \'equations
\[
\mathfrak{H}_i = 0;
\]
on trouvera ensuite, pour les valeurs paires de l'indice,
\[
\mathfrak{H}_{2i} = h_{i+\nu},
\]
%-----134.png--------------------
et enfin, pour les valeurs impaires sup\'erieures \`a $2\nu - 1$,
\[
\mathfrak{H}_{2i+2\nu+1} = Ch'_i.
\]
Telles sont les relations, en nombre illimit\'e, qui doivent toutes r\'esulter
des deux que nous avons donn\'ees en premier lieu, \`a savoir:
\[
\mathfrak{H}_1 = 0, \qquad \mathfrak{H}_0 = -h_{\nu};
\]
on est amen\'e ainsi \`a se demander si leurs premiers membres, $\mathfrak{H}_i$, $\mathfrak{H}_{2i} - h_{i+\nu}$,
$\mathfrak{H}_{2i+2\nu+1} - Ch'_i$, ne s'exprimeraient point, sous forme rationnelle et enti\`ere,
par les fonctions $\mathfrak{H}_1$ et $\mathfrak{H}_0-h_{\nu}$. Mais je laisserai enti\`erement de c\^ot\'e cette
question difficile, et j'arrive imm\'ediatement \`a la r\'esolution des \'equations
relatives au cas de $n = 3$.


\mysection{XLVII.}


Ces \'equations ont \'et\'e donn\'ees au \S~XXXV, et sont
\begin{align*}
 H_2 + h_1 H_0 &= 0, \\
3H_3 + h_1 H_1 &= h_2.
\end{align*}

Si l'on met en \'evidence les quantit\'es $\Omega$, et qu'on fasse $h_1 = \frac{l}{2}$, ce qui
donne
\begin{align*}
h &= -4(1 + k^2) - 5l,\\
h_2 &= \frac{5l^2}{24} - s_1,
\end{align*}
elles prennent la forme suivante:
\begin{align*}
\lambda^3 - 3\Omega\lambda   - 2\Omega_1 + 3l\lambda &= 0, \\
\lambda^4 - 8\Omega\lambda^2 - 8\Omega_1\lambda - 3\Omega_2 + 2l\lambda^2 &= \frac{5l^2}{3} - 8s_1.
\end{align*}

Cela \'etant, j'emploie ces identit\'es, \`a savoir:
\begin{align*}
\Omega^2 - \Omega_2 &= 4s_1, \\
\Omega\Omega_2 - \Omega_1^2 &= \Omega s_1 + 7s_2,
\end{align*}
et je remarque qu'on en tire, par l'\'elimination de $\Omega_1$ et $\Omega_2$, deux \'equations
du second degr\'e en $\Omega$. Mais il convient d'introduire $H_1$ au lieu de $\Omega$; en
%-----135.png-------------------------
faisant alors, pour un moment,
\begin{align*}
a &= 1 - k^2 + k^4, \\
b &= 2 - 3k^2 - 3k^4 + 2k^6,
\end{align*}
ces relations seront
\begin{align*}
36H_1^2 - 12l H_1 + 36l\lambda^2 + 5l^2 - 4a      &= 0, \\
72l H_1^2 - 6(5l^2 - a) H_1 + 72l^2 \lambda^2 - b &= 0.
\end{align*}

\'Eliminons $\lambda^2$, elles donnent imm\'ediatement % original has no acute on E; cf 146.png
\[
H_1 = -\frac{10l^3 - 3al - b}{6(l^2-a)};
\]
nous obtenons ensuite
\[
\lambda^2 = -\frac{4(l^2 - a)^3 + (11 l^3 - 9al - b)^2}{36 l (l^2-a)^2},
\]
ou bien
\[
\lambda^2 = -\frac{\varphi(l)}{36 l(l^2 -a)^2},
\]
si l'on pose, pour abr\'eger,
\[
\varphi(l) = 125l^6 - 210al^4 - 22bl^3 + 93a^2l^2 + 18abl + b^2 - 4a^3,
\]
soit encore
\begin{align*}
\psi(l) &= 5l^6 + 6al^4 - 10bl^3 - 3a^2l^2 + 6abl + b^2 - 4a^3 \\
        &= \varphi(l) - 12l(l^2 - a)(10l^3 - 8al - b);
\end{align*}
de la relation $\lambda^2 - 2H_1 = \Omega$ on conclura:
\[
\Omega = k^2 \sn^2 \omega - \frac{1+k^2}{3} = -\frac{\psi(l)}{36l(l^2 - a)^2}.
\]

Enfin j'observe qu'on d\'eduit des \'equations propos\'ees la valeur de $\Omega_1$
exprim\'ee en $\Omega$ et $\lambda$, par cette formule,
\[
2\Omega_1 = (\lambda^2 - 3\Omega + 3l) \lambda;
\]
faisant donc
\[
\chi(l) = l^6 - 6al^4 + 4bl^3 - 3a^2l^2 - b^2 + 4a^2,
\]
nous parvenons encore \`a la relation
\[
\Omega_1 = k^2 \sn \omega \cn \omega \dn \omega = -\frac{\chi(l)\lambda}{36l(l^2-a)^2}.
\]

Le signe de $\lambda$ se trouve ainsi d\'etermin\'e par celui de $\omega$, et la solution
%-----136.png-------------------------
compl\`ete de l'\'equation de Lam\'e dans le cas de $n = 3$ est obtenue sans
aucune ambigu\"it\'e au moyen de la fonction
\[
\frac{H(x + \omega)}{\Theta(x)}\,
  e^{ \left[ \lambda - \frac{\Theta'(\omega)}{\Theta(\omega)} \right] x}.
\]

On n'a toutefois pas mis en \'evidence dans les formules pr\'ec\'edentes les
valeurs de la constante $l$ qui donnent les solutions doublement p\'eriodiques,
ou les fonctions particuli\`eres de seconde esp\`ece de M.~Mittag-Leffler,
comme nous l'avons fait dans le cas de $n = 2$.

Voici, dans ce but, les nouvelles expressions qu'on en d\'eduit.

Posons, en premier lieu,
\begin{align*}
P &= 5l^2 - 2(1+k^2)l - 3(1-k^2)^2, \\
Q &= 5l^2 - 2(1-2k^2)l - 3,\\
R &= 5l^2 - 2(k^2-2)l - 3k^4,\\
S &= 36l,
\end{align*}
et, d'autre part,
\begin{align*}
A &= l^2 - (1+k^2)l - 3k^2,\\
B &= l^2 - (1-2k^2)l + 3(k^2-k^4),\\
C &= l^2 - (k^2-2)l - 3(1-k^2),\\
D &= l^2 - 1 + k^2 - k^4,
\end{align*}
on aura
\begin{align*}
\lambda^2        &= -\frac{PQR}{SD^2},  \\
k^2 \sn^2 \omega &= -\frac{PA^2}{SD^2}, \\
k^2 \cn^2 \omega &= +\frac{QB^2}{SD^2}, \\
\dn^2 \omega     &= +\frac{RC^2}{SD^2},
\end{align*}
et enfin, pour \'etablir la correspondance des signes entre $\omega$ et $\lambda$, l'\'equation
\[
k^2 \sn \omega \cn \omega \dn \omega = -\frac{ABC \lambda}{SD^2}.
\]

Cela \'etant, ce sont les conditions $P = 0$, $Q = 0$, $R = 0$, $S = 0$ qui
donnent les solutions doublement p\'eriodiques, au nombre de sept, tandis
qu'on obtient les fonctions de M.~Mittag-Leffler en posant $A = 0$, $B = 0$,
$C = 0$, $D = 0$. Mais je laisse de c\^ot\'e l'\'etude d\'etaill\'ee de ces formules, en
%-----137.png----------------------
me bornant \`a la remarque suivante, sur laquelle je reviendrai plus tard.
Exprimons les quantit\'es $k^2\sn^2\omega$, $k^2\cn^2\omega$, $\dn^2\omega$, en partant de l'\'equation
\[
k^2\sn^2\omega - \frac{1+k^2}{3} = -\frac{\psi(l)}{36l(l^2-a)^2},
\]
de cette nouvelle mani\`ere, \`a savoir:
\begin{align*}
  k^2\sn^2\omega &= \frac{12l(l^2-a)^2(1+k^2)-\psi(l)}{36l(l^2-a)^2},\\
  k^2\cn^2\omega &= \frac{12l(l^2-a)^2(2k^2-1)+\psi(l)}{36l(l^2-a)^2},\\
  \dn^2\omega &= \frac{12l(l^2-a)^2(2-k^2)+\psi(l)}{36l(l^2-a)^2}.
\end{align*}
On conclura facilement de l'\'egalit\'e
\[
k^4\sn^2\omega\cn^2\omega\dn^2\omega =
  \frac{\varphi(l)\chi^2(l)}{\left[36l(l^2-a)^2\right]^3}
\]
la relation que voici:
\[
\psi^3(l) - 3\cntrdot 12^2al^2(l^2-a)^4\psi(l)+12^3bl^3(l^2-a)^6 = \varphi(l)\chi^2(l).
\]
Or elle conduit \`a cette cons\'equence, qu'en posant
\[
y = \frac{\psi(l)}{12l(l^2-a)^2},
\]
on a
\[
\int\frac{dy}{\sqrt{y^3-3ay+b}} =
  2\sqrt{3}\int\frac{(5l^2-a)\,dl}{\sqrt{l\varphi(l)}};
\]
c'est donc un exemple de r\'eduction d'une int\'egrale hyperelliptique de seconde
classe \`a l'int\'egrale elliptique de premi\`ere esp\`ece.


\mysection{XLVIII.}


La m\'ethode g\'en\'erale que je vais exposer maintenant pour la d\'etermi\-nation
des constantes $\omega$ et $\lambda$ repose principalement sur la consid\'eration du
produit des solutions de l'\'equation de Lam\'e, qui viennent d'\^etre repr\'esent\'ees
par $F(x)$ et $F(-x)$. Et, d'abord, on remarquera que, ayant
\begin{alignat*}{2}
& F(x+2K)   &&= \mu F(x),\\
& F(x+2iK') &&= \mu'F(x)
\end{alignat*}
%-----138.png-------------------------
et, par suite,
\begin{alignat*}{2}
& F(-x-2K)   &&= \frac{1}{\mu} F(-x),\\
& F(-x-2iK') &&= \frac{1}{\mu'}F(-x),
\end{alignat*}
ce produit est une fonction doublement p\'eriodique de premi\`ere esp\`ece,
qui a pour p\^ole unique $x = iK'$. Voici, en cons\'equence, comment s'obtient
son expression sous forme enti\`erement explicite.

Soit
\[
\Phi(x) = (-1)^n \mu' F(x) F(-x),
\]
le facteur $\mu'$ ayant \'et\'e introduit, pour pouvoir \'ecrire
\begin{align*}
\Phi(iK' + \varepsilon) &= (-1)^n \mu' F(iK' + \varepsilon) F(-iK'-\varepsilon) \\
                        &= (-1)^n      F(iK' + \varepsilon) F( iK'-\varepsilon).
\end{align*}
Cela \'etant et posant, pour abr\'eger,
\begin{align*}
S   &= \frac{1}{\varepsilon^n} + \frac{h_1}{\varepsilon^{n-2}} +
  \frac{h_2}{\varepsilon^{n-4}} + \ldots,\\
S_1 &= C \left(\varepsilon^{n+1} + h'_1 \varepsilon^{n+3} +
  h'_2 \varepsilon^{n+5} + \ldots\right),
\end{align*}
nous aurons
\begin{align*}
F(iK' + \varepsilon) &= S + S_1, \\
F(iK' - \varepsilon) &= (-1)^n (S - S_1),
\end{align*}
d'o\`u, par cons\'equent,
\[
\Phi(iK' + \varepsilon) = S^2 - S_1^2.
\]
On voit ainsi que la partie principale de d\'eveloppement suivant les puissances
croissantes de $\varepsilon$ est donn\'ee par le premier terme $S^2$, et ne d\'epend
point de la constante $C$, entrant dans le second terme, que nous ne connaissons
pas encore. Faisons donc
\[
S^2 = \frac{1}{\varepsilon^{2n}}
    + \frac{A_1}{\varepsilon^{2n-2}}
    + \frac{A_2}{\varepsilon^{2n-4}} + \ldots
    + \frac{A_{n-1}}{\varepsilon^2} + \ldots;
\]
les coefficients $A_1$, $A_2$, \ldots\ seront
\begin{align*}
A_1 &= 2h_1, \\
A_2 &= 2h_2 + h^2_1, \\
A_3 &= 2h_3 + 2h_1h_2, \\
\dotfillalign
\end{align*}
et l'on en conclut que, $h_i$ \'etant un polyn\^ome de degr\'e $i$ en $h_1$, il en est de
%-----139.png------------------------
m\^eme, en g\'en\'eral, pour un coefficient de rang quelconque $A_i$. Maintenant
l'expression cherch\'ee d\'ecoule de la formule de d\'ecomposition en \'el\'ements
simples, qui a \'et\'e donn\'ee au \S~II. Nous obtenons ainsi
\begin{align*}
\Phi(x) = -   \frac{D_x^{2n-1} \left[ \frac{\Theta'(x)}{\Theta(x)} \right]}{\Gamma(2n)}
    &-A_1\frac{D_x^{2n-3}   \left[ \frac{\Theta'(x)}{\Theta(x)} \right]}{\Gamma(2n-2)}
     -A_2\frac{D_x^{2n-5}   \left[ \frac{\Theta'(x)}{\Theta(x)} \right]}{\Gamma(2n-4)} -\ldots \\
    &-A_{n-1} D_x         \left[ \frac{\Theta'(x)}{\Theta(x)} \right] + \textrm{const.}
\end{align*}
La relation \'el\'ementaire
\[
\frac{\Theta'(x)}{\Theta(x)} = \frac{J}{K} - k^2 \sn^2 x
\]
donnera ensuite, sous une autre forme, en d\'esignant par $A$ une nouvelle
constante,
\begin{align*}
\Phi(x) =     \frac{D_x^{2n-2} (k^2 \sn^2 x)}{\Gamma(2n)}
    &+ A_1 \frac{D^{2n-4}   (k^2 \sn^2 x)}{\Gamma(2n-2)}
     + A_2 \frac{D^{2n-6}   (k^2 \sn^2 x)}{\Gamma(2n-4)} + \ldots\\
    &+ A_{n-1}              (k^2 \sn^2 x) + A.
\end{align*}
Pour la d\'eterminer, nous emploierons, en outre de la partie principale de
la s\'erie $S^2$, le terme ind\'ependant de $\varepsilon$, qui sera d\'esign\'e par $A_n$. En d\'eduisant
ce m\^eme terme de l'expression de $\Phi(x)$, et se rappelant qu'on a fait
\[
\frac{1}{\sn^2 \varepsilon} = \frac{1}{\varepsilon^2}
  + s_0 + s_1 \varepsilon^2 + \ldots + s_i \varepsilon^{2i} + \ldots,
\]
nous trouvons imm\'ediatement
\[
A = A_n - A_{n-1} s_0 - A_{n-2} \frac{s_1}{3} - \ldots
  - A_1\frac{s_{n-2}}{2n-3} - \frac{s_{n-1}}{2n - 1}.
\]

Beaucoup d'autres expressions s'obtiennent par un proc\'ed\'e semblable
en fonction lin\'eaire de d\'eriv\'ees successives de $k^2 \sn^2 x$, celles-ci, par exemple,
\[
D_x^{\alpha} F(x) \cntrdot D_x^{\beta} F(-x),
\]
que je vais consid\'erer dans le cas particulier de $\alpha = 1$, $\beta = 1$.

Soit alors
\[
\Phi_1(x) = (-1)^{n+1} \mu' F'(x) F'(-x),
\]
et d\'esignons par $S'$ et $S'_1$ les d\'eriv\'ees par rapport \`a $\varepsilon$ des s\'eries $S$ et $S_1$, de
sorte qu'on ait
\begin{align*}
F'(iK' + \varepsilon) &= S' + S'_1, \\
F'(iK' - \varepsilon) &= (-1)^{n+1}(S' - S'_1).
\end{align*}
%-----140.png------------------------
De la relation
\[
\Phi_1(iK' + \varepsilon) = (-1)^{n+1} F'(iK' + \varepsilon) F'(iK' - \varepsilon),
\]
on conclura cette expression, savoir
\[
\Phi_1(iK' + \varepsilon) = S'^2 - S'^2_1.
\]
Faisant donc, comme tout \`a l'heure,
\[
S'^2 = \frac{n^2}{\varepsilon^{2n+2}} + \frac{B_1}{\varepsilon^{2n}}
     + \frac{B_2}{\varepsilon^{2n-2}} + \ldots + \frac{B_n}{\varepsilon^2}
     + B_{n+1} + \ldots,
\]
o\`u le coefficient $B_i$ est encore un polyn\^ome en $h_1$ de degr\'e $i$, nous aurons
\begin{align*}
\Phi_1(x) = n^2 \frac{D_x^{2n}  (k^2 \sn^2 x)}{\Gamma(2n + 2)}
& + B_1 \frac{D_x^{2n-2}(k^2 \sn^2 x)}{\Gamma(2n)}
  + B_2 \frac{D_x^{2n-4}  (k^2 \sn^2 x)}{\Gamma(2n - 2)} + \ldots\\
& + B_n                 (k^2 \sn^2 x) + B,
\end{align*}
et la constante sera donn\'ee par la formule
\[
B = B_{n+1} - B_n s_0 - B_{n-1} \frac{s_1}{3} - \ldots
  - B_1\frac{s_{n-1}}{2n-1} - n^2\frac{s_n}{2n+1}.
\]

J'envisage enfin le d\'eterminant fonctionnel form\'e avec les solutions
$F(x)$ et $F(-x)$ de l'\'equation de Lam\'e, et je pose
\[
\Phi_2(x) = (-1)^{n+1} \mu' \left[ F(x) F'(-x) + F'(x) F(-x) \right].
\]

La relation suivante, qui s'obtient ais\'ement, et dont le second membre
ne contient que des termes entiers en $\varepsilon$, \`a savoir
\[
\Phi_2(iK' + \varepsilon) = 2(SS'_1 - S'S_1) = 2(2n + 1)C + \ldots,
\]
donne, comme on le voit, la proposition bien connue que cette fonction
est constante; nous allons en obtenir la valeur en la mettant sous la forme
\[
(2n + 1)C = \sqrt{N},
\]
que nous garderons d\'esormais.


\mysection{XLIX.}


J'observe, \`a cet effet, que de l'identit\'e
\[
(SS' - S_1 S'_1)^2 = (SS'_1 - S_1 S')^2 + (S^2 - S_1^2)(S'^2 - S'^2_1)
\]
on conclut imm\'ediatement, entre les fonctions dont il vient d'\^etre question,
%-----141.png------------------
la relation suivante
\[
\frac{1}{4}\Phi'^2(iK'+\varepsilon) =
  \frac{1}{4}\Phi_2^2(iK'+\varepsilon) +
  \Phi(iK'+\varepsilon)\Phi_1(iK'+\varepsilon)
\]
et, par cons\'equent,
\[
\tfrac{1}{4}\Phi'^2(x) = N + \Phi(x)\Phi_1(x).
\]

Elle fait voir qu'en attribuant \`a la variable une valeur particuli\`ere, en
supposant, par exemple, $x = 0$, $N$ \label{page133a}
s'obtient comme un polyn\^ome entier en $h_1$ du degr\'e $2n + 1$,
puisque cette quantit\'e entre, comme on l'a vu, au degr\'e $n$ dans
$\Phi(x)$, et au degr\'e $n + 1$ dans $\Phi_1(x)$. Ce point \'etabli, nous
remarquons que, en posant la condition $N = 0$, le d\'eterminant
fonctionnel $\Phi_2(x)$ est nul, de sorte que le quotient
$\frac{F(x)}{F(-x)}$ se r\'eduit alors \`a une constante.
D\'esignons-la pour un instant par $A$, on voit que le changement de
$x$ en $-x$ donne $A = \frac{1}{A}$; on a donc $A = \pm1$, et, par cons\'equent,
\[
F(-x) = \pm F(x).
\]

Rempla\c{c}ons ensuite $x$ par $x + 2K$ et $x + 2iK'$: le quotient se reproduit
multipli\'e par $\mu^2$ et $\mu'^2$, ainsi il faut poser $\mu^2 = 1$, $\mu'^2 = 1$, c'est-\`a-dire
$\mu = \pm1$, $\mu' = \pm1$.

La condition $N = 0$ d\'etermine donc les valeurs de $h$, pour lesquelles
l'\'equation de Lam\'e est v\'erifi\'ee par des fonctions doublement p\'eriodiques.
Ce sont ces solutions, auxquelles est attach\'e \`a jamais le nom du grand
g\'eom\`etre, et dont les propri\'et\'es lui ont permis de traiter pour la premi\`ere
fois le pro\-bl\`eme difficile de la d\'etermination des temp\'eratures d'un
ellipso\"ide, lorsque l'on donne en chaque point la temp\'erature de la surface.
Elles s'offrent en ce moment comme un cas singulier de l'\'equation
diff\'erentielle, o\`u l'int\'egrale cesse d'\^etre repr\'esent\'ee par la
formule\label{page133}
\[
y = CF(x) + C'F(-x)
\]
et subit un changement de forme analytique. Je me borne \`a les signaler
sous ce point de vue, devant bient\^ot y revenir, et je reprends, pour en
tirer une nouvelle cons\'equence, l'\'equation
\[
\tfrac{1}{4}\Phi'^2(x) = N + \Phi(x)\Phi_1(x).
\]

Introduisons $\sn^2x$ pour variable, en posant $\sn^2x = t$; on voit que
$\Phi(x)$ et $\Phi_1(x)$, ne contenant que des d\'eriv\'ees d'ordre pair $\sn^2x$, deviendront
des polyn\^omes entiers en $t$ des degr\'es $n$ et $n + 1$, que je d\'esignerai
%-----142.png----------------------
par $\Pi(t)$ et $\Pi_1(t)$. Soit encore
\[
R(t) = t(1-t)(1-k^2t);
\]
la relation consid\'er\'ee prend cette forme
\[
R(t)\Pi'^2(t) = N + \Pi(t)\Pi_1(t);
\]
et voici la remarque, importante pour notre objet, \`a laquelle elle donne
lieu.

D\'eveloppons la fonction rationnelle $\frac{\Pi'(t)}{\Pi(t)}$ en fraction continue, et distinguons,
dans la s\'erie des r\'eduites, celle dont le d\'enominateur est du
degr\'e $\nu$, dans les deux cas de $n=2\nu$ et $n=2\nu-1$. Si on la repr\'esente
par $\frac{\theta(t)}{\varphi(t)}$, le d\'eveloppement, suivant les puissances d\'ecroissantes de $t$, de la
diff\'erence
\[
\frac{\Pi'(t)}{\Pi(t)}\varphi(t)-\theta(t),
\]
commencera ainsi par un terme en $\frac{1}{t^{\nu+1}}$, et, en posant
\[
\Pi'(t)\varphi(t) - \Pi(t)\theta(t) = \psi(t),
\]
on voit que, dans le premier cas, $\psi(t)$ sera un polyn\^ome de degr\'e $\nu - 1$,
et, dans le second, de degr\'e $\nu - 2$. Cela \'etant, je consid\`ere l'expression
suivante
\[
N\varphi^2(t) - R(t)\psi^2(t);
\]
on trouve d'abord ais\'ement, en employant la relation propos\'ee et la valeur
de $\psi(t)$, qu'elle devient
\[
\Pi(t) \left[-\varphi^2(t)\Pi_1(t) + 2\varphi(t)\theta(t)R(t)\Pi'(t)
  - \theta^2(t)R(t)\Pi(t) \right],
\]
et contient, par cons\'equent, en facteur, le polyn\^ome $\Pi(t)$. On v\'erifie ensuite
qu'elle est de degr\'e $n+1$ en $t$, dans les deux cas de $n= 2\nu$ et
$n= 2\nu-1$; nous pouvons ainsi poser
\[
N\varphi^2(t)-R(t)\psi^2(t)=\Pi(t)(gt-g'),
\]
et nous allons voir que $\omega$ est donn\'e par la formule
\[
\sn^2\omega = \frac{g'}{g},
\]
o\`u le second membre est une fonction rationnelle de $h$.
%-----143.png----------------------


\mysection{L.}\label{page135}


Consid\'erons dans ce but une nouvelle fonction doublement p\'eriodique
d\'efinie de la mani\`ere suivante
\[
\Psi(x) = -\mu' f(-x)F(x),
\]
en faisant toujours
\[
f(x) = e^{\lambda(x-iK')}\chi(x),
\]
de sorte que les deux facteurs $f(-x)$ et $F(x)$ soient encore des fonctions de
seconde esp\`ece \`a multiplicateurs r\'eciproques. Nous aurons d'abord
\[
\Psi(x)\Psi(-x) = \mu'^2 f(x)f(-x)F(x)F(-x),
\]
et, en employant l'\'egalit\'e, qu'il est facile d'\'etablir:
\[
\mu'f(x)f(-x) = -k^2(\sn^2x - \sn^2\omega),
\]
on parvient \`a cette relation
\[
\Psi(x)\Psi(-x) = (-1)^{n+1}k^2
  \left(\sn^2x-\sn^2\omega\right) \Phi(x),
\]
dont on va voir l'importance. Formons \`a cet effet l'expression de $\Psi(x)$
qui s'obtiendra sous forme lin\'eaire au moyen des d\'eriv\'ees successives
de $k^2\sn^2 x$, puisque cette fonction, comme celles qui ont \'et\'e pr\'ec\'edemment
introduites, a pour seul p\^ole $x = iK'$. Nous d\'eduirons pour cela un
d\'eveloppement, suivant les puissances croissantes de $\varepsilon$, de l'\'equation
\begin{align*}
\Psi(iK'+\varepsilon) &= -f(iK'-\varepsilon)F(iK'+\varepsilon) \\
  &= \left( \frac{1}{\varepsilon}-H_0+H_1\varepsilon-H_2\varepsilon^2+\ldots\right)
     \left( \frac{1}{\varepsilon^n} + \frac{h_1}{\varepsilon^{n-2}} +
            \frac{h_2}{\varepsilon^{n-4}} + \ldots \right),
\end{align*}
d\'eveloppement que je repr\'esenterai par la formule
\[
\Psi(iK'+\varepsilon) = \frac{1}{\varepsilon^{n+1}} + \frac{\alpha_0}{\varepsilon^n} +
  \frac{\alpha_1}{\varepsilon^{n-1}} + \ldots + \frac{\alpha_i}{\varepsilon^{n-i}} + \ldots,
\]
en posant
\[
\alpha_0 = -H_0, \quad \alpha_1 = H_1, \quad \ldots,
\]
et nous observerons imm\'ediatement que cette s\'erie ne contient point le
terme $\frac{\alpha_{n-1}}{\varepsilon}$. On a effectivement, pour $n = 2\nu$,
\[
\alpha_{n-1} = H_{2\nu-1} + h_1H_{2\nu-3} + h_2H_{2\nu-5} +
  \ldots + h_{\nu-1}H_1+h_{\nu},
\]
puis, en supposant $n = 2\nu-1$,
\[
\alpha_{n-1} = -(H_{2\nu-2} + h_1H_{2\nu-4} + h_2H_{2\nu-6} + \ldots + h_{\nu-1}H_0).
\]
%-----144.png----------------------
Or on voit que, d'apr\`es les \'equations obtenues pour la d\'etermination de $\omega$
et $\lambda$, au \S~XLV, le coefficient $\alpha_{n-1}$ est nul dans les deux cas. La partie
principale du d\'eveloppement de $\Psi(iK'+\varepsilon)$, \`a laquelle nous joindrons le
terme ind\'ependant de $\varepsilon$, est donc
\[
\frac{1}{\varepsilon^{n+1}} + \frac{\alpha_0}{\varepsilon^n}
  +\frac{\alpha_1}{\varepsilon^{n-1}} + \ldots +
  \frac{\alpha_{n-2}}{\varepsilon^2} + \alpha_n.
\]
On en conclut, quand $n = 2\nu$,
\begin{align*}
  \Psi(x) = &-\frac{D_x^{2\nu-1}(k^2\sn^2x)}{\Gamma(2\nu+1)} +
            \alpha_0\frac{D_x^{2\nu-2}(k^2\sn^2x)}{\Gamma(2\nu)} \\
            &-\alpha_1\frac{D_x^{2\nu-3}(k^2\sn^2x)}{\Gamma(2\nu-1)} + \ldots +
            \alpha_{2\nu-2}(k^2\sn^2x)+ \alpha,
\end{align*}
la constante ayant pour valeur
\[
\alpha = \alpha_{2\nu} - \alpha_{2\nu-2}s_0 - \alpha_{2\nu-4}\frac{s_1}{3}
         - \ldots - \alpha_0\frac{s_{\nu-1}}{2\nu-1},
\]
puis, dans le cas de $n = 2\nu-1$,
\begin{align*}
  \Psi(x) = & + \frac{D_x^{2\nu-2}(k^2\sn^2x)}{\Gamma(2\nu)} -
            \alpha_0\frac{D_x^{2\nu-3}(k^2\sn^2x)}{\Gamma(2\nu-1)} \\
            & + \alpha_1\frac{D_x^{2\nu-4}(k^2\sn^2x)}{\Gamma(2\nu-3)} - \ldots +
            \alpha_{2\nu-3}(k^2\sn^2x) + \alpha,
\end{align*}
en posant
\[
\alpha = \alpha_{2\nu-1} - \alpha_{2\nu-3}s_0 - \alpha_{2\nu-5}\frac{s_1}{3}
         - \ldots - \alpha_1\frac{s_{\nu-2}}{2\nu-3} - \frac{s_{\nu-1}}{2\nu-1}.
\]
Soit maintenant $\sn^2x=t$; les expressions auxquelles nous venons de
parvenir prendront cette nouvelle forme, \`a savoir
\[
\Psi(x) = G(t) + \sqrt{R(t)}G_1(t),
\]
o\`u $G(t)$ et $G_1(t)$ sont des polyn\^omes entiers en $t$ des degr\'es $\nu$ et $\nu-1$ dans
le premier cas, $\nu$ et $\nu-2$ dans le second. Observons aussi que, le radical
$\sqrt{R(t)}$ changeant de signe avec $x$, d'apr\`es la condition
\[
\sqrt{R(t)} = \sn x\cn x\dn x,
\]
on aura
\[
\Psi(-x) = G(t) - \sqrt{R(t)}G_1(t);
\]
nous concluons donc de l'\'egalit\'e donn\'ee plus haut
\[
\Psi(x)\Psi(-x) = (-1)^{n+1}k^2
  \left(\sn^2x-\sn^2\omega\right) \Phi(x),
\]
%-----145.png----------------------
la suivante:
\[
G^2(t) - R(t)G_1^2(t) = (-1)^{n+1}k^2
  \left(t-\sn^2\omega\right) \Pi(t).
\]
Cette forme de relation est bien connue par le th\'eor\`eme d'Abel pour l'addition
des int\'egrales elliptiques, et l'on sait que les polyn\^omes $G(t)$,
$G_1(t)$, \'etant des degr\'es donn\'es tout \`a l'heure, se trouvent, \`a un facteur
constant pr\`es, d\'etermin\'es par la condition que l'expression
\[
G^2(t) - R(t)G_1^2(t)
\]
soit divisible par $\Pi(t)$. Il suffit, par cons\'equent, de nous reporter \`a l'\'equation
obtenue au \S~XLIX, \`a savoir:
\[
N\varphi^2(t) - R(t)\psi^2(t) = \Pi(t) \left(gt-g'\right),
\]
pour en conclure le r\'esultat que nous avons annonc\'e
\[
\sn^2\omega = \frac{g'}{g}.
\]
Mais nous voyons, de plus, qu'on peut poser
\[
\rho\left[G(t) + \sqrt{R(t)}G_1(t)\right] =
  \sqrt{N}\varphi(t) + \sqrt{R(t)}\psi(t),
\]
$\rho$ d\'esignant une constante. Voici maintenant les cons\'equences \`a tirer de
cette relation.

Je supposerai que l'on ait $n = 2\nu$; les polyn\^omes $\varphi(t)$ et $\psi(t)$, dont
les coefficients doivent \^etre regard\'es comme connus et, si l'on veut, exprim\'es
sous forme enti\`ere en $h$, seront alors des degr\'es $\nu$ et $\nu-1$. Cela
\'etant, revenons \`a la variable primitive en faisant $t=\sn^2x$, on pourra
mettre $\sqrt{R(t)}\psi(t)$ et $\varphi(t)$ sous la forme suivante, \`a savoir:
\begin{align*}
  \sqrt{R(t)}\psi(t) &= {}- a \frac{D_x^{2\nu-1}(k^2\sn^2x)}{\Gamma(2\nu+1)}
                        - a' \frac{D_x^{2\nu-3}(k^2\sn^2x)}{\Gamma(2\nu-1)} - \ldots \\
          \varphi(t) &= {}+ b \frac{D_x^{2\nu-2}(k^2\sn^2x)}{\Gamma(2\nu)}
                        + b' \frac{D_x^{2\nu-4}(k^2\sn^2x)}{\Gamma(2\nu-2)} + \ldots.
\end{align*}
Nous aurons donc cette expression de la fonction $\Psi(x)$:
\begin{multline*}
\rho\Psi(x) =
  - a \frac{D_x^{2\nu-1}(k^2\sn^2x)}{\Gamma(2\nu+1)}
  - a'\frac{D_x^{2\nu-3}(k^2\sn^2x)}{\Gamma(2\nu-1)} - \ldots \\
+ \sqrt{N}\left[
    b \frac{D_x^{2\nu-2}(k^2\sn^2x)}{\Gamma(2\nu)}
  + b'\frac{D_x^{2\nu-4}(k^2\sn^2x)}{\Gamma(2\nu-2)} + \ldots
\right],
\end{multline*}
o\`u les constantes $a$, $a'$, \ldots, $b$, $b'$, \ldots\ sont d\'etermin\'ees lin\'eairement par
les coefficients de $\varphi(t)$ et $\psi(t)$.
%-----146.png---------------------

Or on en d\'eduit, en faisant $x=iK'+\varepsilon$ et se rappelant qu'on a
suppos\'e $n = 2\nu$, l'\'egalit\'e suivante:
\[
\rho\left( \frac{1}{\varepsilon^{n+1}} + \frac{\alpha_0}{\varepsilon^n} +
  \frac{\alpha_1}{\varepsilon^{n-1}} + \ldots \right) =
  \frac{a}{\varepsilon^{n+1}} + \frac{a'}{\varepsilon^{n-1}} + \ldots + \sqrt{N}
  \left( \frac{b}{\varepsilon^n} + \frac{b'}{\varepsilon^{n-2}} + \ldots\right),
\]
d'o\`u nous tirons
\begin{alignat*}{2}
  & \rho &&= a, \\
  & \rho \alpha_0 &&= b \rlap{$\sqrt{N},$} \\
  & \rho \alpha_1 &&= a', \\
\dotfillalignat
\end{alignat*}
\'Eliminons l'ind\'etermin\'ee $\rho$ et rempla\c{c}ons les coefficients $\alpha_0$,
$\alpha_1$, \ldots\ par leurs valeurs (\S~L, p.~\pageref{page135}); on
aura ces relations
\begin{gather*}
\lambda = - \frac{b\sqrt{N}}{a}, \\
h_1 + \tfrac{1}{2}(\lambda^2 - \Omega) = \frac{a'}{a},\\
\dotfillgather
\end{gather*}
La premi\`ere donne l'expression de $\lambda$, et nous reconnaissons, par cette voie,
qu'elle ne contient d'autre irrationalit\'e que $\sqrt{N}$. On obtiendrait la m\^eme
conclusion dans le cas de $n = 2\nu-1$, et c'est le r\'esultat que j'avais principalement
en vue d'\'etablir, apr\`es avoir d\'emontr\'e que $\sn^2\omega$ est une fonction
rationnelle de $h$. L'\'etude des solutions de Lam\'e qui correspondent aux
racines de l'\'equation $N = 0$ nous permettra, comme on va le voir, d'aller
plus loin et d'approfondir davantage la nature de ces expressions de $\lambda$ et
$\sn^2\omega$.


\mysection{LI.}


On a vu au \S~XLIX (p.~\pageref{page133}) que l'int\'egrale g\'en\'erale de l'\'equation
diff\'eren\-tielle n'est plus repr\'esent\'ee, lorsqu'on a $N = 0$, par la formule
\[
y = CF(x) + C'F(-x),
\]
le rapport $\frac{F(x)}{F(-x)}$ se r\'eduisant alors \`a une constante, et, comme cons\'equence,
nous avons \'etabli que les multiplicateurs de la fonction de seconde
esp\`ece deviennent, au signe pr\`es, \'egaux \`a l'unit\'e. Suivant les diverses
combinaisons des signes de $\mu$ et $\mu'$, nous pouvons donc avoir des solutions
%-----147.png------------------------
particuli\`eres de quatre esp\`eces, caract\'eris\'ees par les relations suivantes:
\begin{align*}
\tag*{(I)}  F(x+2K) &= -F(x),  & F(x+2iK') &= +F(x), \\
\tag*{(II)} F(x+2K) &= -F(x),  & F(x+2iK') &= -F(x), \\
\tag*{(III)}F(x+2K) &= +F(x),  & F(x+2iK') &= -F(x), \\
\tag*{(IV)} F(x+2K) &= +F(x),  & F(x+2iK') &= +F(x).
\end{align*}
Toutes existent en effet, et les trois premi\`eres, o\`u $F(x)$ a successivement la
p\'eriodicit\'e de $\sn x$, $\cn x$, $\dn x$, s'obtiennent en faisant, dans l'expression
g\'en\'erale de cette formule, $\lambda = 0$, conjointement avec $\omega=0$, $\omega=K$,
$\omega=K+iK'$. Nous remarquerons, pour l'\'etablir, que, les valeurs de l'\'el\'ement
simple
\[
f(x) = e^{\lambda(x-iK')} \chi(x)
\]
\'etant alors $f(x) = k\sn x, ik\cn x, i\dn x$, dans ces trois cas, les d\'eveloppe\-ments
en s\'erie de $f(iK'+ \varepsilon)$ ne contiennent que des puissances impaires
de $\varepsilon$, de sorte que les coefficients d\'esign\'es par $H_i$ s'\'evanouissent tous pour
des valeurs paires de l'indice. Des deux conditions obtenues au \S~XLV
(p.~\pageref{page123}), pour la d\'etermination de $\omega$ et
$\lambda$, \`a savoir:
\begin{align*}
H_{2\nu-1} + h_1 H_{2\nu-3} + h_2 H_{2\nu-5} + \ldots + h_{\nu-1} H_1 + h_{\nu} &= 0,\\
2\nu H_{2\nu} + (2\nu - 2)h_1 H_{2\nu-2} + (2\nu-4)h_2 H_{2\nu-4} + \ldots + 2h_{\nu-1} H_2 &= 0,
\end{align*}
dans le cas de $n=2\nu$; puis, en supposant $n=2\nu - 1$,
\begin{align*}
H_{2\nu-2} + h_1 H_{2\nu-4} + h_2 H_{2\nu-6} + \ldots + h_{\nu-1} H_0 &= 0,\\
(2\nu-1) H_{2\nu-1} + (2\nu-3) h_1 H_{2\nu-3} + \ldots + h_{\nu-1} H_1 - h_{\nu} &= 0;
\end{align*}
on voit ainsi qu'une seule subsiste et d\'etermine la constante $h$, l'autre \'etant
satisfaite d'elle-m\^eme.

Mais soit, pour plus de pr\'ecision,
\begin{align*}
k\sn(iK' + \varepsilon)  &= \frac{1}{\varepsilon} + p_1 \varepsilon + p_2 \varepsilon^3 + \ldots + p_i \varepsilon^{2i-1} + \ldots,\\
ik\cn(iK' + \varepsilon) &= \frac{1}{\varepsilon} + q_1 \varepsilon + q_2 \varepsilon^3 + \ldots + q_i \varepsilon^{2i-1} + \ldots,\\
i\dn(iK' + \varepsilon)  &= \frac{1}{\varepsilon} + r_1 \varepsilon + r_2 \varepsilon^3 + \ldots + r_i \varepsilon^{2i-1} + \ldots;
\end{align*}
je poserai, dans le cas de $n = 2\nu$,
\begin{alignat*}{5}
P &= p_{\nu} &&+ h_1 p_{\nu-1} &&+ h_2 p_{\nu-2} &&+ \ldots + h_{\nu-1} p_1 &&+ h_{\nu},\\
Q &= q_{\nu} &&+ h_1 q_{\nu-1} &&+ h_2 q_{\nu-2} &&+ \ldots + h_{\nu-1} q_1 &&+ h_{\nu},\\
R &= r_{\nu} &&+ h_1 r_{\nu-1} &&+ h_2 r_{\nu-2} &&+ \ldots + h_{\nu-1} r_1 &&+ h_{\nu};
\end{alignat*}
%-----148.png------------------------
puis, en supposant $n=2\nu - 1$,
\begin{alignat*}{4}
P &= (2\nu-1) p_{\nu} &&+ (2\nu-3) h_1 p_{\nu-1} &&+ \ldots + h_{\nu-1} p_1 &&- h_{\nu},\\
Q &= (2\nu-1) q_{\nu} &&+ (2\nu-3) h_1 q_{\nu-1} &&+ \ldots + h_{\nu-1} q_1 &&- h_{\nu},\\
R &= (2\nu-1) r_{\nu} &&+ (2\nu-3) h_1 r_{\nu-1} &&+ \ldots + h_{\nu-1} r_1 &&- h_{\nu};
\end{alignat*}
cela \'etant, les \'equations
\[
P=0, \quad Q=0, \quad R=0
\]
d\'etermineront les valeurs particuli\`eres de $h$ auxquelles correspondent les
trois esp\`eces de solutions que nous avons consid\'er\'ees, et l'on voit que dans
les deux cas elles sont toutes du degr\'e $\nu$.

Il ne nous reste plus maintenant qu'\`a obtenir les solutions de la quatri\`eme
esp\`ece dont la p\'eriodicit\'e est celle de $\sn^2x$, mais elles se d\'eduisent
moins imm\'ediatement que les pr\'ec\'edentes de l'expression g\'en\'erale de $F(x)$;
il est n\'ecessaire, en effet, de supposer alors la constante $\lambda$ et $\sn \omega$ infinis;
je donnerai en premier lieu une m\'ethode plus directe et plus facile pour
y parvenir.

Soit d'abord $n=2\nu$; je remarque que toute solution de l'\'equation
diff\'erentielle par une fonction doublement p\'eriodique de premi\`ere esp\`ece
r\'esulte du d\'eveloppement
\[
y = \frac{1}{\varepsilon^{2\nu}} + \frac{h_1}{\varepsilon^{2\nu-2}} + \ldots + \frac{h_{\nu-1}}{\varepsilon^2} + h_{\nu},
\]
et sera donn\'ee par l'expression
\begin{multline*}
F(x) =     \frac{D_x^{2\nu-2} (k^2\sn^2 x)}{\Gamma(2\nu)}
     + h_1 \frac{D_x^{2\nu-4} (k^2\sn^2 x)}{\Gamma(2\nu-2)} + \ldots
     + h_{\nu-1}              (k^2\sn^2 x) \\
     + h_{\nu} - h_{\nu-1} s_0 - h_{\nu-2} \frac{s_1}{3} - \ldots
     - h_1\frac{s_{\nu-2}}{2\nu-3} - \frac{s_{\nu-1}}{2\nu-1}.
\end{multline*}

Cela \'etant, disposons de $h$ de mani\`ere \`a avoir
\[
F(iK' + \varepsilon) = \frac{1}        {\varepsilon^{2\nu}}
                     + \frac{h_1}      {\varepsilon^{2\nu-2}} + \ldots
                     + \frac{h_{\nu-1}}{\varepsilon^{2}} + h_{\nu}
                     + h_{\nu+1} \varepsilon^2,
\]
ce qui donne la condition
\[
\nu s_{\nu} + (\nu-1)h_1 s_{\nu-1} + (\nu-2)h_2 s_{\nu-2} + \ldots + h_{\nu-1} s_1 = h_{\nu+1};
\]
je dis que la fonction doublement p\'eriodique
\[
D_x^2 F(x) - \left[n(n+1)k^2 \sn^2 x + h\right] F(x)
\]
%-----149.png-------------------------
est n\'ecessairement nulle. Si, apr\`es avoir pos\'e $x=iK'+\varepsilon$, on la d\'eveloppe
en effet suivant les puissances croissantes de $\varepsilon$, non seulement la partie
principale, mais le terme ind\'ependant dispara\^itront, comme on l'a vu au
\S~XLIV, p.~\pageref{page121}. De ce que la partie principale n'existe pas, on conclut
que la fonction est constante; enfin cette constante elle-m\^eme est nulle,
puisqu'elle s'exprime lin\'eairement et sous forme homog\`ene par le terme
ind\'ependant de $\varepsilon$, et les coefficients des divers termes en $\frac{1}{\varepsilon}$.

Soit ensuite $n=2\nu-1$; le d\'eveloppement qu'on tire de l'\'equation
diff\'erentielle, \`a savoir
\[
y = \frac{1}{\varepsilon^{2\nu-1}} + \frac{h_1}{\varepsilon^{2\nu-3}} + \ldots
  + \frac{h_{\nu-1}}{\varepsilon} + \ldots,
\]
contenant un terme en $\frac{1}{\varepsilon}$, on doit tout d'abord le faire dispara\^itre en posant
$h_{\nu-1} = 0$, pour en d\'eduire une fonction doublement p\'eriodique de
premi\`ere esp\`ece, qui sera de cette mani\`ere
\[
F(x) =  -\frac{D_x^{2\nu-3} (k^2 \sn^2 x)}{\Gamma(2\nu-1)}
  -h_1\frac{D_x^{2\nu-5} (k^2 \sn^2 x)}{\Gamma(2\nu-3)} - \ldots
  -h_{\nu-2}D_x          (k^2 \sn^2 x).
\]
Cela \'etant, et en nous bornant \`a la partie principale, on aura
\[
F(iK' + \varepsilon) =
    \frac{1}{\varepsilon^{2\nu-1}}
  + \frac{h_1}{\varepsilon^{2\nu-3}} + \ldots
  + \frac{h_{\nu-2}}{\varepsilon^3};
\]
il en r\'esulte que, si on laisse ind\'etermin\'ee la constante $h$, le d\'eveloppement
de l'expression
\[
D_x^2 F(x) - \left[ n(n+1)k^2 \sn^2 x + h \right] F(x),
\]
apr\`es avoir pos\'e $x=iK'+\varepsilon$, commencera par un terme en $\frac{1}{\varepsilon^3}$. Mais faisons
$h_{\nu-1} = 0$; comme on peut \'ecrire alors
\[
F(iK' + \varepsilon) =
    \frac{1}{\varepsilon^{2\nu-1}}
  + \frac{h_1}{\varepsilon^{2\nu-3}} + \ldots
  + \frac{h_{\nu-2}}{\varepsilon^3}
  + \frac{h_{\nu-1}}{\varepsilon},
\]
on voit que ce d\'eveloppement commencera par un terme en $\frac{1}{\varepsilon}$, qui lui-m\^eme
doit n\'ecessairement s'\'evanouir, et il est ainsi prouv\'e que, sous la
condition pos\'ee, le r\'esultat de la substitution de la fonction $F(x)$, dans le
premier membre de l'\'equation diff\'erentielle, ne peut \^etre qu'une constante.
J'ajoute que cette constante est nulle, le r\'esultat de la substitution \'etant,
comme $F(x)$, une fonction qui change de signe avec la variable. Soit
%-----150.png-------------------------
donc, dans le cas de $n = 2\nu$,
\[
S = \nu s_{\nu} + (\nu-1) h_1 s_{\nu-1} + (\nu-2) h_2 s_{\nu-2} + \ldots
  + h_{\nu-1} s_1 - h_{\nu+1};
\]
puis, en supposant $n = 2\nu-1$,
\[
S = h_{\nu-1},
\]
on voit que les \'equations
\[
P = 0, \quad Q = 0, \quad R = 0, \quad S=0
\]
d\'eterminent les valeurs de $h$ auxquelles correspondent les quatre esp\`eces
de solutions doublement p\'eriodiques d\'ecouvertes par Lam\'e, ces solutions
ne se trouvant plus distingu\'ees par leur expression alg\'ebrique, comme l'a
fait l'illustre auteur, mais d'apr\`es la nature de leur p\'eriodicit\'e. On voit
aussi que la condition $N = 0$, d'o\`u elles ont \'et\'e tir\'ees, se pr\'esente sous la
forme
\[
PQRS = 0,
\]
et l'on v\'erifie imm\'ediatement que le produit des quatre facteurs, dans les
deux cas de $n = 2\nu$ et $n = 2\nu-1$, est bien du degr\'e $2n+1$ et $h$, comme
nous l'avons \'etabli pour $N$ au \S~XLIX, page~\pageref{page133a}.

Voici maintenant le proc\'ed\'e que j'ai annonc\'e pour d\'eduire les solutions
de la quatri\`eme esp\`ece de la solution g\'en\'erale.


\mysection{LII.}


Je reviens \`a l'\'el\'ement simple
\[
f(x) = \frac{H'(0) H(x+\omega)}{\Theta(x)\Theta(\omega)} \,
    e^{ \left[ \lambda - \frac{\Theta'(\omega)}{\Theta(\omega)} \right]
        (x-iK') + \frac{i\pi\omega}{2K} },
\]
o\`u $\lambda$ et $\sn\omega$ sont des fonctions d\'etermin\'ees de $h$; je les suppose infinies
l'une et l'autre pour une certaine valeur de cette constante, et je me propose
de reconna\^itre ce que devient, lorsqu'on attribue \`a $h$ cette valeur, l'expression
de $f(x)$. Concevons, \`a cet effet, que $\lambda$ soit exprim\'e au moyen
de $\omega$; je ferai
\[
\omega = iK' + \delta,
\]
ce qui donne, apr\`es une r\'eduction facile,
\[
f(x) = \frac{H'(0) \Theta(x+\delta)}{\Theta(x) H(\delta)} \,
    e^{ \left[ \lambda - \frac{H'(\delta)}{H(\delta)} \right]
       (x-iK') + \frac{i\pi\delta}{2K} }.
\]
%-----151.png-------------------------
Or nous avons, en d\'eveloppant suivant les puissances croissantes de~$\delta$
\[
\frac{H'(\delta)}{H(\delta)} = \frac{1}{\delta} - \left( s_0 - \frac{J}{K} \right) \delta
  - \frac{s_1 \delta^3}{3} - \frac{s_2 \delta^5}{5} - \ldots;
\]
cela \'etant, pour que l'exponentielle
\[
e^{\left[ \lambda - \frac{H'(\delta)}{H(\delta)} \right] (x-iK') }
\]
soit finie lorsqu'on fera $\delta = 0$, on voit que $\lambda$ doit s'exprimer de telle mani\`ere
en $\omega$ qu'on ait, en supposant $\omega = iK' + \delta$,
\[
\lambda = \frac{1}{\delta} + \lambda_0 + \lambda_1\delta + \ldots.
\]
Cette forme de d\'eveloppement nous donne, en effet,
\[
\lambda - \frac{H'(\delta)}{H(\delta)} = \lambda_0
      + \left( \lambda_1 + s_0 - \frac{J}{K} \right) \delta + \ldots;
\]
on a d'ailleurs imm\'ediatement
\begin{gather*}
\frac{H'(0)}{H(\delta)} = \frac{1}{\delta}
  + \left( s_0 - \frac{J}{K} \right) \frac{\delta}{2} + \ldots,\\
\frac{\Theta(x+\delta)}{\Theta(x)} = 1 + \frac{\Theta'(x)}{\Theta(x)} \delta + \ldots,
\end{gather*}
et nous en concluons l'expression
\[
f(x) = e^{\lambda_0 (x-iK')}
\left( \frac{1}{\delta} + X + X_1 \delta + \ldots \right),
\]
o\`u le terme ind\'ependant de $\delta$, qui sera seul \`a consid\'erer, est
\[
X = \left( \lambda_1 + s_0 - \frac{J}{K} \right) (x-iK')
  + \frac{i\pi}{2K} + \frac{\Theta'(x)}{\Theta(x)}.
\]
Elle fait voir que les formules, pour $n = 2\nu$ et $n = 2\nu-1$,
\begin{align*}
F(x) &= -   \frac{D_x^{2\nu-1} f(x)}{\Gamma(2\nu)}
       -h_1\frac{D_x^{2\nu-3} f(x)}{\Gamma(2\nu-2)} - \ldots
       -h_{\nu-1} D_x f(x),
\intertext{puis}
F(x) &= +\frac{D_x^{2\nu-2} f(x)}{\Gamma(2\nu-1)}
       +h_1 \frac{D_x^{2\nu-4} f(x)}{\Gamma(2\nu-3)} + \ldots
       +h_{\nu-1} f(x),
\end{align*}
contiennent chacune un terme en $\frac{1}{\delta}$, qui est, pour la premi\`ere,
\[
-e^{\lambda_0 (x-iK')}
  \left[ \frac{\lambda_0^{2\nu-1}}{\Gamma(2\nu)}
    + h_1 \frac{\lambda_0^{2\nu-3}}{\Gamma(2\nu-2)} + \ldots + h_{\nu-1} \lambda_0 \right],
\]
%-----152.png-------------------------
et dans la seconde
\[
e^{\lambda_0 (x-iK')}\left[ \frac{\lambda_0^{2\nu-2}}{\Gamma(2\nu-1)}
                         +h_1\frac{\lambda_0^{2\nu-4}}{\Gamma(2\nu-3)} + \ldots
                         +h_{\nu-1} \right].
\]
Il est donc n\'ecessaire, afin d'obtenir des quantit\'es finies en faisant $\delta = 0$,
que $\lambda_0$ satisfasse \`a ces \'equations
\begin{align*}
\frac{\lambda_0^{2\nu-1}}{\Gamma(2\nu)}
  + h_1\frac{\lambda_0^{2\nu-3}}{\Gamma(2\nu-2)} + \ldots + h_{\nu-1}\lambda_0 &= 0,\\
\frac{\lambda_0^{2\nu-2}}{\Gamma(2\nu-1)} + h_1\frac{\lambda_0^{2\nu-4}}{\Gamma(2\nu-3)}
  + \ldots + h_{\nu-1} &= 0.
\end{align*}
Cela \'etant, les expressions de $F(x)$ se transforment de la mani\`ere suivante.

Soit, en g\'en\'eral,
\[
f(x) = e^{\lambda x} X,
\]
en d\'esignant par $\lambda$ et $X$ une constante et une fonction quelconques. On
voit ais\'ement que la quantit\'e
\[
AD^n_x f(x) + A_1 D_x^{n-1} f(x) + \ldots + A_n f(x),
\]
si l'on admet la relation
\[
A \lambda^n + A_1 \lambda^{n-1} + \ldots + A_n = 0,
\]
s'exprime, au moyen de la nouvelle fonction
\[
f_1(x) = e^{\lambda x} D_x X,
\]
par la formule
\begin{align*}
AD_x^{n-1} f_1(x) &+ (A\lambda + A_1) D_x^{n-2} f_1(x) + \ldots\\
  &+(A \lambda^{n-1} + A_1 \lambda^{n-2} + \ldots + A_{n-1}) f_1(x).
\end{align*}

Dans le cas auquel nous avons \'et\'e conduit, on tire imm\'ediatement de
la valeur de $X$ l'expression
\[
f_1(x) = e^{\lambda_0 (x-iK')} (\lambda_1 + s_0 - k^2 \sn^2 x),
\]
et nous obtenons par cons\'equent pour $F(x)$ le produit, par l'exponentielle
$e^{\lambda_0 x}$, d'une fonction doublement p\'eriodique de premi\`ere esp\`ece,
compos\'ee lin\'eairement avec les d\'eriv\'ees de $\sn^2x$. L'analyse pr\'ec\'edente,
en \'etablissant l'existence de ce genre de solutions de l'\'equation diff\'erentielle,
les rattache aux valeurs de $h$ qui rendent \`a la fois infinies les constantes $\lambda$ et $\sn\omega$; on voit aussi que, dans le cas particulier o\`u $\lambda_0$ est nul,
elles donnent bien les fonctions que je me suis propos\'e de d\'eduire de la
%-----153.png------------------------
solution g\'en\'erale. Mais revenons \`a la premi\`ere forme qui a \'et\'e obtenue au
moyen de la fonction
\[
f(x) = e^{\lambda_0 (x-iK')} \left( \frac{1}{\delta} + X + X_1\delta + \ldots \right).
\]

Le terme $\frac{1}{\delta}(e^{\lambda_0(x-iK')})$ disparaissant, comme nous l'avons vu dans l'expression
de $F(x)$, il est permis de prendre plus simplement \`a la limite,
pour $\delta = 0$,
\[
f(x) = e^{\lambda_0(x-iK')} X.
\]

Cette fonction joue donc le r\^ole d'\'el\'ement simple; il est facile, lorsqu'on
fait $x = iK'+\varepsilon$, d'obtenir son d\'eveloppement et d'avoir ainsi les
quantit\'es qui remplacent, dans le cas pr\'esent, les coefficients d\'esign\'es en
g\'en\'eral par $H_0$, $H_1$, etc. Nous avons en effet, pour $x = iK'+\varepsilon$,
\[
X = \left( \lambda_1 + s_0 - \frac{J}{K} \right) \varepsilon
  + \frac{H'(\varepsilon)}{H(\varepsilon)} = \frac{1}{\varepsilon}
  + \lambda_1 \varepsilon - \frac{s_1 \varepsilon^3}{3} - \frac{s_2 \varepsilon^5}{5}
  - \ldots.
\]
Multiplions par $e^{\lambda_0 \varepsilon}$ les deux membres, et soit
\[
e^{\lambda_0\varepsilon} X = \tfrac{1}{2} + S_0 + S_1\varepsilon + \ldots + S_i\varepsilon^i,
\]
nous trouverons
\begin{align*}
S_0 &= \lambda_0,    \\
S_1 &= \frac{\lambda^2_0}{1 \cntrdot 2} + \lambda_1,\\
S_2 &= \frac{\lambda^3_0}{1 \cntrdot 2 \cntrdot 3} + \lambda_1 \lambda_0, \\
S_3 &= \frac{\lambda^4_0}{1\cntrdot 2\cntrdot 3\cntrdot 4}
     + \lambda_1\frac{\lambda^2_0}{1 \cntrdot 2} - \frac{s_1}{3},\\
\dotfillalign,
\end{align*}
$S_i$ \'etant, en g\'en\'eral, un polyn\^ome du degr\'e $i + 1$ en $\lambda_0$,
o\`u n'entrent que des puissances impaires ou des puissances paires, suivant
que l'indice est pair ou impair. Les conditions donn\'ees au \S~XLV
(p.~\pageref{page123a}) conduisent donc, dans les deux cas de
$n=2\nu$, $n = 2\nu-1$, en y joignant l'\'equation en $\lambda_0$
pr\'ec\'edemment trouv\'ee, \`a ces trois relations
\begin{align*}
     \frac{\lambda_0^{2\nu-1}}{\Gamma(2\nu  )}
+ h_1\frac{\lambda_0^{2\nu-3}}{\Gamma(2\nu-2)} + \ldots
   + h_{\nu-1} \lambda_0 &= 0, \\
S_{2\nu-1} + h_1 S_{2\nu-3} + h_2 S_{2\nu-5} + \ldots + 2h_{\nu-1}S_1 + h_{\nu} &= 0,\\
2\nu S_{2\nu} + (2\nu-2)h_1 S_{2\nu-2} + (2\nu-4)h_2 S_{2\nu-4} + \ldots
  + 2h_{\nu-1} S_2 &=0,
\end{align*}
%-----154.png------------------------
lorsque l'on suppose $n=2\nu$, puis
\begin{gather*}
         \frac{\lambda^{2\nu-2}_0}{\Gamma(2\nu-1)}
   + h_1 \frac{\lambda^{2\nu-4}_0}{\Gamma(2\nu-3)} + \ldots + h_{\nu-1} = 0, \\
S_{2\nu-2} + h_1 S_{2\nu-4} + h_2 S_{2\nu-6} + \ldots + h_{\nu-1} S_0 = 0,\\
(2\nu-1)S_{2\nu-1} + (2\nu-3)h_1 S_{2\nu-3} + \ldots + h_{\nu-1} S_1 - h_{\nu} = 0
\end{gather*}
pour $n = 2\nu-1$. Elles donnent le moyen d'obtenir directement, et sans
supposer la connaissance de la solution g\'en\'erale, les trois quantit\'es $\lambda_0$, $\lambda_1$
et $h$. Elles montrent aussi qu'on a en particulier la valeur $\lambda_0=0$, \`a laquelle
correspondent les solutions de Lam\'e. Effectivement, lorsque $\lambda_0$ est suppos\'e
nul, on obtient
\[
S_{2i} = 0, \quad S_1 = \lambda_1, \quad S_{2i+1} = -\frac{s_i}{2i+1};
\]
cela \'etant, dans le cas de $n = 2\nu$, la premi\`ere et la troisi\`eme \'equation sont
satisfaites d'elles-m\^emes; la deuxi\`eme, devenant
\[
-\frac{s_{\nu-1}}{2\nu-1} - h_1\frac{s_{\nu-2}}{2\nu-3} - h_2\frac{s_{\nu-3}}{2\nu-5}
  + \ldots + h_{\nu-1} \lambda_1 + h_{\nu} = 0,
\]
ne d\'etermine que $\lambda_1$. Il est donc n\'ecessaire de recourir \`a l'une
des relations en nombre infini qui ont \'et\'e donn\'ees au \S~XLVI
(p.~\pageref{page125}), sous ces formes:
\[
\mathfrak{H}_i = 0, \quad \mathfrak{H}_{2i} = h_{i+\nu}, \quad
   \mathfrak{H}_{2i+2\nu+1} = Ch'_i.
\]

La plus simple est
\[
\mathfrak{H}_2 = h_{\nu+1},
\]
ou bien
\begin{multline*}
-\nu(2\nu+1) H_{2\nu+1} + (\nu-1)(2\nu-1) h_1 H_{2\nu-1} \\
+ (\nu-2)(2\nu-3) h_2 H_{2\nu-3} + \ldots + 3h_{\nu-1} H_3 + h_{\nu+1} = 0,
\end{multline*}
et nous en tirons imm\'ediatement
\[
-\nu s_{\nu} - (\nu-1)h_1 s_{\nu-1} - (\nu-2)h_2 s_{\nu-2} - \ldots - h_{\nu-1} s_1
   +h_{\nu+1} = 0,
\]
ce qui est l'\'equation en $h$ pr\'ec\'edemment trouv\'ee.
\medskip

En dernier lieu et pour le cas de $n = 2\nu-1$, nos trois relations se
trouvent v\'erifi\'ees si l'on fait $h_{\nu-1}=0$; on retrouve donc encore de cette
mani\`ere le r\'esultat auquel nous \'etions pr\'ec\'edemment parvenu par une m\'ethode
toute diff\'erente.

\begin{center}
FIN
\end{center}


\clearpage
\markright{\footnotesize\upshape LICENSING}
\advance\oddsidemargin-0.25in % spread out into the margins a bit
\advance\evensidemargin-0.25in % alternatively, use a smaller fontsize than \small
\advance\hsize.5in
\advance\textwidth.5in
\small
\begin{verbatim}
End of the Project Gutenberg EBook of Sur quelques applications des
fonctions elliptiques, by Charles Hermite

*** END OF THIS PROJECT GUTENBERG EBOOK SUR QUELQUES APPLICATIONS ***

***** This file should be named 25227-pdf.pdf or 25227-pdf.zip *****
This and all associated files of various formats will be found in:
        http://www.gutenberg.org/2/5/2/2/25227/

Produced by K.F. Greiner, Joshua Hutchinson and the Online
Distributed Proofreading Team at http://www.pgdp.net (This
file was produced from images generously made available
by Cornell University Digital Collections)


Updated editions will replace the previous one--the old editions
will be renamed.

Creating the works from public domain print editions means that no
one owns a United States copyright in these works, so the Foundation
(and you!) can copy and distribute it in the United States without
permission and without paying copyright royalties.  Special rules,
set forth in the General Terms of Use part of this license, apply to
copying and distributing Project Gutenberg-tm electronic works to
protect the PROJECT GUTENBERG-tm concept and trademark.  Project
Gutenberg is a registered trademark, and may not be used if you
charge for the eBooks, unless you receive specific permission.  If you
do not charge anything for copies of this eBook, complying with the
rules is very easy.  You may use this eBook for nearly any purpose
such as creation of derivative works, reports, performances and
research.  They may be modified and printed and given away--you may do
practically ANYTHING with public domain eBooks.  Redistribution is
subject to the trademark license, especially commercial
redistribution.



*** START: FULL LICENSE ***

THE FULL PROJECT GUTENBERG LICENSE
PLEASE READ THIS BEFORE YOU DISTRIBUTE OR USE THIS WORK

To protect the Project Gutenberg-tm mission of promoting the free
distribution of electronic works, by using or distributing this work
(or any other work associated in any way with the phrase "Project
Gutenberg"), you agree to comply with all the terms of the Full Project
Gutenberg-tm License (available with this file or online at
http://gutenberg.org/license).


Section 1.  General Terms of Use and Redistributing Project Gutenberg-tm
electronic works

1.A.  By reading or using any part of this Project Gutenberg-tm
electronic work, you indicate that you have read, understand, agree to
and accept all the terms of this license and intellectual property
(trademark/copyright) agreement.  If you do not agree to abide by all
the terms of this agreement, you must cease using and return or destroy
all copies of Project Gutenberg-tm electronic works in your possession.
If you paid a fee for obtaining a copy of or access to a Project
Gutenberg-tm electronic work and you do not agree to be bound by the
terms of this agreement, you may obtain a refund from the person or
entity to whom you paid the fee as set forth in paragraph 1.E.8.

1.B.  "Project Gutenberg" is a registered trademark.  It may only be
used on or associated in any way with an electronic work by people who
agree to be bound by the terms of this agreement.  There are a few
things that you can do with most Project Gutenberg-tm electronic works
even without complying with the full terms of this agreement.  See
paragraph 1.C below.  There are a lot of things you can do with Project
Gutenberg-tm electronic works if you follow the terms of this agreement
and help preserve free future access to Project Gutenberg-tm electronic
works.  See paragraph 1.E below.

1.C.  The Project Gutenberg Literary Archive Foundation ("the Foundation"
or PGLAF), owns a compilation copyright in the collection of Project
Gutenberg-tm electronic works.  Nearly all the individual works in the
collection are in the public domain in the United States.  If an
individual work is in the public domain in the United States and you are
located in the United States, we do not claim a right to prevent you from
copying, distributing, performing, displaying or creating derivative
works based on the work as long as all references to Project Gutenberg
are removed.  Of course, we hope that you will support the Project
Gutenberg-tm mission of promoting free access to electronic works by
freely sharing Project Gutenberg-tm works in compliance with the terms of
this agreement for keeping the Project Gutenberg-tm name associated with
the work.  You can easily comply with the terms of this agreement by
keeping this work in the same format with its attached full Project
Gutenberg-tm License when you share it without charge with others.

1.D.  The copyright laws of the place where you are located also govern
what you can do with this work.  Copyright laws in most countries are in
a constant state of change.  If you are outside the United States, check
the laws of your country in addition to the terms of this agreement
before downloading, copying, displaying, performing, distributing or
creating derivative works based on this work or any other Project
Gutenberg-tm work.  The Foundation makes no representations concerning
the copyright status of any work in any country outside the United
States.

1.E.  Unless you have removed all references to Project Gutenberg:

1.E.1.  The following sentence, with active links to, or other immediate
access to, the full Project Gutenberg-tm License must appear prominently
whenever any copy of a Project Gutenberg-tm work (any work on which the
phrase "Project Gutenberg" appears, or with which the phrase "Project
Gutenberg" is associated) is accessed, displayed, performed, viewed,
copied or distributed:

This eBook is for the use of anyone anywhere at no cost and with
almost no restrictions whatsoever.  You may copy it, give it away or
re-use it under the terms of the Project Gutenberg License included
with this eBook or online at www.gutenberg.org

1.E.2.  If an individual Project Gutenberg-tm electronic work is derived
from the public domain (does not contain a notice indicating that it is
posted with permission of the copyright holder), the work can be copied
and distributed to anyone in the United States without paying any fees
or charges.  If you are redistributing or providing access to a work
with the phrase "Project Gutenberg" associated with or appearing on the
work, you must comply either with the requirements of paragraphs 1.E.1
through 1.E.7 or obtain permission for the use of the work and the
Project Gutenberg-tm trademark as set forth in paragraphs 1.E.8 or
1.E.9.

1.E.3.  If an individual Project Gutenberg-tm electronic work is posted
with the permission of the copyright holder, your use and distribution
must comply with both paragraphs 1.E.1 through 1.E.7 and any additional
terms imposed by the copyright holder.  Additional terms will be linked
to the Project Gutenberg-tm License for all works posted with the
permission of the copyright holder found at the beginning of this work.

1.E.4.  Do not unlink or detach or remove the full Project Gutenberg-tm
License terms from this work, or any files containing a part of this
work or any other work associated with Project Gutenberg-tm.

1.E.5.  Do not copy, display, perform, distribute or redistribute this
electronic work, or any part of this electronic work, without
prominently displaying the sentence set forth in paragraph 1.E.1 with
active links or immediate access to the full terms of the Project
Gutenberg-tm License.

1.E.6.  You may convert to and distribute this work in any binary,
compressed, marked up, nonproprietary or proprietary form, including any
word processing or hypertext form.  However, if you provide access to or
distribute copies of a Project Gutenberg-tm work in a format other than
"Plain Vanilla ASCII" or other format used in the official version
posted on the official Project Gutenberg-tm web site (www.gutenberg.org),
you must, at no additional cost, fee or expense to the user, provide a
copy, a means of exporting a copy, or a means of obtaining a copy upon
request, of the work in its original "Plain Vanilla ASCII" or other
form.  Any alternate format must include the full Project Gutenberg-tm
License as specified in paragraph 1.E.1.

1.E.7.  Do not charge a fee for access to, viewing, displaying,
performing, copying or distributing any Project Gutenberg-tm works
unless you comply with paragraph 1.E.8 or 1.E.9.

1.E.8.  You may charge a reasonable fee for copies of or providing
access to or distributing Project Gutenberg-tm electronic works provided
that

- You pay a royalty fee of 20% of the gross profits you derive from
     the use of Project Gutenberg-tm works calculated using the method
     you already use to calculate your applicable taxes.  The fee is
     owed to the owner of the Project Gutenberg-tm trademark, but he
     has agreed to donate royalties under this paragraph to the
     Project Gutenberg Literary Archive Foundation.  Royalty payments
     must be paid within 60 days following each date on which you
     prepare (or are legally required to prepare) your periodic tax
     returns.  Royalty payments should be clearly marked as such and
     sent to the Project Gutenberg Literary Archive Foundation at the
     address specified in Section 4, "Information about donations to
     the Project Gutenberg Literary Archive Foundation."

- You provide a full refund of any money paid by a user who notifies
     you in writing (or by e-mail) within 30 days of receipt that s/he
     does not agree to the terms of the full Project Gutenberg-tm
     License.  You must require such a user to return or
     destroy all copies of the works possessed in a physical medium
     and discontinue all use of and all access to other copies of
     Project Gutenberg-tm works.

- You provide, in accordance with paragraph 1.F.3, a full refund of any
     money paid for a work or a replacement copy, if a defect in the
     electronic work is discovered and reported to you within 90 days
     of receipt of the work.

- You comply with all other terms of this agreement for free
     distribution of Project Gutenberg-tm works.

1.E.9.  If you wish to charge a fee or distribute a Project Gutenberg-tm
electronic work or group of works on different terms than are set
forth in this agreement, you must obtain permission in writing from
both the Project Gutenberg Literary Archive Foundation and Michael
Hart, the owner of the Project Gutenberg-tm trademark.  Contact the
Foundation as set forth in Section 3 below.

1.F.

1.F.1.  Project Gutenberg volunteers and employees expend considerable
effort to identify, do copyright research on, transcribe and proofread
public domain works in creating the Project Gutenberg-tm
collection.  Despite these efforts, Project Gutenberg-tm electronic
works, and the medium on which they may be stored, may contain
"Defects," such as, but not limited to, incomplete, inaccurate or
corrupt data, transcription errors, a copyright or other intellectual
property infringement, a defective or damaged disk or other medium, a
computer virus, or computer codes that damage or cannot be read by
your equipment.

1.F.2.  LIMITED WARRANTY, DISCLAIMER OF DAMAGES - Except for the "Right
of Replacement or Refund" described in paragraph 1.F.3, the Project
Gutenberg Literary Archive Foundation, the owner of the Project
Gutenberg-tm trademark, and any other party distributing a Project
Gutenberg-tm electronic work under this agreement, disclaim all
liability to you for damages, costs and expenses, including legal
fees.  YOU AGREE THAT YOU HAVE NO REMEDIES FOR NEGLIGENCE, STRICT
LIABILITY, BREACH OF WARRANTY OR BREACH OF CONTRACT EXCEPT THOSE
PROVIDED IN PARAGRAPH F3.  YOU AGREE THAT THE FOUNDATION, THE
TRADEMARK OWNER, AND ANY DISTRIBUTOR UNDER THIS AGREEMENT WILL NOT BE
LIABLE TO YOU FOR ACTUAL, DIRECT, INDIRECT, CONSEQUENTIAL, PUNITIVE OR
INCIDENTAL DAMAGES EVEN IF YOU GIVE NOTICE OF THE POSSIBILITY OF SUCH
DAMAGE.

1.F.3.  LIMITED RIGHT OF REPLACEMENT OR REFUND - If you discover a
defect in this electronic work within 90 days of receiving it, you can
receive a refund of the money (if any) you paid for it by sending a
written explanation to the person you received the work from.  If you
received the work on a physical medium, you must return the medium with
your written explanation.  The person or entity that provided you with
the defective work may elect to provide a replacement copy in lieu of a
refund.  If you received the work electronically, the person or entity
providing it to you may choose to give you a second opportunity to
receive the work electronically in lieu of a refund.  If the second copy
is also defective, you may demand a refund in writing without further
opportunities to fix the problem.

1.F.4.  Except for the limited right of replacement or refund set forth
in paragraph 1.F.3, this work is provided to you 'AS-IS' WITH NO OTHER
WARRANTIES OF ANY KIND, EXPRESS OR IMPLIED, INCLUDING BUT NOT LIMITED TO
WARRANTIES OF MERCHANTIBILITY OR FITNESS FOR ANY PURPOSE.

1.F.5.  Some states do not allow disclaimers of certain implied
warranties or the exclusion or limitation of certain types of damages.
If any disclaimer or limitation set forth in this agreement violates the
law of the state applicable to this agreement, the agreement shall be
interpreted to make the maximum disclaimer or limitation permitted by
the applicable state law.  The invalidity or unenforceability of any
provision of this agreement shall not void the remaining provisions.

1.F.6.  INDEMNITY - You agree to indemnify and hold the Foundation, the
trademark owner, any agent or employee of the Foundation, anyone
providing copies of Project Gutenberg-tm electronic works in accordance
with this agreement, and any volunteers associated with the production,
promotion and distribution of Project Gutenberg-tm electronic works,
harmless from all liability, costs and expenses, including legal fees,
that arise directly or indirectly from any of the following which you do
or cause to occur: (a) distribution of this or any Project Gutenberg-tm
work, (b) alteration, modification, or additions or deletions to any
Project Gutenberg-tm work, and (c) any Defect you cause.


Section  2.  Information about the Mission of Project Gutenberg-tm

Project Gutenberg-tm is synonymous with the free distribution of
electronic works in formats readable by the widest variety of computers
including obsolete, old, middle-aged and new computers.  It exists
because of the efforts of hundreds of volunteers and donations from
people in all walks of life.

Volunteers and financial support to provide volunteers with the
assistance they need, is critical to reaching Project Gutenberg-tm's
goals and ensuring that the Project Gutenberg-tm collection will
remain freely available for generations to come.  In 2001, the Project
Gutenberg Literary Archive Foundation was created to provide a secure
and permanent future for Project Gutenberg-tm and future generations.
To learn more about the Project Gutenberg Literary Archive Foundation
and how your efforts and donations can help, see Sections 3 and 4
and the Foundation web page at http://www.pglaf.org.


Section 3.  Information about the Project Gutenberg Literary Archive
Foundation

The Project Gutenberg Literary Archive Foundation is a non profit
501(c)(3) educational corporation organized under the laws of the
state of Mississippi and granted tax exempt status by the Internal
Revenue Service.  The Foundation's EIN or federal tax identification
number is 64-6221541.  Its 501(c)(3) letter is posted at
http://pglaf.org/fundraising.  Contributions to the Project Gutenberg
Literary Archive Foundation are tax deductible to the full extent
permitted by U.S. federal laws and your state's laws.

The Foundation's principal office is located at 4557 Melan Dr. S.
Fairbanks, AK, 99712., but its volunteers and employees are scattered
throughout numerous locations.  Its business office is located at
809 North 1500 West, Salt Lake City, UT 84116, (801) 596-1887, email
business@pglaf.org.  Email contact links and up to date contact
information can be found at the Foundation's web site and official
page at http://pglaf.org

For additional contact information:
     Dr. Gregory B. Newby
     Chief Executive and Director
     gbnewby@pglaf.org


Section 4.  Information about Donations to the Project Gutenberg
Literary Archive Foundation

Project Gutenberg-tm depends upon and cannot survive without wide
spread public support and donations to carry out its mission of
increasing the number of public domain and licensed works that can be
freely distributed in machine readable form accessible by the widest
array of equipment including outdated equipment.  Many small donations
($1 to $5,000) are particularly important to maintaining tax exempt
status with the IRS.

The Foundation is committed to complying with the laws regulating
charities and charitable donations in all 50 states of the United
States.  Compliance requirements are not uniform and it takes a
considerable effort, much paperwork and many fees to meet and keep up
with these requirements.  We do not solicit donations in locations
where we have not received written confirmation of compliance.  To
SEND DONATIONS or determine the status of compliance for any
particular state visit http://pglaf.org

While we cannot and do not solicit contributions from states where we
have not met the solicitation requirements, we know of no prohibition
against accepting unsolicited donations from donors in such states who
approach us with offers to donate.

International donations are gratefully accepted, but we cannot make
any statements concerning tax treatment of donations received from
outside the United States.  U.S. laws alone swamp our small staff.

Please check the Project Gutenberg Web pages for current donation
methods and addresses.  Donations are accepted in a number of other
ways including checks, online payments and credit card donations.
To donate, please visit: http://pglaf.org/donate


Section 5.  General Information About Project Gutenberg-tm electronic
works.

Professor Michael S. Hart is the originator of the Project Gutenberg-tm
concept of a library of electronic works that could be freely shared
with anyone.  For thirty years, he produced and distributed Project
Gutenberg-tm eBooks with only a loose network of volunteer support.


Project Gutenberg-tm eBooks are often created from several printed
editions, all of which are confirmed as Public Domain in the U.S.
unless a copyright notice is included.  Thus, we do not necessarily
keep eBooks in compliance with any particular paper edition.


Most people start at our Web site which has the main PG search facility:

     http://www.gutenberg.org

This Web site includes information about Project Gutenberg-tm,
including how to make donations to the Project Gutenberg Literary
Archive Foundation, how to help produce our new eBooks, and how to
subscribe to our email newsletter to hear about new eBooks.
\end{verbatim}
% %%%%%%%%%%%%%%%%%%%%%%%%%%%%%%%%%%%%%%%%%%%%%%%%%%%%%%%%%%%%%%%%%%%%%%% %
%                                                                         %
% End of the Project Gutenberg EBook of Sur quelques applications des     %
% fonctions elliptiques, by Charles Hermite                               %
%                                                                         %
% *** END OF THIS PROJECT GUTENBERG EBOOK SUR QUELQUES APPLICATIONS ***   %
%                                                                         %
% ***** This file should be named 25227-t.tex or 25227-t.zip *****        %
% This and all associated files of various formats will be found in:      %
%         http://www.gutenberg.org/2/5/2/2/25227/                         %
%                                                                         %
% %%%%%%%%%%%%%%%%%%%%%%%%%%%%%%%%%%%%%%%%%%%%%%%%%%%%%%%%%%%%%%%%%%%%%%% %

\end{document}

This is pdfeTeX, Version 3.141592-1.21a-2.2 (Web2C 7.5.4) (format=pdflatex 2007.10.4)  30 APR 2008 04:54
entering extended mode
**25227-t.tex
(./25227-t.tex
LaTeX2e <2003/12/01>
Babel <v3.8d> and hyphenation patterns for american, french, german, ngerman, b
ahasa, basque, bulgarian, catalan, croatian, czech, danish, dutch, esperanto, e
stonian, finnish, greek, icelandic, irish, italian, latin, magyar, norsk, polis
h, portuges, romanian, russian, serbian, slovak, slovene, spanish, swedish, tur
kish, ukrainian, nohyphenation, loaded.
(/usr/share/texmf-tetex/tex/latex/base/book.cls
Document Class: book 2004/02/16 v1.4f Standard LaTeX document class
(/usr/share/texmf-tetex/tex/latex/base/leqno.clo
File: leqno.clo 1998/08/17 v1.1c Standard LaTeX option (left equation numbers)
) (/usr/share/texmf-tetex/tex/latex/base/bk11.clo
File: bk11.clo 2004/02/16 v1.4f Standard LaTeX file (size option)
)
\c@part=\count79
\c@chapter=\count80
\c@section=\count81
\c@subsection=\count82
\c@subsubsection=\count83
\c@paragraph=\count84
\c@subparagraph=\count85
\c@figure=\count86
\c@table=\count87
\abovecaptionskip=\skip41
\belowcaptionskip=\skip42
\bibindent=\dimen102
)

LaTeX Warning: You have requested, on input line 55, version
               `2005/09/16' of document class book,
               but only version
               `2004/02/16 v1.4f Standard LaTeX document class'
               is available.

(/usr/share/texmf-tetex/tex/latex/amsmath/amsmath.sty
Package: amsmath 2000/07/18 v2.13 AMS math features
\@mathmargin=\skip43
For additional information on amsmath, use the `?' option.
(/usr/share/texmf-tetex/tex/latex/amsmath/amstext.sty
Package: amstext 2000/06/29 v2.01
(/usr/share/texmf-tetex/tex/latex/amsmath/amsgen.sty
File: amsgen.sty 1999/11/30 v2.0
\@emptytoks=\toks14
\ex@=\dimen103
)) (/usr/share/texmf-tetex/tex/latex/amsmath/amsbsy.sty
Package: amsbsy 1999/11/29 v1.2d
\pmbraise@=\dimen104
) (/usr/share/texmf-tetex/tex/latex/amsmath/amsopn.sty
Package: amsopn 1999/12/14 v2.01 operator names
)
\inf@bad=\count88
LaTeX Info: Redefining \frac on input line 211.
\uproot@=\count89
\leftroot@=\count90
LaTeX Info: Redefining \overline on input line 307.
\classnum@=\count91
\DOTSCASE@=\count92
LaTeX Info: Redefining \ldots on input line 379.
LaTeX Info: Redefining \dots on input line 382.
LaTeX Info: Redefining \cdots on input line 467.
\Mathstrutbox@=\box26
\strutbox@=\box27
\big@size=\dimen105
LaTeX Font Info:    Redeclaring font encoding OML on input line 567.
LaTeX Font Info:    Redeclaring font encoding OMS on input line 568.
\macc@depth=\count93
\c@MaxMatrixCols=\count94
\dotsspace@=\muskip10
\c@parentequation=\count95
\dspbrk@lvl=\count96
\tag@help=\toks15
\row@=\count97
\column@=\count98
\maxfields@=\count99
\andhelp@=\toks16
\eqnshift@=\dimen106
\alignsep@=\dimen107
\tagshift@=\dimen108
\tagwidth@=\dimen109
\totwidth@=\dimen110
\lineht@=\dimen111
\@envbody=\toks17
\multlinegap=\skip44
\multlinetaggap=\skip45
\mathdisplay@stack=\toks18
LaTeX Info: Redefining \[ on input line 2666.
LaTeX Info: Redefining \] on input line 2667.
) (/usr/share/texmf-tetex/tex/latex/amsfonts/amssymb.sty
Package: amssymb 2002/01/22 v2.2d
(/usr/share/texmf-tetex/tex/latex/amsfonts/amsfonts.sty
Package: amsfonts 2001/10/25 v2.2f
\symAMSa=\mathgroup4
\symAMSb=\mathgroup5
LaTeX Font Info:    Overwriting math alphabet `\mathfrak' in version `bold'
(Font)                  U/euf/m/n --> U/euf/b/n on input line 132.
)) (/usr/share/texmf-tetex/tex/generic/babel/babel.sty
Package: babel 2004/11/20 v3.8d The Babel package
(/usr/share/texmf-tetex/tex/generic/babel/frenchb.ldf
Language: french 2004/04/02 v1.6f French support from the babel system
(/usr/share/texmf-tetex/tex/generic/babel/babel.def
File: babel.def 2004/11/20 v3.8d Babel common definitions
\babel@savecnt=\count100
\U@D=\dimen112
)
Package babel Info: Making : an active character on input line 219.
Package babel Info: Making ; an active character on input line 220.
Package babel Info: Making ! an active character on input line 221.
Package babel Info: Making ? an active character on input line 222.
\parindentFFN=\dimen113
\std@mcc=\count101
\dec@mcc=\count102
*************************************
* Local config file frenchb.cfg used
*
(/usr/share/texmf-tetex/tex/generic/babel/frenchb.cfg)))

LaTeX Warning: You have requested, on input line 58, version
               `2005/11/23' of package babel,
               but only version
               `2004/11/20 v3.8d The Babel package'
               is available.

(/usr/share/texmf-tetex/tex/latex/base/inputenc.sty
Package: inputenc 2004/02/05 v1.0d Input encoding file
(/usr/share/texmf-tetex/tex/latex/base/latin1.def
File: latin1.def 2004/02/05 v1.0d Input encoding file
))

LaTeX Warning: You have requested, on input line 59, version
               `2006/05/05' of package inputenc,
               but only version
               `2004/02/05 v1.0d Input encoding file'
               is available.

(./25227-t.aux)
\openout1 = `25227-t.aux'.

LaTeX Font Info:    Checking defaults for OML/cmm/m/it on input line 85.
LaTeX Font Info:    ... okay on input line 85.
LaTeX Font Info:    Checking defaults for T1/cmr/m/n on input line 85.
LaTeX Font Info:    ... okay on input line 85.
LaTeX Font Info:    Checking defaults for OT1/cmr/m/n on input line 85.
LaTeX Font Info:    ... okay on input line 85.
LaTeX Font Info:    Checking defaults for OMS/cmsy/m/n on input line 85.
LaTeX Font Info:    ... okay on input line 85.
LaTeX Font Info:    Checking defaults for OMX/cmex/m/n on input line 85.
LaTeX Font Info:    ... okay on input line 85.
LaTeX Font Info:    Checking defaults for U/cmr/m/n on input line 85.
LaTeX Font Info:    ... okay on input line 85.
LaTeX Info: Redefining \dots on input line 85.

Overfull \hbox (7.4968pt too wide) in paragraph at lines 111--111
[]\OT1/cmtt/m/n/10 The Project Gutenberg EBook of Sur quelques applications des
 fonctions[] 
 []


Overfull \hbox (12.74675pt too wide) in paragraph at lines 111--111
[]\OT1/cmtt/m/n/10 *** START OF THIS PROJECT GUTENBERG EBOOK SUR QUELQUES APPLI
CATIONS ***[] 
 []

[1


{/var/lib/texmf/fonts/map/pdftex/updmap/pdftex.map}] [2

] [3

] [4]
LaTeX Font Info:    Try loading font information for U+msa on input line 197.
(/usr/share/texmf-tetex/tex/latex/amsfonts/umsa.fd
File: umsa.fd 2002/01/19 v2.2g AMS font definitions
)
LaTeX Font Info:    Try loading font information for U+msb on input line 197.
(/usr/share/texmf-tetex/tex/latex/amsfonts/umsb.fd
File: umsb.fd 2002/01/19 v2.2g AMS font definitions
) [1

] [2] [3] [4] [5] [6] [7] [8] [9] [10] [11] [12]
Overfull \hbox (0.37708pt too wide) in paragraph at lines 858--864
\OT1/cmr/m/n/10.95 sans qu'il soit be-soin d'in-tro-duire un fac-teur constant 
dans le se-cond membre,
 []

[13] [14] [15] [16] [17] [18] [19] [20] [21] [22]
LaTeX Font Info:    Try loading font information for OMS+cmr on input line 1334
.
(/usr/share/texmf-tetex/tex/latex/base/omscmr.fd
File: omscmr.fd 1999/05/25 v2.5h Standard LaTeX font definitions
)
LaTeX Font Info:    Font shape `OMS/cmr/m/n' in size <10.95> not available
(Font)              Font shape `OMS/cmsy/m/n' tried instead on input line 1334.

[23] [24] [25] [26] [27] [28]
LaTeX Font Info:    Try loading font information for U+euf on input line 1666.
(/usr/share/texmf-tetex/tex/latex/amsfonts/ueuf.fd
File: ueuf.fd 2002/01/19 v2.2g AMS font definitions
) [29] [30] [31] [32] [33] [34] [35] [36] [37] [38] [39]
LaTeX Font Info:    Try loading font information for U+lasy on input line 2285.

(/usr/share/texmf-tetex/tex/latex/base/ulasy.fd
File: ulasy.fd 1998/08/17 v2.2e LaTeX symbol font definitions
) [40] [41] [42] [43] [44] [45] [46] [47] [48] [49] [50] [51] [52] [53] [54] [5
5] [56] [57] [58] [59] [60] [61] [62] [63] [64] [65] [66] [67] [68] [69] [70] [
71] [72] [73] [74] [75] [76] [77] [78] [79] [80] [81] [82] [83] [84] [85] [86] 
[87] [88] [89] [90] [91] [92] [93] [94] [95] [96] [97] [98] [99] [100] [101] [1
02] [103] [104] [105] [106] [107] [108] [109] [110] [111] [112] [113] [114] [11
5] [116] [117] [118] [119] [120] [121] [122] [123] [124] [125] [126] [127] [128
] [129] [130] [131] [132] [133] [134] [135] [136] [137] [138] [139] [140] [141]
[142] [143] [144] [145] [146] [147] [148] [149] [150] [151] [152] [153

] [154] [155] [156] [157] [158] [159] [160] [161] (./25227-t.aux) ) 
Here is how much of TeX's memory you used:
 1868 strings out of 94500
 21561 string characters out of 1175771
 90845 words of memory out of 1000000
 5025 multiletter control sequences out of 10000+50000
 16157 words of font info for 62 fonts, out of 500000 for 2000
 580 hyphenation exceptions out of 8191
 26i,21n,24p,246b,367s stack positions out of 1500i,500n,5000p,200000b,5000s
PDF statistics:
 607 PDF objects out of 300000
 0 named destinations out of 131072
 1 words of extra memory for PDF output out of 65536
</usr/share/
texmf-tetex/fonts/type1/bluesky/euler/eufm7.pfb></usr/share/texmf-tetex/fonts/t
ype1/bluesky/cm/cmsy5.pfb></usr/share/texmf-tetex/fonts/type1/bluesky/cm/cmr5.p
fb></usr/share/texmf-tetex/fonts/type1/bluesky/cm/cmmi5.pfb></usr/share/texmf-t
etex/fonts/type1/bluesky/symbols/msam10.pfb></usr/share/texmf-tetex/fonts/type1
/bluesky/latex/lasy10.pfb></usr/share/texmf-tetex/fonts/type1/bluesky/cm/cmsy9.
pfb></usr/share/texmf-tetex/fonts/type1/bluesky/cm/cmmi9.pfb></usr/share/texmf-
tetex/fonts/type1/bluesky/euler/eufm10.pfb></usr/share/texmf-tetex/fonts/type1/
bluesky/cm/cmbx12.pfb></usr/share/texmf-tetex/fonts/type1/bluesky/cm/cmti10.pfb
></usr/share/texmf-tetex/fonts/type1/bluesky/cm/cmmi6.pfb></usr/share/texmf-tet
ex/fonts/type1/bluesky/cm/cmsy6.pfb></usr/share/texmf-tetex/fonts/type1/public/
tt2001/fmex8.pfb></usr/share/texmf-tetex/fonts/type1/bluesky/cm/cmsy10.pfb></us
r/share/texmf-tetex/fonts/type1/bluesky/cm/cmr9.pfb></usr/share/texmf-tetex/fon
ts/type1/bluesky/cm/cmti9.pfb></usr/share/texmf-tetex/fonts/type1/bluesky/cm/cm
r6.pfb></usr/share/texmf-tetex/fonts/type1/bluesky/cm/cmsy8.pfb></usr/share/tex
mf-tetex/fonts/type1/bluesky/cm/cmmi8.pfb></usr/share/texmf-tetex/fonts/type1/b
luesky/cm/cmex10.pfb></usr/share/texmf-tetex/fonts/type1/bluesky/cm/cmr8.pfb></
usr/share/texmf-tetex/fonts/type1/bluesky/cm/cmmi10.pfb></usr/share/texmf-tetex
/fonts/type1/bluesky/cm/cmcsc10.pfb></usr/share/texmf-tetex/fonts/type1/bluesky
/cm/cmr12.pfb></usr/share/texmf-tetex/fonts/type1/bluesky/cm/cmr17.pfb></usr/sh
are/texmf-tetex/fonts/type1/bluesky/cm/cmr10.pfb></usr/share/texmf-tetex/fonts/
type1/bluesky/cm/cmtt10.pfb>
Output written on 25227-t.pdf (165 pages, 725173 bytes).
